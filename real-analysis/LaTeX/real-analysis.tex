\documentclass[12pt, a4paper]{article}
\usepackage{ctex}

\usepackage[margin=1in]{geometry}
\usepackage{
  color,
  clrscode,
  amssymb,
  ntheorem,
  amsmath,
  listings,
  fontspec,
  xcolor,
  supertabular,
  multirow,
  mathtools,
  mathrsfs
}
\definecolor{bgGray}{RGB}{36, 36, 36}
\usepackage[
  colorlinks,
  linkcolor=bgGray,
  anchorcolor=blue,
  citecolor=green
]{hyperref}
\newfontfamily\courier{Courier}

\theoremstyle{margin}
\theorembodyfont{\normalfont}
\newtheorem{thm}{定理}
\newtheorem{cor}[thm]{推论}
\newtheorem{pos}[thm]{命题}
\newtheorem{lemma}[thm]{引理}
\newtheorem{defi}[thm]{定义}
\newtheorem{std}[thm]{标准}
\newtheorem{imp}[thm]{实现}
\newtheorem{alg}[thm]{算法}
\newtheorem{exa}[thm]{例}
\newtheorem{prob}[thm]{问题}
\DeclareMathOperator{\sft}{E}
\DeclareMathOperator{\idt}{I}
\DeclareMathOperator{\spn}{span}
\DeclareMathOperator*{\agm}{arg\,min}
\newcommand{\pr}{\prime}
\newcommand{\tr}{^\intercal}
\newcommand{\st}{\text{s.t.}}
\newcommand{\hp}{^\prime}
\newcommand{\ms}{\mathscr}
\newcommand{\mn}{\mathnormal}
\newcommand{\tbf}{\textbf}
\newcommand{\mbf}{\mathbf}
\newcommand{\fl}{\mathnormal{fl}}
\newcommand{\f}{\mathnormal{f}}
\newcommand{\g}{\mathnormal{g}}
\newcommand{\R}{\mathbf{R}}
\newcommand{\Q}{\mathbf{Q}}
\newcommand{\JD}{\textbf{D}}
\newcommand{\rd}{\mathrm{d}}
\newcommand{\str}{^*}
\newcommand{\vep}{\varepsilon}
\newcommand{\lhs}{\text{L.H.S}}
\newcommand{\rhs}{\text{R.H.S}}
\newcommand{\con}{\text{Const}}
\newcommand{\oneton}{1,\,2,\,\dots,\,n}
\newcommand{\aoneton}{a_1a_2\dots a_n}
\newcommand{\xoneton}{x_1,\,x_2,\,\dots,\,x_n}
\newcommand\thmref[1]{定理~\ref{#1}}
\newcommand\lemmaref[1]{引理~\ref{#1}}
\newcommand\defref[1]{定义~\ref{#1}}
\newcommand\posref[1]{命题~\ref{#1}}
\newcommand\secref[1]{节~\ref{#1}}
\newcommand\equref[1]{(\ref{#1})}
\newcommand\figref[1]{图 \ref{#1}}
\newcommand\corref[1]{推论~\ref{#1}}
\newcommand\exaref[1]{例~\ref{#1}}
\newcommand\algref[1]{算法~\ref{#1}}
\newcommand{\remark}{\paragraph{评注}}
\newcommand{\example}{\paragraph{例}}
\newcommand{\proof}{\paragraph{证明}}


\title{实分析$\,$笔记}
\author{任云玮}
\date{}


\begin{document}
\maketitle
\tableofcontents
\newpage

\section{测度论}

\subsection{绪论}

  \begin{lemma}
    设矩形$R$由有限个几乎不相交的矩形组成,即$R=\bigcup_{k=1}^N R_k$,则
    $|R|=\sum_{k=1}^N|R_k|$.
  \end{lemma}

  \begin{lemma}
    \label{lemma: 矩形并}
    设$R,R_1,\dots,R_N$为矩形且$R\subset\bigcup_{k=1}^N R_k$,则
    $|R|\le \sum_{k=1}^N|R_k|$.
  \end{lemma}

  \begin{thm}
    设$\mcal{O}\subset\R$为开集,则$\mcal{O}$可以被唯一地写成可数个不相交开区间的并.
  \end{thm}
  \proof
    首先证明可以用不相交开区间覆盖. 分为两步,首先逐点地构造覆盖它的最大开区间,接着
    证明若两个“最大开区间”相交,则它们相等. 而对于可数,只需要用从每个区间中取出一个
    有理数作为编号即可.$\quad\blacksquare$
  \remark
    这一定义给出了定义$R$中区间的测度的思路,而定义$R^n$中集合的测度的思路,也和这个是
    类似的,但是需要经过一定的修改.

  \begin{thm}
    $R^d$中的开集$\mcal{O}$,$d\ge 1$,可被表示成可数个几乎不相交的立方体的并.
  \end{thm}
  \proof
    TODO
  \remark
    注意,并不能保证这种表示方法是唯一的,所以并不能直接用它来定义“面积”.

\subsection{外测度}

  \begin{defi}[外测度]
    \label{defi: 外测度}
    对于$E\subset\R^n$,定义$E$的外测度为
    \[
      m_*(E) = \inf\sum_{j=1}^\infty |Q_j|.
    \]
    其中$\{Q_j\}$取遍$E$的所有可数闭立方覆盖,$|Q_j|$为立方体$Q_j$的体积.
  \end{defi}
  \remark
    这一定义的动机在于,立方体$|Q_j|$的定义是明确,而我们尝试用立方体来覆盖原来的集合,
    我们可以通过把立方体取得足够小,来使得覆盖尽可能地精确. 而这一行为的极限,即其下
    确界,可以认为就是原来集合的测度.

  \begin{pos}
    $\,$
    \begin{enumerate}
      \item $\R^n$中的点的测度为零.
      \item 闭立方$Q$的测度$m_*(Q)$等于它的体积$|Q|$,开立方也是.
      \item $\R^n$的测度为$\infty$.
    \end{enumerate}
  \end{pos}
  \proof
    下给出[2.]的证明的概要. 需要证明$m_*(Q)\le|Q|$及其反向. 对于前者,由于$Q$可以被其
    自身覆盖,所以根据外测度定义中的“下确界”,$m_*(Q)\le|Q|$成立. 而对于后者,考虑证明
    $|Q|\le (1+\vep)m_*(Q)$对任意$\vep>0$成立. 对于任意$Q$的闭立方覆盖$\{Q_j\}$,
    可以取一个每个立方都比原来稍大的开立方覆盖$\{S_j\}$,$|S_j|=(1+\vep)|Q_j|$. 由于
    $Q$是紧集,所以可以取出有限子覆盖. 取对应的闭包再将对应的体积相加即可证明.\par
    这一证明的动机在于,$m_*$的定义中是下确界,很难证明下确界大于等于某个东西,所以考虑证明
    对于任意的立方覆盖,都成立大于等于号,从而导出下确界大于等于. 而由于\lemmaref{lemma:
    矩形并}中只给出了有限的情况,所以需要紧集取出一个有限覆盖.$\quad\blacksquare$

  \begin{lemma}
    对任意$\vep>0$,存在一个覆盖$E\subset\bigcup_{j=1}^\infty Q_j$,成立
    \[
      \sum_{j=1}^\infty m_*(Q_j) \le m_*(E)+\vep.
    \]
  \end{lemma}

  \begin{thm}[外测度的性质]
    \label{thm: 外测度的性质}
    $\,$
    \begin{enumerate}
      \item 单调性:若$E_1\subset E_2$,则$m_*(E_1)\le m_*(E_2)$.
      \item 可数可加性:若$E=\bigcup_{j=1}^\infty E_j$,则$m_*(E)\le
        \sum_{j=1}^\infty m_*(E_j)$.
      \item 设$E\subset\R^d$,则$m_*(E)=\inf m_*(\mcal{O})$,其中
        $\mcal{O}$取遍所有包含$E$的开集.
      \item 若$E=E_1\cup E_2$且$d(E_1,E_2)>0$,则$m_*(E)=m_*(E_1)+m_*(E_2)$.
      \item 若$E$是可数个几乎不相交的立方体$\{Q_j\}$的并,则
        $m_*(E)=\sum_{j=1}^\infty|Q_j|$.
    \end{enumerate}
  \end{thm}
  \remark
    其中[3.]表示任意集合的外测度可以用开集的测度逼近.
  \proof
    关于[2.],按照定义展开$m_*(E_j)$,取满足$\sum_{j=1}^\infty|Q_{k,j}
    \le m_*(E_j) + \frac{\vep}{2^j}|$的立方覆盖. 则$\{Q_{k,j}\}$构成了
    $E$的一个覆盖,再利用$m_*(E)$的定义以及几何级数即可证明.\par
    对于[3.],$m_*(E)\le\inf m_*(\mcal{O})$是显然的,下仅证明另一方向.
    首先考虑$\rhs$的含义,它是$m_*(\mcal{O})$的下确界,要证明某个东西大于它,
    只需要构造出一个$\mcal{O}$并让那个东西大于那一特定的$m_*(\mcal{O})$即可.
    再考虑$lhs$,和之前相同,可以将它看作$m_*(E)+\vep$,而这又可以换成一个稍小
    一点的立方覆盖. 将这一立方覆盖里的每一个立方稍稍放大,即得到了一个开立方覆盖.
    它们的并是一个开集,这就是之前所需要构造的$\mcal{O}$.\par
    [4.]和[5.]的证明中所用的技巧和之前的是相似的.$\quad\blacksquare$

\subsection{可测集和Lebesgue测度}

  \begin{defi}[Lebesgue测度]
    称$E\subset\R^d$为Lebesgue可测的,若对于任意$\vep>0$,存在开集
    $\mcal{O}\supset E$,成立$m_*(\mcal{O}-E)\le\vep$. 对于可测集
    $E$,定义Lebesgue测度$m(E)=m_*(E)$.
  \end{defi}
  \remark
    虽然每一个集合$E$都有外测度,但是只有特定的一类集合的外测度有一些比较好的
    性质,因此我们需要定义Lebesgue测度.

  \begin{lemma}
    若$F$为闭集,$K$为紧集,且$F$和$K$不相交,则$d(F,K)>0$.
  \end{lemma}

  \begin{pos}[可测集的性质]
    $\,$
    \begin{enumerate}
      \item $\R^d$中的开集都可测.
      \item 若$F$是一个外测度为零的集合的子集,则$F$可测.
      \item 可数个可测集的并可测.
      \item 闭集可测.
      \item 可测集的补集可测.
      \item 可测集的可数交可测.
    \end{enumerate}
  \end{pos}
  \proof
    [1.]是显然的. [2.]可以利用\thmref{thm: 外测度的性质}[3.]证明. \par
    对于[3.]只需要证明紧集可测即可,因为闭集可以被表达成可数个紧集的并的形式,之后
    利用[2.]即可完成证明. 而证明紧集$F$,考虑定义,即对任意$\vep>0$构造$\mcal{O}$
    使成立$m_*(\mcal{O}-F)\le\vep$. 根据\thmref{thm: 外测度的性质}[3.]可以
    构造出$m_*(\mcal{O})\le m_*(F)+\vep$的$\mcal{O}$,但是由于$m_*$和集合
    的相减不兼容,所以仍需要进一步的处理. 考虑$\mcal{O}-F$,它实际上是包在$F$外
    的一层集合,它可以被可数个几乎不相交的立方覆盖,而根据\thmref{thm: 外测度的性质}
    [3.],它们是从$m_*$中拆出来的. 再利用前任意有限个,进一步处理即可.$\quad\blacksquare$

  \begin{thm}
    设$E_1,E_2,\dots$为不相交可测集且$E=\bigcup_{j=1}^\infty E_j$,则
    \[
      m(E) = \sum_{j=1}^\infty m(E_j).
    \]
  \end{thm}

  \begin{defi}[$\sigma$-代数]
    称一个$\R^d$的子集组成的集合为一个$\sigma$-代数,若它在可数交、可数并以及
    取补集操作下封闭.
  \end{defi}

  \begin{defi}[Borel集]
    Borel集$\mcal{B}_{\R^d}$是指含有所有$\R^d$中开集的最小的$\sigma$-代数,
    即对任意含有所有开集的集合$S$,成立$\mcal{B}_{\R^d}\subset S$.
  \end{defi}
  \remark
    由于$\sigma$-代数的交依然是$\sigma$-代数,所以也可以定义Borel集为所有含有
    所有开集的$\sigma$-代数的交,这给出了存在性和唯一性.

  \begin{defi}
    定义可数个开集的交集的全体为$G_{\delta}$;定义可数个闭集的并集的全体
    为$F_{\sigma}$.
  \end{defi}

  \begin{pos}
    设$E\subset\R^d$,以下命题等价:
    \begin{enumerate}
      \item $E$可测.
      \item $E$与$G_\delta$仅相差一个零测集.
      \item $E$与$F_\sigma$仅相差一个零测集.
    \end{enumerate}
  \end{pos}


\end{document}
