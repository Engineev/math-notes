\documentclass[12pt, a4paper]{article}
\usepackage{ctex}

\usepackage[margin=1in]{geometry}
\usepackage{
  color,
  clrscode,
  amssymb,
  ntheorem,
  amsfonts,
  amsmath,
  listings,
  fontspec,
  xcolor,
  supertabular,
  multirow,
  mathtools,
  mathrsfs,
}
\definecolor{bgGray}{RGB}{36, 36, 36}
\usepackage[
  colorlinks,
  linkcolor=bgGray,
  anchorcolor=blue,
  citecolor=green
]{hyperref}
\newfontfamily\courier{Courier}

\theoremstyle{margin}
\theorembodyfont{\normalfont}
\newtheorem{thm}{定理}
\newtheorem{cor}[thm]{推论}
\newtheorem{pos}[thm]{命题}
\newtheorem{lemma}[thm]{引理}
\newtheorem{defi}[thm]{定义}

\DeclareMathOperator{\rank}{rank}
\DeclareMathOperator{\adj}{adj}
\DeclareMathOperator{\tr}{tr}
\DeclareMathOperator{\diag}{diag}
\DeclareMathOperator{\nul}{null}
\DeclareMathOperator{\range}{range}
\DeclareMathOperator{\spn}{span}
% \DeclareMathOperator{\deg}{deg}

\newcommand{\hp}{^\prime}
\newcommand{\vep}{\varepsilon}
\newcommand{\inv}{^{-1}}
\newcommand{\rd}{\mathrm{d}}

\renewcommand{\Im}{\text{Im}}
\renewcommand{\Re}{\text{Re}}



\title{实分析$\,$笔记}
\author{任云玮}
\date{}


\begin{document}
\maketitle
\tableofcontents

\newpage
\setcounter{section}{2}
\section{Lebesgue Measure}
  \paragraph{56. Definition of the outer measure}
    注意定义中的下确界部分,这表明如果我们要证明$m^*A\ge k$,则只需要证明对任意$\{I_n\}$,
    $\sum l(I_n)\ge k$即可;同时这表示在$m^*A<\infty$的情况下,总可以找到可列个开区间,
    使得$\sum l(I_k) \le m^*A + K\vep$.\par
    注意,外侧度并非可数可加测度. 
  % end

  \paragraph{56. Proof of Proposition 1}
    关于$[a,b]$情况中$(a_i,b_i)$的构造:将$I_n$中的线段从左到右排列,依次取覆盖了前一个
    $b_i$的线段. 对于任意的有限区间,用一个闭区间从内部逼近它并利用之前的结论,这样可以得到一
    边的结论. 同时利用$m^*I = m^*\bar{I}=l(\bar{I})=l(I)$来得到另一边,注意其中用到了
    $\bar{I}\backslash I$至多包含两个点且单个点的外侧度为零. 对于无穷区间的情况,同样是利用
    内部的闭区间. 
  % end

  \paragraph{p58. Proposition}
    这一命题表明任意Borel集都可以用开集在外侧度意义上从外部逼近;如果考虑是用$G_\delta$集,
    则可以直接从外部在外侧度意义上相等. 
  % end

  \paragraph{p58. Definition}
    实际上这一定义意味着,如果我们定义相应的“内测度”$m_*$,则有$m^*(E)=m_*(E)$. 首先对于$E$,
    一个十分符合直觉的定义“内测度”的方法为$m_*(E)=m_*(X)-m_*(X\backslash E)$,其中$X$是
    某个包含$X$的集合. 而这一定义的是实际上就是说对于任意$X$(可能不包含$E$),这一性质都是成
    立的. \par
    同时注意$\lhs\le\rhs$是自然成立的,所以只需要验证$\lhs\ge\rhs$即可. 
  % end

  \paragraph{p60. Lemma 11}
    注意我们还没有证明开区间都是Lebesgue可测的,但我们已经证明了对于区间,外测度和长度是一致
    的. 
  % end

  \paragraph{p61. Proof of Theorem 12}
    定理10表明$\mathfrak{M}$是一个$\sigma$-代数且根据引理11,它包含所有$(a,\infty)$;
    同时易证Boreal集$\mathfrak{B}$可由全体$(a,\infty)$生成,即它是包含所以$(a,\infty)$
    的最小的$\sigma$-代数. 因此$\mathfrak{B}\subset\mathfrak{M}$,所以Borel集都可测. 
  % end

  \paragraph{p64. Proposition 15}
    这一命题表明可测集可以用开集从外部逼近,用闭集从内部逼近. 如果考虑$G_\delta$集和
    $F_\sigma$,则上述可以在测度意义下近似. 同时若外侧度有限,则可以用有\textbf{限开}区间的
    并来逼近. 上述内容反之亦成立. 
  % end

  \paragraph{p68. Proof of Proposition 19}
    这里$\bigcup_r\cdots$可以被诠释为:枚举所有有理数$r$,对于每一个$r$,它对应了$\{x\,:
    \,f(x)+g(x)<\alpha\}$这一集和中的一部分. 取它们的并集则得到了完整的集合. 
  % end

  \paragraph{p69. Proposition 22}
    可测函数可以用简单函数近似;简单函数可以用阶梯函数近似;阶梯函数可以用连续函数近似. 具体见
    习题23. 另外,这里的近似指的是误差超过$\vep$的集合很小,在习题31中给出了另一种形式:不相
    等的集合很小. 
  % end 

  \paragraph{p72. Proposition 23}
    注意这里的结论并非一致收敛,这里$A$的选取是可以依赖于$\vep$的,即“任意$\vep$,$\delta$,
    存在$A$,……”;但是一致收敛的结论实际上也是成立的,具体内容见习题30.
  % end

  \paragraph{p85. Definition}
    注意在之前的章节中,对于$f$我们不仅要求有界,还要求了finitely-supported.
  % end

  \paragraph{p86. Theorem 9}
    首先根据命题6,在$f$有界的情况下才可以保证$\int f_n$的极限是存在的,所以$\rhs$在这里仅
    仅是下极限. \par
    一个表明不等号可以严格成立的例子是:$f_n(x)=n\chi_{0,1/n}$,它几乎处处收敛于$0$,但是
    $\int_[0,1] f_n$始终为$1$. 
  % end

% end

\newpage
\section{The Lebesgue Integral}
  \paragraph{p79. Proof of Proposition 3}
    在已知上下积分相等的前提下证明$f$可测:条件意味着可以有一列
    \[
      \int \{\psi_n(x)-\varphi_n(x)\}\rd x < \frac{1}{n}.
    \]
    这一积分的不等式意味着满足$m\Delta_v<v/n$,即$\varphi_n$和$\psi_n$相差太多的点不会很
    多. 考虑$\psi^*$和$\varphi^*$,证明在极限情况下,这样的点(在此即不相等的点)为一个零测
    集. 
  % end

  \paragraph{p92.}
    这里之所以说大部分情况下,只需要有$|f_n|\le g$,则之前的结论仍然成立的原因在于,$g\pm 
    f_n$是非负的,而积分和极限都有线性性. 
  % end
% end

\newpage
\section{Differentiation and Integration}
  \paragraph{p97. Overview}
    \begin{equation}
      \label{eq:ch5-overview-1}
      \frac{\rd}{\rd x}\int_a^x f(y)\rd y = f(x)
    \end{equation}
    几乎处处成立,注意我们已知对于连续点它是成立的. 仅对某一类函数$f$,可以定义$f\hp$使成立
    \begin{equation}
      \label{eq:ch5-overview-2}
      \int_a^b f\hp(x)\rd x = f(b)-f(a).
    \end{equation}
  % end

  \paragraph{p98. Lemma 1}
    Vitali覆盖的十分常见的例子是考虑对$E$的每个点的所有领域,或者是类似于这样的集合。这一引理
    的作用在于把一个\href{https://www.quora.com/
    What-is-the-significance-of-the-Vitali-covering-lemma}
    {全局的问题化为局部的问题}. 注意在这里的描述中并没有“3倍”的概念,所以在估计上界的时候(?)
  % end

  \paragraph{p100. Theorem 3}
    这一定理对于单调递增的函数解决了p97中提出的问题:首先$f$几乎处处可微(从而几乎处处连续),
    从而解决了\eqref{eq:ch5-overview-1},同时也给出了唯一自然的$f\hp$的定义(由于是
    Lebesgue积分,所以我们无需考虑那些不可微的点);同时表明在仅有单调性的情况下,对
    于\eqref{eq:ch5-overview-2},只能成立不等式. \par
    证明的关键点在于证明$E_{u,v}$是零测集. 注意到函数值的差就是微分的积分,所以在这里我们实际
    上在估计$sv$和$su$. 
  % end

  \paragraph{p103. Lemma 4}
    这一引理可用于在$f$和$P$,$N$以及$T$之间转换. 
  % end

  \paragraph{p103. Theorem}
    通常来讲,这一结果只有对于有线性性的时候才真正有用,在诸如模长之类的只有三角不等式的场合则
    只能用于进行宽松的有界性估计。
  % end
  
  \paragraph{p105. Proof of Lemma 8}
    思路:反证法. 考虑满足$f>0$的非零测子集,通过一些近似得到满足$\int_{a_n}^{b_n}f\ne 0$
    的区间,从而推得矛盾. 
  % end

  \paragraph{p106. Proof of Lemma 9}
    我们只需要证明$F\hp$几乎处处存在且对任意$y$成立$\int_a^y(F\hp-f)=0$并应用引理8即可. 
    其中$F\hp$的几乎处处存在性由它的定义式及引理7得到. 下考虑$\int(F\hp-f)$. 首先把$F\hp$
    化为更方便处理的形式,由于我们已知其存在性,所以对于几乎所有$x$,成立
    \[
      F\hp(x) = \lim_{n\to\infty}\frac{F(x+1/n)-F(x)}{1/n}=\lim_{n\to\infty}f_n.
    \]
    首先我们尝试交换积分和极限运算. 由于这是一个一般的函数,我们考虑利用Lebesgue收敛定理,即我
    们需要证明$|f_n|\le g$对某个可积函数成立. 已知$|f|\le K<\infty$,所以
    \[
      |f_n(x)| = n\left|\int_x^{x+1/n}f(t)\rd t\right| \le K.
    \]
    从而有
    \[
      \int_a^y F\hp = \lim_{n\to\infty} n\int_a^y\{F(x+1/n)-F(x)\}\rd x.
    \]
    接下来处理该积分,我们有
    \[
      \int_a^y\{F(x+1/n)-F(x)\}\rd x = \int_y^{y+1/n}F - \int_a^{a+1/n}F.
    \]
    由于$F$的连续性(由引理7保证),我们有$n\int_y^{y+1/n}F\to F(y)$. 所以我们
    \[
      \int_a^y F\hp = F(y)-F(a)=\int_a^c f.
    \]
    证毕.
  % end

  \paragraph{p108. Fail to be absolutely continuous}
    首先如果一个函数不是几乎处处可微的,那么它不是绝对连续的。对于一个几乎处处可微的函数,定理
    14表明若它要是绝对连续的,那么它需要在微分后可以积回来。考虑Cantor函数,虽然它几乎处处可
    微,但是它导数全是零。直观上来讲,它的变化全集中在了不可微的点处。
  % end

  \paragraph{p108. Proof of Lemma 11}
    取$\vep=1$,把$[a,b]$分成长度$\le\delta$的子区间,插入这些子区间的端点。则对于每一组,
    成立$\sum|f(x_i)-f(x_{i-1})|<\vep$。从而整体上有$t\le K\vep$,其中$K$是任意大于组
    数的数。
  % end

  \paragraph{p109. Proof of Lemma 13}
    考虑绝对连续的定义,对$|f(x\hp_i)-f(x_i)|$的求和天然的把一个全局性质化为了一个局部性质.
    我们要证对于任意$c\in[a,b]$,成立$f(c)=f(a)$. 考虑Vitali覆盖定理,我们把$[a,c]$分为
    两类区间,其中一类是$f$在其上的变化充分小的,这由几乎处处$f\hp(x)=0$保证,另一方面则是总
    长度充分小的区间,在其上利用绝对连续性得到在其上的变化亦充分小。
  % end

  \paragraph{Sufficient condition of absolute continuity}
    Lipchitz continuity or being an indefinite integral.
  % end

  \paragraph{p113. Lemma 16}
    即把区间往右移动或单纯地向右拉伸,对应的弦的斜率增长。
  % end

  \paragraph{p113/114. Proposition 17/18}
    这两个命题描述了函数的凸性和其导数的关系。
  % end
  
% end


\end{document}
