\documentclass[12pt, a4paper]{article}
\usepackage{ctex}

\usepackage[margin=1in]{geometry}
\usepackage{
  color,
  clrscode,
  amssymb,
  ntheorem,
  amsmath,
  listings,
  fontspec,
  xcolor,
  supertabular,
  multirow,
  mathtools,
  mathrsfs
}
\definecolor{bgGray}{RGB}{36, 36, 36}
\usepackage[
  colorlinks,
  linkcolor=bgGray,
  anchorcolor=blue,
  citecolor=green
]{hyperref}
\newfontfamily\courier{Courier}

\theoremstyle{margin}
\theorembodyfont{\normalfont}
\newtheorem{thm}{定理}
\newtheorem{cor}[thm]{推论}
\newtheorem{pos}[thm]{命题}
\newtheorem{lemma}[thm]{引理}
\newtheorem{defi}[thm]{定义}
\newtheorem{std}[thm]{标准}
\newtheorem{imp}[thm]{实现}
\newtheorem{alg}[thm]{算法}
\newtheorem{exa}[thm]{例}
\newtheorem{prob}[thm]{问题}
\DeclareMathOperator{\sft}{E}
\DeclareMathOperator{\idt}{I}
\DeclareMathOperator{\spn}{span}
\DeclareMathOperator*{\agm}{arg\,min}
\newcommand{\pr}{\prime}
\newcommand{\tr}{^\intercal}
\newcommand{\st}{\text{s.t.}}
\newcommand{\hp}{^\prime}
\newcommand{\ms}{\mathscr}
\newcommand{\mn}{\mathnormal}
\newcommand{\tbf}{\textbf}
\newcommand{\mbf}{\mathbf}
\newcommand{\fl}{\mathnormal{fl}}
\newcommand{\f}{\mathnormal{f}}
\newcommand{\g}{\mathnormal{g}}
\newcommand{\R}{\mathbf{R}}
\newcommand{\Q}{\mathbf{Q}}
\newcommand{\JD}{\textbf{D}}
\newcommand{\rd}{\mathrm{d}}
\newcommand{\str}{^*}
\newcommand{\vep}{\varepsilon}
\newcommand{\lhs}{\text{L.H.S}}
\newcommand{\rhs}{\text{R.H.S}}
\newcommand{\con}{\text{Const}}
\newcommand{\oneton}{1,\,2,\,\dots,\,n}
\newcommand{\aoneton}{a_1a_2\dots a_n}
\newcommand{\xoneton}{x_1,\,x_2,\,\dots,\,x_n}
\newcommand\thmref[1]{定理~\ref{#1}}
\newcommand\lemmaref[1]{引理~\ref{#1}}
\newcommand\defref[1]{定义~\ref{#1}}
\newcommand\posref[1]{命题~\ref{#1}}
\newcommand\secref[1]{节~\ref{#1}}
\newcommand\equref[1]{(\ref{#1})}
\newcommand\figref[1]{图 \ref{#1}}
\newcommand\corref[1]{推论~\ref{#1}}
\newcommand\exaref[1]{例~\ref{#1}}
\newcommand\algref[1]{算法~\ref{#1}}
\newcommand{\remark}{\paragraph{评注}}
\newcommand{\example}{\paragraph{例}}
\newcommand{\proof}{\paragraph{证明}}


\title{实分析$\,$笔记}
\author{任云玮}
\date{}


\begin{document}
\maketitle
\tableofcontents

\newpage
\setcounter{section}{2}
\section{Lebesgue Measure}
  \paragraph{56. Definition of the outer measure}
    注意定义中的下确界部分,这表明如果我们要证明$m^*A\ge k$,则只需要证明对任意$\{I_n\}$,
    $\sum l(I_n)\ge k$即可;同时这表示在$m^*A<\infty$的情况下,总可以找到可列个开区间,
    使得$\sum l(I_k) \le m^*A + K\vep$.\par
    注意,外侧度并非可数可加测度。
  % end

  \paragraph{56. Proof of Proposition 1}
    关于$[a,b]$情况中$(a_i,b_i)$的构造:将$I_n$中的线段从左到右排列,依次取覆盖了前一个
    $b_i$的线段。对于任意的有限区间,用一个闭区间从内部逼近它并利用之前的结论,这样可以得到一
    边的结论。同时利用$m^*I = m^*\bar{I}=l(\bar{I})=l(I)$来得到另一边,注意其中用到了
    $\bar{I}\backslash I$至多包含两个点且单个点的外侧度为零。对于无穷区间的情况,同样是利用
    内部的闭区间。
  % end

  \paragraph{p58. Proposition}
    这一命题表明任意Borel集都可以用开集在外侧度意义上从外部逼近;如果考虑是用$G_\delta$集,
    则可以直接从外部在外侧度意义上相等。
  % end

  \paragraph{p58. Definition}
    实际上这一定义意味着,如果我们定义相应的“内测度”$m_*$,则有$m^*(E)=m_*(E)$. 首先对于$E$,
    一个十分符合直觉的定义“内测度”的方法为$m_*(E)=m_*(X)-m_*(X\backslash E)$,其中$X$是
    某个包含$X$的集合。而这一定义的是实际上就是说对于任意$X$(可能不包含$E$),这一性质都是成
    立的。\par
    同时注意$\lhs\le\rhs$是自然成立的,所以只需要验证$\lhs\ge\rhs$即可。
  % end

  \paragraph{p60. Lemma 11}
    注意我们还没有证明开区间都是Lebesgue可测的,但我们已经证明了对于区间,外测度和长度是一致
    的。
  % end

  \paragraph{p61. Proof of Theorem 12}
    定理10表明$\mathfrak{M}$是一个$\sigma$-代数且根据引理11,它包含所有$(a,\infty)$;
    同时易证Boreal集$\mathfrak{B}$可由全体$(a,\infty)$生成,即它是包含所以$(a,\infty)$
    的最小的$\sigma$-代数。因此$\mathfrak{B}\subset\mathfrak{M}$,所以Borel集都可测。
  % end

  \paragraph{p64. Proposition 15}
    这一命题表明可测集可以用开集从外部逼近,用闭集从内部逼近。如果考虑$G_\delta$集和
    $F_\sigma$,则上述可以在测度意义下近似。同时若外侧度有限,则可以用有\textbf{限开}区间的
    并来逼近。上述内容反之亦成立。
  % end

  \paragraph{p68. Proof of Proposition 19}
    这里$\bigcup_r\cdots$可以被诠释为:枚举所有有理数$r$,对于每一个$r$,它对应了$\{x\,:
    \,f(x)+g(x)<\alpha\}$这一集和中的一部分。取它们的并集则得到了完整的集合。
  % end

  \paragraph{p69. Proposition 22}
    可测函数可以用简单函数近似;简单函数可以用阶梯函数近似;阶梯函数可以用连续函数近似. 具体见
    习题23. 另外,这里的近似指的是误差超过$\vep$的集合很小,在习题31中给出了另一种形式:不相
    等的集合很小。
  % end 

  \paragraph{p72. Proposition 23}
    注意这里的结论并非一致收敛,这里$A$的选取是可以依赖于$\vep$的,即“任意$\vep$,$\delta$,
    存在$A$,……”;但是一致收敛的结论实际上也是成立的,具体内容见习题30.
  % end
% end


\end{document}
