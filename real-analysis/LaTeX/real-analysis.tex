\documentclass[12pt, a4paper]{article}
\usepackage{ctex}

\usepackage[margin=1in]{geometry}
\usepackage{
  color,
  clrscode,
  amssymb,
  ntheorem,
  amsmath,
  listings,
  fontspec,
  xcolor,
  supertabular,
  multirow,
  mathtools,
  mathrsfs
}
\definecolor{bgGray}{RGB}{36, 36, 36}
\usepackage[
  colorlinks,
  linkcolor=bgGray,
  anchorcolor=blue,
  citecolor=green
]{hyperref}
\newfontfamily\courier{Courier}

\theoremstyle{margin}
\theorembodyfont{\normalfont}
\newtheorem{thm}{定理}
\newtheorem{cor}[thm]{推论}
\newtheorem{pos}[thm]{命题}
\newtheorem{lemma}[thm]{引理}
\newtheorem{defi}[thm]{定义}
\newtheorem{std}[thm]{标准}
\newtheorem{imp}[thm]{实现}
\newtheorem{alg}[thm]{算法}
\newtheorem{exa}[thm]{例}
\newtheorem{prob}[thm]{问题}
\DeclareMathOperator{\sft}{E}
\DeclareMathOperator{\idt}{I}
\DeclareMathOperator{\spn}{span}
\DeclareMathOperator*{\agm}{arg\,min}
\newcommand{\pr}{\prime}
\newcommand{\tr}{^\intercal}
\newcommand{\st}{\text{s.t.}}
\newcommand{\hp}{^\prime}
\newcommand{\ms}{\mathscr}
\newcommand{\mn}{\mathnormal}
\newcommand{\tbf}{\textbf}
\newcommand{\mbf}{\mathbf}
\newcommand{\fl}{\mathnormal{fl}}
\newcommand{\f}{\mathnormal{f}}
\newcommand{\g}{\mathnormal{g}}
\newcommand{\R}{\mathbf{R}}
\newcommand{\Q}{\mathbf{Q}}
\newcommand{\JD}{\textbf{D}}
\newcommand{\rd}{\mathrm{d}}
\newcommand{\str}{^*}
\newcommand{\vep}{\varepsilon}
\newcommand{\lhs}{\text{L.H.S}}
\newcommand{\rhs}{\text{R.H.S}}
\newcommand{\con}{\text{Const}}
\newcommand{\oneton}{1,\,2,\,\dots,\,n}
\newcommand{\aoneton}{a_1a_2\dots a_n}
\newcommand{\xoneton}{x_1,\,x_2,\,\dots,\,x_n}
\newcommand\thmref[1]{定理~\ref{#1}}
\newcommand\lemmaref[1]{引理~\ref{#1}}
\newcommand\defref[1]{定义~\ref{#1}}
\newcommand\posref[1]{命题~\ref{#1}}
\newcommand\secref[1]{节~\ref{#1}}
\newcommand\equref[1]{(\ref{#1})}
\newcommand\figref[1]{图 \ref{#1}}
\newcommand\corref[1]{推论~\ref{#1}}
\newcommand\exaref[1]{例~\ref{#1}}
\newcommand\algref[1]{算法~\ref{#1}}
\newcommand{\remark}{\paragraph{评注}}
\newcommand{\example}{\paragraph{例}}
\newcommand{\proof}{\paragraph{证明}}


\title{实分析$\,$笔记}
\author{任云玮}
\date{}


\begin{document}
\maketitle
\tableofcontents
\newpage

\section{集合论导引}

  \begin{thm}
    \label{thm: 开集的闭方块表示}
    所有$\R^n$中的开集$\mcal{O}$都可以写作可数个几乎不相交的闭方块的并的形式.
  \end{thm}

  \begin{pos}[对称差]
    \label{pos: 对称差}
    $\,$
    \begin{enumerate}
      \item $(A_1\cup A_2)\Delta(B_1\cup B_2)\subset(A_1\Delta B_1)\cup(A_2\Delta B_2)$.
      \item TODO
    \end{enumerate}
  \end{pos}

\newpage
\section{测度论导引}

\newpage
\section{$\R^n$上的Lebesgue测度}

\subsection{方块的测度}

\subsection{$\vep^n$的测度}

\subsection{外测度}

  \begin{lemma}
    对于任意的$A_1,A_2\in\mcal{M}^*(\R^n)$,成立
    $|\mu^*A_1 - \mu^*A_2|\le\mu^*(A_1\Delta A_2)$.
  \end{lemma}

\subsection{Lebesgue测度}
  \begin{defi}[Lebesgue测度]
    称集合$A\in\mcal{M}^*(\R^n)$为Lebesgue可测,若对任意$\vep>0$,存在
    初等集$B$,使得
    \begin{equation}
      \label{equ: Lebesgue测度}
      \mu^*(A\Delta B) < \vep.
    \end{equation}
    记Lebesgue可测的集合全体为$\mcal{M}(\R^n)$. 定义$\mu$为$\mu^*$限制在
    $\mcal{M}(\R^n)$上的函数,称其为Lebesgue测度.
  \end{defi}
  \remark
    之所以既需要外测度也需要定义Lebesgue测度是因为虽然外测度是定义在所有集合上的,
    但是它不满足可加性;而Lebesgue虽然有诸如可加性等很好的性质,但它只是定义在$\R^n$的
    一个子集上面. 而如此定义可测的原因在于,对于初等集而言,外测度的性质是良好且直观的.

  \begin{pos}[Lebesgue测度的性质]
    Lebesgue测度满足如下性质
    \begin{enumerate}
      \item 对任意$A\in\mcal{M}(\R^n)$,成立$\mu A\ge 0$.
      \item $\vep^n\subset\mcal{M}(\R^n)$且若$A\in\vep^n$,则$\mu A=m A$.
    \end{enumerate}
  \end{pos}

  \begin{thm}
    $\mcal{M}(\R^n)$为环.
  \end{thm}
  \proof
    利用\posref{pos: 对称差}即可. $\quad\blacksquare$

  \begin{cor}
    $\mcal{M}([0, 1])$为一个代数.
  \end{cor}

  \begin{thm}
    $\mu$在$\mcal{M}(\R^n)$上满足可加性.
  \end{thm}
  \proof
    即证明,若$A=A_1\cupdot A_2$,则$\mu A = \mu A_1 + \mu A_2$. 其中$\mu A
    \le \mu A_1 + \mu A_2$可以由外测度的性质直接推出. 下考虑另一方向,尝试证明
    $\mu A > \mu A_1 + \mu A_2 - \vep$.\par
    对于固定的$\vep$,首先根据Lebesgue可测的定义取出基础集$B_i$,满足$\mu*(A_i
    \Delta B_i)<\vep/6$,同时可以证明$m(B_1\cap B_2)<\vep/3$. 用$B=B_1\cup
    B_2$来估计$A$即可完成证明.$\quad\blacksquare$

  \begin{thm}
    环$\mcal{M}(\R^n)$上的测度$\mu$是$\sigma$-可加的.
  \end{thm}

  \begin{thm}
    \label{thm: 可数并可测}
    设$A\in \mcal{M}^*(\R^n)$且$A=\bigcup_{k=1}^\infty A_k$,其中
    $A_k\in\mcal{M}(\R^n)$. 则$A\in\mcal{M}(\R^n)$.
  \end{thm}
  \remark
    注意,对于Lebesgue可测集,它的测度就是外测度.
  \proof
    首先将$A$重写为不相交集$A_k\hp=A_k\backslash\bigcup_{j=1}^{k-1}A_j$.
    考虑级数$\sum\mu A_k\hp$,根据条件,它是收敛的. 之后对于它前充分多的有限项
    所对应的集合的并,它是Lebesgue可测的,考虑它对应的初等集$B$. 而对于剩下的集合,
    它们的外测度足够小. $\quad\blacksquare$

  \begin{cor}
    可数个零测集的并依然是零测集.
  \end{cor}
  \proof
    利用外测度的$\sigma$-半可加性和\thmref{thm: 可数并可测}即可证明.

  \begin{defi}[完备]
    称定义在环$\mathcal{K}$上的测度$\mu$是完备的,若任意零测集的子集的测度也为零.
  \end{defi}

  \begin{thm}
    Lebesgue测度是完备的.
  \end{thm}

\subsection{广义测度}

  \begin{defi}
    定义$\hat{Q}_l$为中心在原点、边长为$2l$的$n$维立方块.
  \end{defi}

  \begin{defi}[$\sigma$-可测]
    \label{defi: sigma-可测}
    称$A\subset\R^n$广义Lebesgue可测或$\sigma$-可测,若对于任意
    $l\in\mathbb{N}$,集合$A\cap\hat{Q}_l$ Lebesgue可测. 同时,对于
    $\sigma$-可测的集合,定义
    \begin{equation}
      \label{equ: sigma-测度}
      \mu A = \lim_{l\to\infty}\mu(A\cap \hat{Q}_l).
    \end{equation}
    记$\sigma$可测的集合全体为$\mcal{M}_{\sigma}(\R^n)$.
  \end{defi}

  \begin{thm}
    $\mcal{M}(\R^n)\subset\mcal{M}_{\sigma}(\R^n)$.
  \end{thm}

  \begin{thm}
    若对于$A\in\mcal{M}_\sigma(\R^n)$,\equref{equ: sigma-测度}的极限为有限值,
    则$A\in\mcal{M}(\R^n)$.
  \end{thm}
  \remark
    这一定理表明,$\mcal{M}_\sigma(\R^n)$只是$\mcal{M}(\R^n)$加上了原来测度
    为无穷的那些集合. 即对于测度有限部分,这两种的按照不同方式定义的测度是相同的.
  \proof
    为证明$A\in\mcal{M}(\R^n)$,只需要利用\thmref{thm: 可数并可测},即验证
    $A\in\mcal{M}^*(\R^n)$且$A$可以拆成可数个Lebesgue可测集的并的形式.\par
    首先将利用$\{A_l\backslash A_{l-1}\}$拆分$A$,可以证明它们都是Lebesgue可测
    的. 之后只需要证明$A$的外测度有限即可. 注意
    \[
      \mu^* A \le \sum_{l=1}^\infty\mu^*(A_l\backslash A_{l-1})
      = \sum_{l=1}^\infty\mu(A_l\backslash A_{l-1}).
    \]
    而利用级数相关知识,可知$\rhs$收敛于有限值.$\quad\blacksquare$

  \begin{thm}
    $\mcal{M}_\sigma(\R^n)$是$\sigma$-代数,它的单位元为$\R^n$.
  \end{thm}

  \begin{cor}
    设$X\subset\R^n$为紧集,则$\mcal{M}(X)$为$\sigma$-代数.
  \end{cor}

  \begin{thm}
    $\R^n$中的开集都$\sigma$-可测.
  \end{thm}
  \proof
    利用$\mcal{M}_\sigma(\R^n)$是$\sigma$-代数这一事实以及\thmref{thm:
    开集的闭方块表示}即可.$\quad\blacksquare$

\subsection{Borel $\sigma-$代数}

  \begin{defi}[Borel $\sigma$-代数]
    \label{defi: Borel代数}
    定义$\R^n$上的Borel $\sigma$-代数是由开集全体生成的$\sigma$-代数,即
    \[
      \mcal{B}(\R^n) = \sigma(\mcal{T}(\R^n)) =
      \bigcap\{ \mcal{A}\subset\R^n\,|\, \mcal{A}\supset\mcal{T}(\R^n),
      \,\text{且$\mcal{A}$为$\sigma$-代数}\}.
    \]
    其中$\mcal{T}(\R^n)$表示$\R^n$上的开集全体.
  \end{defi}
  \remark
    这一定义表明Borel $\sigma$-代数是最小的包含了所有开集的$\sigma$-代数. 首先,
    对$\mcal{A}$的要求表明它是包含了所有开集的$\sigma$-代数,而$\bigcap$则表明
    它是最小的. 与此同时,由于$\sigma$-代数对于取补集操作封闭,所以也可以说Borel
    $\sigma$-代数是由全体闭集生成的. 另,常简称Borel $\sigma$-代数为Borel代数
    或Borel集.

  \begin{thm}
    Borel代数由$\R^n$中的闭方块全体$\mcal{K}(\R^n)$生成. 且Borel集Lebesgue可测.
  \end{thm}
  \remark
    这一定理意味着只需要定义了方块以及方块上的测度,那么对应的Borel集永远是可测的.
  \proof
    首先考虑定理的前半部分. 根据\thmref{thm: 开集的闭方块表示}可知,$\mcal{T}(\R^n)
    \subset\sigma(\mcal{K}(\R^n))$,从而$\sigma(\mcal{T}(\R^n))\subset
    \sigma(\mcal{K}(\R^n))$. 而根据\defref{defi: Borel代数}的评注,可知反向也
    成立. 因此$\mcal{B}(\R^n) = \sigma(\mcal{K}(\R^n))$.\par



\end{document}
