\documentclass[12pt, a4paper]{article}
\usepackage{ctex}

\usepackage[margin=1in]{geometry}
\usepackage{
  color,
  clrscode,
  amssymb,
  ntheorem,
  amsmath,
  listings,
  fontspec,
  xcolor,
  supertabular,
  multirow,
  mathtools,
  mathrsfs
}
\definecolor{bgGray}{RGB}{36, 36, 36}
\usepackage[
  colorlinks,
  linkcolor=bgGray,
  anchorcolor=blue,
  citecolor=green
]{hyperref}
\newfontfamily\courier{Courier}

\theoremstyle{margin}
\theorembodyfont{\normalfont}
\newtheorem{thm}{定理}
\newtheorem{cor}[thm]{推论}
\newtheorem{pos}[thm]{命题}
\newtheorem{lemma}[thm]{引理}
\newtheorem{defi}[thm]{定义}
\newtheorem{std}[thm]{标准}
\newtheorem{imp}[thm]{实现}
\newtheorem{alg}[thm]{算法}
\newtheorem{exa}[thm]{例}
\newtheorem{prob}[thm]{问题}
\DeclareMathOperator{\sft}{E}
\DeclareMathOperator{\idt}{I}
\DeclareMathOperator{\spn}{span}
\DeclareMathOperator*{\agm}{arg\,min}
\newcommand{\pr}{\prime}
\newcommand{\tr}{^\intercal}
\newcommand{\st}{\text{s.t.}}
\newcommand{\hp}{^\prime}
\newcommand{\ms}{\mathscr}
\newcommand{\mn}{\mathnormal}
\newcommand{\tbf}{\textbf}
\newcommand{\mbf}{\mathbf}
\newcommand{\fl}{\mathnormal{fl}}
\newcommand{\f}{\mathnormal{f}}
\newcommand{\g}{\mathnormal{g}}
\newcommand{\R}{\mathbf{R}}
\newcommand{\Q}{\mathbf{Q}}
\newcommand{\JD}{\textbf{D}}
\newcommand{\rd}{\mathrm{d}}
\newcommand{\str}{^*}
\newcommand{\vep}{\varepsilon}
\newcommand{\lhs}{\text{L.H.S}}
\newcommand{\rhs}{\text{R.H.S}}
\newcommand{\con}{\text{Const}}
\newcommand{\oneton}{1,\,2,\,\dots,\,n}
\newcommand{\aoneton}{a_1a_2\dots a_n}
\newcommand{\xoneton}{x_1,\,x_2,\,\dots,\,x_n}
\newcommand\thmref[1]{定理~\ref{#1}}
\newcommand\lemmaref[1]{引理~\ref{#1}}
\newcommand\defref[1]{定义~\ref{#1}}
\newcommand\posref[1]{命题~\ref{#1}}
\newcommand\secref[1]{节~\ref{#1}}
\newcommand\equref[1]{(\ref{#1})}
\newcommand\figref[1]{图 \ref{#1}}
\newcommand\corref[1]{推论~\ref{#1}}
\newcommand\exaref[1]{例~\ref{#1}}
\newcommand\algref[1]{算法~\ref{#1}}
\newcommand{\remark}{\paragraph{评注}}
\newcommand{\example}{\paragraph{例}}
\newcommand{\proof}{\paragraph{证明}}


\title{泛函分析$\,$笔记}
\author{任云玮}
\date{}


\begin{document}
\maketitle
\tableofcontents
\newpage

\setcounter{section}{1}
\section{Normed Spaces. Banach Spaces}
  \paragraph{p58. Normed Spaces and metric spaces}
    首先它们的共同点在于在其上都定义了“距离”的概念,而不同点在于,metric space的定义中就包含
    了“距离”metric,而对于normed space,它\textbf{首先}是一个向量空间,而对应的“距离”也是
    通过向量的“长度”norm来定义的. 
  % end

  \paragraph{p68. Schauder basis and Hamel basis}
    在Hamel basis的语境下,张成是指用\textbf{有限}的线性组合来张成,即$x=\sum_{i=1}^n
    \alpha_ib_i$. 而在Schauder basis的语境中,我们利用范数定义了收敛级数的概念,这允许我们
    用$\textbf{可数无限}$的线性组合来张成,即$x=\sum_{n=1}^\infty\alpha_ib_i$.
  % end

  \paragraph{p72. Lemma 2.4-1}
    这一定理的用处在于,通过范数得到各个坐标(及其和)的上界. 
    关于证明:
    \begin{enumerate}
      \item 假设存在$\|y_m\|\to 0$,得到坐标向量各个分量序列的有界性. 
      \item 利用有界性逐次得到一个坐标向量所有分量都收敛的子列$(y_{m_n})$,设收敛于$y$.
      \item 利用范数的连续性推出矛盾. 
    \end{enumerate}
  % end

  \paragraph{p75.}
    注意一个拓扑空间即为一个集合$X$和由它的满足一定性质的子集组成的集合$\mathcal{T}$. 例如
    一个度量空间和其中的开集全体. 而这里的意思是,对于两个等价范数,对应的拓扑空间是相同的. 
  % end
 
  \paragraph{p78. F. Riesz's Lemma}
    考虑$X=Z=\mathbb{R}^2$,$Y=\spn(x)$,则这一引理即是在说:在单位圆上我们可以找到一个
    点$z$,使得它离$x$轴上的任意点都有一定距离. 这一例子对于理解证明的思路也是很有帮助的. 
  % end

  \paragraph{p118. Proof of Theorem 2.10-2}
    利用$Y$的完备性逐点地定义$T$,证明$T\in B(X,Y)$且$T_n\to T$. 
  % end

  \paragraph{p121. Examples}
    在这些例子总所做的事情是:(1) 求出dual space中对应的norm;(2) 构造一个的线性双射. (3)
    证明该双射保norm. 其中(1)和(2)的顺序可以互换,而(1)时常被并入(3). 
    另外注意在此并没有dual Schauder basis的概念. 
  % end

  \paragraph{p121. 2.10-6}
    首先,我们构造从${l^1}\hp$到$l^\infty$的双射. 设$(e_k)=(\delta_{kj})$为$l^1$的一个
    Schauder基,则对于任意$f\in {l^1}\hp$,定义$Tf=(\gamma_k)=(f(e_k))$. 易得$T$是线
    性映射. 由于$f$的有界性,有
    \begin{equation}
      \label{eq:121-1}
      \sup_k|f(e_k)|\le \sup_k\|f\|\|e_k\|\le \|f\|,
    \end{equation}
    从而$(\gamma_k)\in l^\infty$. 同时对于任意$(\beta_k)\in l^\infty$,我们可以定义
    $g(x)=\sum_{k=1}^\infty x_k\beta_k \in {l^1}^*$. 同时由于$|x_k\beta_k| \le 
    (\sup_k\beta_k)|x_k|$而$\sum|x_k|<\infty$,所以$g\in{l^1}\hp$. 因此$T$是一个线性
    双射. \par
    接下来我们求出${l^1}\hp$上的对应范数. 首先注意到级数$\sum x_kf(e_k)$的收敛是一致的且
    $f$是连续的,从而有$f(x)=f(\sum x_ke_k)=\sum x_kf(e_k)$. 因此
    \[
      \|f\|=\sup_{\|x\|=1}|f(x)|=
      \sup_{\|x\|=1}\left|\sum_{k=1}^\infty x_kf(e_k)\right|\le
      \sup_k|f(e_k)|\sum_{k=1}^\infty|x_k|=\sup_k|f(e_k)|.
    \]
    而同时根据\equref{eq:121-1},有$\|f\|=\sup_k|f(e_k)|$.\par
    最后显然线性双射$T$是保范数的,从而它是一个同构. 因此$l^1$的对偶空间为$l^\infty$.
  % end

  \paragraph{p122. 2.10-7}
    首先和之前一样,取$(e_k)$作为$l^p$的一组Schauder基,对于任意$f\in {l^p}\hp$,定义$
    Tf=(f(e_k))$. 首先我们证明$(\gamma_k)=(f(e_k))\in l^q$. 即证明$\sum_{k=1}^n|
    \gamma_k|^q$关于$n$有界. 注意对于任意$x$,有
    \[
      \|f\|\|x\| \ge |f(x)| = \sum_{k=1}^\infty \xi_k\gamma_k.
    \]
    所以我们只需要取如(11)所示的$x_n$,即可使得不等式两边有所需的形式. 从而得$(\gamma_k)\in
    l^q$. 而反之对于$b=(\beta_k)\in l^q$,定义$g(x)=\sum_{k=1}^\infty\xi_k\beta_k$.
    $g$的线性性是显然的,而有界性有Hölder不等式保证. 所以$T$是一个双射,而$T$的线性性是显然
    的. 剩余的证明略. 
  % end

% end

\end{document}