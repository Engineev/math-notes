\documentclass[12pt, a4paper]{article}
\usepackage{ctex}

\usepackage[margin=1in]{geometry}
\usepackage{
  color,
  clrscode,
  amssymb,
  ntheorem,
  amsmath,
  listings,
  fontspec,
  xcolor,
  supertabular,
  multirow,
  mathtools,
  mathrsfs
}
\definecolor{bgGray}{RGB}{36, 36, 36}
\usepackage[
  colorlinks,
  linkcolor=bgGray,
  anchorcolor=blue,
  citecolor=green
]{hyperref}
\newfontfamily\courier{Courier}

\theoremstyle{margin}
\theorembodyfont{\normalfont}
\newtheorem{thm}{定理}
\newtheorem{cor}[thm]{推论}
\newtheorem{pos}[thm]{命题}
\newtheorem{lemma}[thm]{引理}
\newtheorem{defi}[thm]{定义}
\newtheorem{std}[thm]{标准}
\newtheorem{imp}[thm]{实现}
\newtheorem{alg}[thm]{算法}
\newtheorem{exa}[thm]{例}
\newtheorem{prob}[thm]{问题}
\DeclareMathOperator{\sft}{E}
\DeclareMathOperator{\idt}{I}
\DeclareMathOperator{\spn}{span}
\DeclareMathOperator*{\agm}{arg\,min}
\newcommand{\pr}{\prime}
\newcommand{\tr}{^\intercal}
\newcommand{\st}{\text{s.t.}}
\newcommand{\hp}{^\prime}
\newcommand{\ms}{\mathscr}
\newcommand{\mn}{\mathnormal}
\newcommand{\tbf}{\textbf}
\newcommand{\mbf}{\mathbf}
\newcommand{\fl}{\mathnormal{fl}}
\newcommand{\f}{\mathnormal{f}}
\newcommand{\g}{\mathnormal{g}}
\newcommand{\R}{\mathbf{R}}
\newcommand{\Q}{\mathbf{Q}}
\newcommand{\JD}{\textbf{D}}
\newcommand{\rd}{\mathrm{d}}
\newcommand{\str}{^*}
\newcommand{\vep}{\varepsilon}
\newcommand{\lhs}{\text{L.H.S}}
\newcommand{\rhs}{\text{R.H.S}}
\newcommand{\con}{\text{Const}}
\newcommand{\oneton}{1,\,2,\,\dots,\,n}
\newcommand{\aoneton}{a_1a_2\dots a_n}
\newcommand{\xoneton}{x_1,\,x_2,\,\dots,\,x_n}
\newcommand\thmref[1]{定理~\ref{#1}}
\newcommand\lemmaref[1]{引理~\ref{#1}}
\newcommand\defref[1]{定义~\ref{#1}}
\newcommand\posref[1]{命题~\ref{#1}}
\newcommand\secref[1]{节~\ref{#1}}
\newcommand\equref[1]{(\ref{#1})}
\newcommand\figref[1]{图 \ref{#1}}
\newcommand\corref[1]{推论~\ref{#1}}
\newcommand\exaref[1]{例~\ref{#1}}
\newcommand\algref[1]{算法~\ref{#1}}
\newcommand{\remark}{\paragraph{评注}}
\newcommand{\example}{\paragraph{例}}
\newcommand{\proof}{\paragraph{证明}}


\title{矩阵分析$\,$笔记}
\author{任云玮}
\date{}

\begin{document}
\maketitle
\tableofcontents

\newpage
\section{Eigenvalues, Eigenvectors, and Similarity}
\setcounter{subsection}{-1}
\subsection{Introduction}
% end
\subsection{The eigenvalue-eigenvector equation}
  \paragraph{p46}
    注意对于矩阵$A$和常数$k$,有
    \[
      (A+kI)x=\lambda\hp x \quad\Rightarrow\quad
      Ax = (\lambda\hp-k)x.
    \]
    因此在计算特征值的时候可以先给矩阵加上$I$的某个倍数来得到一个特征值更方便计算的矩阵. 
  % end

  \paragraph{p46. Theorem 1.1.6}
    这一定理的第一部分可以用来证明$\sigma(p(A))$非空并给出其中的某些值;而第二部分则限定了
    $\sigma(p(A))$中的值的可选范围. 如果我们在已知$p(A)$的特征值情况下讨论$A$,则这两部分的
    效果是反过来的. 
  % end

  \paragraph{p47. The proof of Theorem 1.1.9}
    我们的目的是证明任意$A\in M_n$有至少一个特征值$\lambda$,且它所对应的特征向量可以被表示
    为$g(A)y$的形式,其中$g(t)\in\mathbb{P}_{n-1}$.\par
    按照定理1.1.6,我们可以找一个多项式$p$,找$p(A)$的特征值. 考虑之前的观察以及$p$有常数项
    这件事,我们只需要说明$0\in\sigma(p(A)))$,即证$p(A)$是奇异的. 对于一个$p(A)$形式的矩
    阵,十分自然地引入了一组向量$(A^ny, \dots, y)$. 对于任意的$y\ne 0$,我们可以取恰当的
    $p$使得这组向量是线性相关,从而$p(A)$是奇异的. 将上述讨论反过来叙述即证明了命题的第一部
    分,其中关于$p$需选取度数最小的那一个. \par
    为得到所需要的$g$,我们只需要考虑把$p(A)y=0$进行因式分解,得
    \[
      (A-\lambda I)(g(A)y) = 0.
    \]
    由于$p$是满足线性相关的条件的多项式中度数最小的,所以$g(A)y$非奇异,从而它是一个特征向量. 
  % end
% end
\subsection{The characteristic polynomial and algebraic multiplicity}
  \paragraph{p54. Notes on Definition 1.2.14}
    $S_k(\lambda_1,\dots,\lambda_n)$可以理解为从$\lambda_1,\dots,\lambda_n$这$n$个
    数中取出$k$个,将它们相乘,然后把所有取法的结果相加. 
  % end
  \paragraph{p55. } TODO
% end
\subsection{Similarity}
  \paragraph{p59. Exercise 1.}
    由于相似变换不改变矩阵的特征值,所以有$S_k(S^{-1}AS)=S_k(A)$,因此成立$E_k(S^{-1}A
    S)=S_{k}(S^{-1}AS)=S_k(A)=E_k(A)$. 
  % end

  \paragraph{p59. Theorem 1.3.7}
    这一定理表明,如果我们有$k$个线形无关的特征向量,那么我们就可以把一个矩阵的前$k$列相似地
    化成为一个对角线是对应特征值的对角阵;反之亦成立,而这就保证了只要一个矩阵可对角化,我们就
    可以通过恰当地摆放这些特征向量来得到$S$,是的$S^{-1}AS$的对角线恰为$\lambda_1,\dots,
    \lambda_n$的任意一个排列,其中不同的$\lambda_i$可能相同. 

  % end

  \paragraph{p60. Exercise 3.}
    $\rank(A-\lambda I)>n-m$意味着$\dim(\nullspace (A-\lambda I)) < m$,即和
    $\lambda$相关联的线形无关的特征向量不足$m$个. 所以$A$的线性无关特征向量总数不足$n$个. 
  % end

  \paragraph{p60. Exercise 4.}
    根据秩-零化度定理,$\rank(A-\lambda I)\le n-k$. 设$\lambda$的代数重数为$m$,则根据
    Theorem 1.2.18,有$\rank(A-\lambda I)\ge n-m$. 所以有
    \[
      n-m \le n-k \quad\Rightarrow\quad m\ge k.
    \]
  % end

  \paragraph{p60. Lemma 1.3.8} 
    不同特征值对应的特征向量组成的向量组线性无关. 
  % end
  \paragraph{p60. The proof of Lemma 1.3.8}
    设$(\lambda_i, u_i)$为对应的特征值-特征向量对,令$x_1u_1+\cdots+x_ku_k=0$.
    将$A$作用于上式两端$0,1,\dots,k-1$次,得矩阵
    \[
      \begin{bmatrix}
        x_1u_1 & \cdots x_ku_k
      \end{bmatrix}
      \begin{bmatrix}
        1          & 1 & \cdots & 1 \\
        \lambda_1  & \lambda_2 & \cdots & \lambda_k \\
        \vdots & \vdots & \vdots & \vdots \\
        \lambda_1^{k-1} & \lambda_2^{k-1} & \cdots & \lambda_k^{k-1}
      \end{bmatrix}
      =0.
    \]
    由于$\lambda_i$各不相同,所以其中第二个矩阵为Vandermonde矩阵,是非奇异的,所以在两边右
    乘它的逆后我们得到$[x_1u_1, \cdots, x_ku_k] = 0$. 由于特征向量不是零向量,所以有$x_i
    =0$. 从而$\{u_i\}$线性无关. 
  % end

  \paragraph{p62. Theorem 1.3.12}
    之所以可以让$A$有如此形式是有Theorem 1.3.7保证的. 
  % end

  \paragraph{p62. Definitions 1.3.16}
    称$W\subset\mathbb{C}^n$为$F$-invariant, 若将$\mathcal{F}\subset M_n$左乘作用于
    $W$,其值域被包含于$W$,即在$\mathcal{F}$中元素的作用下封闭. \par
    疑问:$A$和$B$可交换是否意味着它们对应的线性变换可交换?
  % end

  \paragraph{p63. Exercise 1.}
    Suppose the one-dimensional subspace $W \subset \mathbb{C}^n$ is 
    $A$-invariant, then for any $x\in W$, 
    \[
      Ax\in W \quad\Rightarrow\quad Ax=kx,\quad\text{For some $k\in\mathbb{C}$.}
    \]
    Hence, $x$ is an eigenvector of $A$.
  % end

  \paragraph{p63. Observation 1.3.18}
    有$k$维不变子空间$W$等价于可以相似地将化为$\begin{bsmallmatrix}B & C \\ 0 & D
    \end{bsmallmatrix}$,其中$B\in M_k$. 由于相似矩阵有相同的特征多项式,所以它同时表明
    $p_A(t)=p_B(t)p_D(t)$. \par
    同时在该不变子空间中可以找到一个$A$的特征向量. \footnote{我意识到为了得出这一结论,下面
    的大部分讨论实际上是不需要,真正重要的只有之后第二节的后一半. }
    关于这一结论,首先我们考虑这样一件事情,我们是否可以找到这样一个不变子空间$W$,
    使得$A|_W:W\to W$是一个满射(从而是一个双射). 
    答案是可行的,首先考虑将$A$作用于$W$,其值域$\tilde{W}$是一个$W$的子空间,如果$\dim
    \tilde{W}=\dim W$,则有$\tilde{W}=W$. 若$\dim\tilde{W}<\dim W$,则我们注意到
    $\tilde{W}$仍是一个$A$的不变子空间,所以我们可以再做一遍同样的操作. 由于$\dim W$是一个
    有限的非负整数,所以迟早会有一个满足要求的$\tilde{W}$. 接下来我们来考虑这一$\tilde{W}$
    的维度的问题,考虑在$W$非零的情况下,它是否是一定是非零的. 不幸的是这是不能保证的,显然
    $W=\ker A$是一个不变子空间,且它可以不为零,但是按照这种方式找到的$\tilde{W}$是
    $\{0\}$. \par
    但是我们仍然可以从这一结论出发得出结论. 考虑非零不变子空间$W$和对应的$\tilde{W}$. 如果
    $\tilde{W}=\{0\}$,则我们知道$A$至少把一个$W$中的向量$x$映到了$0$上,从而$x$是$A$的
    与$A$相关联的特征向量. 而若$\tilde{W}\ne\{0\}$,我们不妨设$A=\tilde{W}$. 接下来我们
    考虑$S_1$和$B=[\beta_1,\dots,\beta_k]$的含义. 首先$S_1$是一个从$\mathbb{C}^k$到
    $W\subset\mathbb{C}^n$的线形双射. 而$AS_1=S_1B$意味着$As_i=S_1\beta_i$,即意味着
    对于任意$x=\sum x_is_i \in W$,有
    \[
      Ax = \sum_{i=1}^k x_i(As_i) = 
      S_1\left( \sum_{i=1}^k x_i\beta_i \right) = S_1B[x_1,\dots,x_k]^T.
    \]
    这说明我们可以把$A$作用于$x\in W$的效果拆分成一次$B$作用于对应坐标组成的向量上的结果再
    加上一次用$S_1$把坐标向量转成对应的$W$中的向量. 我们知道$B$一定有一个特征值$\lambda$以
    及一个相关的特征向量$[x_1^*, \dots, x_k^*]$,代入上式即得到
    \[
      Ax^*= S_1B[x_1^*,\dots,x_k^*]^T = \lambda S_1[x_1^*,\dots,x_k^*]^T
      =\lambda x.
    \]
    从而我们的到了一个特征向量. \par
    这一讨论也说明了对于一个不变子空间,我们可以把$A$对于其中向量的作用效果用一个$M_k$中的
    矩阵来描述. 
  % end
  \paragraph{p63. The proof of Observation 1.3.18}
    首先假设$k$维子空间$W\subset\mathbb{C}^n$是$A$-invariant,我们所要做的即找非奇异
    的$S$,使得$S^{-1}AS$为左上角块属于$M_k$. 设$s_1,\dots,s_k$是$W$的一组基,$S_1=
    [s_1,\dots,s_k]$由于$W$ $A$-invariant,所以对于任意$W$中的向量$x$,$Ax$是$s_1,
    \dots,s_k$的线形组合,分别取$x=s_k$,我们有
    \[
      As_i = \sum_{j=1}^k b_{ij}s_j = S_1\beta_i,\quad 1\le i \le k.
    \]
    设$B=[\beta_1,\dots,\beta_k]$,则有$AS_1 = S_1B$. 令$S=[S_1, S_2]$,注意到
    $S_2$实际上并不是十分重要,所以我们只需要确保$S$非奇异即可,我们可以从$s_1,\dots, s_k$
    开始扩充出一组$\mathbb{C}^n$的基,并用这组基的后半部分部分组成$S_2$. 我们有
    \[
      S^{-1}AS = [S^{-1}S_1B S^{-1}AS_2].
    \]
    其中$S^{-1}S_1 = [S^{-1}s_1,\dots,S^{-1}s_k] = [e_1,\dots,e_k]$.\par
    反之,如果我们已知相似于(1.3.17)的形式,设变换矩阵为$S=[S_1,S_2]$,可以证明$S_1$的列
    向量张成的空间为$A$-invariant. \par
  % end

  \paragraph{p63. Lemma 1.3.19}
    两两可交换的矩阵共有至少一个特征向量. 
  % end
  \paragraph{p64. The proof of Lemma 1.3.19}
    首先考虑交换族有什么性质. 注意到有$A(Bx)=B(Ax)$,若$x$是$A$的与$\lambda$相关联的特征
    向量,则有$B(Ax)=\lambda(Bx)$,从而$Bx$也是$A$的一个特征向量. 即对于任意一个$A\in
    \mathcal{F}$和它的任意特征值$\lambda$,设$W_{A,\lambda}$是其对应的特征向量全体,则
    $W_{A,\lambda}$是一个$\mathcal{F}$-不变子空间. \par
    同时我们分析$W_{A,\lambda}$的性质. 取定$x_0\in W_{A,\lambda}$,则对于TODO
  % end

  \paragraph{p64. Exercise 1.}
    TODO: Why commuting implies $\dim W=1$.
  % end
  \paragraph{p64. Theorem 1.3.21}
    通常我们会把某一个可对角化的矩阵$A$相似地化成$\diag(\lambda_1,\dots,\lambda_n)$的形
    式,而此时的问题在于用这一个$S$,是否能够将其他的$B$化成对角阵. 注意命题的第一部分仅保证了
    使同时对角化的阵的存在性,没有说明是否某个给定的矩阵可以做到这件事情,而命题的第二部分保证
    了这个$S$可以对角化其他所有阵. \par
    首先根据Theorem 1.3.7我们知道这个这个$S$由$A$的$n$个线性无关的特征向量组成,而由于对任
    意$B\in\mathcal{B}$,$AB=BA$,所以$B$是一个和$A$共形的对角分块矩阵,我们只需要证明每
    一个矩阵块都是对角阵即可. 而考虑到在$B$本身不是对角阵的情况,每个矩阵块的大小都要比$n
    \times n$要小,所以自然的可以想到用归纳法来完成整个证明. 
  % end

  \paragraph{p66. Example 1.3.24}
    TODO
  % end

  \paragraph{p67 Lemma 1.3.28}
    假设我们已经有了一个非奇异的复矩阵$A=C+iD$,以及一个和它相关的等式,则我们常常可以按照实部
    虚部拆分成两个等式. 之后我们利用这个引理,对于虚部的等式乘上系数$\eta$,在拼回成一个实矩
    阵. 
  % end

  \paragraph{p68 Theorem 1.3.31} TODO

% end
\subsection{Left and right eigenvectors and geometric multiplicity}
  \paragraph{p76. Exercise 3.}
    对于任意$x\in W$,$Ax = \lambda x \in W$,所以$W$是一个$A$-不变子空间. 注意对于任意
    $A$,$\mathbb{C}^n$一定是它的一个不变子空间,但不一定是某个特征空间. 由于对于任意特征向
    量$x$,$\text{span}(x)$是一个不变子空间,而$1$又是最小的正整数,所以最小的不变子空间一
    定是这样一个由单个特征向量张成的子空间. 
  % end

  \paragraph{p77. Exercise 3.}
    \[
      I = S\inv S =
      \begin{bmatrix}
        y_1^* \\ \vdots \\ y_n^*
      \end{bmatrix}
      \begin{bmatrix}
        x_1 & \dots & x_n
      \end{bmatrix}
      = [y_i^*x_j].
    \]
  % end

  \paragraph{p78. Theorem 1.4.7}
    对于一个$y$,可以认为它是某族平面的法向量,而$y^*\mapsto y^*A$是一个从平面到平面的映
    射. 因此在之后的证明中会出现$y$的正交补之类的东西. \par
    注意在仅知道$\lambda$是某个右特征向量的特征值时,仅可以确保$A$相似$\begin{bsmallmatrix}
    \lambda & * \\ 0 & *  \end{bsmallmatrix}$,而这里进一步保证了右上方块为零. \par
    关于(b)的证明:不妨设$y^*x = 1$.
    首先考虑$y$所确定的超平面$W$,设它由$S\in M_{n,n-1}$的列向量张成,对于
    $x$,由于有$y^*x\ne 0$,即$y$和$x$不正交,从而$x$在$W$的“外面”,从而$S=[x\, S]$是一
    个非奇异的矩阵. \par
    接下来我们证明$A$在这组基下的表示$S\inv AS$有所需要的形式. 首先我们来求$S\inv$. 按照$S$
    的形状对$S\inv$进行划分,设$S\inv = \begin{bsmallmatrix} \eta^* \\ B
    \end{bsmallmatrix}$,其中$\eta\in\mathbb{C}^n$. 则根据$I_n=S\inv S$可以推得
    \[
      \begin{bmatrix}
        \eta^* x & \eta^*S_1 \\ B^* x & B^* S_1
      \end{bmatrix}
      =
      \begin{bmatrix}
        1 x & 0 \\ 0 & I_{n-1}
      \end{bmatrix}.
    \]
    从而$\eta$是和超平面$W$正交的且满足$\eta^* x =1$的向量,即$\eta=y$. 最后直接计算
    $S\inv AS$即可完成证明. 
  % end
  \paragraph{p78. Theorem 1.4.7}
    $y$的正交补和$x$张成的子空间都是$A$的不变子空间. 
  % end



% end

\newpage
\section{Unitary Similarity and Unitary Equivalence}
\subsection{Unitary matrices and the QR factorization}
  \paragraph{p84. Theorem 2.1.4}
    酉阵即为标准正交基组成的矩阵. 同时可以把标准正交基看做一个直角坐标系. 
  % end

  \paragraph{p85. Definition 2.1.5}
    等距同构:不改变两点间距离的变换,即满足$d(p,q) =d(\varphi(p), \varphi(q))$的变换.
  % end

  \paragraph{p86. Theorem 2.1.9}
    从线性变换的角度考虑这一个定理. 设$A$是线性变换$\varphi$在标准正交基$\{e_i\}$下的矩阵,
    则$A\inv$即为$\varphi\inv$在$\{e_i\}$下的矩阵,$\varphi^*$在$\{e_i\}$的矩阵为
    $A^*$. 而$A\inv\sim A^*$表明,存在一个可逆线性变换$\psi$,成立$\varphi\inv=\psi
    \inv\varphi^*\psi$,即有$\psi=\varphi^*\psi\varphi$. 
  % end

  \paragraph{p87. Example 2.1.12}
    以下内容中,我们可以假设$w$为单位向量。Householder变换的几何含义为:做该向量关于平面
    $w^\perp$的镜像。首先由于它是等距变化,所以是酉阵;同时有$\langle U_wu,v\rangle = 
    \langle u,U_wv\rangle$,所以它是Hermite阵。同样的,对于实向量空间也有相应的结论的。
    而由于它是酉阵(等距变化),所以特征值的模都为$1$。根据谱定理,我们知道它有$n$个不同的一维
    度不变子空间,其中一个是$v$所在直线,它对应的特征值是$-1$,而$v^\perp$中的$n-1$个一维不
    变子空间对应的特征向量都是$1$,所以其行列式为$-1$。
  % end

  \paragraph{p89. Theorem 2.1.14}
    我们来依此解释这些结论中$Q$所对应的含义。首先$A\in M_{n,m}$是一个从$\mathbb{F}^m$到
    $\mathbb{F}^n$的线性映射。(a)表明可以在$\mathbb{F}^n$中找到$m$个坐标轴,把$A$的行为
    拆分成现在$\mathbb{F}^n$中作用,在变换到对应的$m$个坐标轴上。而对于(d),则是一个相反的
    拆分,先映射到$\mathbb{F}^m$中,再换成另一个坐标系。同时这些$Q$的列向量标准正交意味着这
    些都是直角坐标系。\par
    而$R$则表示我们拆分出的那个线性变换,它对角线非负意味着至少它不会把某条坐标轴映到几乎相反的
    方向。
  % end

  \paragraph{p90. Exercise(Cholesky)}
    设$A=QR$,其中$Q$为酉阵而$R$为对角线非负的上三角阵,则$B=A^*A=R^*Q^*QR=R^*R$,令$L=
    R^*$即可。当$A$非奇异时,设$B=\tilde{L}\tilde{L}^*$,则有
    \[
      LL^*=\tilde{L}\tilde{L}^* \quad\Rightarrow\quad
      L\inv\tilde{L} = L^*\tilde{L}^{-*}.
    \]
    其中LHS是下三角阵而RHS是上三角阵,所以它们都是对角阵。根据定理2.1.14证明中的讨论,它们都
    为$I$,所以$L=\tilde{L}$.
  % end
  
% end

\newpage
\section{Cheat Sheet}
  \begin{thm}[rank-nullity]
    Let $A\in M_{m,n}(\mathbb{F})$ be given. $\rank A + \dim(\nullspace A) = n$.
  \end{thm}

  \begin{lemma}[full-rank factorization]
    Suppose $A\in M_{m,n}(\mathbb{F})$, then $\rank A = k$ iff $A=XY^T$ for some
    $X\in M_{m,k}(\mathbb{F})$ and $Y\in M_{n,k}(\mathbb{F})$ that each have 
    independent columns.
  \end{lemma}

  \begin{lemma}[rank-one perturbation]
    $\det(A+xy^*) = \det A + y^*(\adj A)x$.
  \end{lemma}

  \begin{lemma}[特征多项式系数]
    设$p_A(t)=t^n+a_{n-1}t^{n-1}+\cdots+a_1t+a_0$,有
    \[
      a_k = \frac{1}{k!}p_A^{(k)}(0) = (-1)^{n-k}E_{n-k}(A).
    \]
  \end{lemma}

  \begin{lemma}
    \[
      \begin{bmatrix}
        I_m & X \\
        0   & I_n
      \end{bmatrix}^{-1}
      = \begin{bmatrix}
        I_m & -X \\
        0   & I_n
      \end{bmatrix}.
    \]
  \end{lemma}
% end

\end{document}
