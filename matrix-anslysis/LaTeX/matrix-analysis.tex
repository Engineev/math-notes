\documentclass[12pt, a4paper]{article}
\usepackage{ctex}

\usepackage[margin=1in]{geometry}
\usepackage{
  color,
  clrscode,
  amssymb,
  ntheorem,
  amsmath,
  listings,
  fontspec,
  xcolor,
  supertabular,
  multirow,
  mathtools,
  mathrsfs
}
\definecolor{bgGray}{RGB}{36, 36, 36}
\usepackage[
  colorlinks,
  linkcolor=bgGray,
  anchorcolor=blue,
  citecolor=green
]{hyperref}
\newfontfamily\courier{Courier}

\theoremstyle{margin}
\theorembodyfont{\normalfont}
\newtheorem{thm}{定理}
\newtheorem{cor}[thm]{推论}
\newtheorem{pos}[thm]{命题}
\newtheorem{lemma}[thm]{引理}
\newtheorem{defi}[thm]{定义}
\newtheorem{std}[thm]{标准}
\newtheorem{imp}[thm]{实现}
\newtheorem{alg}[thm]{算法}
\newtheorem{exa}[thm]{例}
\newtheorem{prob}[thm]{问题}
\DeclareMathOperator{\sft}{E}
\DeclareMathOperator{\idt}{I}
\DeclareMathOperator{\spn}{span}
\DeclareMathOperator*{\agm}{arg\,min}
\newcommand{\pr}{\prime}
\newcommand{\tr}{^\intercal}
\newcommand{\st}{\text{s.t.}}
\newcommand{\hp}{^\prime}
\newcommand{\ms}{\mathscr}
\newcommand{\mn}{\mathnormal}
\newcommand{\tbf}{\textbf}
\newcommand{\mbf}{\mathbf}
\newcommand{\fl}{\mathnormal{fl}}
\newcommand{\f}{\mathnormal{f}}
\newcommand{\g}{\mathnormal{g}}
\newcommand{\R}{\mathbf{R}}
\newcommand{\Q}{\mathbf{Q}}
\newcommand{\JD}{\textbf{D}}
\newcommand{\rd}{\mathrm{d}}
\newcommand{\str}{^*}
\newcommand{\vep}{\varepsilon}
\newcommand{\lhs}{\text{L.H.S}}
\newcommand{\rhs}{\text{R.H.S}}
\newcommand{\con}{\text{Const}}
\newcommand{\oneton}{1,\,2,\,\dots,\,n}
\newcommand{\aoneton}{a_1a_2\dots a_n}
\newcommand{\xoneton}{x_1,\,x_2,\,\dots,\,x_n}
\newcommand\thmref[1]{定理~\ref{#1}}
\newcommand\lemmaref[1]{引理~\ref{#1}}
\newcommand\defref[1]{定义~\ref{#1}}
\newcommand\posref[1]{命题~\ref{#1}}
\newcommand\secref[1]{节~\ref{#1}}
\newcommand\equref[1]{(\ref{#1})}
\newcommand\figref[1]{图 \ref{#1}}
\newcommand\corref[1]{推论~\ref{#1}}
\newcommand\exaref[1]{例~\ref{#1}}
\newcommand\algref[1]{算法~\ref{#1}}
\newcommand{\remark}{\paragraph{评注}}
\newcommand{\example}{\paragraph{例}}
\newcommand{\proof}{\paragraph{证明}}


\title{矩阵分析$\,$笔记}
\author{任云玮}
\date{}

\begin{document}
\maketitle
\tableofcontents

\newpage
\section{Eigenvalues, Eigenvectors, and Similarity}
\subsection{Introduction}
% end
\subsection{The eigenvalue-eigenvector equation}
  \paragraph{p46}
    注意对于矩阵$A$和常数$k$,有
    \[
      (A+kI)x=\lambda\hp x \quad\Rightarrow\quad
      Ax = (\lambda\hp-k)x.
    \]
    因此在计算特征值的时候可以先给矩阵加上$I$的某个倍数来得到一个特征值更方便计算的矩阵。
  % end

  \paragraph{p46. Theorem 1.1.6}
    这一定理的第一部分可以用来证明$\sigma(p(A))$非空并给出其中的某些值;而第二部分则限定了
    $\sigma(p(A))$中的值的可选范围。如果我们在已知$p(A)$的特征值情况下讨论$A$,则这两部分的
    效果是反过来的。
  % end

  \paragraph{p47. The proof of Theorem 1.1.9}
    我们的目的是证明任意$A\in M_n$有至少一个特征值$\lambda$,且它所对应的特征向量可以被表示
    为$g(A)y$的形式,其中$g(t)\in\mathbb{P}_{n-1}$.\par
    按照定理1.1.6,我们可以找一个多项式$p$,找$p(A)$的特征值。考虑之前的观察以及$p$有常数项
    这件事,我们只需要说明$0\in\sigma(p(A)))$,即证$p(A)$是奇异的。对于一个$p(A)$形式的矩
    阵,十分自然地引入了一组向量$(A^ny, \dots, y)$。对于任意的$y\ne 0$,我们可以取恰当的
    $p$使得这组向量是线性相关,从而$p(A)$是奇异的。将上述讨论反过来叙述即证明了命题的第一部
    分,其中关于$p$需选取度数最小的那一个。\par
    为得到所需要的$g$,我们只需要考虑把$p(A)y=0$进行因式分解,得
    \[
      (A-\lambda I)(g(A)y) = 0.
    \]
    由于$p$是满足线性相关的条件的多项式中度数最小的,所以$g(A)y$非奇异,从而它是一个特征向量。
  % end
% end

\newpage
\section{Cheat Sheet}
  \begin{thm}[rank-nullity]
    Let $A\in M_{m,n}(\mathbb{F})$ be given. $\rank A + \dim(\nullspace A) = n$.
  \end{thm}

  \begin{lemma}[full-rank factorization]
    Suppose $A\in M_{m,n}(\mathbb{F})$, then $\rank A = k$ iff $A=XY^T$ for some
    $X\in M_{m,k}(\mathbb{F})$ and $Y\in M_{n,k}(\mathbb{F})$ that each have 
    independent columns.
  \end{lemma}
% end

\end{document}
