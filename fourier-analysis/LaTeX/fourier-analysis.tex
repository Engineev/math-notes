\documentclass[12pt, a4paper]{article}
\usepackage{ctex}

\usepackage[margin=1in]{geometry}
\usepackage{
  color,
  clrscode,
  amssymb,
  ntheorem,
  amsmath,
  listings,
  fontspec,
  xcolor,
  supertabular,
  multirow,
  mathtools,
  mathrsfs
}
\definecolor{bgGray}{RGB}{36, 36, 36}
\usepackage[
  colorlinks,
  linkcolor=bgGray,
  anchorcolor=blue,
  citecolor=green
]{hyperref}
\newfontfamily\courier{Courier}

\theoremstyle{margin}
\theorembodyfont{\normalfont}
\newtheorem{thm}{定理}
\newtheorem{cor}[thm]{推论}
\newtheorem{pos}[thm]{命题}
\newtheorem{lemma}[thm]{引理}
\newtheorem{defi}[thm]{定义}
\newtheorem{std}[thm]{标准}
\newtheorem{imp}[thm]{实现}
\newtheorem{alg}[thm]{算法}
\newtheorem{exa}[thm]{例}
\newtheorem{prob}[thm]{问题}
\DeclareMathOperator{\sft}{E}
\DeclareMathOperator{\idt}{I}
\DeclareMathOperator{\spn}{span}
\DeclareMathOperator*{\agm}{arg\,min}
\newcommand{\pr}{\prime}
\newcommand{\tr}{^\intercal}
\newcommand{\st}{\text{s.t.}}
\newcommand{\hp}{^\prime}
\newcommand{\ms}{\mathscr}
\newcommand{\mn}{\mathnormal}
\newcommand{\tbf}{\textbf}
\newcommand{\mbf}{\mathbf}
\newcommand{\fl}{\mathnormal{fl}}
\newcommand{\f}{\mathnormal{f}}
\newcommand{\g}{\mathnormal{g}}
\newcommand{\R}{\mathbf{R}}
\newcommand{\Q}{\mathbf{Q}}
\newcommand{\JD}{\textbf{D}}
\newcommand{\rd}{\mathrm{d}}
\newcommand{\str}{^*}
\newcommand{\vep}{\varepsilon}
\newcommand{\lhs}{\text{L.H.S}}
\newcommand{\rhs}{\text{R.H.S}}
\newcommand{\con}{\text{Const}}
\newcommand{\oneton}{1,\,2,\,\dots,\,n}
\newcommand{\aoneton}{a_1a_2\dots a_n}
\newcommand{\xoneton}{x_1,\,x_2,\,\dots,\,x_n}
\newcommand\thmref[1]{定理~\ref{#1}}
\newcommand\lemmaref[1]{引理~\ref{#1}}
\newcommand\defref[1]{定义~\ref{#1}}
\newcommand\posref[1]{命题~\ref{#1}}
\newcommand\secref[1]{节~\ref{#1}}
\newcommand\equref[1]{(\ref{#1})}
\newcommand\figref[1]{图 \ref{#1}}
\newcommand\corref[1]{推论~\ref{#1}}
\newcommand\exaref[1]{例~\ref{#1}}
\newcommand\algref[1]{算法~\ref{#1}}
\newcommand{\remark}{\paragraph{评注}}
\newcommand{\example}{\paragraph{例}}
\newcommand{\proof}{\paragraph{证明}}


\title{Fourier分析$\,$笔记}
\author{任云玮}
\date{}


\begin{document}
\maketitle
\tableofcontents

\newpage
\section{The Genesis of Fourier Analysis}

\section{Basic Property of Fourier Series}

\subsection{Examples and formulation of the problem}
% end

\subsection{Uniqueness of Fourier series}
  \paragraph{p37. }

  \paragraph{p41. Notes on Theorem 2.1}
    这一命题表明对于连续函数,只需要验证它们的Fourier系数是否相等即可. 
  % end

  \paragraph{p40. proof of Theorem 2.1}
    先考虑$f$为实函数的情况. 首先条件中有$\hat{f}(n)=0$,按照定义它意味着
    \[
      \frac{1}{2\pi}\int_{-\pi}^\pi f(\theta)e^{\iu n\theta} \rd\theta=0.
    \]
    由于积分的线形性,所以我们有对于任意三角多项式$p_k$,成立
    \[
      \int_{-\pi}^\pi f(\theta)p_k(\theta)\rd\theta = 0.
    \]
    我们考虑利用反证法来证明此命题. 不失一般性的,我们假设定义在$[-\pi, \pi]$上,
    在$\theta_0=0$处连续且为$f(0)>0$. 我们尝试构造一列三角多项式$\{p_k\}$,让它
    在$0$附近为正且在其他地方迅速衰减,则我们即可得到$\int f(\theta)p_k(\theta)>0$,
    而这与之前的讨论矛盾.\par
    首先按照之前的想法,我们取充分小的$[-\delta,\delta]\subset [-\pi,\pi]$,满足
    在其中$f(\theta)>f(0)/2$. 接下来我们构造一个三角多项式$p$,满足如下条件:
    \begin{enumerate}
      \item 在$[-\delta,\delta]$外$|p(\theta)|<s<1$.
      \item 在$[-\delta,\delta]$上$p(\theta)\ge 0$.
      \item 在某个$[-\eta,\eta]\subset[-\delta,\delta]$上$p(\theta)>r>1$.
    \end{enumerate}
    这样我们只需要令$p_k(\theta) = [p(\theta)]^k$,在进行一下估计即可. 我们可以设
    \[
      p(\theta) = \vep + \cos\theta.
    \]
    其中$\vep>0$充分小以满足[1.]. 同时显然只要最初选择的$|\delta|<\pi/2$,它就满足[2.].
    而对于[3.],只需要$|\eta|$充分小,也是可以成立的. \par
    接下来我们对积分$\int f(\theta)p_k(\theta)\rd\theta$进行估计. 我们有
    \[
      \left|\int_{-\pi}^\pi\right| \ge 
      \left|\int_{-\eta}^\eta + \int_{\eta\le|\theta|<\delta}\right|
      -\left|\int_{\delta\le|\theta|\le \pi}\right|
      \ge \left|\int_{-\eta}^\eta\right|
      -\left|\int_{\delta\le|\theta|\le \pi}\right|.
    \]
    由于$f$ Riemann可积,所以有界,即$|f|<B$. 从而有
    \begin{gather*}
      \left|\int_{\delta\le|\theta|\le \pi}f(\theta)p_k(\theta)\rd\theta\right|
       \le 2(\pi-\delta)Bs^k,\quad
      \int_{-\eta}^\eta f(\theta)p_k(\theta)\rd\theta \ge 2\eta \frac{f(0)}{2}r^k.
    \end{gather*}
    当$k$充分大时,有$|\int_{-\pi}^\pi|>0$,与已知矛盾. \par
    而对于$f$是复函数的情况,只需要利用$\bar{\hat{f}}(n) = \overline{\hat{f}(-n)}$即可.
  % end

  \paragraph{p41. Notes on Corollary 2.3}
    这一命题表明一定条件下,Fourier系数的绝对收敛性可以保证Fourier级数的一致收敛
    形. 而之后的命题([Corollary 2.4])则给出了通过函数的光滑程度导出Fourier系
    数衰减速度的方法.
  % end

  \paragraph{p42. [1.]}
    注意由于$e^{in\theta}$的周期性,设$b-a=2\pi$,有
    \[
      \frac{1}{2\pi}\int_{a}^be^{in\theta}\rd\theta = 
      \begin{cases}
        1, & n = 0, \\
        0, & n \ne 0.
      \end{cases}
    \]
  % end
% end

\subsection{Convolutions}
  \paragraph{p47. Notes on Lemma 3.2}
    这一引理表明可以用一列有界连续函数$\{f_k\}$在积分平均的含义下去逼近
    一个Riemann可积函数$f$. 在处理非连续函数的Fourier级数时,常先使用
    此引理. 而对于连续的情况,则直接使用Corollary 5.4即可,它声称连续
    函数可以用三角多项式一致逼近.
  % end
% end

\subsection{Good kernels}
  \paragraph{p49. Proof of Theorem 4.1}
    直观地想,由于good kernel的性质,当$n$充分大时,$[x-\pi,x+\pi]$上的加权
    平均的结果应该和$f(x)$是差不多的,自然而然就会想到把积分分为$|y|\le\delta$和
    $\delta\le|y|\le\pi$两部分. 利用$f$在$x$处的连续性等性质估计
    \[
      (f\ast K_n)(x) - f(x) = \frac{1}{2\pi}\int_{-\pi}^\pi
      K_n(y)[f(x-y)-f(x)]\rd y
    \]
    即可证明在连续点处的收敛性. 而一致收敛性则有闭区间上的连续函数的一致连续性保证.
  % end
% end

\subsection{Cesàro and Abel summability: applications to Fourier series}
  \paragraph{说明}
    提出这些不同的可求和性的原因在Fourier级数在某些点常常不是收敛的,所以在此给出了
    其他的定义级数收敛的方法,使得Fourier级数在那些点也有值. 而这些不同收敛方式则定义
    了对应的不同形式的kernel.
  % end

  

% end

\subsection{Exercises}
  \paragraph{积分公式}
  \[
    \int_0^\pi\sin n\theta\rd\theta = \frac{1-(-1)^n}{n}.
  \]
% end

\newpage
\section{Convergence of Fourier Series}

\subsection{Mean-square convergence of Fourier series}
  \paragraph{p77. 说明}
    首先注意到这些事实
    \begin{enumerate}
      \item $\{e_n=e^{\iu n\theta}\}_{|n|\le N}$是一组标准正交基,$S_N(f)$是这
        组基生成的子空间$S_N$中的元素.
      \item 由于是标准正交基,所以有
        \[
          S_N(f)=\sum_{n=-N}^N (f,e_n)e_n  \quad\text{且}\quad 
          \|S_N(f)\|^2= \sum_{n=-N}^N |(f,e_n)|^2
        \]
      \item 由于$S_{N-1}\subset S_N$,所以在$S_N$中一定可以比在$S_{N-1}$中逼近地更好.
    \end{enumerate}
    根据[2.],我们可以知道$S_N(f)$就是$f$在$S_N$中的投影,所以是最佳平方逼近函数. 对于
    满足一定条件的$f$,我们可以导出其最佳平方逼近函数收敛于$f$.
  % end

  \paragraph{Proof of Theorem 1.1}
    总体思路为找一列平方收敛于$f$的三角多项式并利用最佳逼近引理说明Fourier级数比它们
    逼近地更好,所以也平方收敛与$f$.\par
    由于可以用三角多项式一致地逼近一个连续函数,所以对于$f$连续情况的证明是方便的. 对于
    $f$仅仅是可积的情况. 首先我们知道可以用一列一致有界的连续函数$f_k$,在积分的意义下
    逼近$f$,对于这些连续的函数再重复之前的讨论即可. 注意需要先说明一下积分意义的逼近和
    平方逼近之间的关系.
  % end

  \paragraph{p80. [1]}
    很多并不是某个$\hat{f}(n)$的积分也有这样的形式.
  % end
% end

\subsection{Return to pointwise convergence}
  \paragraph{Proof of Theorem 2.1}
    首先我们对结论进行处理,可知我们仅需证明当$N\to\infty$时,
    \[
      \frac{1}{2\pi}\int_{-\pi}^\pi (f(\theta_0-t)-f(\theta_0))D_N(t)\rd t\to 0.
    \]
    而这直觉上来讲是成立的,因为虽然$D_N$并非good kernel,但是当$N\to\infty$时,它仍集中
    至原点附近,而由于$f$在$\theta_0$处可微,$f(\theta_0-t)-f(\theta_0)$在原点附近的值
    趋于零. 虽然我们(或许)仍可以通过$[\delta,\pi]$的方法来证明,但我们也可以尝试利用
    Riemann-Lebesgue引理,注意到有
    \[\begin{split}
      \int_{-\pi}^\pi (f(\theta_0-t)-f(\theta_0))D_N(t)\rd t
      = \int_{-\pi}^\pi \frac{f(\theta_0-t)-f(\theta_0)}{\sin(t/2)}
        \sin((N+1/2)t)\rd t.
    \end{split}\]
    由于
    \[
      \sin((N+1/2)t) = \sin(t/2)\cos(Nt) + \cos(t/2)\sin(Nt),
    \]
    所以我们只需要说明积分中的第一项可积即可. 而这可由
    \[
      \frac{f(\theta_0-t)-f(\theta_0)}{\sin(t/2)} =
      \frac{f(\theta_0-t)-f(\theta_0)}{t}\frac{t}{\sin(t/2)},
    \]
    以及$f$在$\theta_0$处的可微性导出.
  % end

  \paragraph{Note on Theorem 2.2}
    注意对于$f$和$g$本身,除了Riemann可积以外,没有任何光滑性方面的要求. 
  % end

  \paragraph{p83. Note on Sec 2.2.2}
    
  % end

  \paragraph{p84. [1.]}
    假设确实是某个Riemann可积函数$\tilde{f}$的Fourier级数,考虑该级数的Abel和. 按照
    Abel和的定义可知在$r\to 1$时它发散. 同时我们利用Fourier级数的Abel和积分表示以及
    $\tilde{f}$的有界性,可知当$r\to 1$时它是有界的,从而推出矛盾.
  % end

% end

\subsection{Exercises}
  \paragraph{引理}
    记$e_n=e^{-\iu n\theta}$,对$n\ne 0$,有$|(f,e_n)|=|(f\hp,e_n)|/|n|$.
  % end
% end



\end{document}
