\documentclass[12pt, a4paper]{article}
\usepackage{ctex}

\usepackage[margin=1in]{geometry}
\usepackage{
  color,
  clrscode,
  amssymb,
  ntheorem,
  amsfonts,
  amsmath,
  listings,
  fontspec,
  xcolor,
  supertabular,
  multirow,
  mathtools,
  mathrsfs,
}
\definecolor{bgGray}{RGB}{36, 36, 36}
\usepackage[
  colorlinks,
  linkcolor=bgGray,
  anchorcolor=blue,
  citecolor=green
]{hyperref}
\newfontfamily\courier{Courier}

\theoremstyle{margin}
\theorembodyfont{\normalfont}
\newtheorem{thm}{定理}
\newtheorem{cor}[thm]{推论}
\newtheorem{pos}[thm]{命题}
\newtheorem{lemma}[thm]{引理}
\newtheorem{defi}[thm]{定义}

\DeclareMathOperator{\rank}{rank}
\DeclareMathOperator{\adj}{adj}
\DeclareMathOperator{\tr}{tr}
\DeclareMathOperator{\diag}{diag}
\DeclareMathOperator{\nul}{null}
\DeclareMathOperator{\range}{range}
\DeclareMathOperator{\spn}{span}
% \DeclareMathOperator{\deg}{deg}

\newcommand{\hp}{^\prime}
\newcommand{\vep}{\varepsilon}
\newcommand{\inv}{^{-1}}
\newcommand{\rd}{\mathrm{d}}

\renewcommand{\Im}{\text{Im}}
\renewcommand{\Re}{\text{Re}}



\title{Fourier分析$\,$笔记}
\author{任云玮}
\date{}


\begin{document}
\maketitle
\tableofcontents

\newpage
\section{The Genesis of Fourier Analysis}

\section{Basic Property of Fourier Series}

\subsection{Examples and formulation of the problem}
% end

\subsection{Uniqueness of Fourier series}
  \paragraph{p37. }

  \paragraph{p41. Notes on Theorem 2.1}
    这一命题表明对于连续函数,只需要验证它们的Fourier系数是否相等即可. 
  % end

  \paragraph{p40. proof of Theorem 2.1}
    先考虑$f$为实函数的情况. 首先条件中有$\hat{f}(n)=0$,按照定义它意味着
    \[
      \frac{1}{2\pi}\int_{-\pi}^\pi f(\theta)e^{\iu n\theta} \rd\theta=0.
    \]
    由于积分的线形性,所以我们有对于任意三角多项式$p_k$,成立
    \[
      \int_{-\pi}^\pi f(\theta)p_k(\theta)\rd\theta = 0.
    \]
    我们考虑利用反证法来证明此命题. 不失一般性的,我们假设定义在$[-\pi, \pi]$上,
    在$\theta_0=0$处连续且为$f(0)>0$. 我们尝试构造一列三角多项式$\{p_k\}$,让它
    在$0$附近为正且在其他地方迅速衰减,则我们即可得到$\int f(\theta)p_k(\theta)>0$,
    而这与之前的讨论矛盾.\par
    首先按照之前的想法,我们取充分小的$[-\delta,\delta]\subset [-\pi,\pi]$,满足
    在其中$f(\theta)>f(0)/2$. 接下来我们构造一个三角多项式$p$,满足如下条件:
    \begin{enumerate}
      \item 在$[-\delta,\delta]$外$|p(\theta)|<s<1$.
      \item 在$[-\delta,\delta]$上$p(\theta)\ge 0$.
      \item 在某个$[-\eta,\eta]\subset[-\delta,\delta]$上$p(\theta)>r>1$.
    \end{enumerate}
    这样我们只需要令$p_k(\theta) = [p(\theta)]^k$,在进行一下估计即可. 我们可以设
    \[
      p(\theta) = \vep + \cos\theta.
    \]
    其中$\vep>0$充分小以满足[1.]. 同时显然只要最初选择的$|\delta|<\pi/2$,它就满足[2.].
    而对于[3.],只需要$|\eta|$充分小,也是可以成立的. \par
    接下来我们对积分$\int f(\theta)p_k(\theta)\rd\theta$进行估计. 我们有
    \[
      \left|\int_{-\pi}^\pi\right| \ge 
      \left|\int_{-\eta}^\eta + \int_{\eta\le|\theta|<\delta}\right|
      -\left|\int_{\delta\le|\theta|\le \pi}\right|
      \ge \left|\int_{-\eta}^\eta\right|
      -\left|\int_{\delta\le|\theta|\le \pi}\right|.
    \]
    由于$f$ Riemann可积,所以有界,即$|f|<B$. 从而有
    \begin{gather*}
      \left|\int_{\delta\le|\theta|\le \pi}f(\theta)p_k(\theta)\rd\theta\right|
       \le 2(\pi-\delta)Bs^k,\quad
      \int_{-\eta}^\eta f(\theta)p_k(\theta)\rd\theta \ge 2\eta \frac{f(0)}{2}r^k.
    \end{gather*}
    当$k$充分大时,有$|\int_{-\pi}^\pi|>0$,与已知矛盾. \par
    而对于$f$是复函数的情况,只需要利用$\bar{\hat{f}}(n) = \overline{\hat{f}(-n)}$即可.
  % end

  \paragraph{p41. Notes on Corollary 2.3}
    这一命题表明一定条件下,Fourier系数的绝对收敛性可以保证Fourier级数的一致收敛
    形. 而之后的命题([Corollary 2.4])则给出了通过函数的光滑程度导出Fourier系
    数衰减速度的方法.
  % end

  \paragraph{p42. [1.]}
    注意由于$e^{in\theta}$的周期性,设$b-a=2\pi$,有
    \[
      \frac{1}{2\pi}\int_{a}^be^{in\theta}\rd\theta = 
      \begin{cases}
        1, & n = 0, \\
        0, & n \ne 0.
      \end{cases}
    \]
  % end
% end

\subsection{Convolutions}
  \paragraph{p47. Notes on Lemma 3.2}
    这一引理表明可以用一列有界连续函数$\{f_k\}$在积分平均的含义下去逼近
    一个Riemann可积函数$f$.
  % end
% end

\subsection{Good kernels}
  \paragraph{p49. Proof of Theorem 4.1}
    直观地想,由于good kernel的性质,当$n$充分大时,$[x-\pi,x+\pi]$上的加权
    平均的结果应该和$f(x)$是差不多的,自然而然就会想到把积分分为$|y|\le\delta$和
    $\delta\le|y|\le\pi$两部分. 利用$f$在$x$处的连续性等性质估计
    \[
      (f\ast K_n)(x) - f(x) = \frac{1}{2\pi}\int_{-\pi}^pi
      K_n(y)[f(x-y)-f(x)]\rd y
    \]
    即可证明在连续点处的收敛性. 而一致收敛性则有闭区间上的连续函数的一致连续性保证.
  % end
% end

\subsection{Cesàro and Abel summability: applications to Fourier series}
% end 



\end{document}
