\documentclass[12pt, a4paper]{article}
\usepackage{ctex}

\usepackage[margin=1in]{geometry}
\usepackage{
  color,
  clrscode,
  amssymb,
  ntheorem,
  amsfonts,
  amsmath,
  listings,
  fontspec,
  xcolor,
  supertabular,
  multirow,
  mathtools,
  mathrsfs,
}
\definecolor{bgGray}{RGB}{36, 36, 36}
\usepackage[
  colorlinks,
  linkcolor=bgGray,
  anchorcolor=blue,
  citecolor=green
]{hyperref}
\newfontfamily\courier{Courier}

\theoremstyle{margin}
\theorembodyfont{\normalfont}
\newtheorem{thm}{定理}
\newtheorem{cor}[thm]{推论}
\newtheorem{pos}[thm]{命题}
\newtheorem{lemma}[thm]{引理}
\newtheorem{defi}[thm]{定义}

\DeclareMathOperator{\rank}{rank}
\DeclareMathOperator{\adj}{adj}
\DeclareMathOperator{\tr}{tr}
\DeclareMathOperator{\diag}{diag}
\DeclareMathOperator{\nul}{null}
\DeclareMathOperator{\range}{range}
\DeclareMathOperator{\spn}{span}
% \DeclareMathOperator{\deg}{deg}

\newcommand{\hp}{^\prime}
\newcommand{\vep}{\varepsilon}
\newcommand{\inv}{^{-1}}
\newcommand{\rd}{\mathrm{d}}

\renewcommand{\Im}{\text{Im}}
\renewcommand{\Re}{\text{Re}}



\title{复分析$\,$笔记}
\author{任云玮}
\date{}


\begin{document}
\maketitle
\tableofcontents
\newpage

\section{复数}

\subsection{复数基础}

  \begin{pos}
    设$z=x+\iu y$,其中$x, y\in\R$,则
    \[
      x^2 + y^2 = |z|^2,\quad x = \frac{z+\bar{z}}{2},\quad
      y = \frac{z-\bar{z}}{2i}.
    \]
  \end{pos}
  \remark
    在之后的内容中,将略去“$x,y\in\R$”.

  \begin{pos}[三角表示法]
    $z = x+\iu y = r(\cos\theta + \iu\sin\theta)$,其中
    $r=|z|$,$\theta$满足$x=r\cos\theta$,$y=r\sin\theta$.
    在此表示法下,乘法有公式
    \[
      |z_1z_2| = |z_1||z_2|,\quad \Arg(z_1z_2) = \Arg z_1 + \Arg z_2.
    \]
    对于除法也是类似的. 同时可以定义乘方为
    \[
      z^n = |z|^n(\cos(n\Arg z) + \iu\sin(n\Arg z)).
    \]
    对于$n\le 0$情况的定义是类似的.
  \end{pos}

  \begin{thm}[Moivre公式]
    \label{thm: Moivre公式}
    $(\cos\theta + \iu\sin\theta)^n = \cos n\theta + \iu\sin n\theta$.
  \end{thm}

  \begin{pos}\footnote{习题1.12}
    $\prod\limits_{k=1}^{n-1}\sin\dfrac{k\pi}{n}=\dfrac{n}{2^{n-1}}$.
  \end{pos}
  \proof
    

  \begin{thm}[Lagrange等式]
    设$z_i, w_i \in \mC$,则
    \[
      |\sum_{j=1}^n z_jw_j|^2 = \left(\sum_{j=1}^n|z_j|^2\right)
      \left(\sum_{j=1}^n|w_j|^2\right) - \sum_{1\le j<k\le n}
      |z_j\bar{w}_k - z_k\bar{w}_j|^2.
    \]
  \end{thm}
  \proof
    由于齐次性,所以不妨设$\sum |w_i|^2 = 1$. 记$\eta(j, k) =
    |z_j\bar{w}_k - z_k\bar{w}_j|^2$,显然成立
    $\eta(j, k) = \eta(k, j)$,$\eta(k, k) = 0$且$\eta(j, k)=
    |z_j\bar{w}_k|^2 + |z_k\bar{w}_j|^2 - z_j\bar{w}_k\bar{z}_kw_j
    - \bar{z}_jw_kz_k\bar{w}_j$. 所以
    \[\begin{split}
      \sum_{1\le j<k\le n}\eta(j, k)
      &= \frac{1}{2}\sum_{j=1}^n\sum_{k=1}^n\eta(j, k) \\
      &= \sum_{j=1}^n\sum_{k=1}^n|z_j\bar{w}_j|^2
          - \sum_{j=1}^n\sum_{k=1}^nz_jw_j\bar{z}_k\bar{w}_k \\
      &= \sum_{j=1}^n|z_j|^2 - \big| \sum_{j=1}^nz_jw_j \big|^2
    \end{split}\]
    移项后即得Lagrange等式.$\quad\blacksquare$
  \remark
    TODO: $\sum\eta$的几何解释.

  \begin{cor}[Cauchy不等式]
    设$z_i, w_i \in \mC$,则$|\sum z_jw_j|^2 \le (\sum|z_j|^2)(\sum|w_j|^2)$.
  \end{cor}

\end{document}
