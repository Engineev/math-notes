\documentclass[12pt, a4paper]{article}
\usepackage{ctex}

\usepackage[margin=1in]{geometry}
\usepackage{
  color,
  clrscode,
  amssymb,
  ntheorem,
  amsmath,
  listings,
  fontspec,
  xcolor,
  supertabular,
  multirow,
  mathtools,
  mathrsfs
}
\definecolor{bgGray}{RGB}{36, 36, 36}
\usepackage[
  colorlinks,
  linkcolor=bgGray,
  anchorcolor=blue,
  citecolor=green
]{hyperref}
\newfontfamily\courier{Courier}

\theoremstyle{margin}
\theorembodyfont{\normalfont}
\newtheorem{thm}{定理}
\newtheorem{cor}[thm]{推论}
\newtheorem{pos}[thm]{命题}
\newtheorem{lemma}[thm]{引理}
\newtheorem{defi}[thm]{定义}
\newtheorem{std}[thm]{标准}
\newtheorem{imp}[thm]{实现}
\newtheorem{alg}[thm]{算法}
\newtheorem{exa}[thm]{例}
\newtheorem{prob}[thm]{问题}
\DeclareMathOperator{\sft}{E}
\DeclareMathOperator{\idt}{I}
\DeclareMathOperator{\spn}{span}
\DeclareMathOperator*{\agm}{arg\,min}
\newcommand{\pr}{\prime}
\newcommand{\tr}{^\intercal}
\newcommand{\st}{\text{s.t.}}
\newcommand{\hp}{^\prime}
\newcommand{\ms}{\mathscr}
\newcommand{\mn}{\mathnormal}
\newcommand{\tbf}{\textbf}
\newcommand{\mbf}{\mathbf}
\newcommand{\fl}{\mathnormal{fl}}
\newcommand{\f}{\mathnormal{f}}
\newcommand{\g}{\mathnormal{g}}
\newcommand{\R}{\mathbf{R}}
\newcommand{\Q}{\mathbf{Q}}
\newcommand{\JD}{\textbf{D}}
\newcommand{\rd}{\mathrm{d}}
\newcommand{\str}{^*}
\newcommand{\vep}{\varepsilon}
\newcommand{\lhs}{\text{L.H.S}}
\newcommand{\rhs}{\text{R.H.S}}
\newcommand{\con}{\text{Const}}
\newcommand{\oneton}{1,\,2,\,\dots,\,n}
\newcommand{\aoneton}{a_1a_2\dots a_n}
\newcommand{\xoneton}{x_1,\,x_2,\,\dots,\,x_n}
\newcommand\thmref[1]{定理~\ref{#1}}
\newcommand\lemmaref[1]{引理~\ref{#1}}
\newcommand\defref[1]{定义~\ref{#1}}
\newcommand\posref[1]{命题~\ref{#1}}
\newcommand\secref[1]{节~\ref{#1}}
\newcommand\equref[1]{(\ref{#1})}
\newcommand\figref[1]{图 \ref{#1}}
\newcommand\corref[1]{推论~\ref{#1}}
\newcommand\exaref[1]{例~\ref{#1}}
\newcommand\algref[1]{算法~\ref{#1}}
\newcommand{\remark}{\paragraph{评注}}
\newcommand{\example}{\paragraph{例}}
\newcommand{\proof}{\paragraph{证明}}


\title{复分析$\,$笔记}
\author{任云玮}
\date{}


\begin{document}
\maketitle
\tableofcontents
\newpage

\section{复数}

\subsection{复数基础}

  \begin{pos}
    设$z=x+\iu y$,其中$x, y\in\R$,则
    \[
      x^2 + y^2 = |z|^2,\quad x = \frac{z+\bar{z}}{2},\quad
      y = \frac{z-\bar{z}}{2i}.
    \]
  \end{pos}
  \remark
    在之后的内容中,将略去“$x,y\in\R$”.

  \begin{pos}[三角表示法]
    $z = x+\iu y = r(\cos\theta + \iu\sin\theta)$,其中
    $r=|z|$,$\theta$满足$x=r\cos\theta$,$y=r\sin\theta$.
    在此表示法下,乘法有公式
    \[
      |z_1z_2| = |z_1||z_2|,\quad \Arg(z_1z_2) = \Arg z_1 + \Arg z_2.
    \]
    对于除法也是类似的. 同时可以定义乘方为
    \[
      z^n = |z|^n(\cos(n\Arg z) + \iu\sin(n\Arg z)).
    \]
    对于$n\le 0$情况的定义是类似的.
  \end{pos}

  \begin{thm}[Moivre公式]
    \label{thm: Moivre公式}
    $(\cos\theta + \iu\sin\theta)^n = \cos n\theta + \iu\sin n\theta$.
  \end{thm}
  \remark
    利用Moivre公式来证明三角恒等式时,一般可以利用诸如
    $\cos\theta = \Re(\cos\theta + \iu\sin\theta)$,从而避免
    使用Eular公式.

  \begin{pos}\footnote{习题1.12}
    $\prod\limits_{k=1}^{n-1}\sin\dfrac{k\pi}{n}=\dfrac{n}{2^{n-1}}$.
  \end{pos}
  \proof
    考虑方程$(z+1)^n=1$的根,设它们为$z_k$,$k=0,1,\dots,n-1$,其中$z_0=0$.
    有
    \[
      (z+1)^n-1 = z(z-z_1)\cdots(z-z_{n-1}).
    \]
    比较两边的一次项系数,得$z_1\cdots z_{n-1}=(-1)^{n-1}n$. 求解方程,
    对于$k\ne 0$,有
    \[\begin{split}
      z_k &= \cos\frac{2k\pi}{n}-1 + \iu\sin\frac{2k\pi}{n}
          = -2\sin\frac{k\pi}{n}(\sin\frac{k\pi}{n} - \iu\cos\frac{k\pi}{n}) \\
          &= -2\sin\frac{k\pi}{n}(\cos(\pi/2-k\pi/n) + \iu\sin(k\pi/n-\pi/2)).
    \end{split}\]
    因此成立
    \[\begin{split}
      (-1)^{n-1}n = (-2)^{n-1}\prod_{k=1}^{n-1}\sin\frac{k\pi}{n}(\cos 0 + \iu\sin 0)
      \quad\Rightarrow\quad \prod_{k=1}^{n-1}\sin\frac{k\pi}{n} = \frac{n}{2^{n-1}}.
      \quad\blacksquare
    \end{split}\]

  \begin{thm}[Lagrange等式]
    设$z_i, w_i \in \mC$,则
    \[
      |\sum_{j=1}^n z_jw_j|^2 = \left(\sum_{j=1}^n|z_j|^2\right)
      \left(\sum_{j=1}^n|w_j|^2\right) - \sum_{1\le j<k\le n}
      |z_j\bar{w}_k - z_k\bar{w}_j|^2.
    \]
  \end{thm}
  \proof
    由于齐次性,所以不妨设$\sum |w_i|^2 = 1$. 记$\eta(j, k) =
    |z_j\bar{w}_k - z_k\bar{w}_j|^2$,显然成立
    $\eta(j, k) = \eta(k, j)$,$\eta(k, k) = 0$且$\eta(j, k)=
    |z_j\bar{w}_k|^2 + |z_k\bar{w}_j|^2 - z_j\bar{w}_k\bar{z}_kw_j
    - \bar{z}_jw_kz_k\bar{w}_j$. 所以
    \[\begin{split}
      \sum_{1\le j<k\le n}\eta(j, k)
      &= \frac{1}{2}\sum_{j=1}^n\sum_{k=1}^n\eta(j, k) \\
      &= \sum_{j=1}^n\sum_{k=1}^n|z_j\bar{w}_j|^2
          - \sum_{j=1}^n\sum_{k=1}^nz_jw_j\bar{z}_k\bar{w}_k \\
      &= \sum_{j=1}^n|z_j|^2 - \big| \sum_{j=1}^nz_jw_j \big|^2
    \end{split}\]
    移项后即得Lagrange等式.$\quad\blacksquare$
  \remark
    TODO: $\sum\eta$的几何解释.

  \begin{cor}[Cauchy不等式]
    设$z_i, w_i \in \mC$,则$|\sum z_jw_j|^2 \le (\sum|z_j|^2)(\sum|w_j|^2)$.
  \end{cor}

\end{document}
