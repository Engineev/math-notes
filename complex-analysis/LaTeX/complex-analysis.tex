\documentclass[12pt, a4paper]{article}
\usepackage{ctex}

\usepackage[margin=1in]{geometry}
\usepackage{
  color,
  clrscode,
  amssymb,
  ntheorem,
  amsfonts,
  amsmath,
  listings,
  fontspec,
  xcolor,
  supertabular,
  multirow,
  mathtools,
  mathrsfs,
}
\definecolor{bgGray}{RGB}{36, 36, 36}
\usepackage[
  colorlinks,
  linkcolor=bgGray,
  anchorcolor=blue,
  citecolor=green
]{hyperref}
\newfontfamily\courier{Courier}

\theoremstyle{margin}
\theorembodyfont{\normalfont}
\newtheorem{thm}{定理}
\newtheorem{cor}[thm]{推论}
\newtheorem{pos}[thm]{命题}
\newtheorem{lemma}[thm]{引理}
\newtheorem{defi}[thm]{定义}

\DeclareMathOperator{\rank}{rank}
\DeclareMathOperator{\adj}{adj}
\DeclareMathOperator{\tr}{tr}
\DeclareMathOperator{\diag}{diag}
\DeclareMathOperator{\nul}{null}
\DeclareMathOperator{\range}{range}
\DeclareMathOperator{\spn}{span}
% \DeclareMathOperator{\deg}{deg}

\newcommand{\hp}{^\prime}
\newcommand{\vep}{\varepsilon}
\newcommand{\inv}{^{-1}}
\newcommand{\rd}{\mathrm{d}}

\renewcommand{\Im}{\text{Im}}
\renewcommand{\Re}{\text{Re}}



\title{复分析$\,$笔记}
\author{任云玮}
\date{}


\begin{document}
\maketitle
\tableofcontents

\newpage
\section{绪论}

\subsection{复数与复平面}

  复数相关定义与性质、复平面相关拓扑性质略

\subsection{复平面上函数}

  \paragraph{说明}
    极限、连续函数相关的定义与性质略,
    它们在对于一般度量空间中的函数的讨论中已经讨论过了.

  \begin{defi}[极值]
    对于定义在$\Omega\subset\mC$上的复函数$f$,称$f$在$z_0\in\Omega$
    处达到极大值,若对任意$z\in\Omega$,成立$|f(z)| \le |f(z_0)|$.
    对于极小值同理.
  \end{defi}
  \remark
    由于复数之间是没有大小关系的,所以极值是利用绝对值定义的.

  \begin{defi}[全纯\protect\footnotemark]
    \footnotetext{Holomorphic}
    设$f$是定义在开集$\Omega\subset\mC$的复函数. 称$f$在点$z_0
    \in\Omega$处全纯,若存在有限极限
    \[
      \lim_{h\to 0}\frac{f(z_0+h) - f(z_0)}{h}.
    \]
    其中$h\ne 0$且$z_0+h\in\Omega$. 称$f$在$\Omega$上全纯,若
    它在每个点上全纯. 若$f$在$\mC$上全纯,则称它是整的.
  \end{defi}
  \remark
    这一定义和实变量函数的可导在形式上是一致的,但是实际上这一条件相较于
    可导要强很多. 另外,和、差等函数的相关公式以及链式法则都是成立的,
    在此略去.

  \begin{pos}[有限增量公式]
    设$f$在$z_0$处全纯,则$f(z_0+h)=f(z_0)+f\hp(z_0)h + h\psi(h)$,
    其中$\lim_{h\to 0}\psi(h)=0$.
  \end{pos}

  \paragraph{复函数与映射}
    一个复函数$f$在一定程度上可以看作$F:\R^2\to\R^2$的映射,但是它们仍
    有着本质的区别. 首先$f$的全纯和$F$的可导是不等价的,考虑非全纯的函数
    $f(z)=\bar{z}$,它对应的$F$是无穷次可微的. 另一方面,$f$的复导数
    是一个复数,而$F$的导数则是对应的Jacob矩阵. 但是,它们之间仍是有联系
    的. 它们之间的关联见\thmref{thm: Cauchy-Riemann方程1}以及\thmref{thm:
    Cauchy-Riemann方程2}.

  \begin{defi}
    定义
    $\dfrac{\pt}{\pt z} = \dfrac{1}{2}\left(
    \dfrac{\pt}{\pt x} + \dfrac{1}{i}\dfrac{\pt}{\pt y}
    \right)$,
    $\dfrac{\pt}{\pt \bar{z}} = \dfrac{1}{2}\left(
    \dfrac{\pt}{\pt x} - \dfrac{1}{i}\dfrac{\pt}{\pt y}
    \right)$.
  \end{defi}

  \begin{thm}[Cauchy-Riemann方程]
    \label{thm: Cauchy-Riemann方程1}
    设$f=u+\iu v$在$z_0$处全纯,则
    \begin{equation}
        \label{equ: Cauchy-Riemann方程}
        \frac{\pt f}{\pt \bar{z}}(z_0)=0\quad\text{且}\quad
        f\hp(z_0) = \frac{\pt f}{\pt z}(z_0)
        = 2\frac{\pt u}{\pt z}(z_0).
    \end{equation}
     设$F:\R^2\to\R^2$为$f$对应的映射,则$F$可微且成立
     \[
      \det J_F(x_0,y_0) = |f\hp(z_0)|^2.
     \]
  \end{thm}
  \remark
    设$f=u+\iu v$,则Cauchy-Riemann方程也可以写作
    \begin{equation}
      \label{equ: Cauchy-Riemann方程2}
      \frac{\pt u}{\pt x} = \frac{\pt v}{\pt y},\quad
      \frac{\pt u}{\pt y} = -\frac{\pt v}{\pt x}.
    \end{equation}
  \proof
    $f(z)=f(x+\iu y)=f(x, y)$,分别让$h$沿实轴和虚轴方向趋于零,得到
    两偏导数,根据全纯的定义,它们应相等,从而可得到\equref{equ: Cauchy-Riemann方程}中
    的前者. 同时,将它们相加再除以二即为$\pt f/\pt z$,同时利
    用\equref{equ: Cauchy-Riemann方程2},可以得到后者.\par
    而关于$F$的可微性,只需注意到若利用\equref{equ: Cauchy-Riemann方程2},则有
    \[\begin{split}
      &|F(x_0+h_1,x_0+h_2)-F(x_0,y_0)-J_F(x_0,y_0)(h_1,h_2)\tr|  \\
      =& \left|f(z_0+h)-f(z_0)-
      \left(\frac{\pt u}{\pt x}-\iu\frac{\pt u}{\pt y}\right)(h_1+\iu h_2)\right|
      = |f(z_0+h)-f(z_0)-f\hp(z_0)h|.
    \end{split}\]
    由全纯的定义可知$F$是可微的. 而最后只需要将Cauchy-Riemann方程应用到
    $J_F$行列式的表达式中即可完成所有的证明.$\quad\blacksquare$

  \begin{thm}[Cauchy-Riemann方程]
    \label{thm: Cauchy-Riemann方程2}
    设$f=u+\iu v$是定义在开集$\Omega$上的复函数. 设$u$和$v$连续
    可微且满足Cauchy-Riemann方程\equref{equ: Cauchy-Riemann方程2},
    则$f$在$\Omega$上全纯且$f\hp(z_0) = \pt f/\pt z$.
  \end{thm}
  \proof
    利用$f=u+\iu v$,分别对$u$和$v$利用有限增量公式展开即可.$\quad\blacksquare$

  \begin{thm}[Cauchy-Riemann方程]
    \label{thm: 极坐标Cauchy-Riemann方程}
    在极坐标下,Cauchy-Riemann方程的形式为
    \[
      \frac{\pt u}{\pt r} = \frac{1}{r}\frac{\pt v}{\pt\theta},\quad
      \frac{1}{r}\frac{\pt u}{\pt\theta} = -\frac{\pt v}{\pt r}.
    \]
  \end{thm}
  \proof
    首先将\thmref{thm: 偏导数的极坐标表示}的结果代入\equref{equ:
    Cauchy-Riemann方程2},则有
    \[\begin{split}
      (*1)\qquad\frac{\pt u}{\pt r}\cos\theta - \frac{1}{r}\frac{\pt u}{\pt\theta}\sin\theta
      &= \frac{\pt v}{\pt r}\sin\theta - \frac{1}{r}\frac{\pt v}{\pt\theta}\cos\theta\\
      (*2)\qquad\frac{\pt u}{\pt r}\sin\theta + \frac{1}{r}\frac{\pt u}{\pt\theta}\cos\theta
      &= -\frac{\pt v}{\pt r}\cos\theta + \frac{1}{r}\frac{\pt v}{\pt\theta}\sin\theta.
    \end{split}\]
    计算$(*1)\times r\cos\theta + (*2)\times r\sin\theta$即得结论中的前式,
    计算$(*1)\times r\sin\theta - (*2)\times r\cos\theta$即得结论中的后式.
    $\quad\blacksquare$

  \begin{thm}[链式法则]
    设$U$和$V$是复平面上的开集,$f:U\to V$和$g:V\to\mC$从实变量角度
    可微,定义$h=g\circ f$,则
    \[
      \frac{\pt h}{\pt z}=\frac{\pt g}{\pt z}\frac{\pt f}{\pt z}
      + \frac{\pt g}{\pt\bar{z}}\frac{\pt\bar{f}}{\pt z},\quad
      \frac{\pt h}{\pt\bar{z}} = \frac{\pt g}{\pt z}\frac{\pt f}{\pt\bar{z}}
      + \frac{\pt g}{\pt\bar{z}}\frac{\pt\bar{f}}{\pt\bar{z}}.
    \]
  \end{thm}
  \proof
    TODO

  \begin{defi}[调和]
    对于二阶连续可导的函数,定义Laplace算子为
    \[
      \Delta = \frac{\pt}{\pt x^2} + \frac{\pt}{\pt y^2}.
    \]
    称定义在复平面上开集$\Omega$中的实值函数$f$为调和的,若$\Delta f=0$在
    $\Omega$中成立.
  \end{defi}

  \begin{pos}
    $\Delta = 4\dfrac{\pt}{\pt z}\dfrac{\pt}{\pt\bar{z}}
    = 4\dfrac{\pt}{\pt\bar{z}}\dfrac{\pt}{\pt z}$.
  \end{pos}

  \begin{cor}[调和]
    若$f$在开集$\Omega$上全纯,则它的实部和虚部分别调和.
  \end{cor}

  \begin{defi}[Blaschke因子]
    对于单位圆盘$\mathbb{D}$内的复数$w$,定义Blaschke因子为
    \[
      F: z\mapsto \frac{w-z}{1-\bar{w}z},\quad z\in\mC.
    \]
  \end{defi}

  \begin{thm}[Blaschke因子]
    Blaschke因子满足如下性质:
    \begin{enumerate}
      \item $F(\mathbb{D})\subset\mathbb{D}$,且$F$全纯.
      \item $F(0)=w$且$F(w)=0$.
      \item 若$|z|=1$,则$|F(z)|=1$.
      \item $F:\mathbb{D}\to\mathbb{D}$为双射.
    \end{enumerate}
  \end{thm}
  \proof
    TODO

  \begin{pos}[常数]
    设$f$是定义在连通开集$\Omega$上的全纯函数,若$\Re(f)$,$\Im(f)$或
    $|f|$在$\Omega$上为常数,则$f$也为常数.
  \end{pos}
  \proof
    由于连通开集必然是路径连通的,所以只需要证明对应的$J_f=0$即可. 对于
    前两者,证明是显然的. 对于$|f|=C$的情况,只需要考虑极坐标的形式并利
    用\thmref{thm: 极坐标Cauchy-Riemann方程}即可.$\quad\blacksquare$

\subsection{幂级数}

  \begin{thm}[幂级数收敛半径]
    给定幂级数$\sum_{n=0}^\infty a_nz^n$,在扩充实数域中存在$R$使得
    \begin{enumerate}
      \item 若$|z|<R$,则幂级数绝对收敛.
      \item 若$|z|>R$,则幂级数发散.
    \end{enumerate}
    并且,$R$由Hadamard公式确定
    \[
      \frac{1}{R} = \limsup|a_n|^{1/n}.
    \]
  \end{thm}
  \proof
    对于固定的$z$,取常数$s\in(|z|, R)$,将$\sum|a_n||z|^n$放缩成
    幂级数即可.$\quad\blacksquare$

  \begin{thm}
    设$\{a_n\}$为非零复序列且满足$\lim_{n\to\infty}|a_{n+1}|/|a_n|=L$,
    则$\lim_{n\to\infty}\sqrt[n]{|a_n|}=L$.
  \end{thm}

  \begin{defi}[指数函数]
    \label{defi: 指数函数}
    对于$z\in\mC$,定义$e^z = \sum\limits_{n=0}^\infty\dfrac{z^n}{n!}$.
  \end{defi}
  \remark
    首先由于$\rhs$在$\R$中一致收敛至$e^x$,所以这一定义和原有的定义是一致的.
    与此同时,可以证明对于任意的$\rhs$对任意$z\in\mC$是收敛的,所以这一定义
    是良定义的.

  \begin{defi}[三角函数]
    \label{defi: 三角函数}
    对于$z\in\mC$,定义三角函数
    \[
      \cos z = \sum_{n=0}^\infty(-1)^n\frac{z^{2n}}{(2n)!},\quad
      \sin z = \sum_{n=0}^\infty(-1)^n\frac{z^{2n+1}}{(2n+1)!}.
    \]
  \end{defi}

  \begin{thm}[Euler公式]
    \label{thm: Euler公式}
    对于$z\in\mC$,成立$e^{iz} = \cos z + \iu\sin z$.
  \end{thm}
  \proof
    根据\defref{defi: 指数函数}和\defref{defi: 三角函数},不难验证
    上式的正确性. 需要注意,此两级数相加和换序的正确性是由它们的绝对收敛性
    保证的.$\quad\blacksquare$

  \begin{thm}
    幂级数$f(z)=\sum_{n=0}^\infty a_nz^n$在它的收敛圆盘内定义了一个
    全纯函数. 且$f$可以逐项求导,其导数$f\hp$的收敛半径与$f$的收敛半径相同.
  \end{thm}
  \proof
    关于$f$和$f\hp$的收敛半径相同的验证是简单的,下仅证明$f\hp$的存在性,
    且它可逐项求导得到,即$\sum_{n=0}^\infty na_nz^{n-1}$收敛至$g(z)$
    而$\lim_{h\to 0}\{(f(z+h)-f(z))/h - g(z)\} = 0$.\par
    设$g(z)=\sum_{n=0}^\infty na_nz^{n-1}$,$S_N(z)$为其前$N$
    项和而$E_N(z)$为其余项,对于任意的$z$,考虑
    \[\begin{split}
      \frac{f(z+h)-f(z)}{h} - g(z) = &
      \left( \frac{S_N(z+h)-S_N(z)}{h} - S\hp_n(z) \right)\\
      &+ (S\hp_N(z) - g(z)) + \left( \frac{E_N(z+h)-E_N(z)}{h} \right).
    \end{split}\]
    先让$N\to\infty$,可以证明后两项趋于零. 之后固定充分大的$N$,令$h\to 0$,
    则可以证明第一项趋于零.$\quad\blacksquare$

  \begin{cor}
    幂级数在它的收敛圆盘内定义了一个无穷次可导的复函数,且它的任意
    阶导数都可以通过逐项求导得到.
  \end{cor}

  \begin{defi}[解析]
    称复函数$f$在$z_0\in\mC$处解析,若存在$\delta>0$,在$O_\delta(z_0)$中$f$
    有幂级数展开.
  \end{defi}

\subsection{曲线积分}

  \begin{defi}[参数曲线]
    参数曲线是指映射$z:[a,b]\subset\R\to\mC$. 称它为光滑的,
    若$z$连续可导并且$z(t)\ne 0$. 称两个参数曲线$z$和$\tilde{z}:[c,d]\to\mC$
    是等价的,若存在连续可微的从$[c,d]$到$[a,b]$的双射$s\mapsto t(s)$成立
    $t\hp(s)>0$且$\tilde{z}=z(t(s))$.\par
    称分段光滑的曲线是闭合的,若$z(a)=z(b)$. 称它为简单的,若$z(x)=z(y)$可以
    推得$x=y$或$x,y\in\{a, b\}$. 通常,曲线一词指代分段光滑曲线.
  \end{defi}
  \remark
    注意,$\R^2$中的曲线通常有不止一种参数化方法.

  \begin{defi}[曲线积分]
    给定$\mC$中的光滑曲线$\gamma$,设$z:[a,b]\to\mC$是它的参数化而
    $f$是定义在$\gamma$上的连续函数,则定义$f$沿$\gamma$的积分为
    \[
      \int_\gamma f(z)\rd z = \int_a^b f(z(t))z\hp(t)\rd t.
    \]
  \end{defi}
  \remark
    可以证明$\rhs$的取值与参数化的方法无关,所以它是良定义的.

  \begin{defi}[曲线长度]
    给定$\mC$中的光滑曲线$\gamma$,设$z:[a,b]\to\mC$是它的参数化.
    定义$\gamma$的长度为
    \[
      \len(\gamma) = \int_a^b|z\hp(t)|\rd t.
    \]
  \end{defi}

  \begin{thm}
    连续函数的曲线积分满足如下性质.
    \begin{enumerate}
      \item 线性性.
      \item $\int_\gamma f(z)\rd z = -\int_{\gamma^-}f(z)\rd z$.
      \item $|\int_\gamma f(z)\rd z| \le \sup_{z\in\gamma}|f(z)|\times\len(\gamma)$.
    \end{enumerate}
  \end{thm}

  \begin{thm}
    设连续函数$f$在$\Omega$上有原函数$F$,$\gamma$是$\Omega$中以
    $w_1$为起点$w_2$为终点的曲线,则
    \[
      \int_\gamma f(z)\rd z = F(w_2) - F(w_1).
    \]
  \end{thm}

  \begin{cor}
    \label{cor: 曲线积分1}
    设连续函数$f$在$\Omega$上有原函数$F$,$\gamma$是$\Omega$中的
    闭曲线,则
    \[
      \oint_\gamma f(z)\rd z = 0.
    \]
  \end{cor}

  \begin{cor}
    若$f$在区域$\Omega$中全纯且$f\hp = 0$,则$f$为常值.
  \end{cor}

 
\newpage
\section{Cauchy定理及其应用}

\subsection{Goursat定理}

  \begin{thm}[Goursat]
    \label{thm: Goursat}
    设$\Omega$是$\mC$中的开集,$T\subset\Omega$是一个三角形,且它
    的内部也都在$\Omega$内. 设$f$在$\Omega$中全纯,则
    \[
      \int_T f(z)\rd z = 0.
    \]
  \end{thm}
  \remark
    这一定理同\corref{cor: 曲线积分1}最大的区别在于,它并不要求$f$的
    原函数,甚至相反的,它要求$f$全纯. 另外,如果将条件中的三角形换成
    长方形,结论依然是成立的,只需要将它分割成两个三角形再应用此定理即可
    证明.
  \proof
    通过连接各边中点来四分三角形. 选取其中$|\int f\rd z|$最大者作为
    下一个三角形,使得成立$|I_0|\le 4^n|I_n|$,即用沿着小三角形的积分
    来估计原积分. 另一方面,这些小三角形及其内部构成了一个紧集套,设它
    收缩至$z_0$. 利用在$z_0$处展开$f$至二次项以及\corref{cor:
    曲线积分1}来估计$I_n$. $\quad\blacksquare$

\subsection{局部原函数存在性与圆盘上的Cauchy定理}

  \begin{thm}[局部存在性]
    \label{thm: 局部存在性}
    开圆盘上的全纯函数在该圆盘上有原函数.
  \end{thm}
  \proof
    不失一般性的,可以假设圆盘以原点为中心.
    分为两步完成证明,首先定义一个无歧义的$F(z)$,接着证明$F(z)$
    是$f(z)$的原函数.
    \begin{enumerate}
      \item 选取连接原点和$z$的直角路径,定义沿该路径积分的结果为$F(z)$.
      \item 通过路径积分的相消和\thmref{thm: Goursat},证明$F(z+h)-
        F(z)$即为沿着连接两点的直线段积分的结果. 再进行进一步的估计与证明.
        $\quad\blacksquare$
    \end{enumerate}
  \remark
    这一定理表明对于全纯函数,至少在局部永远是有原函数的.

  \begin{thm}[圆盘上的Cauchy定理]
    设$f$在圆盘上全纯,则对于任意圆盘中的闭曲线$\gamma$,成立
    \[
      \int_\gamma f(z)\rd z = 0.
    \]
  \end{thm}

  \begin{cor}
    \label{cor: Cauchy定理}
    设$f$在开集$\Omega$上全纯,圆$C$及其内部都在$\Omega$中,则
    \[
      \int_C f(z)\rd z = 0.
    \]
  \end{cor}
  \remark
    实际这些定理对于包括锁孔形在内的所有可以方便定义内部的“简单”图形
    都成立.

\subsection{利用Cauchy定理积分}

  通常可以总结为如下步骤:
  \begin{enumerate}
    \item 选取恰当的全纯函数$f$.
    \item 接下来选取恰当的闭曲线$\gamma$,在其上对$f$应用\corref{cor: Cauchy定理}.
    \item 将$\int_\gamma f(z)\rd z$中$\gamma$拆分成不同的段,使得在其上的积分
      或互相抵消,或可以得出结果,或有原所求积分的形式.
    \item 整理之前的结果,进行诸如取极限等操作并得出结论.
  \end{enumerate}
  各个步骤之间的顺序并非一定的.

\subsection{Cauchy积分公式}

  \begin{thm}[Cauchy]
    \label{thm: Cauchy公式}
    设$f$在开集$\Omega$上全纯且$\Omega$包含圆盘$D$的闭包.
    记$C$为$D$的边界且取向为正,则对于任意$z\in D$,成立
    \[
      f(z) = \frac{1}{2\pi\iu}\int_C\frac{f(\zeta)}{\zeta-z}\rd\zeta.
    \]
  \end{thm}
  \remark
    这一定理给出了全纯函数在某个点的函数值的曲线积分表达式.
  \proof
    考虑将$z$排除的锁孔形$\Gamma_{\vep}$,$\vep$为锁孔半径. 则$F(\zeta)
    =f(\zeta)/(\zeta-z)$在$\Gamma$上全纯,所以对应的沿锁孔形的积分为零.
    接下来让走廊的宽度趋于零,由于连续性对应积分的变化也趋于零. 于是就有
    \[
      \int_C \frac{f(\zeta)}{\zeta-z}\rd\zeta
      = \int_{C\hp}\frac{f(\zeta)}{\zeta-z}\rd\zeta,
    \]
    其中$C\hp$为锁眼,即以$z$为圆心,半径为$\vep$的圆. 同时,对右侧做变量代换,令
    $\zeta = z+\vep e^{i\theta}$,同时令$\vep\to 0$,可得$\rhs=2\pi\iu f(z)$.
    $\quad\blacksquare$

  \begin{cor}[Cauchy]
    \label{cor: Cauchy}
    设$f$是在开集$\Omega$上的全纯函数,则$f$在复数含义下无限次可导.
    并且,如果$C\subset\Omega$是一个内部也在$\Omega$中的圆,则
    对于$C$内部的点,成立
    \[
      f^{(n)}(z) = \frac{n!}{2\pi\iu}\int_C\frac{f(\zeta)}{(\zeta-z)^{n+1}}\rd\zeta.
    \]
  \end{cor}
  \proof
    对$n$施归纳法即可.$\quad\blacksquare$
  \remark
    这一命题描述了全纯函数的正则性. 它意味着对于全纯函数而言,积分和微分实际上
    上一样的,例如如果要证明一个函数复数意义下可导,只需要证明它存在原函数即可,
    根据此命题自然而然就得出了该函数的全纯.

  \begin{cor}
    设$f$在开集$\Omega$上全纯,$D$是以$z_0$为圆心的半径为$R$圆盘,且它的闭包在
    $\Omega$内. 则成立
    \[
      |f^{(n)}(z_0)| \le \frac{n!\|f\|_\infty}{R^n}.
    \]
  \end{cor}

  \begin{thm}[幂级数展开]
    设$f$在开集$\Omega$上全纯. $D$是以$z_0$为圆心的圆盘且它的闭包在$\Omega$内,
    则$f$在$z_0$处有幂级数展开,即对任意$z\in D$成立
    \[
      f(z) = \sum_{n=0}^\infty a_n(z-z_0)^n,\qquad
      a_n = \frac{f^{(n)}(z_0)}{n!}.
    \]
  \end{thm}
  \remark
    这一定理表明全纯的条件在很大程度上已经意味着幂级数展开的存在性. 尤其是对于
    整函数,这一定理表明它在整个$\mC$上有幂级数展开.
  \proof
    方法在于首先应用Cauchy公式得到$f(z)$的积分表达式
    \[
      f(z) = \frac{1}{2\pi\iu}\int_C\frac{f(\zeta)}{\zeta-z}\rd\zeta.
    \]
    考虑被积函数,利用一致收敛的几何级数来得到级数的形式,同时化出$(z-z_0)^n$,
    具体方法为
    \[
      \frac{1}{\zeta-z} = \frac{1}{\zeta-z_0}\frac{1}{1-(\frac{z-z_0}{\zeta-z_0})}
      = \frac{1}{\zeta-z_0}\sum_{n=0}^\infty\left(\frac{z-z_0}{\zeta-z_0}\right)^n.
      \quad\blacksquare
    \]

  \begin{cor}[Liouville]
    若整函数$f$有界,则$f$为常值函数.
  \end{cor}

  \begin{cor}
    所有非常值复多项式$P(z)=a_nz^n + \cdots + a_0$在$\mC$中有一个根.
  \end{cor}
  \proof
    只需注意到如果$P$没有根,则$1/P$是一个有界整函数即可.

  \begin{cor}
    任意$n\ge 1$阶多项式$P(z)=a_nz^n + \cdots + a_0$在$\mC$中有恰$n$个根.
    且若设这些根为$w_1,\dots,w_n$,则$P$可被分解为
    \[
      P(z) = a_n(z-w_1)\cdots(z-w_n).
    \]
  \end{cor}
  \proof

  \begin{thm}
    设$f$在区域$\Omega$上全纯且在一列聚点在$\Omega$中的点处取值为零,则$f\equiv 0$.
  \end{thm}
  \proof
    TODO

  \begin{cor}
    设$f$和$g$在区域$\Omega$上全纯且$f(z)=g(z)$对于$\Omega$的某个非空开子集中
    的任意$z$成立,则在$\Omega$上成立$f\equiv g$.
  \end{cor}

\subsection{应用}

  \begin{thm}[Morera]
    \label{thm: Morera}
    设$f$是开圆盘$D$上的连续函数,并且对于包含于$D$中的三角形$T$,成立
    \[
      \int_T f(z)\rd z = 0,
    \]
    则$f$全纯.
  \end{thm}
  \proof
    由于复变函数的正规性,所以我们只需要证明$f$的原函数$F$全纯即可. 我们可以
    按照\thmref{thm: 局部存在性}中的方法构造处$F$,再验证一下即可.$\quad\blacksquare$

  \begin{thm}[级数]
    \label{thm: 级数、全纯}
    设$\{f_n\}$是一列全纯函数. 设在$\Omega$的任意紧子集中,它们一致收敛于$f$,
    则$f$在$\Omega$上全纯.
  \end{thm}
  \proof
    利用\thmref{thm: Morera}即可.$\quad\blacksquare$

  \begin{thm}
    设$\{f_n\}$是一列全纯函数. 设在$\Omega$的任意紧子集中,它们一致收敛于$f$,
    则$\{f_n\hp\}$在任意$\Omega$的紧子集中一致收敛于$f\hp$.
    \footnote{$f\hp$的存在性由\thmref{thm: 级数、全纯}.}
  \end{thm}
  \remark
    只需要反复应用\thmref{thm: 级数、全纯}以及此命题,就可以证明任意阶导数的一致收敛性.
  \proof
    用$|f_n-f|$来估计$|f_n\hp-f\hp|$即可. 若设$\Omega_\delta=
    \{z\in\Omega\,|\, \overline{D_\delta}(z)\subset\Omega\}$为所有离边界
    距离不小于$\delta$的点全体,可以证明成立不等式
    \[
      \sup_{z\in\Omega_\delta}|F\hp(z)| \le \frac{1}{\delta}\sup_{z\in\Omega}|F(z)|.
      \quad\blacksquare
    \]

  \begin{thm}[含参积分]
    设$F(z,s)$定义在$\Omega\times[0, 1]$上,其中$\Omega$是$\mC$中的开集.
    设$F$满足如下条件
    \begin{enumerate}
      \item 对任意固定的$s$,$F(z, s)$全纯.
      \item $F$在$\Omega\times[0, 1]$上连续.
    \end{enumerate}
    则如下定义在$\Omega$上的函数$f$全纯,
    \[
      f(z) = \int_0^1F(z,s)\rd s.
    \]
  \end{thm}
  \proof
    可以利用\thmref{thm: Morera}来证明,但这样需要验证积分换序的条件.
    为了避免这件事,我们可以考虑Riemann积分的定义,定义
    \[
      f_n = \frac{1}{n}\sum_{k=1}^nF(z,k/n).
    \]
    这样我们就将问题化归为\thmref{thm: 级数、全纯}条件的验证.$\quad\blacksquare$

  \begin{thm}[对称原理\protect\footnotemark]
    \footnotetext{关于定理中的记号,见书P58.}
    设$f^+$和$f^-$是分别定义在$\Omega^+$和$\Omega^-$上的全纯函数,且
    可以连续地沿拓到$I$上且成立$f^+(x)=f^-(x)$对任意$x\in I$成立,则
    按如下定义的函数$f$在$\Omega$上全纯
    \[
      f(z)=
      \begin{cases}
        f^+(z), &z\in\Omega^+, \\
        f^+(z)=f^-(z) &z\in I, \\
        f^-(z), & z\in\Omega^-.
      \end{cases}
    \]
  \end{thm}

  \begin{thm}[Schwarz镜像原理]
    设$f$在$\Omega^+$上全纯且其在$I$伤的连续沿拓为实函数. 则存在在整个$\Omega$上
    全纯函数$F$,在$\Omega^+$上成立$F=f$.
  \end{thm}

\subsection{说明}
  对于全纯函数而言,微分和积分是一体的. 如果要证明$f$全纯,则只需要构造处它的原函数
  $F$即可,则根据\corref{cor: Cauchy}可知,$f$是全纯的. 另一方面,若$f$全纯,
  则可以根据Goursat定理等,得到关于它的积分的诸多性质. 而Cauchy定理则意味着,除了
  在实函数中常见的级数展开,还可以使用积分来表示$f$在某一点的值. \par
  如果要证明函数$f$在某个区域$\Omega$内的某个性质,通常可以考虑通过逐点的取一个
  小圆盘的方式来证明,另外,如果这个函数全纯的话,则一般只需要取它的一个小开集即可.


\newpage
\input{ch3-meromorphic-functions-and-logarithm.tex}

\newpage
\section{Fourier变换}

\newpage
\section{整函数}
  \paragraph{说明}
    本节默认$f$不恒为零.
  % end

\subsection{Jensen公式}

  \begin{thm}[Jensen]
    \label{thm: Jensen}
    设$\Omega$是包含$\bar{D}_R$的开集,$f$在$\Omega$中全纯且$f(0)\ne 0$. 同时
    对$z\in C_R$,$f(z)\ne 0$. 记$z_1,\dots,z_N$为$f$在$D_R$中的零点,其中每个
    零点被计数其重数次,则
    \begin{equation}
      \label{equ: Jensen}
      \log|f(0)| = \sum_{k=1}^N\log\left( \frac{|z_k|}{R} \right)
      + \frac{1}{2\pi}\int_0^{2\pi}\log|f(Re^{\iu\theta})|\rd\theta.
    \end{equation}
  \end{thm}
  \remark
    考虑\equref{equ: Jensen}的$\rhs$,这一定理描述了在对数意义下,零点和函数值大小
    的关系. 考虑普通的多项式的情况,若它有$R$个零点,则它的大小差不多就是$Az^R$,这一
    定理说的是类似的事情.
  \proof
    首先根据\corref{cor: 全纯、实部}的评注以及\equref{equ: Jensen}的形式,自然
    可以想到设$F(z)=f(z)/\prod(z-z_k)$. 则有
    \[
      \log\left|\frac{f(0)}{z_1\cdots z_N}\right| = 
      \frac{1}{2\pi}\int_0^{2\pi}\log\left|
      \frac{f(z)}{(z-z_1)\cdots(z-z_N)}\right|\rd\theta.
    \]
    注意到这里的对数是实对数,将它拆开并对上式变形,最后可得
    \[
      \log|f(0)|=\frac{1}{2\pi}\int_0^{2\pi}\log|f(Re^{\iu\theta})|\rd\theta
      + \sum_{k=1}^N\log\left(\frac{|z_k|}{R}\right)
      - \frac{1}{2\pi}\sum_{k-1}^N\int_0^{2\pi}\log\left|
      1-\frac{z_k}{R}e^{\iu\theta}\right|\rd\theta.
    \]
    我们只需要最后证明$\rhs$的最后一项的求和的每一项都为零即可. 采用\corref{cor: 全纯、实部}
    里的记号,注意到被积函数是$u(e^{\iu\theta})$的形式,可以发现
    \[
      u(z) = \log\left|1-\frac{z_k}{R}z\right| = 
      \Re\left(\log\left(1-\frac{z_k}{R}z\right)\right).
    \]
    上式的合理性由$1-zz_k/R$在单位圆盘内无零点保证. 同时注意到$u(0)=0$,从而$\rhs$的最后
    一项为零. $\quad\blacksquare$
  % end

  \begin{lemma}
    条件同前,设$z_1,\dots,z_N$是$f$在$D_R$中的零点,设$\mathfrak{n}(r)$为$f$在
    $D_r$中的零点个数,则
    \[
      \int_0^R \mathfrak{n}(r)\frac{\rd r}{r} = 
      \sum_{k=1}^N\log\left| \frac{R}{z_k} \right|.
    \]
  \end{lemma}
  \proof
    利用
    \[
      \log\left| \frac{R}{z_k} \right|=\int_{|z_k|}^R\frac{\rd r}{r},
      \quad n_k(r) = ( r > |z_k| )?.\quad\blacksquare
    \]
  % end

  \begin{cor}
    \label{cor: 整函数、零点、模}
    条件同前. 此推论仅仅是\thmref{thm: Jensen}的另一形式,它的$\lhs$的被积函数可以理解为
    $D_r$内平均零点个数. 
    \[
      \int_0^R\mathfrak{n}(r)\frac{\rd r}{r} = \frac{1}{2\pi}
      \int_0^{2\pi}\log|f(Re^{\iu\theta})|\rd\theta - \log|f(0)|.
    \]
  \end{cor}

% end

\subsection{有限阶函数}
  \begin{defi}
    \label{defi: 整函数的阶}
    设$f$是整函数. 若存在正数$\rho$和正常量$A$和$B$,使得对任意$z\in\mC$成立
    \begin{equation}
      \label{equ: 整函数的阶、小于等于}
      |f(z)| \le Ae^{B|z|^\rho},
    \end{equation}
    则称$f$增长率的阶$\le\rho$. 定义$f$增长率的阶(简称阶)为
    \[
      \rho_f = \inf\rho.
    \]
    其中$\rho$取遍所有满足\equref{equ: 整函数的阶、小于等于}的值.
  \end{defi}
  \remark
    注意到这样一个事实:设$p<\rho$,则
    \[
      |f(z)|\le Ae^{B|z|^\rho + C|z|^p} \le Ae^{(B+C)|z|^\rho}.
    \]
    即,和多项式的情况类似,我们仅需要考虑指数上的最高次即可. 同样的,对于没有$e^|z|$
    大的因子基本上都可以忽略掉.\par
    阶描述了整函数在无穷远处的行为,如果它的阶为$\rho$,那么在无穷远处,它的增长就
    类似于$e^{|z|^\rho}$. 之所以这一对任意$z\in\mathbb{C}$成立的式子描述了在
    无穷远处的行为,是因为若我们知道对于充分大的$R=|z|$成立
    \equref{equ: 整函数的阶、小于等于},则由于$f$在$D_R$内是有有界的,
    设$|f(z)|\le M$,而$e^x\ge e^{-R}$,所以对任意$z\in\mathbb{C}$成立
    \[
      |f(z)| \le \left(A + Me^R\right)e^{B|z|^\rho}.
    \]
    这也表明我们只需要对于充分大的$|z|$证明\equref{equ: 整函数的阶、小于等于}即可.\par
    要证明某个函数的阶为$\rho$,即证明
    \[
      0 < \lim_{z\to\infty}\frac{|f(z)|}{e^{B|z|^\rho}} < \infty.
    \]
    
  % end

  \begin{thm}
    \label{thm: 整函数的阶}
    设整函数$f$的阶$\le\rho$,则
    \begin{enumerate}
      \item 对于充分大的$r$成立$\mathfrak{n}(r)\le Cr^\rho$.
      \item 设$z_1,z_2,\dots$为$f$的零点且$z_k\ne 0$,则对任意$s>\rho$,成立
        \[
          \sum_{k=1}^\infty\frac{1}{|z_k|^s} < \infty.
        \]
    \end{enumerate}
  \end{thm}
  \remark
    如果有$f(0)\ne 0$,[1.]中的“充分大”的要求可以去掉. 这一定理用于证明\thmref{thm:
    无穷乘积、导数}条件中的收敛性.
  \proof
    对于[1.],不妨设$f(0)\ne 0$. 考虑\corref{cor: 整函数、零点、模}和\defref{defi:
    整函数的阶},可知
    \[\begin{split}
      \rhs \le CR^\rho-\log|f(0)|,\quad
      \lhs \ge \int_{R/2}^R\mathfrak{n}(r)\frac{\rd r}{r}\ge 
      \mathfrak{n}(R/2)\log 2.
    \end{split}\]
    而对于充分大的$R$,$\log|f(0)|$项可忽略.\par
    对于[2.],首先由于有界点列必有聚点而$f$不恒为零,所以可知在单位圆中至多有有限个
    零点,所以我们只需要考虑$|z_k|\ge 1$的零点即可. 将零点按照所处于哪一个$2^j\le
    |z_k|<2^{j+1}$来分类并做恰当的放缩即可. $\quad\blacksquare$
  % end

% end

\subsection{无穷乘积}
  \begin{thm}
    \label{thm: 无穷乘积}
    若$\sum|a_n|<\infty$,则乘积$\prod_{n=1}^\infty(1+a_n)$收敛. 乘积
    收敛于零当且仅当其中一个某一项为零.
  \end{thm}
  \remark
    这一定理的后半部分给出了确定一个用无穷乘积表示的函数的零点的方式. 同时注意,
    前半部分仅仅是一个充分条件. $\prod(1+a_n)$的收敛性在$a_n\ne -1$的情况下
    和$\sum\log(1+a_n)$的收敛性是等价的,而同时注意到
    \[
      \log(1+a_n) = a_n - \frac{a_n^2}{2} + O(a_n^3), 
    \]
    要构造诸如说明非必要的反例,通常可以使用这一估计.
  \proof
    正如同常见的一样,使用对数将乘积转化为求和. 由于$\{a_n\}$绝对收敛,所以不是一般性
    的,可设$|a_n|<1/2$. 所以可以对$1+a_n$取对数主支,即可以有
    \[
      \prod_{n=1}^N(1+a_n) = \prod_{n=1}^ne^{\log(1+a_n)}
      = \exp\left(\sum_{n=1}^N\log(1+a_n)\right).
    \]
    而由于$|\log(1+a_n)|\le 2|a_n|$而$\{|a_n|\}$收敛,所以$\{\log(1+a_n)\}$也
    绝对收敛,设其收敛于$B$. 由于$e^z$的连续性,可知该无穷乘积收敛于$e^B$. 这也意味着
    若$1+a_n\ne 0$,则乘积的结果不为$0$. $\quad\blacksquare$
  % end

  \begin{thm}
    \label{thm: 无穷乘积、导数}
    设$\{F_n\}$是一列开集$\Omega$上全纯函数. 设存在收敛的正项级数$\{c_n\}$,使得
    对任意$z\in\Omega$成立
    \[
      |F_n(z) - 1| \le c_n.
    \]
    则有
    \begin{enumerate}
      \item $\prod_{n=1}^\infty F_n(z)$在$\Omega$上一致收敛于全纯函数$F(z)$.
      \item 若任意$F_n(z)$无零点,则
        \[
          \frac{F\hp(z)}{F(z)} = \sum_{n=1}^\infty\frac{F\hp_n(z)}{F_n(z)}.
        \]
    \end{enumerate}
  \end{thm}
  \remark
    关于[2.],考虑级数情况的求导以及对数将乘法化为加法的性质,即可明白为什么会成这样的
    形式. 这一定理给出了一个证明无穷乘积形式的函数为全纯函数的方法.
  \proof
    和之前的证明相似,我们可知它确实一致收敛.     
  % end

  \begin{thm}
    \label{thm: cot级数展开}
    \[
      \pi\cot\pi z=\frac{1}{z}+\sum_{n=1}^\infty\frac{2z}{z^2-n^2}.
    \]
  \end{thm}
  \proof
    整体思路是证明$\Delta(z) = \lhs(z) -  \rhs(z)$为一个有界整函数. 具体而言,
    首先分别证明两侧的周期性,并描述在原点处极点的性质,从而得出想要的结论,注意对于
    $\rhs$,成立
    \[
      \rhs = \lim_{N\to\infty}\sum_{|n|\le N}\frac{1}{z+n}. \quad\blacksquare
    \]
  % end

  \begin{thm}
    \[
      \frac{\sin\pi z}{\pi} = z\prod_{n=1}^\infty\left(1-\frac{z^2}{n^2}\right).
    \]
  \end{thm}
  \proof
    首先注意到
    \[
      \left(\frac{f}{g}\right) = \frac{f}{g}
      \left( \frac{f\hp}{f} - \frac{g\hp}{g} \right).
    \]
    而对于$f\hp/f$的形式,我们可以有\thmref{thm: 无穷乘积、导数}. 证明$(\lhs/\rhs)\hp=0$
    即可.$\quad\blacksquare$
  % end

% end

\subsection{Weierstrass无穷乘积}

  \begin{defi}[自然因子]
    对于正整数$k$,定义自然因子
    \[
      E_0(z) =1-z,\quad E_k(z)=(1-z)e^{z+z^2/2+\cdots+z^k/k}.
    \]
  \end{defi}
  \remark
    对于$|z|<1$,成立
    \[
      E_k(z) = \exp\left(\log(1-z)+\sum_{n=1}^k\frac{z^n}{n}\right)
      = \exp\left(\sum_{n=k+1}^\infty\frac{z^n}{n}\right).
    \]
    在对自然因子进行估计的时候,上式是常用的. 另外,常见的会对于自然因子的变量按照$1/2$
    来进行分类,通常是$z/a_n$的形式,分别考虑$D_{2R}$外和内的$\{a_n\}$,其中$R=|z|$.
    具体来说,对任意$R$,分别对$|z|=R$的$z$证明,在此过程中$R$即为一个常量,此时按照
    $2R$将$a_n$分类.  同时,很多时候不等式中的常量$c$是放缩后等比数列求和的结果,即它们与
    $z$无关,从而上述按照$|z|=R$分类再拼起来的所取的$c$是可以是一致的.
  % end

  \begin{lemma}[自然因子估计]
    \label{lemma: 自然因子}
    设$|z|\le 1/2$,则存在$c>0$成立$|1-E_k(z)|\le c|z|^{k+1}$. 其中$c$的选取
    与$k$无关.
  \end{lemma}
  \proof
    首先,由于$|z|\le 1/2$,有$E_k(z)=\exp(\log(1-z)+\sum_{n=1}^k z^k/k)=e^w$. 将
    $\log$后级数展开相消得
    \[
      |w| = \left| \sum_{n=k+1}^\infty\frac{z^n}{n} \right|
       = |z|^{k+1}\left| \sum_{n=k+1}^\infty\frac{z^{n-k-1}}{n} \right|
       \le 2|z|^{k+1}.
    \]
    因为$|w|<1$,所以成立
    \[
      |1-E_k(z)| = |1-e^w| \le 2|w|\le 4|z|^{k+1}.\quad\blacksquare
    \]
  % end

  \begin{thm}
    设$\{a_n\}$为任意复数序列且满足当$n\to\infty$时$|a_n|\to\infty$. 存在整
    函数$f$,在且仅在$z=a_n$处取值为零. 任意其他满足条件的函数都有形式$f(z)e^{g(z)}$,
    其中$g$为整函数。
  \end{thm}
  \remark
    这一定理表明,对于给定的零点和重数,可以构造出满足这些条件的整函数\equref{equ:
    Weierstrass乘积}. 这样构造的思路来源于$\sin$的乘积展开,利用自然因子使得它收敛.
  \proof
    对于定理的后半部分,只需要考虑$f_1/f_2$即可. 设$\{a_n\}$中仅有$m$项为零,仍用$\{a_n\}$
    表示去除了这些项以后的序列. 定义Weierstrass乘积为
    \begin{equation}
      \label{equ: Weierstrass乘积}
      f(z)=z^m\prod_{n=1}^\infty E_n(z/a_n),
    \end{equation}
    下面通过考虑它在半径为$R$的圆盘内的行为来证明,即证明它在任意$D_R$内收敛且在且仅在$0$和
    $a_n\in D_R$处有零点. 首先考虑满足$|a_n|<2R$的因子,它们只有有限个,所以取出它们不影响
    收敛性. 它们组成的乘积满足要求. 而对于满足$|a_n|\ge 2R$的因子,对于$z\in D_R$,有
    $|z/a_n|<1/2$,根据\lemmaref{lemma: 自然因子},有估计$|1-E_n(z/a_n)|\le c|z|^{n+1}$.
    所以可知
    \[
      \prod_{|a_n|\ge 2R}E_n(z/a_n)
    \]
    收敛于一个全纯函数且在$D_R$内无零点,从而$f$满足要求.$\quad\blacksquare$
  % end

% end

\subsection{Hadamard分解定理}

  \begin{lemma}[自然因子估计]
    \label{lemma: 自然因子估计2}
    \begin{gather*}
      |E_k(z)|\ge e^{-c|z|^{k+1}},\quad (|z|\le 1/2) \\
      |E_k(z)|\ge |1-z|e^{-c\hp|z|^k},\quad (|z|\ge 1/2).
    \end{gather*}
  \end{lemma}

  \begin{lemma}
    \label{lemma: Hadamard}
    设$\rho<s<k+1$,其中$\rho$为阶,$k=[\rho]$. 设$U$为以$a_n\ne 0$为圆心,
    $|a_n|^{-k-1}$的圆盘的并,则对于$z\notin U$,成立
    \[
      \left| \prod_{n=1}^\infty E_k(z/a_n) \right|\ge e^{-c|z|^s}.
    \]
  \end{lemma}
  \proof
    我们对于任意固定的$z$证明此结论,并在必要时指出证明中的估计中的常量是与$z$无关的.
    取定$z$,按照常见的思路,将$a_n$按照是否在$D_{2|z|}$内分类,即
    \[
      \prod_{n=1}^\infty E_k(z/a_n) = \left(\prod_{|a_n|<2|z|} E_k(z/a_n)\right)
      \left(\prod_{|a_n|\ge2|z|} E_k(z/a_n)\right),
    \]
    并对两式分别估计. 首先考虑第二部分,根据\lemmaref{lemma: 自然因子估计2},成立
    \[\begin{split}
      \left|\prod_{|a_n|\ge2|z|} E_k(z/a_n)\right| 
      &= \prod_{|a_n|\ge2|z|} |E_k(z/a_n)| \\
      &\ge \prod_{|a_n|\ge 2|z|} e^{c|z/a_n|^{k+1}} \\
      &= \exp\left(-c|z|^{k+1}\sum_{|a_n|\ge2|z|}|a_n|^{-k-1} \right).
    \end{split}\]
    其中$|a_n|^{-k-1} = |a_n|^{-s}|a_n|^{s-k-1}$. 而由于$\rho_0<s$,正项级数
    $\sum_{n=1}^\infty|a_n|^{-s}=c_1<\infty$. 同时由于$s<k+1$,所以对于$|a_n|
    \ge 2|z|$有$|a_n|^{s-k-1}\le 2^{s-k-1}|z|^{s-k-1}$. 从而成立
    \[
      \sum_{|a_n|\ge2|z|}|a_n|^{-k-1} \le 2^{s-k-1}c_1|z|^{s-k-1}.
    \]
    因此对于第二部分有估计
    \[
      \left|\prod_{|a_n|\ge2|z|} E_k(z/a_n)\right| \ge
      e^{-c|z|^{k+1} 2^{s-k-1}c_1|z|^{s-k-1}} = e^{-c_2|z|^s},
    \]
    其中$c_2>0$是与$z$无关的常量.\par
    对于第一部分,首先根据\lemmaref{lemma: 自然因子估计2},有
    \begin{equation}
      \label{equ: pord E_k估计}
      \left|\prod_{|a_n|<2|z|} E_k(z/a_n)\right| 
      \ge \prod_{|a_n|<2|z|}\left|1-\frac{z}{a_n}\right|
      \prod_{|a_n|<2|z|}e^{-c\hp|z/a_n|^k}.
    \end{equation}
    其中$|a_n|^{-k}=|a_n|^{-s}|a_n|^{s-k}$. 由于$s\ge k$,所以$|a_n|^{s-k}
    <2^{s-k}|a_n|^{s-k}$,从而
    \[
      \prod_{|a_n|<2|z|}e^{-c\hp|z/a_n|^k} = 
      \exp\left( -c\hp|z|^k\sum_{|a_n|<2|z|}|a_n|^{-k} \right)
      \ge e^{-c\hp c_1|z|^s} = e^{-c_3|z|^s},
    \]
    其中$c_3>0$仍是与$z$无关的常量. 最后考虑\equref{equ: pord E_k估计}的$\rhs$
    的前半部分. 因为$z\notin U$,所以有$|a_n-z|\ge |a_n|^{-k-1}$. 所以成立
    \[
      \prod_{|a_n|<2|z|}\left|1-\frac{z}{a_n}\right| \ge 
      \prod_{|a_n|<2|z|}|a_n|^{-k-2}.
    \]
    注意$a_n$是$f$在$D_{2|z|}$中的零点,对上式取对数,有
    \[\begin{split}
      (-k-2)\sum_{|a_n|<2|z|}\log|a_n| 
      &\ge (-k-2)\mathfrak{n}(2|z|)\log 2|z|.
    \end{split}\]
    其中$\mathfrak{n}$表示$f/z^m$的零点个数,$m$为原点处的零点重数. 注意$f/z^m$和$f$的
    阶是相等的,而对于$f/z^m$,它在原点取值不为零,所以根据\thmref{thm: 整函数的阶},有
    \[
      (-k-2)\mathfrak{n}(2|z|)\log 2|z| 
      \ge (-k-2)C2^\rho|z|^s\log 2|z|.
    \]
    其中$C$仅与$\rho$有关. 由于$s$的选取是任意的,所以可以忽略$\log 2|z|$的效果.
    $\quad\blacksquare$
  % end

  \begin{cor}
    \label{cor: prod E_k的估计}
    存在一列半径$r_1,r_2,\dots$满足$r_m\to\infty$使得
    \[
      \left| \prod_{n=1}^\infty E_k(z/a_n) \right|\ge e^{-c|z|^s},
      \quad |z|=r_k
    \]
  \end{cor}
  \remark
    这一推论表明可以躲开\lemmaref{lemma: Hadamard}中说的那些圆盘. 
  % end

  \begin{lemma}
    \label{lemma: 实部、整函数、多项式}
    设$g$为整函数,$u=\Re(g)$且对于一列$r_n\to\infty$满足
    \[
      u(z) \le Cr_n^s,\quad |z|=r_n.
    \]
    则$g$是多项式且它的次数$\le s$.
  \end{lemma}
  \proof
    将$g$幂级数展开,命题要求证明对于$n>s$成立$a_n=0$. 利用\thmref{thm: 幂级数系数、积分}
    得到$a_n$的积分表达式. 利用$2u=g+\bar{g}$将表达式和实部相联系. 注意由于沿圆积分
    $e^{-\iu n\theta}$结果为零. 所以对$n>0$有
    \[
      a_n = \frac{1}{\pi r^n}\int_0^{2\pi}[u(re^{\iu\theta})-Cr^s]
      e^{\iu n\theta}\rd\theta.
    \]
    令$r\to\infty$即可.$\quad\blacksquare$
  % end

  \begin{thm}[Hadamard分解定理]
    设整函数$f$的阶为$\rho_0$. 设$k=[\rho_0]$,$a_1,a_2,\dots$为$f$的非零零点,则
    \[
      f(z) = e^{P(z)}z^m\prod_{n=1}^\infty E_k(z/a_n),
    \]
    其中$P\in\mathbb{P}_k$,$m$是$f$在原点的零点的重数.
  \end{thm}
  \proof
    这一定理的证明总结了本节的一些常用做法. 首先处理不带$e^{g(z)}$的形式. 定义
    \[
      E(z) = z^m\prod_{n=1}^\infty E_k(z/a_n).
    \]
    首先要证明整,即对于任意$D_R$证明全纯. 而由于$E$是用无穷乘积定义的,所以考虑
    利用\thmref{thm: 无穷乘积、导数},而对于条件中的收敛性,则可利用\thmref{thm:
    整函数的阶}. 在证明中需要对$E_k$进行估计,可以通过按照模是否大于$2R$对$a_n$进行
    分类,接着利用之前证明的诸多关于$E_k$的估计来得出结论. 而关于零点,则直接应用
    \thmref{thm: 无穷乘积}即可.\par
    接着考虑$f/E$,它全纯且无零点,则$f/E=e^g$. 利用\corref{cor: prod E_k的估计}
    来得出\lemmaref{lemma: 实部、整函数、多项式}的条件,从而得出结论.$\quad\blacksquare$
  % end

% end

% end


\newpage
\section{Gamma函数与Zeta函数}

  \paragraph{相关}
    \thmref{thm: 级数、全纯}. 对于通过级数定义的函数,可以通过在任意紧子集中
    证明一致收敛性来导出全纯. 而对于利用普通含参积分定义的函数,则需要连续性以及
    对于任意参数情况下的全纯来导出. 而对于反常含参积分,则一般可以通过定义化为
    一个取极限过程和一个常义含参积分,对于取极限过程,证明它是一致的.\footnote{
      实际上这里就是之前级数相关的内容.
    }
  % end 

\subsection{Gamma函数}

  \begin{defi}[Gamma函数]
    对于$s>0$,定义Gamma函数为
    \begin{equation}
      \label{equ: Gamma函数}
      \Gamma(s) = \int_0^\infty e^{-t}t^{s-1}\rd t.
    \end{equation}
  \end{defi}

  \begin{thm}[Gamma函数]
    Gamma函数可以在半平面$\Re(s)>0$上延拓为一个解析函数,且这个函数的表达式
    依然为\equref{equ: Gamma函数}.
  \end{thm}
  \remark
    直观上来说,之所以会有这样的性质,是因为当$t$充分大时,$e^{-t}$是指数减小的;
    而对于$0$附近的情况,有$|t^{s-1}| = t^{\Re(s)-1}$,因此反常可积性也是可以
    保证的.
  \proof
    由于解析是一个局部性质,所以仅需要证明在任意$S_{\delta,M}=\{\delta<\Re(s)<M\}$上
    \equref{equ: Gamma函数}定义了一个解析函数即可. 而由于\equref{equ: Gamma函数}实际
    上是极限$\lim_{\vep\to\infty}\int_{\vep}^{1/\vep}e^{-t}t^{s-1}\rd t$,所以要证明
    的事情实际上包括:上述含参积分收敛;$F_\vep$解析;$F_\vep\to \Gamma$是一致的. 最后利用
    \thmref{thm: 级数、全纯}完成证明.\par
    设$\sigma=\Re(s)$,由于$|e^{-t}t^{s-1}| = e^{-t}t^{\sigma-1},$且$\rhs$的积分收敛,
    所以原含参积分收敛. 
    同时$f(s,t)=e^{-t}t^{s-1}$在$S_{\delta,M}\times[\vep,1/\vep]$上的连续函数,且
    对于固定的$t$,$f$全纯. 所以根据\thmref{thm: 含参积分、全纯},$F_\vep(s)=
    \int_{\vep}^{1/\vep}f(s,t)\rd t$全纯. 另外对于任意$s$,有估计
    \[
      |\Gamma(s)-F_\vep(s)|
      \le \int_0^\vep e^{-t}t^{\sigma-1}\rd t 
      + \int_{1/\vep}^\infty e^{-t}t^{\sigma-1}\rd t
      \le \int_0^\vep t^{\sigma-1}\rd t + \int_{1/\vep}^\infty e^{-t}t^{M-1}\rd t.
    \]
    当$\vep\to 0$时,$\rhs\to 0$,所以收敛是一致的. 因此根据\thmref{thm: 级数、全纯},在
    $S_{\delta,M}$上$\Gamma(s)$是解析的.$\quad\blacksquare$
  % end

  \begin{lemma}[准周期性]
    \label{lemma: Gamma的准周期性}
    设$\Re(s)>0$,则$\Gamma(s+1)=s\Gamma(s)$. 因此$\Gamma(n+1)=n!$对$n=0,1,2\dots$成立.
  \end{lemma}
  \proof
    对$F_\vep$分部积分,得
    \[\begin{split}
      F_\vep(s) = \frac{1}{s}\int_{\vep}^{1/\vep}e^{-t}\rd t^s 
      = \frac{1}{s}\left\{ e^{-t}t^s\bigg|_{\vep}^{1/\vep} + F_\vep(s+1) \right\}.
    \end{split}\]
    其中$e^{-t}t^s\big|_{\vep}^{1/\vep}$在$\vep\to0$时趋于$0$,因此成立$\Gamma(s+1)=
    s\Gamma(s)$. 同时根据$\Gamma(1)=1$可得$\Gamma(n+1)=n!$.$\quad\blacksquare$
  % end

  \begin{thm}[Gamma函数的延拓]
    定义在$\Re(s)>0$上的Gamma函数可在$\mathbb{C}$上解析延拓为一个亚纯函数,且它在且近在
    非正整数上有简单极点. 它在$s=-n$处的留数为$(-1)^n/n!$.
  \end{thm}
  \remark
    这样发现在另一半平面上,它的表达式和原表达式是不一样的,这是因为$t^{\sigma-1}$
    在$\sigma<0$时不再在$0$附近反常可积. 所以在该半平面上对于$0$附近的部分有另外
    形式的表达式,见\equref{equ: Gamma的延拓}. 这个式子的级数部分在一般情况下
    显然是收敛的,唯一会出现问题的就是在$s=-n$的地方,那特定的一项会变成$\infty$.
    在作估计的时候对于这一项单独处理即可.
  \proof
    不难想到可以利用\lemmaref{lemma: Gamma的准周期性}在$\Re(s)>-m$上定义所需的函数$F_m$,
    接下来只需要证明$F_m$的极点相关性质满足命题所述. 由于这样定义的亚纯函数$F_m$同$\Gamma$
    在定义域内的极点外的部分都是一致的,所以上述定义是良定义的.$\quad\blacksquare$
  % end
  \proof
    下给出另一种证明,这一证明利用原表达式直接给出了$\Gamma$的延拓的表达式,并证明它满足所需
    性质. 对于\equref{equ: Gamma函数},成立
    \begin{equation}\begin{split}
      \label{equ: Gamma的延拓}
      \Gamma(s) &= \int_0^1e^{-t}t^{s-1}\rd t + \int_1^\infty e^{-t}t^{s-1}\rd t \\
      &= \int_0^1\sum_{n=0}^\infty\frac{(-t)^n}{n!}t^{s-1}\rd t 
        + \int_1^\infty e^{-t}t^{s-1}\rd t \\
      &= \sum_{n=0}^\infty\frac{(-1)^n}{n!(n+s)}
        + \int_1^\infty e^{-t}t^{s-1}\rd t.
    \end{split}\end{equation}
    考虑它在整个$\mathbb{C}$上的行为. 首先第二项收敛于一个全纯函数,所以仅需要证明第一项的
    级数收敛于一个亚纯函数并有满足要求的极点. 考虑它在$D_R$中的行为,这样可以暂时不管$D_R$
    外的极点,同时按照$N>2R$将级数拆为两部分,前半部分为所需的亚纯函数而第二部分在$D_R$内
    全纯. 这里之所以选取$N>2R$是为了方便$n+s$的放缩. $\quad\blacksquare$
  % end

  \begin{pos}[沿拓的准周期性]
    仍用$\Gamma$来表示Gamma函数在$\mathbb{C}$上的解析沿拓,它有性质
    \begin{enumerate}
      \item 在非极点处成立$\Gamma(s+1)=s\Gamma(s)$.
      \item $\res_{-(n+1)}\Gamma(s)=-n\res_{-n}\Gamma(s)$.
    \end{enumerate}
  \end{pos}

  \begin{lemma}
    对于$0<a<1$,成立$\int_0^\infty v^{a-1}/(1+v)\rd v = \pi/\sin(\pi a)$.
  \end{lemma}
  \proof
    做代换$v=e^x$.
  % end

  \begin{thm}[Gamma函数的对称性]
    \label{thm: Gamma函数的对称性}
    对于任意$s\in\mathbb{C}$,成立
    \[
      \Gamma(s)\Gamma(1-s) = \frac{\pi}{\sin\pi s}.
    \]
  \end{thm}
  \remark
    代入$s=1/2$,得$\Gamma(1/2)=\sqrt{\pi}$.
  % end

  \begin{thm}
    函数$\Gamma$有如下性质:
    \begin{enumerate}
      \item $1/\Gamma(s)$为整函数,在$s=0,-1,-2,\dots$有简单零点,此外无零点.
      \item 对于$1/\Gamma(s)$有估计
        \[
          \left|\frac{1}{\Gamma(s)}\right|\le c_1e^{c_2|s|\log|s|},
        \]
        从而它的阶为$1$.
    \end{enumerate}
  \end{thm}
  \proof
    对于[1.]只需要利用\thmref{thm: Gamma函数的对称性}即可并利用\equref{equ:
    Gamma的延拓}来证明[2.]. 另外根据\defref{defi: 整函数的阶}的评注,只需要
    考虑充分大的$|s|$即可.\par
    首先有
    \[
      \frac{1}{\Gamma(s)} = \frac{\sin \pi s}{\pi}
      \left\{\sum_{n=0}^\infty\frac{(-1)^n}{n!(n+s)}+
      \int_1^\infty e^{-t}t^{-s}\rd t\right\}.
    \]
    对两项分别估计,首先是积分的部分. 由于对于大的$t$,主要的部分为$e^{-t}$,
    所以对于$t^{-s}$可以进行较大的放缩. 设$|\sigma|\le n<|\sigma|+1$可以进行估计
    \[\begin{split}
      \left|\int_1^\infty e^{-t}t^{-s}\rd t\right|
      &\le \int_0^\infty e^{-t}t^{|\sigma|}\rd t \\
      &\le n! \\
      &\le n^n \\
      &\le e^{n\log n} \\
      &\le e^{(|\sigma|+1)\log(|\sigma|+1)}
    \end{split}\]
    而对于$|\sin\pi s|$,利用Euler公式有估计$|\sin\pi s|\le e^{\pi|s|}$. 由于
    只需要考虑充分大的$|s|$,所以这一部分的估计是成立的.\par
    而对于级数部分的估计,如果不是$s=-n$附近的部分,它的行为主要由$\sin\pi s$确定,
    下考虑$\Im(s)\le 1$的情况,对于$k-1/2\le \Re(s) < k+1/2$的那一项单独处理
    即可.$\quad\blacksquare$
  % end
  
  \begin{thm}
    记$\gamma = \lim_{N\to\infty}\sum_{n=1}^N 1/n - \log N$为Euler常数,
    根据Hadamard定理有
    \[
      \frac{1}{\Gamma(s)} = e^{\gamma s}s\prod_{n=1}^\infty
      \left( 1+\frac{s}{n} \right)e^{-s/n}.
    \]
  \end{thm}

% end

% end

\newpage
\section{附录}

\subsection{不等式}

  \begin{pos}
    对于任意$z\in\mC$,成立$|e^z-1|\le e^{|z|}-1\le |z|e^{|z|}$.
  \end{pos}

  \begin{pos}
    对于任意$z\in\mathbb{D}$,成立$|e^z-1|\le 2|z|$.
  \end{pos}

  \begin{pos}
    对于任意$z\in\mathbb{C}$,成立$|e^z|\ge e^{-|z|}$.
  \end{pos}

% end

\subsection{数项级数}

  \begin{thm}
    考虑复数项级数$\{a_n\}$和$\{c_n\}$,若当$n$充分大时,
    \begin{enumerate}
      \item $|a_n| \le |c_n|$且$c_n$收敛,则$a_n$绝对收敛.
      \item $|a_n| \ge |c_n|$且$c_n$发散,则$a_n$发散.
    \end{enumerate}
  \end{thm}

  \begin{thm}[分部求和]
    \label{thm: 分部求和}
    设有复序列$\{a_n\}$和$\{b_n\}$而$B_n=\sum_{i=1}^nb_n$,则
    \[
      \sum_{n=M}^N a_nb_n = a_NB_N - a_MB_{M-1} -
      \sum_{n=M}^{N-1}(a_{n+1}-a_n)B_n.
    \]
  \end{thm}
% end 

\subsection{函数项级数}

  \begin{thm}
    设$\{f_n\}$是一列函数且对于任意$n$,成立$\sup|f_n-f|\le M_n$且$M_n\to 0$,
    则$f_n$一致收敛于$f$.
  \end{thm}

% end

\subsection{多元微积分}

  \begin{thm}[偏导数的极坐标表示]
    \label{thm: 偏导数的极坐标表示}
    对于$f:E\subset\R^2\to\R$,$z=f(x,y)$,它的偏导数在极坐标下的表示为
    \[\begin{split}
      \frac{\pt z}{\pt x} &= \frac{\pt z}{\pt r}\cos\theta
        - \frac{1}{r}\frac{\pt z}{\pt\theta}\sin\theta,\\
      \frac{\pt z}{\pt y} &= \frac{\pt z}{\pt r}\sin\theta
       + \frac{1}{r}\frac{\pt z}{\pt\theta}\cos\theta.
    \end{split}\]
  \end{thm}
  \proof
    利用链式法则并求解方程,得
    \begin{gather*}
      \begin{pmatrix}
        \pt z/\pt r \\ \pt z/\pt \theta
      \end{pmatrix} =
      \begin{pmatrix}
        \pt x/\pt r & \pt y/\pt r \\
        \pt x/\pt \theta & \pt y/\pt\theta
      \end{pmatrix}
      \begin{pmatrix}
        \pt z/\pt x \\ \pt z/\pt y
      \end{pmatrix} \\
      \Rightarrow\quad
      \begin{pmatrix}
        \pt z/\pt x \\ \pt z/\pt y
      \end{pmatrix} =
      \begin{pmatrix}
        \cos\theta & \sin\theta \\
        -r\sin\theta & r\cos\theta
      \end{pmatrix}^{-1}
      \begin{pmatrix}
        \pt z/\pt r \\ \pt z/\pt \theta
      \end{pmatrix}=
      \begin{pmatrix}
        \cos\theta & -\sin\theta / r \\
        \sin\theta & \cos\theta / r
      \end{pmatrix}
      \begin{pmatrix}
        \pt z/\pt r \\ \pt z/\pt \theta
      \end{pmatrix}.\quad\blacksquare
    \end{gather*}

  \begin{pos}[极坐标]
    对于$f:E\subset\R^2\to\R$,$z=f(x,y)$,成立
    \[
      \left(\frac{\pt z}{\pt x}\right)^2 + \left(\frac{\pt z}{\pt y}\right)^2
      = \left(\frac{\pt z}{\pt r}\right)^2
        + \frac{1}{r^2}\left(\frac{\pt z}{\pt\theta}\right)^2.
    \]
  \end{pos}
  \remark
    需要说明的是,左边的偏导数中将$f$看作了关于$(x,y)$的函数而右边的偏导数中
    将$f$看作的是关于$(r,\theta)$的函数.
  \proof
    将$f$看作自变量为$(r,\theta)$的复合函数,则
    \[\begin{split}
      \frac{\pt z(r,\theta)}{\pt r} &=
      \frac{\pt z}{\pt x}\frac{\pt x}{\pt r} + \frac{\pt z}{\pt y}\frac{\pt y}{\pt r}
      = \frac{\pt z}{\pt x}\cos\theta + \frac{\pt z}{\pt y}\sin\theta, \\
      \frac{\pt z(r,\theta)}{\pt\theta} &=
      \frac{\pt z}{\pt x}\frac{\pt x}{\pt\theta} + \frac{\pt z}{\pt y}\frac{\pt y}{\pt\theta}
      = \frac{\pt z}{\pt x}(-r\cos\theta) + \frac{\pt z}{\pt y}(r\sin\theta).
    \end{split}\]
    将第二个式子两边同除$r$后求上两个式的平方和即可.$\quad\blacksquare$
  % end
% end

\end{document}
