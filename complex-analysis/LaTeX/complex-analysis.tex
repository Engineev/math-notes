\documentclass[12pt, a4paper]{article}
\usepackage{ctex}

\usepackage[margin=1in]{geometry}
\usepackage{
  color,
  clrscode,
  amssymb,
  ntheorem,
  amsfonts,
  amsmath,
  listings,
  fontspec,
  xcolor,
  supertabular,
  multirow,
  mathtools,
  mathrsfs,
}
\definecolor{bgGray}{RGB}{36, 36, 36}
\usepackage[
  colorlinks,
  linkcolor=bgGray,
  anchorcolor=blue,
  citecolor=green
]{hyperref}
\newfontfamily\courier{Courier}

\theoremstyle{margin}
\theorembodyfont{\normalfont}
\newtheorem{thm}{定理}
\newtheorem{cor}[thm]{推论}
\newtheorem{pos}[thm]{命题}
\newtheorem{lemma}[thm]{引理}
\newtheorem{defi}[thm]{定义}

\DeclareMathOperator{\rank}{rank}
\DeclareMathOperator{\adj}{adj}
\DeclareMathOperator{\tr}{tr}
\DeclareMathOperator{\diag}{diag}
\DeclareMathOperator{\nul}{null}
\DeclareMathOperator{\range}{range}
\DeclareMathOperator{\spn}{span}
% \DeclareMathOperator{\deg}{deg}

\newcommand{\hp}{^\prime}
\newcommand{\vep}{\varepsilon}
\newcommand{\inv}{^{-1}}
\newcommand{\rd}{\mathrm{d}}

\renewcommand{\Im}{\text{Im}}
\renewcommand{\Re}{\text{Re}}



\title{复分析$\,$笔记}
\author{任云玮}
\date{}


\begin{document}
\maketitle
\tableofcontents

\newpage
\section{绪论}

\subsection{复数与复平面}

  复数相关定义与性质、复平面相关拓扑性质略

\subsection{复平面上函数}

  \paragraph{说明}
    极限、连续函数相关的定义与性质略,
    它们在对于一般度量空间中的函数的讨论中已经讨论过了.

  \begin{defi}[极值]
    对于定义在$\Omega\subset\mC$上的复函数$f$,称$f$在$z_0\in\Omega$
    处达到极大值,若对任意$z\in\Omega$,成立$|f(z)| \le |f(z_0)|$.
    对于极小值同理.
  \end{defi}
  \remark
    由于复数之间是没有大小关系的,所以极值是利用绝对值定义的.

  \begin{defi}[全纯\protect\footnotemark]
    \footnotetext{Holomorphic}
    设$f$是定义在开集$\Omega\subset\mC$的复函数. 称$f$在点$z_0
    \in\Omega$处全纯,若存在有限极限
    \[
      \lim_{h\to 0}\frac{f(z_0+h) - f(z_0)}{h}.
    \]
    其中$h\ne 0$且$z_0+h\in\Omega$. 称$f$在$\Omega$上全纯,若
    它在每个点上全纯. 若$f$在$\mC$上全纯,则称它是整的.
  \end{defi}
  \remark
    这一定义和实变量函数的可导在形式上是一致的,但是实际上这一条件相较于
    可导要强很多. 另外,和、差等函数的相关公式以及链式法则都是成立的,
    在此略去.

  \begin{pos}[有限增量公式]
    设$f$在$z_0$处全纯,则$f(z_0+h)=f(z_0)+f\hp(z_0)h + h\psi(h)$,
    其中$\lim_{h\to 0}\psi(h)=0$.
  \end{pos}

  \paragraph{复函数与映射}
    一个复函数$f$在一定程度上可以看作$F:\R^2\to\R^2$的映射,但是它们仍
    有着本质的区别. 首先$f$的全纯和$F$的可导是不等价的,考虑非全纯的函数
    $f(z)=\bar{z}$,它对应的$F$是无穷次可微的. 另一方面,$f$的复导数
    是一个复数,而$F$的导数则是对应的Jacob矩阵. 但是,它们之间仍是有联系
    的. 它们之间的关联见\thmref{thm: Cauchy-Riemann方程1}以及\thmref{thm:
    Cauchy-Riemann方程2}.

  \begin{defi}
    定义
    $\dfrac{\pt}{\pt z} = \dfrac{1}{2}\left(
    \dfrac{\pt}{\pt x} + \dfrac{1}{i}\dfrac{\pt}{\pt y}
    \right)$,
    $\dfrac{\pt}{\pt \bar{z}} = \dfrac{1}{2}\left(
    \dfrac{\pt}{\pt x} - \dfrac{1}{i}\dfrac{\pt}{\pt y}
    \right)$.
  \end{defi}

  \begin{thm}[Cauchy-Riemann方程]
    \label{thm: Cauchy-Riemann方程1}
    设$f=u+\iu v$在$z_0$处全纯,则
    \begin{equation}
        \label{equ: Cauchy-Riemann方程}
        \frac{\pt f}{\pt \bar{z}}(z_0)=0\quad\text{且}\quad
        f\hp(z_0) = \frac{\pt f}{\pt z}(z_0)
        = 2\frac{\pt u}{\pt z}(z_0).
    \end{equation}
     设$F:\R^2\to\R^2$为$f$对应的映射,则$F$可微且成立
     \[
      \det J_F(x_0,y_0) = |f\hp(z_0)|^2.
     \]
  \end{thm}
  \remark
    设$f=u+\iu v$,则Cauchy-Riemann方程也可以写作
    \begin{equation}
      \label{equ: Cauchy-Riemann方程2}
      \frac{\pt u}{\pt x} = \frac{\pt v}{\pt y},\quad
      \frac{\pt u}{\pt y} = -\frac{\pt v}{\pt x}.
    \end{equation}
  \proof
    $f(z)=f(x+\iu y)=f(x, y)$,分别让$h$沿实轴和虚轴方向趋于零,得到
    两偏导数,根据全纯的定义,它们应相等,从而可得到\equref{equ: Cauchy-Riemann方程}中
    的前者. 同时,将它们相加再除以二即为$\pt f/\pt z$,同时利
    用\equref{equ: Cauchy-Riemann方程2},可以得到后者.\par
    而关于$F$的可微性,只需注意到若利用\equref{equ: Cauchy-Riemann方程2},则有
    \[\begin{split}
      &|F(x_0+h_1,x_0+h_2)-F(x_0,y_0)-J_F(x_0,y_0)(h_1,h_2)\tr|  \\
      =& \left|f(z_0+h)-f(z_0)-
      \left(\frac{\pt u}{\pt x}-\iu\frac{\pt u}{\pt y}\right)(h_1+\iu h_2)\right|
      = |f(z_0+h)-f(z_0)-f\hp(z_0)h|.
    \end{split}\]
    由全纯的定义可知$F$是可微的. 而最后只需要将Cauchy-Riemann方程应用到
    $J_F$行列式的表达式中即可完成所有的证明.$\quad\blacksquare$

  \begin{thm}[Cauchy-Riemann方程]
    \label{thm: Cauchy-Riemann方程2}
    设$f=u+\iu v$是定义在开集$\Omega$上的复函数. 设$u$和$v$连续
    可微且满足Cauchy-Riemann方程\equref{equ: Cauchy-Riemann方程2},
    则$f$在$\Omega$上全纯且$f\hp(z_0) = \pt f/\pt z$.
  \end{thm}
  \proof
    利用$f=u+\iu v$,分别对$u$和$v$利用有限增量公式展开即可.$\quad\blacksquare$

  \begin{thm}[Cauchy-Riemann方程]
    \label{thm: 极坐标Cauchy-Riemann方程}
    在极坐标下,Cauchy-Riemann方程的形式为
    \[
      \frac{\pt u}{\pt r} = \frac{1}{r}\frac{\pt v}{\pt\theta},\quad
      \frac{1}{r}\frac{\pt u}{\pt\theta} = -\frac{\pt v}{\pt r}.
    \]
  \end{thm}
  \proof
    首先将\thmref{thm: 偏导数的极坐标表示}的结果代入\equref{equ:
    Cauchy-Riemann方程2},则有
    \[\begin{split}
      (*1)\qquad\frac{\pt u}{\pt r}\cos\theta - \frac{1}{r}\frac{\pt u}{\pt\theta}\sin\theta
      &= \frac{\pt v}{\pt r}\sin\theta - \frac{1}{r}\frac{\pt v}{\pt\theta}\cos\theta\\
      (*2)\qquad\frac{\pt u}{\pt r}\sin\theta + \frac{1}{r}\frac{\pt u}{\pt\theta}\cos\theta
      &= -\frac{\pt v}{\pt r}\cos\theta + \frac{1}{r}\frac{\pt v}{\pt\theta}\sin\theta.
    \end{split}\]
    计算$(*1)\times r\cos\theta + (*2)\times r\sin\theta$即得结论中的前式,
    计算$(*1)\times r\sin\theta - (*2)\times r\cos\theta$即得结论中的后式.
    $\quad\blacksquare$

  \begin{thm}[链式法则]
    设$U$和$V$是复平面上的开集,$f:U\to V$和$g:V\to\mC$从实变量角度
    可微,定义$h=g\circ f$,则
    \[
      \frac{\pt h}{\pt z}=\frac{\pt g}{\pt z}\frac{\pt f}{\pt z}
      + \frac{\pt g}{\pt\bar{z}}\frac{\pt\bar{f}}{\pt z},\quad
      \frac{\pt h}{\pt\bar{z}} = \frac{\pt g}{\pt z}\frac{\pt f}{\pt\bar{z}}
      + \frac{\pt g}{\pt\bar{z}}\frac{\pt\bar{f}}{\pt\bar{z}}.
    \]
  \end{thm}
  \proof
    TODO

  \begin{defi}[调和]
    对于二阶连续可导的函数,定义Laplace算子为
    \[
      \Delta = \frac{\pt}{\pt x^2} + \frac{\pt}{\pt y^2}.
    \]
    称定义在复平面上开集$\Omega$中的实值函数$f$为调和的,若$\Delta f=0$在
    $\Omega$中成立.
  \end{defi}

  \begin{pos}
    $\Delta = 4\dfrac{\pt}{\pt z}\dfrac{\pt}{\pt\bar{z}}
    = 4\dfrac{\pt}{\pt\bar{z}}\dfrac{\pt}{\pt z}$.
  \end{pos}

  \begin{cor}[调和]
    若$f$在开集$\Omega$上全纯,则它的实部和虚部分别调和.
  \end{cor}

  \begin{defi}[Blaschke因子]
    对于单位圆盘$\mathbb{D}$内的复数$w$,定义Blaschke因子为
    \[
      F: z\mapsto \frac{w-z}{1-\bar{w}z},\quad z\in\mC.
    \]
  \end{defi}

  \begin{thm}[Blaschke因子]
    Blaschke因子满足如下性质:
    \begin{enumerate}
      \item $F(\mathbb{D})\subset\mathbb{D}$,且$F$全纯.
      \item $F(0)=w$且$F(w)=0$.
      \item 若$|z|=1$,则$|F(z)|=1$.
      \item $F:\mathbb{D}\to\mathbb{D}$为双射.
    \end{enumerate}
  \end{thm}
  \proof
    TODO

  \begin{pos}[常数]
    设$f$是定义在连通开集$\Omega$上的全纯函数,若$\Re(f)$,$\Im(f)$或
    $|f|$在$\Omega$上为常数,则$f$也为常数.
  \end{pos}
  \proof
    由于连通开集必然是路径连通的,所以只需要证明对应的$J_f=0$即可. 对于
    前两者,证明是显然的. 对于$|f|=C$的情况,只需要考虑极坐标的形式并利
    用\thmref{thm: 极坐标Cauchy-Riemann方程}即可.$\quad\blacksquare$

\subsection{幂级数}

  \begin{thm}[幂级数收敛半径]
    给定幂级数$\sum_{n=0}^\infty a_nz^n$,在扩充实数域中存在$R$使得
    \begin{enumerate}
      \item 若$|z|<R$,则幂级数绝对收敛.
      \item 若$|z|>R$,则幂级数发散.
    \end{enumerate}
    并且,$R$由Hadamard公式确定
    \[
      \frac{1}{R} = \limsup|a_n|^{1/n}.
    \]
  \end{thm}
  \proof
    对于固定的$z$,取常数$s\in(|z|, R)$,将$\sum|a_n||z|^n$放缩成
    幂级数即可.$\quad\blacksquare$

  \begin{thm}
    设$\{a_n\}$为非零复序列且满足$\lim_{n\to\infty}|a_{n+1}|/|a_n|=L$,
    则$\lim_{n\to\infty}\sqrt[n]{|a_n|}=L$.
  \end{thm}

  \begin{defi}[指数函数]
    \label{defi: 指数函数}
    对于$z\in\mC$,定义$e^z = \sum\limits_{n=0}^\infty\dfrac{z^n}{n!}$.
  \end{defi}
  \remark
    首先由于$\rhs$在$\R$中一致收敛至$e^x$,所以这一定义和原有的定义是一致的.
    与此同时,可以证明对于任意的$\rhs$对任意$z\in\mC$是收敛的,所以这一定义
    是良定义的.

  \begin{defi}[三角函数]
    \label{defi: 三角函数}
    对于$z\in\mC$,定义三角函数
    \[
      \cos z = \sum_{n=0}^\infty(-1)^n\frac{z^{2n}}{(2n)!},\quad
      \sin z = \sum_{n=0}^\infty(-1)^n\frac{z^{2n+1}}{(2n+1)!}.
    \]
  \end{defi}

  \begin{thm}[Euler公式]
    \label{thm: Euler公式}
    对于$z\in\mC$,成立$e^{iz} = \cos z + \iu\sin z$.
  \end{thm}
  \proof
    根据\defref{defi: 指数函数}和\defref{defi: 三角函数},不难验证
    上式的正确性. 需要注意,此两级数相加和换序的正确性是由它们的绝对收敛性
    保证的.$\quad\blacksquare$

  \begin{thm}
    幂级数$f(z)=\sum_{n=0}^\infty a_nz^n$在它的收敛圆盘内定义了一个
    全纯函数. 且$f$可以逐项求导,其导数$f\hp$的收敛半径与$f$的收敛半径相同.
  \end{thm}
  \proof
    关于$f$和$f\hp$的收敛半径相同的验证是简单的,下仅证明$f\hp$的存在性,
    且它可逐项求导得到,即$\sum_{n=0}^\infty na_nz^{n-1}$收敛至$g(z)$
    而$\lim_{h\to 0}\{(f(z+h)-f(z))/h - g(z)\} = 0$.\par
    设$g(z)=\sum_{n=0}^\infty na_nz^{n-1}$,$S_N(z)$为其前$N$
    项和而$E_N(z)$为其余项,对于任意的$z$,考虑
    \[\begin{split}
      \frac{f(z+h)-f(z)}{h} - g(z) = &
      \left( \frac{S_N(z+h)-S_N(z)}{h} - S\hp_n(z) \right)\\
      &+ (S\hp_N(z) - g(z)) + \left( \frac{E_N(z+h)-E_N(z)}{h} \right).
    \end{split}\]
    先让$N\to\infty$,可以证明后两项趋于零. 之后固定充分大的$N$,令$h\to 0$,
    则可以证明第一项趋于零.$\quad\blacksquare$

  \begin{cor}
    幂级数在它的收敛圆盘内定义了一个无穷次可导的复函数,且它的任意
    阶导数都可以通过逐项求导得到.
  \end{cor}

  \begin{defi}[解析]
    称复函数$f$在$z_0\in\mC$处解析,若存在$\delta>0$,在$O_\delta(z_0)$中$f$
    有幂级数展开.
  \end{defi}

\subsection{曲线积分}

  \begin{defi}[参数曲线]
    参数曲线是指映射$z:[a,b]\subset\R\to\mC$. 称它为光滑的,
    若$z$连续可导并且$z(t)\ne 0$. 称两个参数曲线$z$和$\tilde{z}:[c,d]\to\mC$
    是等价的,若存在连续可微的从$[c,d]$到$[a,b]$的双射$s\mapsto t(s)$成立
    $t\hp(s)>0$且$\tilde{z}=z(t(s))$.\par
    称分段光滑的曲线是闭合的,若$z(a)=z(b)$. 称它为简单的,若$z(x)=z(y)$可以
    推得$x=y$或$x,y\in\{a, b\}$. 通常,曲线一词指代分段光滑曲线.
  \end{defi}
  \remark
    注意,$\R^2$中的曲线通常有不止一种参数化方法.

  \begin{defi}[曲线积分]
    给定$\mC$中的光滑曲线$\gamma$,设$z:[a,b]\to\mC$是它的参数化而
    $f$是定义在$\gamma$上的连续函数,则定义$f$沿$\gamma$的积分为
    \[
      \int_\gamma f(z)\rd z = \int_a^b f(z(t))z\hp(t)\rd t.
    \]
  \end{defi}
  \remark
    可以证明$\rhs$的取值与参数化的方法无关,所以它是良定义的.

  \begin{defi}[曲线长度]
    给定$\mC$中的光滑曲线$\gamma$,设$z:[a,b]\to\mC$是它的参数化.
    定义$\gamma$的长度为
    \[
      \len(\gamma) = \int_a^b|z\hp(t)|\rd t.
    \]
  \end{defi}

  \begin{thm}
    连续函数的曲线积分满足如下性质.
    \begin{enumerate}
      \item 线性性.
      \item $\int_\gamma f(z)\rd z = -\int_{\gamma^-}f(z)\rd z$.
      \item $|\int_\gamma f(z)\rd z| \le \sup_{z\in\gamma}|f(z)|\times\len(\gamma)$.
    \end{enumerate}
  \end{thm}

  \begin{thm}
    设连续函数$f$在$\Omega$上有原函数$F$,$\gamma$是$\Omega$中以
    $w_1$为起点$w_2$为终点的曲线,则
    \[
      \int_\gamma f(z)\rd z = F(w_2) - F(w_1).
    \]
  \end{thm}

  \begin{cor}
    \label{cor: 曲线积分1}
    设连续函数$f$在$\Omega$上有原函数$F$,$\gamma$是$\Omega$中的
    闭曲线,则
    \[
      \oint_\gamma f(z)\rd z = 0.
    \]
  \end{cor}

  \begin{cor}
    若$f$在区域$\Omega$中全纯且$f\hp = 0$,则$f$为常值.
  \end{cor}

 
\newpage
\section{Cauchy定理及其应用}

\subsection{Goursat定理}

  \begin{thm}[Goursat]
    \label{thm: Goursat}
    设$\Omega$是$\mC$中的开集,$T\subset\Omega$是一个三角形,且它
    的内部也都在$\Omega$内. 设$f$在$\Omega$中全纯,则
    \[
      \int_T f(z)\rd z = 0.
    \]
  \end{thm}
  \remark
    这一定理同\corref{cor: 曲线积分1}最大的区别在于,它并不要求$f$的
    原函数,甚至相反的,它要求$f$全纯. 另外,如果将条件中的三角形换成
    长方形,结论依然是成立的,只需要将它分割成两个三角形再应用此定理即可
    证明.
  \proof
    通过连接各边中点来四分三角形. 选取其中$|\int f\rd z|$最大者作为
    下一个三角形,使得成立$|I_0|\le 4^n|I_n|$,即用沿着小三角形的积分
    来估计原积分. 另一方面,这些小三角形及其内部构成了一个紧集套,设它
    收缩至$z_0$. 利用在$z_0$处展开$f$至二次项以及\corref{cor:
    曲线积分1}来估计$I_n$. $\quad\blacksquare$

\subsection{局部原函数存在性与圆盘上的Cauchy定理}

  \begin{thm}[局部存在性]
    \label{thm: 局部存在性}
    开圆盘上的全纯函数在该圆盘上有原函数.
  \end{thm}
  \proof
    不失一般性的,可以假设圆盘以原点为中心.
    分为两步完成证明,首先定义一个无歧义的$F(z)$,接着证明$F(z)$
    是$f(z)$的原函数.
    \begin{enumerate}
      \item 选取连接原点和$z$的直角路径,定义沿该路径积分的结果为$F(z)$.
      \item 通过路径积分的相消和\thmref{thm: Goursat},证明$F(z+h)-
        F(z)$即为沿着连接两点的直线段积分的结果. 再进行进一步的估计与证明.
        $\quad\blacksquare$
    \end{enumerate}
  \remark
    这一定理表明对于全纯函数,至少在局部永远是有原函数的.

  \begin{thm}[圆盘上的Cauchy定理]
    设$f$在圆盘上全纯,则对于任意圆盘中的闭曲线$\gamma$,成立
    \[
      \int_\gamma f(z)\rd z = 0.
    \]
  \end{thm}

  \begin{cor}
    \label{cor: Cauchy定理}
    设$f$在开集$\Omega$上全纯,圆$C$及其内部都在$\Omega$中,则
    \[
      \int_C f(z)\rd z = 0.
    \]
  \end{cor}
  \remark
    实际这些定理对于包括锁孔形在内的所有可以方便定义内部的“简单”图形
    都成立.

\subsection{利用Cauchy定理积分}

  通常可以总结为如下步骤:
  \begin{enumerate}
    \item 选取恰当的全纯函数$f$.
    \item 接下来选取恰当的闭曲线$\gamma$,在其上对$f$应用\corref{cor: Cauchy定理}.
    \item 将$\int_\gamma f(z)\rd z$中$\gamma$拆分成不同的段,使得在其上的积分
      或互相抵消,或可以得出结果,或有原所求积分的形式.
    \item 整理之前的结果,进行诸如取极限等操作并得出结论.
  \end{enumerate}
  各个步骤之间的顺序并非一定的.

\subsection{Cauchy积分公式}

  \begin{thm}[Cauchy]
    \label{thm: Cauchy公式}
    设$f$在开集$\Omega$上全纯且$\Omega$包含圆盘$D$的闭包.
    记$C$为$D$的边界且取向为正,则对于任意$z\in D$,成立
    \[
      f(z) = \frac{1}{2\pi\iu}\int_C\frac{f(\zeta)}{\zeta-z}\rd\zeta.
    \]
  \end{thm}
  \remark
    这一定理给出了全纯函数在某个点的函数值的曲线积分表达式.
  \proof
    考虑将$z$排除的锁孔形$\Gamma_{\vep}$,$\vep$为锁孔半径. 则$F(\zeta)
    =f(\zeta)/(\zeta-z)$在$\Gamma$上全纯,所以对应的沿锁孔形的积分为零.
    接下来让走廊的宽度趋于零,由于连续性对应积分的变化也趋于零. 于是就有
    \[
      \int_C \frac{f(\zeta)}{\zeta-z}\rd\zeta
      = \int_{C\hp}\frac{f(\zeta)}{\zeta-z}\rd\zeta,
    \]
    其中$C\hp$为锁眼,即以$z$为圆心,半径为$\vep$的圆. 同时,对右侧做变量代换,令
    $\zeta = z+\vep e^{i\theta}$,同时令$\vep\to 0$,可得$\rhs=2\pi\iu f(z)$.
    $\quad\blacksquare$

  \begin{cor}[Cauchy]
    \label{cor: Cauchy}
    设$f$是在开集$\Omega$上的全纯函数,则$f$在复数含义下无限次可导.
    并且,如果$C\subset\Omega$是一个内部也在$\Omega$中的圆,则
    对于$C$内部的点,成立
    \[
      f^{(n)}(z) = \frac{n!}{2\pi\iu}\int_C\frac{f(\zeta)}{(\zeta-z)^{n+1}}\rd\zeta.
    \]
  \end{cor}
  \proof
    对$n$施归纳法即可.$\quad\blacksquare$
  \remark
    这一命题描述了全纯函数的正则性. 它意味着对于全纯函数而言,积分和微分实际上
    上一样的,例如如果要证明一个函数复数意义下可导,只需要证明它存在原函数即可,
    根据此命题自然而然就得出了该函数的全纯.

  \begin{cor}
    设$f$在开集$\Omega$上全纯,$D$是以$z_0$为圆心的半径为$R$圆盘,且它的闭包在
    $\Omega$内. 则成立
    \[
      |f^{(n)}(z_0)| \le \frac{n!\|f\|_\infty}{R^n}.
    \]
  \end{cor}

  \begin{thm}[幂级数展开]
    设$f$在开集$\Omega$上全纯. $D$是以$z_0$为圆心的圆盘且它的闭包在$\Omega$内,
    则$f$在$z_0$处有幂级数展开,即对任意$z\in D$成立
    \[
      f(z) = \sum_{n=0}^\infty a_n(z-z_0)^n,\qquad
      a_n = \frac{f^{(n)}(z_0)}{n!}.
    \]
  \end{thm}
  \remark
    这一定理表明全纯的条件在很大程度上已经意味着幂级数展开的存在性. 尤其是对于
    整函数,这一定理表明它在整个$\mC$上有幂级数展开.
  \proof
    方法在于首先应用Cauchy公式得到$f(z)$的积分表达式
    \[
      f(z) = \frac{1}{2\pi\iu}\int_C\frac{f(\zeta)}{\zeta-z}\rd\zeta.
    \]
    考虑被积函数,利用一致收敛的几何级数来得到级数的形式,同时化出$(z-z_0)^n$,
    具体方法为
    \[
      \frac{1}{\zeta-z} = \frac{1}{\zeta-z_0}\frac{1}{1-(\frac{z-z_0}{\zeta-z_0})}
      = \frac{1}{\zeta-z_0}\sum_{n=0}^\infty\left(\frac{z-z_0}{\zeta-z_0}\right)^n.
      \quad\blacksquare
    \]

  \begin{cor}[Liouville]
    若整函数$f$有界,则$f$为常值函数.
  \end{cor}

  \begin{cor}
    所有非常值复多项式$P(z)=a_nz^n + \cdots + a_0$在$\mC$中有一个根.
  \end{cor}
  \proof
    只需注意到如果$P$没有根,则$1/P$是一个有界整函数即可.

  \begin{cor}
    任意$n\ge 1$阶多项式$P(z)=a_nz^n + \cdots + a_0$在$\mC$中有恰$n$个根.
    且若设这些根为$w_1,\dots,w_n$,则$P$可被分解为
    \[
      P(z) = a_n(z-w_1)\cdots(z-w_n).
    \]
  \end{cor}
  \proof

  \begin{thm}
    设$f$在区域$\Omega$上全纯且在一列聚点在$\Omega$中的点处取值为零,则$f\equiv 0$.
  \end{thm}
  \proof
    TODO

  \begin{cor}
    设$f$和$g$在区域$\Omega$上全纯且$f(z)=g(z)$对于$\Omega$的某个非空开子集中
    的任意$z$成立,则在$\Omega$上成立$f\equiv g$.
  \end{cor}

\subsection{应用}

  \begin{thm}[Morera]
    \label{thm: Morera}
    设$f$是开圆盘$D$上的连续函数,并且对于包含于$D$中的三角形$T$,成立
    \[
      \int_T f(z)\rd z = 0,
    \]
    则$f$全纯.
  \end{thm}
  \proof
    由于复变函数的正规性,所以我们只需要证明$f$的原函数$F$全纯即可. 我们可以
    按照\thmref{thm: 局部存在性}中的方法构造处$F$,再验证一下即可.$\quad\blacksquare$

  \begin{thm}[级数]
    \label{thm: 级数、全纯}
    设$\{f_n\}$是一列全纯函数. 设在$\Omega$的任意紧子集中,它们一致收敛于$f$,
    则$f$在$\Omega$上全纯.
  \end{thm}
  \proof
    利用\thmref{thm: Morera}即可.$\quad\blacksquare$

  \begin{thm}
    设$\{f_n\}$是一列全纯函数. 设在$\Omega$的任意紧子集中,它们一致收敛于$f$,
    则$\{f_n\hp\}$在任意$\Omega$的紧子集中一致收敛于$f\hp$.
    \footnote{$f\hp$的存在性由\thmref{thm: 级数、全纯}.}
  \end{thm}
  \remark
    只需要反复应用\thmref{thm: 级数、全纯}以及此命题,就可以证明任意阶导数的一致收敛性.
  \proof
    用$|f_n-f|$来估计$|f_n\hp-f\hp|$即可. 若设$\Omega_\delta=
    \{z\in\Omega\,|\, \overline{D_\delta}(z)\subset\Omega\}$为所有离边界
    距离不小于$\delta$的点全体,可以证明成立不等式
    \[
      \sup_{z\in\Omega_\delta}|F\hp(z)| \le \frac{1}{\delta}\sup_{z\in\Omega}|F(z)|.
      \quad\blacksquare
    \]

  \begin{thm}[含参积分]
    设$F(z,s)$定义在$\Omega\times[0, 1]$上,其中$\Omega$是$\mC$中的开集.
    设$F$满足如下条件
    \begin{enumerate}
      \item 对任意固定的$s$,$F(z, s)$全纯.
      \item $F$在$\Omega\times[0, 1]$上连续.
    \end{enumerate}
    则如下定义在$\Omega$上的函数$f$全纯,
    \[
      f(z) = \int_0^1F(z,s)\rd s.
    \]
  \end{thm}
  \proof
    可以利用\thmref{thm: Morera}来证明,但这样需要验证积分换序的条件.
    为了避免这件事,我们可以考虑Riemann积分的定义,定义
    \[
      f_n = \frac{1}{n}\sum_{k=1}^nF(z,k/n).
    \]
    这样我们就将问题化归为\thmref{thm: 级数、全纯}条件的验证.$\quad\blacksquare$

  \begin{thm}[对称原理\protect\footnotemark]
    \footnotetext{关于定理中的记号,见书P58.}
    设$f^+$和$f^-$是分别定义在$\Omega^+$和$\Omega^-$上的全纯函数,且
    可以连续地沿拓到$I$上且成立$f^+(x)=f^-(x)$对任意$x\in I$成立,则
    按如下定义的函数$f$在$\Omega$上全纯
    \[
      f(z)=
      \begin{cases}
        f^+(z), &z\in\Omega^+, \\
        f^+(z)=f^-(z) &z\in I, \\
        f^-(z), & z\in\Omega^-.
      \end{cases}
    \]
  \end{thm}

  \begin{thm}[Schwarz镜像原理]
    设$f$在$\Omega^+$上全纯且其在$I$伤的连续沿拓为实函数. 则存在在整个$\Omega$上
    全纯函数$F$,在$\Omega^+$上成立$F=f$.
  \end{thm}

\subsection{说明}
  对于全纯函数而言,微分和积分是一体的. 如果要证明$f$全纯,则只需要构造处它的原函数
  $F$即可,则根据\corref{cor: Cauchy}可知,$f$是全纯的. 另一方面,若$f$全纯,
  则可以根据Goursat定理等,得到关于它的积分的诸多性质. 而Cauchy定理则意味着,除了
  在实函数中常见的级数展开,还可以使用积分来表示$f$在某一点的值. \par
  如果要证明函数$f$在某个区域$\Omega$内的某个性质,通常可以考虑通过逐点的取一个
  小圆盘的方式来证明,另外,如果这个函数全纯的话,则一般只需要取它的一个小开集即可.


\newpage
\input{ch3-meromorphic-functions-and-logarithm.tex}

\newpage
\section{Fourier变换}

\newpage
\section{整函数}

\subsection{Jensen公式}

  \begin{thm}[Jensen]
    \label{thm: Jensen}
    设$\Omega$是包含$\bar{D}_R$的开集,$f$在$\Omega$中全纯且$f(0)\ne 0$. 同时
    对$z\in C_R$,$f(z)\ne 0$. 记$z_1,\dots,z_N$为$f$在$D_R$中的零点,其中每个
    零点被计数其重数次,则
    \begin{equation}
      \label{equ: Jensen}
      \log|f(0)| = \sum_{k=1}^N\log\left( \frac{|z_k|}{R} \right)
      + \frac{1}{2\pi}\int_0^{2\pi}\log|f(Re^{\iu\theta})|\rd\theta.
    \end{equation}
  \end{thm}
  \remark
  \proof
    首先注意到如果$f_1$和$f_2$满足\equref{equ: Jensen},则$f_1f_2$也满足.
    所以我们考虑将$f$拆分成更加简单的形式并分别证明. 由于全纯,所以我们可以将它拆成
    \[
      f(z) = (z-z_1)\cdots (z-z_N)g(z).
    \]
    其中$g$在$\bar{D}_R$无零点,之后对于$g$和形如$x-w$的函数分别证明\equref{equ: Jensen}.\par
    对于无零点的$g$,要证的即为
    \[
      \log|g(0)| = \frac{1}{2\pi}\int_0^{2\pi}\log|f(Re^{\iu\theta})|\rd\theta.
    \]
    利用\thmref{thm: 全纯、对数存在性}和\corref{cor: 全纯、实部}的评注即可证明.\par
    而对于形如$z-w$的函数,其中$w$是圆盘内的任意一点,它在$w$处有唯一零点,所要证的式子即为
    \[
      \log|w|=\log\left( \frac{|w|}{R} \right) + \frac{1}{2\pi}\int_0^{2\pi}
      \log|Re^{\iu\theta} - w|\rd\theta \quad\Leftrightarrow\quad
      \int_0^{2\pi}\log|e^{\iu\theta}-a|\rd\theta = 0,
    \]
    其中$a=w/R$. 作变量代换,用$-\theta$替换$\theta$,则上式等价为
    \[
      \int_0^{2\pi}\log|1-ae^{\iu\theta}|\rd\theta = 0.
    \]
    利用和之前相同的方法证明即可.$\quad\blacksquare$
  % end

  \begin{lemma}
    条件同前,设$z_1,\dots,z_N$是$f$在$D_R$中的零点,设$\mathfrak{n}(r)$为$f$在
    $D_r$中的零点个数,则
    \[
      \int_0^R \mathfrak{n}(r)\frac{\rd r}{r} = 
      \sum_{k=1}^N\log\left| \frac{R}{z_k} \right|.
    \]
  \end{lemma}
  \proof
    利用
    \[
      \log\left| \frac{R}{z_k} \right|=\int_{|z_k|}^R\frac{\rd r}{r},
      \quad n_k(r) = ( r > |z_k| )?.\quad\blacksquare
    \]
  % end

  \begin{cor}
    条件同前. 此推论描述了零点和函数值大小之间的关系.
    \[
      \int_0^R\mathfrak{n}(r)\frac{\rd r}{r} = \frac{1}{2\pi}
      \int_0^{2\pi}\log|f(Re^{\iu\theta})|\rd\theta - \log|f(0)|.
    \]
  \end{cor}

% end
% end

\newpage
\section{附录}

\subsection{数项级数}

  \begin{thm}
    考虑复数项级数$\{a_n\}$和$\{c_n\}$,若当$n$充分大时,
    \begin{enumerate}
      \item $|a_n| \le |c_n|$且$c_n$收敛,则$a_n$绝对收敛.
      \item $|a_n| \ge |c_n|$且$c_n$发散,则$a_n$发散.
    \end{enumerate}
  \end{thm}

  \begin{thm}[分部求和]
    \label{thm: 分部求和}
    设有复序列$\{a_n\}$和$\{b_n\}$而$B_n=\sum_{i=1}^nb_n$,则
    \[
      \sum_{n=M}^N a_nb_n = a_NB_N - a_MB_{M-1} -
      \sum_{n=M}^{N-1}(a_{n+1}-a_n)B_n.
    \]
  \end{thm}

\subsection{函数项级数}

  \begin{thm}
    设$\{f_n\}$是一列函数且对于任意$n$,成立$\sup|f_n-f|\le M_n$且$M_n\to 0$,
    则$f_n$一致收敛于$f$.
  \end{thm}

\subsection{多元微积分}

  \begin{thm}[偏导数的极坐标表示]
    \label{thm: 偏导数的极坐标表示}
    对于$f:E\subset\R^2\to\R$,$z=f(x,y)$,它的偏导数在极坐标下的表示为
    \[\begin{split}
      \frac{\pt z}{\pt x} &= \frac{\pt z}{\pt r}\cos\theta
        - \frac{1}{r}\frac{\pt z}{\pt\theta}\sin\theta,\\
      \frac{\pt z}{\pt y} &= \frac{\pt z}{\pt r}\sin\theta
       + \frac{1}{r}\frac{\pt z}{\pt\theta}\cos\theta.
    \end{split}\]
  \end{thm}
  \proof
    利用链式法则并求解方程,得
    \begin{gather*}
      \begin{pmatrix}
        \pt z/\pt r \\ \pt z/\pt \theta
      \end{pmatrix} =
      \begin{pmatrix}
        \pt x/\pt r & \pt y/\pt r \\
        \pt x/\pt \theta & \pt y/\pt\theta
      \end{pmatrix}
      \begin{pmatrix}
        \pt z/\pt x \\ \pt z/\pt y
      \end{pmatrix} \\
      \Rightarrow\quad
      \begin{pmatrix}
        \pt z/\pt x \\ \pt z/\pt y
      \end{pmatrix} =
      \begin{pmatrix}
        \cos\theta & \sin\theta \\
        -r\sin\theta & r\cos\theta
      \end{pmatrix}^{-1}
      \begin{pmatrix}
        \pt z/\pt r \\ \pt z/\pt \theta
      \end{pmatrix}=
      \begin{pmatrix}
        \cos\theta & -\sin\theta / r \\
        \sin\theta & \cos\theta / r
      \end{pmatrix}
      \begin{pmatrix}
        \pt z/\pt r \\ \pt z/\pt \theta
      \end{pmatrix}.\quad\blacksquare
    \end{gather*}

  \begin{pos}[极坐标]
    对于$f:E\subset\R^2\to\R$,$z=f(x,y)$,成立
    \[
      \left(\frac{\pt z}{\pt x}\right)^2 + \left(\frac{\pt z}{\pt y}\right)^2
      = \left(\frac{\pt z}{\pt r}\right)^2
        + \frac{1}{r^2}\left(\frac{\pt z}{\pt\theta}\right)^2.
    \]
  \end{pos}
  \remark
    需要说明的是,左边的偏导数中将$f$看作了关于$(x,y)$的函数而右边的偏导数中
    将$f$看作的是关于$(r,\theta)$的函数.
  \proof
    将$f$看作自变量为$(r,\theta)$的复合函数,则
    \[\begin{split}
      \frac{\pt z(r,\theta)}{\pt r} &=
      \frac{\pt z}{\pt x}\frac{\pt x}{\pt r} + \frac{\pt z}{\pt y}\frac{\pt y}{\pt r}
      = \frac{\pt z}{\pt x}\cos\theta + \frac{\pt z}{\pt y}\sin\theta, \\
      \frac{\pt z(r,\theta)}{\pt\theta} &=
      \frac{\pt z}{\pt x}\frac{\pt x}{\pt\theta} + \frac{\pt z}{\pt y}\frac{\pt y}{\pt\theta}
      = \frac{\pt z}{\pt x}(-r\cos\theta) + \frac{\pt z}{\pt y}(r\sin\theta).
    \end{split}\]
    将第二个式子两边同除$r$后求上两个式的平方和即可.$\quad\blacksquare$


\end{document}
