%!TEX program = xelatex
\documentclass[12pt, a4paper]{article}
\usepackage{ctex}

\usepackage[margin=1in]{geometry}
\usepackage{
  color,
  clrscode,
  amssymb,
  ntheorem,
  amsfonts,
  amsmath,
  listings,
  fontspec,
  xcolor,
  supertabular,
  multirow,
  mathtools,
  mathrsfs,
}
\definecolor{bgGray}{RGB}{36, 36, 36}
\usepackage[
  colorlinks,
  linkcolor=bgGray,
  anchorcolor=blue,
  citecolor=green
]{hyperref}
\newfontfamily\courier{Courier}

\theoremstyle{margin}
\theorembodyfont{\normalfont}
\newtheorem{thm}{定理}
\newtheorem{cor}[thm]{推论}
\newtheorem{pos}[thm]{命题}
\newtheorem{lemma}[thm]{引理}
\newtheorem{defi}[thm]{定义}

\DeclareMathOperator{\rank}{rank}
\DeclareMathOperator{\adj}{adj}
\DeclareMathOperator{\tr}{tr}
\DeclareMathOperator{\diag}{diag}
\DeclareMathOperator{\nul}{null}
\DeclareMathOperator{\range}{range}
\DeclareMathOperator{\spn}{span}
% \DeclareMathOperator{\deg}{deg}

\newcommand{\hp}{^\prime}
\newcommand{\vep}{\varepsilon}
\newcommand{\inv}{^{-1}}
\newcommand{\rd}{\mathrm{d}}

\renewcommand{\Im}{\text{Im}}
\renewcommand{\Re}{\text{Re}}



\title{复分析作业$\,$W1}
\author{\small 任云玮\\\small2016级ACM班\\\small516030910586}
\date{}

\begin{document}
\maketitle

\noindent 1、计算:\\
(1) $(1+\iu)\pm(1-2\iu)$,(并作图)
\ans
  图见\figref{fig1}.
  \[\begin{split}
    (1+\iu)+(1-2\iu) &= 2 - \iu\\
    (1+\iu)-(1-2\iu)&=  3\iu \quad\blacksquare
  \end{split}
  \]
  \begin{figure}[htbp]
    \centering
    \includegraphics[width=8cm]{./image/W1_1_1.png}
    \caption{}
    \label{fig1}
  \end{figure}
\\(2) $\dfrac{\iu}{(\iu-1)(\iu-2)(\iu-3)}$.
\ans
  \[
    \frac{\iu}{(\iu-1)(\iu-2)(\iu-3)}=
    \frac{\iu}{\iu^3 + \iu^2(-1-2-3) + \iu(6+3+2) + 1\times 2\times3}
    =\frac{\iu}{\iu 10} = \frac{1}{10}.\quad\blacksquare
  \]
\\(3) $\sqrt{2}(\cos\alpha + \iu\sin\alpha)(\cos\beta+\iu\sin\beta)$,
  其中$0<\alpha,\beta<\pi/2$,$\alpha=\arctan2$,$\beta=\arctan3$.
\ans
  已知$\tan(\alpha+\beta)=-1$,
  $\cos(\alpha+\beta)=-\sqrt{1/(\tan^2(\alpha+\beta))}=-1/\sqrt{2}$,
  $\sin(\alpha+\beta)=\sqrt{1-\cos^2(\alpha+\beta)}=1/\sqrt{2}$,所以
  \[\begin{split}
    \sqrt{2}(\cos\alpha + \iu\sin\alpha)(\cos\beta+\iu\sin\beta)
    = \sqrt{2}(\cos(\alpha+\beta) + \iu\sin(\alpha+\beta))
    = -1+\iu.\quad\blacksquare
  \end{split}\]

\vspace{1cm}
\par\noindent 3、证明:\\
(1) 当且仅当$z=\bar{z}$,复数$z$为实数.
\proof
  设$z=a+\iu b$,则
  $
    z=\bar{z}\,\Leftrightarrow\, a+\iu b = a-\iu b
    \,\Leftrightarrow\, b = -b \,\Leftrightarrow\, b = 0.
  $
  即$z\in\R$.$\quad\blacksquare$ \\
(2) 设$z_1$和$z_2$为复数,若$z_1+z_2$和$z_1z_2$都是实数,则或$z_1$和
    $z_2$都是实数,或它们是一对共轭复数.
\proof
  若$\bar{z}_1=z_2$,则命题成立. 设$z_1=a+\iu b$,$z_2=c+\iu d$.
  若$\bar{z}_1\ne z_2$,则$a\ne c$或$b\ne -d$. 而$z_1+z_2\in\R$,即
  $b = -d$,所以$a\ne c$. 而$z_1z_2\in\R$,即$ad=-bc$. 所以$b=-d=0$,
  即$z_1,z_2\in\R$. 综上,证毕. $\quad\blacksquare$

\vspace{1cm}
\par\noindent 4、求复数$\dfrac{z-1}{z+1}$的实部及虚部.
\ans
  设$z=a+\iu b$,
  \[
    \frac{z-1}{z+1} = 1 - \frac{2}{a+1+\iu b} =
    1 - \frac{2(a+1) - \iu 2b}{(a+1)^2 + b^2} \quad\Rightarrow\quad
    \begin{cases}
      \Re\left(\dfrac{z-1}{z+1}\right) = \dfrac{a^2+b^2-1}{(a+1)^2+b^2}\\
      \Im\left(\dfrac{z-1}{z+1}\right) = \dfrac{2b}{(a+1)^2+b^2}.
    \end{cases}\quad\blacksquare
  \]


\vspace{1cm}
\par\noindent 5、设$z_1,z_2\in\bC$,求证\\
(1) $|z_1-z_2|^2=|z_1|^2+|z_2|^2-2\Re(z_1\bar{z}_2)$.
\proof
  $
    |z_1-z_2|^2 = (z_1-z_2)(\bar{z}_1-\bar{z}_2) =
    z_1\bar{z}_2 + z_2\bar{z}_2 - (z_1\bar{z_2} + \overline{z_1\bar{z}_2})
    = |z_1|^2 + |z_2|^2 - 2\Re(z_1\bar{z}_2).\quad\blacksquare
  $
(2) $|z_1-z_2| \ge ||z_1| - |z_2||$.
\proof
  首先$\Re(z_1\bar{z}_2) \le |z_1\bar{z}_2| = |z_1||z_2|$,而
  \[
    |z_1-z_2|^2 = |z_1|^2 + |z_2|^2 - 2\Re(z_z\bar{z}_2)
    \ge |z_1|^2 + |z_2|^2 - 2|z_1||z_2| = ||z_1|-|z_2||^2.
    \quad\blacksquare
  \]
(3) $|z_1+z_2|^2 + |z_1-z_2|^2 = 2(|z_1|^2 + |z_2|^2)$,并说明其几何意义.
\proof
  几何意义:平行四边形的两对角线长度的平方和等于四边长度的平方和.
  \[
    |z_1+z_2|^2 + |z_1-z_2|^2 = |z_1|^2+|z_2|^2 + 2\Re(z_1\bar{z}_2)
    + |z_1|^2+|z_2|^2 - 2\Re(z_1\bar{z}_2) = 2(|z_1|^2+|z_2|^2).
    \quad\blacksquare
  \]

\vspace{1cm}
\par\noindent 8、如果$|z_1|=|z_2|=|z_3|=1$,且$z_1+z_2+z_3=0$,证明
  $z_1$,$z_2$,$z_3$是内接于单位元内的一个正三角形的顶点.
\proof
  由于$|z_i|=1$,所以$z_i$在单位圆上. $z_1+z_2+z_3 = 0$,所以
  $|z_1+z_2| = |-z_3| = 1$. 而
  \[
    |z_1+z_2|^2+|z_1-z_2|^2 = 2(|z_1|^2+|z_2|^2) = 4
    \quad\Rightarrow\quad
    |z_1-z_2|^2 = 3.
  \]
  同理,$|z_2-z_3|^2=|z_3-z_1|^2=3$. 因此$z_i$在内接于单位圆的等边三角形的
  顶点.$\quad\blacksquare$

\vspace{1cm}
\par\noindent 14 设$|z_0|<1$,证明:\\
(A) 若$|z|=1$,那么 $\left|\dfrac{z-z_0}{1-\bar{z}_0z}\right|=1$.\\
(B) 若$|z|<1$,证明:\\
(1) $\left|\dfrac{z-z_0}{1-\bar{z}_0z}\right|<1$.\\
(2) $1-\left|\dfrac{z-z_0}{1-\bar{z}_0z}\right|^2 =
     \dfrac{(1-|z_0|^2)(1-|z|^2)}{|1-\bar{z}_0z|^2}$.\\
(3) $\dfrac{||z|-|z_0||}{1-|z_0||z|} \le
     \left|\dfrac{z-z_0}{1-\bar{z}_0z}\right| \le
      \dfrac{|z|+|z_0|}{1+|z_0||z|}$.\\
(4) $\left|\dfrac{z-z_0}{1-\bar{z}_0z}\right| \le |z|+|z_0|$.
\proof
  首先,根据5(1),成立
  \begin{equation}
    \label{eq1}
    |z-z_0|^2 = |z|^2 + |z_0|^2 - 2\Re(\bar{z}_0z)
    ,\quad
    |1-\bar{z}_0z|^2 = 1 + |z_0|^2|z|^2 - 2\Re(\bar{z}_0z).
  \end{equation}
(A) 将$|z|=1$代入\equref{eq1},得
  \[
    |z-z_0|^2 = 1 + |z_0|^2 - 2\Re(\bar{z}_0z) = |1-\bar{z}_0z|^2
    \quad\Rightarrow\quad \left|\frac{z-z_0}{1-\bar{z}_0z}\right| = 1. \quad\blacksquare
  \]
(B1) 将$|z|<1$代入\equref{eq1},有
  \begin{gather*}
    |z-z_0|^2 - |1-\bar{z}_0z|^2 = |z|^2 + |z_0|^2 - 1 - |z_0|^2|z|^2
     = (|z_0|^2-1)(1-|z|^2) < 0 \\
    \Rightarrow\quad
    \left|\frac{z-z_0}{1-\bar{z}_0z}\right| < 1.\quad\blacksquare
  \end{gather*}
(B2) 将\equref{eq1}代入$\lhs$,得
  \[
    \lhs = \frac{1+|z_0|^2|z|^2 - |z|^2 - |z_0|^2}{|1-z_0z|^2} = \rhs
    \quad\blacksquare
  \]
(B3) 已知$\Re(\bar{z}_0z) \le |z_0||z|$,根据糖水不等式,有
  \[
    \frac{|z-z_0|^2}{|1-z_0z|^2} =
    \frac{|z|^2+|z_0|^2-2\Re(\bar{z}_0z)}{1+|z_0|^2|z|^2-2\Re(\bar{z}_0z)}
    \ge \frac{|z|^2+|z_0|^2 - 2|z_0||z|}{1+|z_0|^2|z|^2-2|z_0||z|} =
    \frac{||z|-|z_0||^2}{|1-\bar{z}_0z|^2}.
  \]
  不等式的另一边同理. $\quad\blacksquare$\\
(B4) 根据(B3),成立
  \[
    \frac{|z-z_0|^2}{|1-z_0z|^2} \le \frac{|z|+|z_0|}{1+|z_0||z|} \le |z|+|z_0|.
    \quad\blacksquare
  \]

\vspace{1cm}
\par\noindent 15、设有限复数$z_1$和$z_2$在复球面上表示为$P_1$和$P_2$
  两点. 求证$P_1$及$P_2$的距离,$P_1$及$N$的距离分别为
  \[
    |P_1P_2| = \frac{2|z_1-z_2|}{\sqrt{(1+|z_1|^2)(1+|z_2|^2)}},\quad
    |P_1N| = \frac{2}{\sqrt{1+|z_1|^2}}.
  \]
\proof
  已知$\overrightarrow{OP_i} = (z_i+\bar{z}_i,-\iu(z_i-\bar{z}_i),|z_i|^2-1)/(|z_i|^2+1)$,
  所以
  \[\begin{split}
    |P_1P_2|^2 &= |OP_1|^2+|OP_2|^2 - 2\overrightarrow{OP_1}\cdot\overrightarrow{OP_2}
                = 2(1-\overrightarrow{OP_1}\cdot\overrightarrow{OP_2})\\
              &= \frac{2\left\{
                (1+|z_1|^2)(1+|z_2|^2) - 4\Re(z_1)\Re(z_2) - 4\Im(z_1)\Im(z_2)
                - (|z_1|^2-1)(|z_2|^2-1)\right\}}{(1+|z_1|^2)(1+|z_2|^2)}\\
              &= \frac{4}{(1+|z_1|^2)(1+|z_2|^2)}(|z_1|^2+|z_2|^2
                -2(\Re z_1\Re z_2 - \Im z_1\Im z_2)) \\
              &= \frac{4|z_1-z_2|^2}{(1+|z_1|^2)(1+|z_2|^2)}.
  \end{split}\]
  同时,令$z_2\to\infty$,则有
  \[
    |P_1N|^2 = \lim_{z_2\to\infty} \frac{4|z_1-z_2|^2}{(1+|z_1|^2)(1+|z_2|^2)}
    = \lim_{z_2\to\infty}\frac{4}{1+|z_1|^2}
      \frac{|z_1|^2+|z_2|^2-2\Re(z_1\bar{z}_2)}{1+|z_2|^2}
  \]
  由于$|\Re(z_1\bar{z}_2)| \le |z_1||z_2|$,所以,
  \[
    |P_1N|^2 = \frac{4}{1+|z_1|^2} \quad\Rightarrow\quad
    |P_1N| = \frac{2}{\sqrt{1+|z_1|^2}}.\quad\blacksquare
  \]


\end{document}
