\documentclass[12pt, a4paper]{article}
\usepackage{ctex}

\usepackage[margin=1in]{geometry}
\usepackage{
  color,
  clrscode,
  amssymb,
  ntheorem,
  amsmath,
  listings,
  fontspec,
  xcolor,
  supertabular,
  multirow,
  mathtools,
  mathrsfs
}
\definecolor{bgGray}{RGB}{36, 36, 36}
\usepackage[
  colorlinks,
  linkcolor=bgGray,
  anchorcolor=blue,
  citecolor=green
]{hyperref}
\newfontfamily\courier{Courier}

\theoremstyle{margin}
\theorembodyfont{\normalfont}
\newtheorem{thm}{定理}
\newtheorem{cor}[thm]{推论}
\newtheorem{pos}[thm]{命题}
\newtheorem{lemma}[thm]{引理}
\newtheorem{defi}[thm]{定义}
\newtheorem{std}[thm]{标准}
\newtheorem{imp}[thm]{实现}
\newtheorem{alg}[thm]{算法}
\newtheorem{exa}[thm]{例}
\newtheorem{prob}[thm]{问题}
\DeclareMathOperator{\sft}{E}
\DeclareMathOperator{\idt}{I}
\DeclareMathOperator{\spn}{span}
\DeclareMathOperator*{\agm}{arg\,min}
\newcommand{\pr}{\prime}
\newcommand{\tr}{^\intercal}
\newcommand{\st}{\text{s.t.}}
\newcommand{\hp}{^\prime}
\newcommand{\ms}{\mathscr}
\newcommand{\mn}{\mathnormal}
\newcommand{\tbf}{\textbf}
\newcommand{\mbf}{\mathbf}
\newcommand{\fl}{\mathnormal{fl}}
\newcommand{\f}{\mathnormal{f}}
\newcommand{\g}{\mathnormal{g}}
\newcommand{\R}{\mathbf{R}}
\newcommand{\Q}{\mathbf{Q}}
\newcommand{\JD}{\textbf{D}}
\newcommand{\rd}{\mathrm{d}}
\newcommand{\str}{^*}
\newcommand{\vep}{\varepsilon}
\newcommand{\lhs}{\text{L.H.S}}
\newcommand{\rhs}{\text{R.H.S}}
\newcommand{\con}{\text{Const}}
\newcommand{\oneton}{1,\,2,\,\dots,\,n}
\newcommand{\aoneton}{a_1a_2\dots a_n}
\newcommand{\xoneton}{x_1,\,x_2,\,\dots,\,x_n}
\newcommand\thmref[1]{定理~\ref{#1}}
\newcommand\lemmaref[1]{引理~\ref{#1}}
\newcommand\defref[1]{定义~\ref{#1}}
\newcommand\posref[1]{命题~\ref{#1}}
\newcommand\secref[1]{节~\ref{#1}}
\newcommand\equref[1]{(\ref{#1})}
\newcommand\figref[1]{图 \ref{#1}}
\newcommand\corref[1]{推论~\ref{#1}}
\newcommand\exaref[1]{例~\ref{#1}}
\newcommand\algref[1]{算法~\ref{#1}}
\newcommand{\remark}{\paragraph{评注}}
\newcommand{\example}{\paragraph{例}}
\newcommand{\proof}{\paragraph{证明}}


\title{统计$\,$笔记}
\author{任云玮}
\date{}

\begin{document}
\maketitle
\tableofcontents

\newpage
\section{常见分布}
\subsection{离散分布}
  \subsubsection{Hypergeometric Distribution}
    \paragraph{含义}
      设有$N$个球,其中$M$个为红色,$N-M$个为绿色,它们除颜色以外无区别. 从中选取$K$个,
      考虑恰有$x$个是红球的概率分布. 
    \paragraph{公式}
      \begin{align*}
        &P(X=x|N,M,K) = \frac{\binom{M}{x}\binom{N-M}{K-x}}{\binom{N}{K}}.\\
        &\E X = \frac{KM}{N}. \\
        &\Var X = \frac{KM}{N}\frac{(N-M)(N-K)}{N(N-1)}.\\
        &x = 0,1,\dots,K.
      \end{align*}
  % end
  \subsubsection{Binomial Distribution}
    \paragraph{含义}
    考虑$n$次相同的成功概率为$p$的Bernoulli试验,考虑其中恰有$y$次成功的概率分布. 
    \paragraph{公式}
    \begin{align*}
      & P(Y=y|n,p)=\binom{n}{y}p^y(1-p)^{n-y}. \\
      & \E X = np \\
      & \Var X = np(1-p). \\
      & M_X(t) = [pe^t + (1-p)]^n.
    \end{align*}
  % end
  \subsubsection{Poisson Distribution}
    \paragraph{含义}
    Poisson分布可以用来描述在一段时间内,某事件发生的次数的概率分布,例如类似于等车的行为. 
    设$t\ge 0$而$N_t$是关于$t$的整数值随机变量,设它满足如下性质:
    \begin{enumerate}
      \item 初始无到达:$N_0=0$;
      \item 不相交时间区间内的达到次数相互独立:设$s<t$,则$N_s$和$N_t-N_s$独立;
      \item 到达次数近于区间长度有关:$N_s$和$N_{t+s}-N_t$是同分布的;
      \item 当时间区间充分小时,到达可能性正比于区间长度:$\lim_{t\to 0}
            \dfrac{P(N_t=1)}{t}=\lambda$;
      \item 不会出现同时到达的情况:$\lim_{t\to 0}\dfrac{P(N_t>1)}{t}=0$;
    \end{enumerate}
    经常的,我们会进行归一化,同时只考虑$t=1$的情况,即“仅等一单位时间情况下,等到车的数量”. 
    
    \paragraph{公式}
    \begin{align*}
      & P(N_t=n|\lambda) = e^{-\lambda t}\frac{(\lambda t)^n}{n!}. \\
      & \E N_1 = \lambda. \\
      & \Var N_1 = \lambda. \\
      & M_{N_1}(t) = e^{\lambda(e^t-1)}.
    \end{align*}

    \paragraph{推导}
    我们仅考虑归一化的结果,即一个单位时间的情况. 我们把该单位时间等分成足够多的$n$,则对于每一
    个小时间段,可以认为互相独立的Bernoulli试验,其成功的概率是$\lambda/n$. 则该单位时间的
    到达次数,即为这$n$次Bernoulli试验的成功次数,即为一个二次分布,我们令$n\to\infty$,即
    得到Poisson分布的pmf,
    \[
      f(x) 
      = \lim_{n\to\infty} \binom{n}{x}(\lambda/n)^x (1-\lambda/n)^{1-x}
      = e^{-\lambda}\frac{\lambda^x}{x!}.
    \]

    \paragraph{性质}
    \begin{thm}
      设$X\sim\text{Poisson}(\theta)$,$Y\sim\text{Poisson}(\lambda)$且$X$和$Y$独
      立,则$X+Y\sim\text{Poisson}(\theta+\lambda)$.
    \end{thm}

    \paragraph{利用Poisson分布近似}
    注意到Poisson描述的是连续的时间段中事件发生的情况,在某些离散的情况下,亦可以使用Poisson
    分布来近似. 考虑二项分布的情况,我们可以把Bernoulli试验的成功理解成Poisson分布中的“到
    达”,同时把单个的Bernoulli试验理解成一个单位时间,则自然的$p$就对应着$\lambda$. 若$p$
    充分小,则可以认为这些离散的Bernoulli试验已经和连续的情况差不多乐,所以此时可以有
    \[
      P(Y=y|n,p) \sim P(N_n=y|p) = P(N_1=y|np).
    \]
    我们可以通过比较它们的递推式来证明这一结论. 
  % end
  \subsubsection{Negative Binomial Distribution}
    \paragraph{含义}
    考虑$n$次独立的Bernoulli$(p)$试验,设随机变量$X$表示在第$X$轮发生了等$r$次成功,即描述
    一个成功了$r$次后才会停止的试验. 或者考虑一种等价的形式,设随机变量$Y$表示在第$r$次成功前
    的失败的次数,显然有$Y=X-r$. 
    
    \paragraph{公式}
    \begin{align*}
      &P(X=x|r,p) = \binom{x-1}{r-1}p^r(1-p)^{x-r} \\
      &P(Y=y|r,p) = \binom{r+y-1}{y}p^r(1-p)^y = (-1)^y\binom{-r}{y}p^r(1-p)^y.\\
      &\E Y = r\frac{1-p}{p}. \\
      &\Var Y = r\frac{1-p}p^2 = \mu+\frac{1}{2}\mu^2.
    \end{align*}

    \paragraph{利用负二项分布近似Poisson分布}
    注意到如果我们令$p\to 1$而$r\to\infty$且同时满足$r(1-p)\to\lambda$,则期望与方差都
    会趋于$\lambda$,和Poisson分布的结果一致. 这暗示了在这一条件下,负二项分布或许可以近似
    Poisson分布,而事实确实如此. \par
    我们可以这样考虑这个事情. 当$r$变大时候,一次试验在其中所占的份就变小了,从而变得更加的连续
    了,而在这里把一次失败理解成一次到达,则$p\to 1$的条件同Poisson分布中关于概率与区间长度的
    要求是一致的,而$r(1-p)\to\lambda$的要求可以从归一化之后的角度看待. 
  % end
  \subsubsection{Geometric Distribution}
    \paragraph{含义}
    负二项分布中,取$r=1$的特殊情况,即一种成功就停止的试验. 

    \paragraph{公式}
    \begin{align*}
      &P(X=x|p) = p(1-p)^{x-1}. \\
      &\E X = \frac{1}{p}. \\
      &\Var X = \frac{1-p}{p^2}. 
    \end{align*}

    \paragraph{性质}
    之前的失败对于之后的概率没有影响,即有
    \[
      P(X>s|X>t) = P(X>s-t).
    \]
    这也意味着如果某种概率是随着试验次数/时间的增长而有变化时,几何分布是不适用的. 
  % end
% end
\subsection{连续分布}
  \subsubsection{Exponential Distribution}
    \paragraph{含义}
    已知在单位时间内事件发生的次数为$\lambda$,$W$为第一次事件发生前所经过的时间. 

    \paragraph{推导}
    我们下求$W$的cdf并求导以得其pdf. 设$P_\lambda$为Poisson分布,则
    \begin{align*}
      & F(x)=1-P(W>x) = 1-P_\lambda(N_x=0) = 1-e^{-\lambda x} \\
      \Rightarrow\quad &
        f(x) = F\hp(x) = \lambda e^{-\lambda x}.
    \end{align*}

    \paragraph{公式}
    设$\theta=1/\lambda$表示所需等时间的期望,则
    \begin{align*}
      & f(x) = \frac{1}{\theta}e^{-x/\theta}, \\
      & \E(x) = \theta, \\
      & \Var(x) = \theta^2, \\
      & x\ge 0, \theta>0.
    \end{align*}
      
  % end
  \subsubsection{Gamma Distribution}
    \paragraph{含义}
    设单位时间内事件发生的次数为$\lambda$,$W$为第$\alpha$次事件发生的时间. 记$\theta=1/
    \lambda$. 
    
    \paragraph{公式}
    \begin{align*}
      &f(x) = \frac{1}{\Gamma(\alpha)\theta^\alpha} x^{\alpha-1}e^{-x/\theta},\\
      &E(W) = \alpha\theta \\
      &\Var(W) = \alpha\theta^2 \\
      &x \ge 0, \alpha,\theta>0.
    \end{align*}
    

  % end
  \subsubsection{正态分布}
    \paragraph{公式}
    其参数即为其均值$\mu$和方差$\sigma^2$.
    \[
      f(x|\mu,\sigma^2)=\frac{1}{\sqrt{2\pi}\sigma}e^{-(x-\mu)^2/(2\sigma^2)},
      -\infty<x<\infty.
    \]
    若$X\sim\text{n}(\mu,\sigma^2)$,则称$Z=(X-\mu)/\sigma\sim n(0,1)$呈
    \textbf{标准正态分布}. 
    \paragraph{性质}
    正态分布的随机变量的pdf在$x=\mu$处达到最大值,在$x=\mu\pm\sigma$处变换凹凸性.
    \paragraph{利用正态分布近似二项分布}
    设$X\sim\text{binomial}(n, p)$且$Y\sim \text{n}(np, np(1-p))$,则
    \[
      P(X\le x)\approx P(Y\le x+1/2), \quad 
      P(X\ge x)\approx P(Y\ge x-1/2).
    \]
    其中$\pm 1/2$为连续性修正. 
  % end
  
% end

\newpage
\section{Cheat Sheet}

  \begin{lemma}[独立]
    设$(X,Y)$为二元随机向量的pmf为$f(x,y)$. 则随机变量$X$和$Y$独立当且仅当存在$g(x)$和
    $h(y)$使得对任意$x,y\in\mathbb{R}$成立$f(x,y)=g(x)h(y)$.
  \end{lemma}

  \begin{thm}[变换的pdf]
    设$(U,V)=g(X,Y)$是从$\mathcal{A}=\{(x,y)\,:\,f_{X,Y}(x,y)>0\}$到$\mathcal{B}=
    g(\mathcal{A})$的双射. 设$h=g^{-1}$而$J$是$h$的Jacob行列式且$J$不恒为$0$,则
    \[
      f_{U,V}(u,v) = f_{X,Y}(h(u, v))|J|.
    \]
    TODO: p161
  \end{thm}

  \begin{lemma}[独立]
    设$X$和$Y$为两个独立随机变量且$g$和$h$分别是仅关于$x$和$y$的函数,则随机变量$U=g(X)$和
    $V=h(Y)$也独立. 
  \end{lemma}

  \begin{lemma}[协方差]
    $\Cov(X,Y) = \E((X-\mu_X)(Y-\mu_Y)) = \E XY - \mu_x\mu_Y$.
  \end{lemma}

  \begin{lemma}
    设$X_1,\dots,X_n$是随机采样且$g(x)$是一个满足$\E g(X_1)$和$\Var g(X_1)$存在的函数,
    则有
    \begin{align*}
      \E\left(\sum_{i=1}^n g(X_i) \right) &= n(\E g(X_1)), \\
      \Var\left( \sum_{i=1}^n g(X_i) \right) &= n\left(\Var g(X_1) \right).
    \end{align*}
    这一引理的$\lhs$中的变换的形式(求和),是常见的统计的形式,因此此引理可用于分析统计的均值
    和方差. 
  \end{lemma}

% end

\newpage
\section{注解}
  \paragraph{p212 Definition 5.2.2/3}
    \textbf{随机变量}是指一个从样本空间$S$到$\mathbb{R}$的映射,而一个随机采样则是$n$个
    iid 的随机变量. 而一个\textbf{统计}是指一个从一个随机采样到$\mathbb{R}^n$的映射. 而我
    们所定义的\textbf{统计均值}和\textbf{统计方差}都是一个统计,即是一个从映射(随机变量)到
    $\mathbb{R}^n$的映射. 如果我们取定随机采样,则可以认为它是一个从样本空间到
    $\mathbb{R}^n$的映射,即它也是一个随机变量. 从而我们可以讨论它的均值或者方差,讨论它的分
    布. 这里要注意,统计均值/方差和均值/方差是完全不同的东西,前者(在取定随机采样后)是一个随
    机变量,而后者则是一个常数. 
  % end

  \paragraph{p214 Theorem 5.2.6}
    这里给出了之所以在定义统计方差时,分母用$n-1$而非$n$. 只有这样才可以使得$S^2$的期望恰好为
    $\sigma^2$——各随机变量的各自的方差. 
  % end

  \paragraph{p232 Definition 5.5.1}
    首先再次注意随机变量是一个从样本空间到$\mathbb{R}$的映射,而$P(|X_n-X|\ge\vep)$表示
    这样一件事情:如果对于样本空间里的一个元素,如果$X_n$和$X$之间的差不小于$\vep$,则把这个
    元素“对应”的概率加上,把所有这些元素都加上的结果即为$P(|X_n-X|\ge\vep)$. 而此定义表明
    当$n$充分大的时候,这些概率累加的结果可以充分小. 
  % end

  \paragraph{p234 Definition 5.5.6}
    对于一个具体的式子,这一定义和Definition 5.5.1的一个区别在于,在后者中如果我们展开极限
    以外的部分,我们会得到一个关于$\vep$和$n$的函数,需要这个函数在$n\to\infty$时候趋于零
    (或者$1$). 而在这一定义中,我们将不得不先处理极限,然后考察不满足要求的样本集合对应的
    概率. 这一定义相较于5.5.1要更强,它表示函数“从概率角度”几乎处处点态收敛,而在5.5.1中,
    它实际上是一个数列的收敛性,可以理解为它是某个“概括”了$X_n$的数量的收敛性. 
  % end

  \paragraph{p235 Definition 5.5.10}
    即对应的cdf列在$F_X$的连续点集合上点态收敛. 
  % end

  \paragraph{p236 Theorem 5.5.13}
    "converges in distribution to $\mu$" 的直观含义就是结果一定是$\mu$. 
  % end

  \paragraph{p236 Theorem 5.5.14}
    这一定理表明在一定条件下,$\sqrt{n}(\bar{X}_n-\mu)/\sigma$趋于$n(0, 1)$. 这一定理
    的作用在于,假设我们从一个分布中进行随机采样,则当样本足够多时,样本均值的分布可以用正态
    分布(的一点变形)来估计. 仅一步的,结合Theorem 5.5.17,在一定条件下我们可以用样本方差
    来代替方差. (见Example 5.5.18)
  % end

  \paragraph{p240 Example 5.5.19}
    这里问题是给定一个随机采样$X_1,\dots,X_n$,以及一个统计,在此为$\hat{p}/(1-\hat{p})
    $,其中$\hat{p}=\sum_i X_i/n$. 而我们希望做的就是得到这个统计的诸如方差等性质. 
  % end

  \paragraph{p242 (5.5.9)}
    注意这里的$\theta$是各随机变量的期望组成的向量.
  % end

  \paragraph{p242 Example 5.5.22}
    设$g(x)=x/(1-x)$,注意$E\hat{p} = p$,因此
    \[
      \Var\left(\frac{\hat{p}}{1-\hat{p}}\right) =
      \Var_p g(\hat{p}) \approx
      [g\hp(p)]^2\Var_p\hat{p}.
    \]
  % end

  \paragraph{p243 Example 5.5.25}
    TODO
  % end

  \paragraph{p245 Sec 5.6}
    本节的目的可以理解为,在假设可以生成uniform distributed的随机数的情况下,考虑如何生成满
    足特定分布的随机数. 具体的步骤分为两步,首先给出一个生成随机数的算法,接下来利用大数定理或者
    通过说明cdf相等来说明生成的随机数确实满足某个分布. 
  % end

  \paragraph{p251 Example 5.6.7}
    我们注意到这样一个事情,我们可以不需要直接地生成满足对应分布的随机数,而是生成一些诸如一致
    分布的随机数,然后按照一定的规则选择是否要接受这个生成出的点,这样我们只需要确保我们最后接
    受了的点是满足所需分布的即可. \par
    接下来考虑这个具体的例子。首先如果我们可以随机的生成曲线下的点,则对应的$x$坐标就满足对应的
    分布。由于生成曲线下的点通常是困难的,但确定一个点是否在曲线下方是容易的,所以我们可以考虑生
    成在一个足够大的直线$y=c$下的点,注意到由于$y$的分布是一致的,所以如果我们拒绝掉所有不在曲
    线下的点,由于$y$的分布是一致的,所以最后我们接受的点的$x$坐标满足所需分布。在这个例子中,
    $V$对应的是$x$,$U$对应的是$y$(比例)。
  % end

  \paragraph{p253 Theorem 5.6.8}
    从Example 5.6.7出发,一个十分直观的想法就是,如果$V$的分布和$Y$更像一点,我们就可以更加
    高效的生成可以取的点。所以假设我们已经有了一个可以生产的分布,然后我们通过乘一个常数,使得
    整条曲线都在目标分布的曲线的上方,然后在按照上述同样的做法就可以更方便的得出结果。
  % end

% end

% TODO: p156 normal distribution

\end{document}
