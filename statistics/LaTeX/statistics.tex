\documentclass[12pt, a4paper]{article}
\usepackage{ctex}

\usepackage[margin=1in]{geometry}
\usepackage{
  color,
  clrscode,
  amssymb,
  ntheorem,
  amsmath,
  listings,
  fontspec,
  xcolor,
  supertabular,
  multirow,
  mathtools,
  mathrsfs
}
\definecolor{bgGray}{RGB}{36, 36, 36}
\usepackage[
  colorlinks,
  linkcolor=bgGray,
  anchorcolor=blue,
  citecolor=green
]{hyperref}
\newfontfamily\courier{Courier}

\theoremstyle{margin}
\theorembodyfont{\normalfont}
\newtheorem{thm}{定理}
\newtheorem{cor}[thm]{推论}
\newtheorem{pos}[thm]{命题}
\newtheorem{lemma}[thm]{引理}
\newtheorem{defi}[thm]{定义}
\newtheorem{std}[thm]{标准}
\newtheorem{imp}[thm]{实现}
\newtheorem{alg}[thm]{算法}
\newtheorem{exa}[thm]{例}
\newtheorem{prob}[thm]{问题}
\DeclareMathOperator{\sft}{E}
\DeclareMathOperator{\idt}{I}
\DeclareMathOperator{\spn}{span}
\DeclareMathOperator*{\agm}{arg\,min}
\newcommand{\pr}{\prime}
\newcommand{\tr}{^\intercal}
\newcommand{\st}{\text{s.t.}}
\newcommand{\hp}{^\prime}
\newcommand{\ms}{\mathscr}
\newcommand{\mn}{\mathnormal}
\newcommand{\tbf}{\textbf}
\newcommand{\mbf}{\mathbf}
\newcommand{\fl}{\mathnormal{fl}}
\newcommand{\f}{\mathnormal{f}}
\newcommand{\g}{\mathnormal{g}}
\newcommand{\R}{\mathbf{R}}
\newcommand{\Q}{\mathbf{Q}}
\newcommand{\JD}{\textbf{D}}
\newcommand{\rd}{\mathrm{d}}
\newcommand{\str}{^*}
\newcommand{\vep}{\varepsilon}
\newcommand{\lhs}{\text{L.H.S}}
\newcommand{\rhs}{\text{R.H.S}}
\newcommand{\con}{\text{Const}}
\newcommand{\oneton}{1,\,2,\,\dots,\,n}
\newcommand{\aoneton}{a_1a_2\dots a_n}
\newcommand{\xoneton}{x_1,\,x_2,\,\dots,\,x_n}
\newcommand\thmref[1]{定理~\ref{#1}}
\newcommand\lemmaref[1]{引理~\ref{#1}}
\newcommand\defref[1]{定义~\ref{#1}}
\newcommand\posref[1]{命题~\ref{#1}}
\newcommand\secref[1]{节~\ref{#1}}
\newcommand\equref[1]{(\ref{#1})}
\newcommand\figref[1]{图 \ref{#1}}
\newcommand\corref[1]{推论~\ref{#1}}
\newcommand\exaref[1]{例~\ref{#1}}
\newcommand\algref[1]{算法~\ref{#1}}
\newcommand{\remark}{\paragraph{评注}}
\newcommand{\example}{\paragraph{例}}
\newcommand{\proof}{\paragraph{证明}}


\title{统计$\,$笔记}
\author{任云玮}
\date{}

\begin{document}
\maketitle
\tableofcontents

\newpage
\section{常见分布}
\subsection{离散分布}
  \subsubsection{Hypergeometric Distribution}
    \paragraph{含义}
      设有$N$个球,其中$M$个为红色,$N-M$个为绿色,它们除颜色以外无区别。从中选取$K$个,
      考虑恰有$x$个是红球的概率分布。
    \paragraph{公式}
      \begin{align*}
        &P(X=x|N,M,K) = \frac{\binom{M}{x}\binom{N-M}{K-x}}{\binom{N}{K}}.\\
        &\E X = \frac{KM}{N}. \\
        &\Var X = \frac{KM}{N}\frac{(N-M)(N-K)}{N(N-1)}.\\
        &x = 0,1,\dots,K.
      \end{align*}
  % end
  \subsubsection{Binomial Distribution}
    \paragraph{含义}
    考虑$n$次相同的成功概率为$p$的Bernoulli试验,考虑其中恰有$y$次成功的概率分布。
    \paragraph{公式}
    \begin{align*}
      & P(Y=y|n,p)=\binom{n}{y}p^y(1-p)^{n-y}. \\
      & \E X = np \\
      & \Var X = np(1-p). \\
      & M_X(t) = [pe^t + (1-p)]^n.
    \end{align*}
  % end
  \subsubsection{Poisson Distribution}
    \paragraph{含义}
    Poisson分布可以用来描述在一段时间内,某事件发生的次数的概率分布,例如类似于等车的行为。
    设$t\ge 0$而$N_t$是关于$t$的整数值随机变量,设它满足如下性质:
    \begin{enumerate}
      \item 初始无到达:$N_0=0$;
      \item 不相交时间区间内的达到次数相互独立:设$s<t$,则$N_s$和$N_t-N_s$独立;
      \item 到达次数近于区间长度有关:$N_s$和$N_{t+s}-N_t$是同分布的;
      \item 当时间区间充分小时,到达可能性正比于区间长度:$\lim_{t\to 0}
            \dfrac{P(N_t=1)}{t}=\lambda$;
      \item 不会出现同时到达的情况:$\lim_{t\to 0}\dfrac{P(N_t>1)}{t}=0$;
    \end{enumerate}
    经常的,我们会进行归一化,同时只考虑$t=1$的情况,即“仅等一单位时间情况下,等到车的数量”。
    
    \paragraph{公式}
    \begin{align*}
      & P(N_t=n|\lambda) = e^{-\lambda t}\frac{(\lambda t)^n}{n!}. \\
      & \E N_1 = \lambda. \\
      & \Var N_1 = \lambda. \\
      & M_{N_1}(t) = e^{\lambda(e^t-1)}.
    \end{align*}

    \paragraph{利用Poisson分布近似}
    注意到Poisson描述的是连续的时间段中事件发生的情况,在某些离散的情况下,亦可以使用Poisson
    分布来近似。考虑二项分布的情况,我们可以把Bernoulli试验的成功理解成Poisson分布中的“到
    达”,同时把单个的Bernoulli试验理解成一个单位时间,则自然的$p$就对应着$\lambda$。若$p$
    充分小,则可以认为这些离散的Bernoulli试验已经和连续的情况差不多乐,所以此时可以有
    \[
      P(Y=y|n,p) \sim P(N_n=y|p) = P(N_1=y|np).
    \]
    我们可以通过比较它们的递推式来证明这一结论。
  % end
  \subsubsection{Negative Binomial Distribution}
    \paragraph{含义}
    考虑$n$次独立的Bernoulli$(p)$试验,设随机变量$X$表示在第$X$轮发生了等$r$次成功,即描述
    一个成功了$r$次后才会停止的试验。或者考虑一种等价的形式,设随机变量$Y$表示在第$r$次成功前
    的失败的次数,显然有$Y=X-r$。
    
    \paragraph{公式}
    \begin{align*}
      &P(X=x|r,p) = \binom{x-1}{r-1}p^r(1-p)^{x-r} \\
      &P(Y=y|r,p) = \binom{r+y-1}{y}p^r(1-p)^y = (-1)^y\binom{-r}{y}p^r(1-p)^y.\\
      &\E Y = r\frac{1-p}{p}. \\
      &\Var Y = r\frac{1-p}p^2 = \mu+\frac{1}{2}\mu^2.
    \end{align*}

    \paragraph{利用负二项分布近似Poisson分布}
    注意到如果我们令$p\to 1$而$r\to\infty$且同时满足$r(1-p)\to\lambda$,则期望与方差都
    会趋于$\lambda$,和Poisson分布的结果一致。这暗示了在这一条件下,负二项分布或许可以近似
    Poisson分布,而事实确实如此。\par
    我们可以这样考虑这个事情。当$r$变大时候,一次试验在其中所占的份就变小了,从而变得更加的连续
    了,而在这里把一次失败理解成一次到达,则$p\to 1$的条件同Poisson分布中关于概率与区间长度的
    要求是一致的,而$r(1-p)\to\lambda$的要求可以从归一化之后的角度看待。
  % end
  \subsubsection{Geometric Distribution}
    \paragraph{含义}
    负二项分布中,取$r=1$的特殊情况,即一种成功就停止的试验。

    \paragraph{公式}
    \begin{align*}
      &P(X=x|p) = p(1-p)^{x-1}. \\
      &\E X = \frac{1}{p}. \\
      &\Var X = \frac{1-p}{p^2}. 
    \end{align*}

    \paragraph{性质}
    之前的失败对于之后的概率没有影响,即有
    \[
      P(X>s|X>t) = P(X>s-t).
    \]
    这也意味着如果某种概率是随着试验次数/时间的增长而有变化时,几何分布是不适用的。
  % end
% end

\end{document}
