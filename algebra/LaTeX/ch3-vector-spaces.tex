\section{向量空间}

\subsection{$\R^n$的子空间}

\subsection{域}

  \begin{defi}[域]
    设集合$F$定义了如下被称作加法和乘法的复合律
    \[
      F\times F\to^{+} F \quad\text{和}\quad F\times F\to^{\times}F.
    \]
    称$F$为域若它满足如下性质
    \begin{enumerate}
      \item 加法使$F$构成了一个Abel群$F^+$,记其单位元为$0$.
      \item 乘法满足交换律,且乘法使$F$的非零元素全体构成了一个Abel群$F^\times$,记其单位元为$1$.
      \item 满足分配律:$a(b+c)=ab+ac$.
    \end{enumerate}
  \end{defi}
  \remark
    [2.]需要单独说明满足交换律,只要是为了处理与$0$相乘的情况. 另外可以证明:
    \begin{enumerate}
      \item $0\ne 1$.
      \item $0a=a0=0$.
      \item 乘法满足结合律,$1$是$F$全体\footnote{定义中仅说是非
      零子集的单位元.}的乘法单位元.
    \end{enumerate}

  \begin{defi}[子域]
    设$F$为域,称$L\subset F$为其子域,若它对加减乘除封闭且包含乘法单位元.
  \end{defi}
  \remark
    可以证明$0=1-1$成立,从而$0\in L$.

  \begin{thm}
    设$p$为素数,则任意非零的模$p$同余类有逆元,因此$\F_p=\Z/p\Z$为
    一个阶为$p$的域.
  \end{thm}
  \remark
    这一定理声称$\F_p^\times$是一个群. 如果它结论成立的话,那么我们
    可以知道$|\F_p^\times|=p-1$为一有限数,所以它的任意元素$\bar{a}$
    的阶也是一个有限数,设为$r$. 即有$\bar{a}^r=1$,从而它的逆元为
    $\bar{a}^{r-1}$.
  \proof
    首先可以证明$\F_p^\times$有消去律,即若$\bar{a}\bar{b}=0$,则
    $\bar{a}=0$或$\bar{b}=0$;若$\bar{a}\ne -$且$\bar{a}\bar{b}
    =\bar{a}\bar{c}$,则$\bar{b}=\bar{c}$. 接下来只需要对于
    $\bar{1},\bar{a},\bar{a}^2,\dots$以及$\F_p^\times$施鸽巢原理
    以及消去律即可.$\quad\blacksquare$

  \begin{defi}[特征\protect\footnotemark]
    \footnotetext{Characteristic}
    定一个域$F$的特征为它的乘法单位元$1$作为加法群中元素时的阶数,
    若其阶数为无穷,则定义特征为零.
  \end{defi}

  \begin{lemma}
    域的特征或为零,或为一个素数.
  \end{lemma}
  \proof
    设特征为$m\ne 0$. 记乘法单位元为$\bar{1}$,$\bar{n}$为$n$个$\bar{1}$
    相加的结果. $\bar{1}$在加法意义下生成的子群
    $\langle\bar{1}+\rangle=\{\bar{0},\bar{1},\dots,\overline{m-1}\}$.
    若$m$不为素数,则$m=rs$,其中$r,s\ne 1$. 即成立
    \[
      \bar{0}=\bar{m}=\overline{rs}=\sum_1^{rs}\bar{1}
      =\left(\sum_1^r\bar{1}\right)=\left(\sum_1^s\bar{1}\right)
      =\bar{r}\bar{s}.
    \]
    而$\bar{r},\bar{s}\in F^\times$但$\bar{0} \notin F^\times$,同时
    $F^\times$是一个群,矛盾.$\quad\blacksquare$

  \begin{thm}[乘法群的结构]
    \label{thm: 乘法群的结构}
    设$p$为素数. 则$\F_p^\times$是$p-1$阶循环群.
  \end{thm}
  \remark
    证明见之后的内容.

  \begin{cor}[Fermat]
    设$p$为素数,则对任意$a\in\Z$,成立$a^p\equiv a\mod p$.
  \end{cor}

  \begin{cor}[Wilson]
    设$p$为素数,则$(p-1)!\equiv -1 \mod p$.
  \end{cor}
  \proof
    考虑$\F_p^\times=\{1,2,\dots,p-1\}$,根据\thmref{thm:
    乘法群的结构}它是$p-1$阶循环群,所以
    \[
      (p-1)! = \prod_{k=0}^{p-2}(p-1)^k = (p-1)^{(p-2)(p-1)/2}
      \equiv p-1 \equiv -1 \mod p.
    \]

  \begin{defi}[原根]
    能生成$\F_p^\times$的数被称作模$p$的原根.
  \end{defi}

\subsection{向量空间}

  \begin{defi}[向量空间]
    在域$F$上的向量空间$V$是指定义了加法和数乘的一个集合,且这两个运算满足
    \begin{enumerate}
      \item 加法是的$V$构成一个Abel群$V^+$.
      \item $1v=v$.
      \item $(ab)v=a(bv)$,其中$a,b\in F$,$v\in V$.
      \item 分配律.
    \end{enumerate}
  \end{defi}

\subsection{基与维度}
  \paragraph{说明}
    本节仅记录了部分命题.

  \begin{thm}
    设$A\in F^{m\times n}$,$B\in F^n$. 则$AX=B$有解当且仅当$B$
    位于$A$的列空间中.
  \end{thm}

  \begin{thm}
    设$S$和$L$是向量空间$V$的有限子集. 设$S$张成$V$且$L$线性无关,
    则$|S|\ge |L|$.
  \end{thm}

\subsection{利用基计算}

  \begin{pos}
    考虑在域$F$上的向量空间$V$,在给定它的一组基$\mbf{B}$后,
    映射$\psi: F^n\to V,\, X\mapsto\mbf{B}X$为一个向量空间的同构.
  \end{pos}

  \begin{pos}
    设$(v_1,\dots,v_n)$为向量空间$V$的一个子集,$\psi: F^n\to V$
    为$\psi(X)=SX$. 则
    \begin{enumerate}
      \item $\psi$为单射当且仅当$S$线性无关.
      \item $\psi$为满射当且仅当$S$张成$V$.
      \item $\psi$为双射当且仅当$S$是$V$的一组基.
    \end{enumerate}
  \end{pos}

  \begin{cor}
    任意域$F$上的$n$维向量空间$V$同构于$F^n$.
  \end{cor}

  \begin{defi}[换基矩阵]
    对于向量空间$V$的基$\mbf{B}=(v_1,\dots,v_n)$和$\mbf{B}\hp=(v_1\hp,
    \dots,v_n\hp)$. 称$P$为换基矩阵,若成立$\mbf{B}\hp=\mbf{B}P$.
  \end{defi}
  \remark
    设$v\in V$在$\mbf{B}$下的坐标为$X$,则在$\mbf{B}\hp$下的坐标为
    $X\hp = P^{-1}X$.

  \begin{pos}
    $\,$
    \begin{enumerate}
      \item $\mbf{B}$和$\mbf{B}\hp$是向量空间$V$的两组基. 则换基
        矩阵$P$可逆且由$\mbf{B}$和$\mbf{B}\hp$唯一确定.
      \item 设$\mbf{B}$是$V$的一组基,则$V$的所有基组成的全体为
        $\{\mbf{B}P\,|\, P\in GL_n(F)\}$.
    \end{enumerate}
  \end{pos}

\subsection{直和}

  \begin{defi}[直和]
    定义$V$的子空间$W_1,\dots,W_n$的直和为
    \[
      W_1 + \cdots W_n = \{v\in V\,|\, v=w_1+\cdots+w_n,\,w_i\in W_i\}.
    \]
  \end{defi}

  \begin{defi}[独立]
    称子空间$W_1,\dots,W_n$独立,若对于$w_i\in W_i$,$w_1+\cdots+w_n=0$
    意味着$w_i=0$.
  \end{defi}
  \remark
    子空间的直和是向量的张成的类比,或者说推广. 考虑向量的独立,它等价于各个向量
    张成的子空间的独立.

  \begin{pos}
    设$W_1,\dots,W_n$是有限维向量空间$V$的子空间,$\mbf{B}_i$是$W_i$的基.
    \begin{enumerate}
      \item $W_i$独立且$\sum W_i=V$等价于$(\mbf{B}_1,\dots,\mbf{B}_n)$是$V$的基.
      \item $\dim(\sum W_i)\le\sum\dim W_i$,当且仅当独立时等号成立.
      \item 若$W_i\hp$是$W_i$的子空间,则$W_i\hp$独立.
    \end{enumerate}
  \end{pos}
  \remark
    若[1.]的条件成立,则称$V=W_1\oplus\cdots\oplus W_n$为直和.

  \begin{thm}
    设$W_1$和$W_2$是有限维向量空间$V$的子空间.
    \begin{enumerate}
      \item $\dim W_1 + \dim W_2 = \dim(W_1\cap W_2)+\dim(W_1+W_2)$.
      \item $W_1$和$W_2$独立当且仅当$W_1\cap W_2=\{0\}$.
      \item 若$V=W_1+W_2$,则存在$W_2$的子空间$W_2\hp$成立$W_1\oplus W_2\hp=V$.
    \end{enumerate}
  \end{thm}
