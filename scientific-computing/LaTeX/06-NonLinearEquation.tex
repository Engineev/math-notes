\section{非线性方程求根}
\subsection{二分法}
  \begin{thm}[闭区间套定理]
    设$\{[a_n, b_n]\}$满足
    \begin{enumerate}
      \item $[a_n, b_n] \subset [a_{n-1}, b_{n-1}]$,
      \item 当$n\to\infty$时,$|b_n - a_n|\to 0$.
    \end{enumerate}
    则存在$\xi$成立
    \[
      \bigcap_{k=0}^{\infty}[a_n, b_n] = \{\xi\}.
    \]
  \end{thm}

  \begin{thm}[连续函数零点定理]
    \label{thm: 连续函数零点定理}
    设$\f\in\ms{C}[a, b]$且$\f(a)\f(b)<0$,则存在$c\in[a, b]$,
    成立$\f(c) = 0$.
  \end{thm}
  \remark
    这一定理是多种方程求根方法的基础,下给出构造性的证明,这一证明
    本身实际上描述了二分法求根的过程.
  \proof
    构造如下闭区间套,令$[a_0, b_0]=[a, b]$,$c_n=(a_n+b_n)/2$,
    如果$\f(c_n)=0$,则结论成立,否则$c_n$至少与$a_n$和$b_n$中的
    一个异号,取该半个区间为$[a_{n+1}, b_{n+1}]$. 由上述构造可知
    \[
      [a_{n+1}, b_{n+1}]\subset[a_n,b_n],\quad
      |b_n - a_n| = \frac{b-a}{2^n}\to 0.
    \]
    从而存在$\xi\in\bigcap_{k=0}^\infty[a_n, b_n]$.
    若$\f(\xi)\ne 0$,不妨设$\f(\xi) = r>0$,则存在$\delta>0$,
    在$[\xi-\delta,\xi+\delta]$上$\f(x)>0$恒成立. 取足够大的$n$,
    即可使$[a_n,b_n]\subset[\xi-\delta,\xi+\delta]$,其中
    $\f(a_n)\f(b_n)<0$,与$\f$在$[\xi-\delta,\xi+\delta]$保号
    矛盾,从而$\f(\xi)=0$. $\blacksquare$

  \begin{alg}[二分法求解零点]
    对区间$n$等分,对每个满足端点函数值异号的区间$[a_k,a_{k+1}]$,
    按照\thmref{thm: 连续函数零点定理}证明中的方法二分,直到满足
    $b_n-a_n<\vep$,即与零点误差小于$\vep$为止.
  \end{alg}
  \remark
    二分法实现简单,但是在高维情况下,由于不再有“区间端点”的概念,
    所以难以推广.

\subsection{不动点法}
  \begin{defi}[压缩映射]
    设$\f$将$E\subset\R$映射到$E$上. 称$\f$为压缩映射,
    若存在常数$l<1$,对任意$x,y\in E$成立
    \[
      |\f(x) - \f(y)| \le l|x-y|.
    \]
  \end{defi}

  \begin{thm}[压缩映射定理]
    \label{thm: 压缩映射定理}
    设$\varphi$是$[a,b]$上的连续压缩映射,则存在唯一的$x\in[a, b]$,
    成立$\varphi(x) = x$.
  \end{thm}
  \proof
    对于唯一性. 设成立$x_1=\varphi(x_1)$,$x_2=\varphi(x_2)$,则
    \[
      |x_1-x_2| = |\varphi(x_1)-\varphi(x_2)|\le l|x_1-x_2|
      \quad\Rightarrow\quad x_1=x_2.
    \]
    对于存在性. 令$F(x)=x-\varphi(x)\in\ms{C}[a, b]$. 若
    $a=\varphi(a)$或$\varphi(b)$,则得证. 否则由于$\f$映射到$[a,b]$
    自身,所以成立$a<\varphi(a), \varphi(b)<b$. 则成立
    $F(a)F(b)<0$,由\thmref{thm: 连续函数零点定理}可知,存在$\xi\in[a,b]$,
    成立$F(\xi)=0$,即$\xi=\varphi(\xi)$. $\blacksquare$
  \remark
    此定理对于任意的完备度量空间都是成立的,且条件中的连续性要求可以略去,
    证明方法是构造迭代数列$x_{n+1}=\varphi(x_n)$.

  \begin{thm}[余项估计]
    设$\varphi$是$[a, b]$上压缩常数为$l$的连续压缩映射. 则数列
    \[
      x_0 = a,\quad x_{n+1} = \varphi(x_n)
    \]
    收敛至$x_*$. 且有估计式
    \[
      |x_n - x_*| \le \frac{l^n}{1-l}|x_1-x_0|.
    \]
  \end{thm}
  \proof
    下证明$\{x_n\}$为Cauchy序列.
    \[
      |x_{k+1}-x_k| = |\varphi(x_k) - \varphi(x_{k-1})|
      \le l|x_k-x_{k-1}| \le \cdots \le l^k|x_1-x_0|.
    \]
    所以对于任意的$n$和$p>0$,成立
    \[
      |x_{n+p}-x_n|\le \sum_{k=0}^{p-1} |x_{n+k+1}-x_{n+k}|
      \le |x_1-x_0|\sum_{k=0}^{p-1}l^{n+k}.
    \]
    由于$l>0$,根据几何级数的性质,成立
    \[
      |x_{n+p}-x_n| \le \frac{l^n}{1-l}|x_1-x_0|.
    \]
    当$n\to\infty$时,$\rhs\to 0$,从而$\{x_n\}$是Cauchy序列,
    所以收敛. 令$p\to\infty$,即得估计式
    \[
      |x_n-x_*| \le \frac{l^n}{1-l}|x_1-x_0|.
      \quad\blacksquare
    \]

  \begin{defi}[局部收敛]
    设$\varphi(x)$有不动点$x_*$,则对于迭代法
    \begin{equation}
      \label{equ: 迭代法}
      x_{n+1}=\varphi(x_n).
    \end{equation}
    如果存在$x_*$的某个领域$O_\delta(x)$,使得任意$x_0\in O_\delta(x)$,
    \equref{equ: 迭代法}收敛至$x_*$,则称\equref{equ: 迭代法}局部收敛.
  \end{defi}

  \begin{thm}[局部收敛的条件]
    设$x_*$是$\varphi\in\ms{C}^1$的不动点,且$|\varphi\hp(x_*)|<1$,
    则\equref{equ: 迭代法}在$x_*$处局部收敛.
  \end{thm}
  \proof
    由于$\varphi\hp$是连续的,所以存在$O_\delta(x_*)$,对任意的$x\in
    O_\delta(x_*)$,成立
    \[
      |\varphi\hp(x)| < l < 1.
    \]
    从而对于任意的$x,y\in O_\delta(x_*)$,根据中值定理,成立
    \[
      |\varphi(x) - \varphi(y)| = |\varphi\hp(\xi)||x-y| < l|x-y|.
    \]
    即$\varphi(x)$在$O_\delta(x_*)$上是压缩映射,从而对任意$x_0\in
    O_\delta(x_*)$,\equref{equ: 迭代法}收敛至$x_*$. $\blacksquare$

  \begin{pos}[不局部收敛的条件]
    设$x_*$是$\varphi(x)\in\ms{C}^1[a, b]$的不动点,若
    $\varphi(x)$在$[a, b]$上单调且$|\varphi\hp(x_*)|\ge h>1$,则
    $\varphi(x)$不局部收敛.
  \end{pos}
  \proof
    由于$\varphi\hp(x_*)>1$且连续,所以存在$\delta>0$,使得在$O_\delta(x_*)$
    中成立
    \[
      |\varphi(x)-x_*| = |\varphi\hp(\xi)(x-x_*)| \ge h|x-x_*|.
    \]
    所以对任意的$0<\vep<\delta$,对任意的$x_0\in O_\delta(x_*)
    \backslash\{x_*\}$,迭代
    足够多次后成立$|x_n-x_*|>\vep$. 由于$\varphi(x)$单调,所以仅有
    $x=x_*$满足$\varphi(x)=x_*$. 从而若存在$x_0\in[a, b]$,使迭代数列收敛
    至$x_*$,则对任意$\vep>0$,存在$n$,使得$x_n\in O_{\vep}(x_*)\backslash\{x_*\}$.
    对于这样的$n$,继续迭代足够多次后,成立$|x_m-x_*|>\vep$,与收敛矛盾,从而不存在
    $x_0\in[a, b]$,使得迭代数列收敛. $\blacksquare$

  \begin{defi}[$p$阶收敛]
    \label{defi: p阶收敛}
    设$x_*$是$\varphi(x)$的不动点,记迭代误差$e_k=x_k-x_*$,若
    当$k\to\infty$时,成立
    \[
      \frac{e_{k+1}}{e_k^p} \to C \ne 0.
    \]
    则称\equref{equ: 迭代法} $p$阶收敛.
  \end{defi}
  \remark
    此定义描述了迭代式收敛的速度.

  \begin{thm}[$p$阶收敛条件]
    \label{thm: p阶收敛条件}
    设$x_*$是迭代过程$x_{n+1}=\varphi(x_n)$的不动点,若对正整数$p$,
    $\varphi^{(p)}$在$x_*$附加连续,且成立
    \[\begin{split}
      \varphi\hp(x_*) = \cdots = \varphi^{(p-1)}(x_*) = 0,\quad
      \varphi^{(p)}(x_*) \ne 0.
    \end{split}\]
    则\equref{equ: 迭代法}在$x_*$附近$p$阶收敛.
  \end{thm}
  \proof
    在$x_*$处将$\varphi$ Taylor展开即可. 存在
    $\xi$在$x_k$和$x_*$之间,成立
    \begin{equation}
      \label{equ: 导数为零情况余项}
      \frac{x_{k+1}-x_*}{(x_k-x_*)^p} = \frac{\varphi^{(p)}(\xi)}{p!}.
      \quad\blacksquare
    \end{equation}
  \remark
    经实践证明,由于当$k\to\infty$时,$\xi\to x_*$,所以对于前$p-1$阶导数
    为零的迭代式,使用\equref{equ: 导数为零情况余项}来计算\defref{defi: p阶收敛}中的常数是十分方便的.


  \begin{alg}[不动点法]
    对于方程$\f(x) = 0$,将其变形成等价的$x=\varphi(x)$的形式,
    且$\varphi$满足在零点处局部收敛,则可以利用迭代数列
    $x_{n+1} = \varphi(x_n)$来求解方程的根.
  \end{alg}

  \begin{alg}[Aitken $\Delta^2$加速法]
    设$\{x_n\}$为一收敛的迭代数列,令
    \[
      \bar{x}_{k+1} = x_k - \frac{(x_{k+1}-x_k)^2}{x_k-2x_kx_{k+1}+x_{k+2}}
      = x_k - \frac{(\Delta x_k)^2}{\Delta^2x_k}.
    \]
    $\{\bar{x}_k\}$收敛得比$\{x_k\}$更快,即满足
    \[
      \lim_{k\to\infty}\frac{\bar{x}_{k+1}-x_*}{x_k-x_*} = 0.
    \]
  \end{alg}
  \remark
    这一加速法的思路在于
    \[
      x_{n+1}-x_* = \varphi(x_n) - \varphi(x_*) = \varphi\hp(\xi)(x_n-x_*).
    \]
    当区间足够小以至$\varphi\hp$变化不大时,可以近似地将$\varphi\hp(\xi)$看作常量.
    另外,这一做法实际上只是数据的后处理,实际上如果我需要的是$\bar{x}_{n+1}$,那无需计算
    $\bar{x}_1$,$\dots$,$\bar{x}_n$,而只需利用$x_{n}$,$x_{n+1}$,$x_{n+2}$
    来计算$\bar{x}_{n+1}$即可.

  \begin{alg}[Steffensen迭代法]
    \[\begin{cases}
      y_k = \varphi(x_k),\quad z_k = \varphi(y_k), \\
      x_{k+1} = x_k - \dfrac{(y_k-x_k)^2}{z_k-2y_k+x_k}.
    \end{cases}\]
  \end{alg}
  \remark
    这一算法实际上只是将Aitken加速法中的$\bar{x}_n$也加入了
    迭代序列中.

  \begin{thm}[Steffensen迭代法]
    若$x_*$是迭代函数
    \[
      \psi = x - \frac{(\varphi - x)^2}{\varphi\circ\varphi - 2\varphi + x}
    \]
    的不动点,则$x_*$也是$\varphi$的不动点. 反之,若$x_*$是$\varphi$的不动点,
    $\varphi^{\pr\pr}$存在,且$\varphi\hp(x_*)\ne 1$,则$x_*$也是$\psi$的
    不动点,且Steffensen迭代法二阶收敛.
  \end{thm}
  \remark
    这一定理表明,就算$\varphi$所对应的迭代序列不收敛,但是对应
    的Steffensen迭代法的迭代序列还是有可能收敛的.

\subsection{Newton法}
  \begin{thm}[Newton法]
    设$x_*$是$\f(x)$的零点且$\f\hp(x_*)\ne 0$,则
    \[
      \varphi(x) = x - \frac{\f(x)}{\f\hp(x)}
    \]
    当$\f\hp(x_*)\ne 0$时,在$x_*$处局部平方收敛,否则线性收敛.
  \end{thm}
  \remark
    Newton法的动机在于在零点附近,用一个线性函数来近似原函数,即
    \[
      \f(x) \approx \f(x_k) + \f\hp(x_k)(x-x_k) \quad\Rightarrow\quad
      x_{k+1} = x = x_k - \frac{\f(x_k)}{\f\hp(x_k)}.
    \]
    则该线性方程的解是原方程的一个近似解. 证明只需利用
    \thmref{thm: p阶收敛条件}即可.

  \begin{alg}[Newton法]
    Newton法求根的步骤如下:
    \begin{enumerate}
      \item 选定初值$x_0$,计算$\f(x_0)$和$\f\hp(x_0)$;
      \item 按照公式$x_{n+1} = x_n - \f_n/\f\hp_n$,并
            计算$\f(x_{n+1})$和$\f\hp(x_{n+1})$;
      \item 如果$x_n$满足$|\delta|<\vep_1$或$|\f|<\vep_2$,则
        终止迭代,以$x_n$作为所求根,否则进入步骤4. 此处
        $\vep_1$,$\vep_2$是允差,而
        \[
          \delta =
          \begin{cases}
            |x_{n}-x_{n-1}|, & |x_1|<C \\
            \dfrac{|x_n-x_{n-1}|}{x_n}, |x_1|\ge C
          \end{cases}
        \]
        其中$C$是取绝对误差或相对误差的控制常数,一般取$C=1$.
      \item 如果迭代次数达到预先规定的$N$,或$\f\hp_n=0$,则方法
        失败,否则继续迭代.
    \end{enumerate}
  \end{alg}

  \begin{thm}[重根情况的Newton法]
    设$x_*$是$\f(x)$的$m$重根,则根据
    \[
      \varphi(x) = x - m \frac{\f(x)}{\f\hp(x)}
    \]
    所得的迭代序列,在$x_*$附近平方收敛.
  \end{thm}

  \begin{thm}[重根情况的Newton法]
    设$x_*$是$\f(x)$的根,则根据
    \[
      \varphi(x) = x - \frac{\f\f\hp}{(\f\hp)^2 - \f\f^{\pr\pr}}
    \]
    所得的迭代序列在$x_*$附近平方收敛.
  \end{thm}
  \remark
    这一函数实际上是令
    \[
      \mu(x) = \frac{(x-x_*)\g(x)}{m\g(x) + (x-x_*)\g\hp(x)}.
    \]
    在考虑$\varphi(x) = x - \mu/\mu\hp$所得到的. 其缺点在于
    涉及了$\f$的二阶导数.

  \begin{thm}[Newton法的收敛区间]
    设$\f\in\ms{C}^2[a, b]$,且满足$\f(a)\f(b)<0$,在$[a, b]$上
    $\f\hp(x)\ne 0$,且$\f^{\pr\pr}$不变号,以及初值$x_0$使得
    $\f(x_0)\f^{\pr\pr}(x_0)>0$成立. 则Newton法所得序列单调收敛于
    唯一根$x_*$.
  \end{thm}

  \begin{pos}[方程组情况的推广]
    对于$\mbf{F}:\,\R^n\to\R^n$,设$\mbf{x}_0$是它的近似根,
    则可以构造方程组的Newton方法
    \[
      \mbf{x}_{k+1} = \mbf{x}_k - [\mbf{F}\hp(\mbf{x}_k)]^{-1}\mbf{F}(\mbf{x}_k).
    \]
    其中$\mbf{F}\hp$为$\mbf{F}$的Jacobi矩阵.
  \end{pos}
  \remark
    这一算法的最大工作量在于求解Jacobi矩阵的逆,即求解线性方程组.
