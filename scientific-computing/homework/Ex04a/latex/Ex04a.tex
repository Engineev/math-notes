\documentclass[12pt, a4paper]{article}
\usepackage{ctex}

\usepackage[margin=1in]{geometry}
\usepackage{
  color,
  clrscode,
  amssymb,
  ntheorem,
  amsmath,
  listings,
  fontspec,
  xcolor,
  supertabular,
  multirow,
  mathtools,
  mathrsfs
}
\definecolor{bgGray}{RGB}{36, 36, 36}
\usepackage[
  colorlinks,
  linkcolor=bgGray,
  anchorcolor=blue,
  citecolor=green
]{hyperref}
\newfontfamily\courier{Courier}

\theoremstyle{margin}
\theorembodyfont{\normalfont}
\newtheorem{thm}{定理}
\newtheorem{cor}[thm]{推论}
\newtheorem{pos}[thm]{命题}
\newtheorem{lemma}[thm]{引理}
\newtheorem{defi}[thm]{定义}
\newtheorem{std}[thm]{标准}
\newtheorem{imp}[thm]{实现}
\newtheorem{alg}[thm]{算法}
\newtheorem{exa}[thm]{例}
\newtheorem{prob}[thm]{问题}
\DeclareMathOperator{\sft}{E}
\DeclareMathOperator{\idt}{I}
\DeclareMathOperator{\spn}{span}
\DeclareMathOperator*{\agm}{arg\,min}
\newcommand{\pr}{\prime}
\newcommand{\tr}{^\intercal}
\newcommand{\st}{\text{s.t.}}
\newcommand{\hp}{^\prime}
\newcommand{\ms}{\mathscr}
\newcommand{\mn}{\mathnormal}
\newcommand{\tbf}{\textbf}
\newcommand{\mbf}{\mathbf}
\newcommand{\fl}{\mathnormal{fl}}
\newcommand{\f}{\mathnormal{f}}
\newcommand{\g}{\mathnormal{g}}
\newcommand{\R}{\mathbf{R}}
\newcommand{\Q}{\mathbf{Q}}
\newcommand{\JD}{\textbf{D}}
\newcommand{\rd}{\mathrm{d}}
\newcommand{\str}{^*}
\newcommand{\vep}{\varepsilon}
\newcommand{\lhs}{\text{L.H.S}}
\newcommand{\rhs}{\text{R.H.S}}
\newcommand{\con}{\text{Const}}
\newcommand{\oneton}{1,\,2,\,\dots,\,n}
\newcommand{\aoneton}{a_1a_2\dots a_n}
\newcommand{\xoneton}{x_1,\,x_2,\,\dots,\,x_n}
\newcommand\thmref[1]{定理~\ref{#1}}
\newcommand\lemmaref[1]{引理~\ref{#1}}
\newcommand\defref[1]{定义~\ref{#1}}
\newcommand\posref[1]{命题~\ref{#1}}
\newcommand\secref[1]{节~\ref{#1}}
\newcommand\equref[1]{(\ref{#1})}
\newcommand\figref[1]{图 \ref{#1}}
\newcommand\corref[1]{推论~\ref{#1}}
\newcommand\exaref[1]{例~\ref{#1}}
\newcommand\algref[1]{算法~\ref{#1}}
\newcommand{\remark}{\paragraph{评注}}
\newcommand{\example}{\paragraph{例}}
\newcommand{\proof}{\paragraph{证明}}


\title{科学计算作业$\,$练习$4a$}
\author{\small 任云玮\\\small2016级ACM班\\\small516030910586}
\date{}

\begin{document}
\lstset{
  numbers=left,
  basicstyle=\scriptsize,
  numberstyle=\tiny\color{red!89!green!36!blue!36},
  language=Matlab,
  breaklines=true,
  keywordstyle=\color{blue!70},commentstyle=\color{red!50!green!50!blue!50},
  morekeywords={},
  stringstyle=\color{purple},
  frame=shadowbox,
  rulesepcolor=\color{red!20!green!20!blue!20}
}
\maketitle

\section{习题}

\noindent8. 对权函数$\rho(x)=1+x^2$……
\ans
  设$\psi_n = x^n$,$\eta_n = \varphi_n / \|\varphi_n\|$,则
  \begin{align*}
    \psi_0 &= 1, & \varphi_0 &= 1, & \eta_0 &= \frac{3}{8}, \\
    \psi_1 &= x, & \varphi_1 &= x, & \eta_1 &= \frac{15}{16}x, \\
    \psi_2 &= x^2, & \varphi_2 &=  x^2-\frac{2}{5}, & \eta_2 &= \frac{525}{136}\left(x^2-\frac{2}{5}\right), \\
    \psi_3 &= x^3, & \varphi_3 &= x^3-\frac{9}{14}x. & &
  \end{align*}

\vspace{1cm}
\par\noindent10. 证明对每一个Chebyshev多项式……
\proof
  \[
    P = \int_{-1}^1\frac{T^2_n(x)}{\sqrt{1-x^2}}\rd x
    = \int_{-1}^1\cos^2(n\arccos x)\rd \arccos x
  \]
  另$x = \cos t$,$u = nt$,根据$\cos^2x$的周期性,有
  \[
    P = \int_0^\pi \cos(nt) \rd t =
    \int_0^{n\pi}\cos^2u\rd \frac{u}{n} =
    \int_0^\pi\cos^2u\rd u = \frac{\pi}{2}.\quad\blacksquare
  \]

\vspace{1cm}
\par\noindent11. 用$T_3(x)$的零点做插值点……
\ans
  $T_3$的零点为
  \[
    x_1 = \cos\frac{\pi}{6} = \frac{\sqrt{3}}{2},\quad
    x_2 = \cos\frac{\pi}{2} = 0,\quad
    x_3 = \cos\frac{5\pi}{6} = -\frac{\sqrt{3}}{2}.
  \]
  在零点处的函数值分别为
  \[
    \f_1 = e^{\sqrt{3}/2},\quad \f_2 = 1,\quad \f_3 = e^{-\sqrt{3}/2}.
  \]
  则二次插值多项式为
  \[\begin{split}
    P_2(x) &= \f[x_1] + \f[x_1,x_2](x-x_1) +
    \f[x_1, x_2, x_3](x-x_1)(x-x_2) \\
    &= e^{\sqrt{3}/2} + \frac{e^{\sqrt{3}/2}-1}{\sqrt{3}/2}(x-\frac{\sqrt{3}}{2})
    + \frac{2}{3}\left(e^{\sqrt{3}/2} - 2 + e^{-\sqrt{3}/2}\right)(x-\frac{\sqrt{3}}{2})x\\
    &= \frac{2}{3}\left( e^{\frac{\sqrt{3}}{2}} + e^{-\frac{\sqrt{3}}{2}} - 2\right)x^2
    + \left( \frac{e^{\frac{\sqrt{3}}{2}} - e^{-\frac{\sqrt{3}}{2}}}{\sqrt{3}} \right)x + 1
  \end{split}\]
  对于它的误差,满足
  \[
    \vep \le \frac{1}{2^2(2+1)!}\|(e^x)^{(3)}\|_{\infty}
    = \frac{e}{24} \quad\blacksquare
  \]

\vspace{1cm}
\par\noindent12. 设$\f(x) = x^2 + 3x + 2$……
\ans
  另$x=\frac{1}{2}(t+1)$,则
  \[
    \f(x) = \g(t) = \frac{1}{4}t^2+2t+\frac{15}{4},\quad t\in[-1, 1].
  \]
  设$\widetilde{P}_n$是首项为$1$的Legendre多项式,$\g^*$
  为$\g$的在$S_2 = \spn\{1,x\}$上的最佳平方逼近函数,则
  \[\begin{split}
    & \g-\g^* = \frac{1}{4} \widetilde{P}_2\quad
    \Rightarrow\quad \g^* = \g-\frac{1}{4}\widetilde{P}_2
    =\frac{1}{4}t^2+2t+\frac{15}{4} - \frac{1}{4}(t^2-\frac{1}{3})
    = 2t + \frac{23}{6}
  \end{split}\]
  又$t = 2x-1$,所以$\f$的在$S_2$上的最佳平方逼近函数为
  \[
    \f_2^* = 4x + \frac{11}{6}.
  \]
  易知,$\f$在$S_3 = \spn\{1,x,x^2\}$上的最佳平方逼近函数为
  \[
    \f_3^* = \f = x^2 + 3x + 2.\quad\blacksquare
  \]

\end{document}
