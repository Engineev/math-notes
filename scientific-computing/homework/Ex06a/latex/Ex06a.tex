\documentclass[12pt, a4paper]{article}
\usepackage{ctex}

\usepackage[margin=1in]{geometry}
\usepackage{
  color,
  clrscode,
  amssymb,
  ntheorem,
  amsmath,
  listings,
  fontspec,
  xcolor,
  supertabular,
  multirow,
  mathtools,
  mathrsfs
}
\definecolor{bgGray}{RGB}{36, 36, 36}
\usepackage[
  colorlinks,
  linkcolor=bgGray,
  anchorcolor=blue,
  citecolor=green
]{hyperref}
\newfontfamily\courier{Courier}

\theoremstyle{margin}
\theorembodyfont{\normalfont}
\newtheorem{thm}{定理}
\newtheorem{cor}[thm]{推论}
\newtheorem{pos}[thm]{命题}
\newtheorem{lemma}[thm]{引理}
\newtheorem{defi}[thm]{定义}
\newtheorem{std}[thm]{标准}
\newtheorem{imp}[thm]{实现}
\newtheorem{alg}[thm]{算法}
\newtheorem{exa}[thm]{例}
\newtheorem{prob}[thm]{问题}
\DeclareMathOperator{\sft}{E}
\DeclareMathOperator{\idt}{I}
\DeclareMathOperator{\spn}{span}
\DeclareMathOperator*{\agm}{arg\,min}
\newcommand{\pr}{\prime}
\newcommand{\tr}{^\intercal}
\newcommand{\st}{\text{s.t.}}
\newcommand{\hp}{^\prime}
\newcommand{\ms}{\mathscr}
\newcommand{\mn}{\mathnormal}
\newcommand{\tbf}{\textbf}
\newcommand{\mbf}{\mathbf}
\newcommand{\fl}{\mathnormal{fl}}
\newcommand{\f}{\mathnormal{f}}
\newcommand{\g}{\mathnormal{g}}
\newcommand{\R}{\mathbf{R}}
\newcommand{\Q}{\mathbf{Q}}
\newcommand{\JD}{\textbf{D}}
\newcommand{\rd}{\mathrm{d}}
\newcommand{\str}{^*}
\newcommand{\vep}{\varepsilon}
\newcommand{\lhs}{\text{L.H.S}}
\newcommand{\rhs}{\text{R.H.S}}
\newcommand{\con}{\text{Const}}
\newcommand{\oneton}{1,\,2,\,\dots,\,n}
\newcommand{\aoneton}{a_1a_2\dots a_n}
\newcommand{\xoneton}{x_1,\,x_2,\,\dots,\,x_n}
\newcommand\thmref[1]{定理~\ref{#1}}
\newcommand\lemmaref[1]{引理~\ref{#1}}
\newcommand\defref[1]{定义~\ref{#1}}
\newcommand\posref[1]{命题~\ref{#1}}
\newcommand\secref[1]{节~\ref{#1}}
\newcommand\equref[1]{(\ref{#1})}
\newcommand\figref[1]{图 \ref{#1}}
\newcommand\corref[1]{推论~\ref{#1}}
\newcommand\exaref[1]{例~\ref{#1}}
\newcommand\algref[1]{算法~\ref{#1}}
\newcommand{\remark}{\paragraph{评注}}
\newcommand{\example}{\paragraph{例}}
\newcommand{\proof}{\paragraph{证明}}


\title{科学计算作业$\,$练习$6a$}
\author{\small 任云玮\\\small2016级ACM班\\\small516030910586}
\date{}

\begin{document}
\lstset{
  numbers=left,
  basicstyle=\scriptsize,
  numberstyle=\tiny\color{red!89!green!36!blue!36},
  language=Matlab,
  breaklines=true,
  keywordstyle=\color{blue!70},commentstyle=\color{red!50!green!50!blue!50},
  morekeywords={},
  stringstyle=\color{purple},
  frame=shadowbox,
  rulesepcolor=\color{red!20!green!20!blue!20}
}
\maketitle

\noindent1. 确定下列积分公式中的待定系数……
\ans
  (1) 分别代入$\f(x) = 1$,$\f(x) = x$,$\f(x) = x^2$,
  \[
   \begin{cases}
     &A_{-1} + A_0 + A_1 = 2h \\
     &-hA_{-1} + hA_1 = 0 \\
     &h^2A_{-1} + h^2A_1 = 2h^3/3
   \end{cases}
   \quad\Rightarrow\quad
   \begin{cases}
     & A_{-1} = h/3 \\
     & A_0 = 4h/3 \\
     & A_1 = h/3
   \end{cases}
  \]
  所以至少有$2$次代数精度,代入$\f(x) = x^3$和$\f(x) = x^4$,有
  \[\begin{split}
    I(x^3) = 0 &= 0 = Q(x^3) \\
    I(x^4) = \frac{2}{5}h^5 &\ne \frac{2}{3}h^5 = Q(x^4)
  \end{split}\]
  综上,具有$3$次代数精度. $\blacksquare$\\
  (2) 分别代入$\f(x) = 1$,$\f(x) = x$,$\f(x) = x^2$,
  \[\begin{cases}
    &A_{-1} + A_0 + A_1 = 4h \\
    &-hA_{-1} + hA_1 = 0 \\
    &h^2A_{-1} + h^2A_1 = 16h^2/3
    \end{cases}
    \quad\Rightarrow\quad
    \begin{cases}
      & A_{-1} = 8h/3 \\
      & A_0 = -4h/3 \\
      & A_1 = 8h/3
    \end{cases}
  \]
  所以至少有$2$次代数精度,代入$\f(x)=x^3$和$\f(x)=x^4$,有
  \[\begin{split}
    I(x^3) = 0 &= 0 = Q(x^3)\\
    I(x^4) = \frac{64}{5}h^5 &\ne \frac{16}{3}h^5 = Q(x^4)
  \end{split}\]
  综上,具有$3$次代数精度. $\blacksquare$\\
  (3) 分别代入$\f(x) = 1$,$\f(x) = x$,$\f(x) = x^2$,
  \[
    \begin{cases}
      & (1+2+3)/3 = 2 \\
      & (-1 + 2x_1 + 3x_2)/3 = 0\\
      & (1 + 2x_1^2 + 3x_2^2)/3 = 2/3
    \end{cases}
    \quad\Rightarrow\quad
    \begin{cases}
      &x_1 = (1\mp\sqrt{6})/5 \\
      &x_2 = (3\pm 2\sqrt{6})/15
    \end{cases}
  \]
  所以至少有$2$次代数精度,对于这两组解,代入$\f(x)=x^3$,有
  \[\begin{split}
    I(x^3) = 0 \ne Q(x^3)
  \end{split}\]
  综上,具有$2$次代数精度. $\blacksquare$\\
  (4) 分别代入$\f(x)=1$,$\f(x)=x$,$\f(x)=x^2$
  \[
    \begin{cases}
      & h(1 + 1)/2 + 0 = h \\
      & h^2/2 + 0 = h^2/2 \\
      & h^3/2 - 2ah^3 = h^3/3
    \end{cases}
    \quad\Rightarrow\quad
    a = \frac{1}{12}
  \]
  所以至少有$2$次代数精度,代入$\f(x)=x^3$,$\f(x)=x^4$,
  \[\begin{split}
    I(x^3) = \frac{1}{4}h^4 &= \frac{1}{4}h^4 = Q(x^4)\\
    I(x^4) = \frac{1}{5}h^5 &\ne \frac{1}{6}h^5 = Q(x^5)
  \end{split}\]
  综上,具有$3$次代数精度. $\blacksquare$

\vspace{1cm}
\par\noindent 6. 若用复合梯形公式……
\ans
  由于$(e^x)^{(n)} = e^x$对任意$n = 0,1,\dots$恒成立,
  所以在区间$[0,1]$上,成立$|(e^x)^{(n)}| \le e$. 对于
  复合梯形公式,
  \[
    \frac{1}{12}\left(\frac{1}{n}\right)^2e \le \frac{1}{2} \times 10^{-5}
    \quad\Rightarrow\quad n\ge 213.
  \]
  对于复合Simpson公式,
  \[
    \frac{1}{180}\left(\frac{1}{2n}\right)^4e \le \frac{1}{2} \times 10^{-5}
    \quad\Rightarrow\quad n \ge 4.
  \]
  即分别至少等分为$213$份和$4$份. $\blacksquare$

\vspace{1cm}
\par\noindent 10. 试构造Gauss求积公式……
\ans
  对于权函数$\rho(x) = 1/\sqrt{x}$,对应的正交多项式$\varphi_n$的
  前三项为
  \[
    \varphi_0 = 1,\quad \varphi_1 = x-\frac{1}{3},\quad
    \varphi_2 = x^2 - \frac{6}{7}x + \frac{3}{35}
  \]
  其中$\varphi_2$的零点为
  \[
    x_0 = \frac{15-2\sqrt{30}}{35},\quad
    x_1 = \frac{15+2\sqrt{30}}{35}.
  \]
  代入$\f(x) = 1$和$\f(x) = x$,得
  \[
  \begin{cases}
    & A_0 + A_1 = 2 \\
    & A_0x_0 + A_1x_1 = 2/3
  \end{cases}
  \quad\Rightarrow\quad
  \begin{cases}
    & A_0 = 1 + \sqrt{30}/18 \\
    & A_1 = 1 - \sqrt{30}/18.
  \end{cases}
  \]
  综上,对应的Gauss求积公式为
  \[
    Q(\f) = \left( 1+\frac{\sqrt{30}}{18} \right)\f\left(\frac{15-2\sqrt{30}}{35}\right)
    + \left(1 - \frac{\sqrt{30}}{18} \right) \f\left(\frac{15+2\sqrt{30}}{35}\right).
    \quad\blacksquare
  \]

\end{document}
