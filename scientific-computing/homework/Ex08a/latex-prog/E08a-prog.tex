\documentclass[12pt, a4paper]{article}
\usepackage{ctex}

\usepackage[margin=1in]{geometry}
\usepackage{
  color,
  clrscode,
  amssymb,
  ntheorem,
  amsfonts,
  amsmath,
  listings,
  fontspec,
  xcolor,
  supertabular,
  multirow,
  mathtools,
  mathrsfs,
}
\definecolor{bgGray}{RGB}{36, 36, 36}
\usepackage[
  colorlinks,
  linkcolor=bgGray,
  anchorcolor=blue,
  citecolor=green
]{hyperref}
\newfontfamily\courier{Courier}

\theoremstyle{margin}
\theorembodyfont{\normalfont}
\newtheorem{thm}{定理}
\newtheorem{cor}[thm]{推论}
\newtheorem{pos}[thm]{命题}
\newtheorem{lemma}[thm]{引理}
\newtheorem{defi}[thm]{定义}

\DeclareMathOperator{\rank}{rank}
\DeclareMathOperator{\adj}{adj}
\DeclareMathOperator{\tr}{tr}
\DeclareMathOperator{\diag}{diag}
\DeclareMathOperator{\nul}{null}
\DeclareMathOperator{\range}{range}
\DeclareMathOperator{\spn}{span}
% \DeclareMathOperator{\deg}{deg}

\newcommand{\hp}{^\prime}
\newcommand{\vep}{\varepsilon}
\newcommand{\inv}{^{-1}}
\newcommand{\rd}{\mathrm{d}}

\renewcommand{\Im}{\text{Im}}
\renewcommand{\Re}{\text{Re}}



\title{E08a 编程作业解答}
\author{姓名:任云玮$\quad$学号:516030910586}
\date{}

\usepackage{titlesec}

\titleformat*{\section}{\large\bfseries}
\titleformat*{\subsection}{\normalsize\bfseries}
\titleformat*{\subsubsection}{\small\bfseries}
\newcommand{\ilc}{\texttt}

\begin{document}
\lstset{
  numbers=left,
  basicstyle=\scriptsize,
  numberstyle=\tiny\color{red!89!green!36!blue!36},
  language=Matlab,
  breaklines=true,
  keywordstyle=\color{blue!70},commentstyle=\color{red!50!green!50!blue!50},
  morekeywords={},
  stringstyle=\color{purple},
  frame=shadowbox,
  rulesepcolor=\color{red!20!green!20!blue!20}
}
\maketitle


\section{问题1的解答}
  求解下列方程的实根
  \begin{align}
    & x^2-3x+2-e^x = 0 \\
    & x^3+2x^2+10x-20 =0
  \end{align}
\subsection{设计不动点迭代}
  对方程(1)和(2), 分别设计不动点迭代,即给出相应的$\varphi(x)$,并说明其收敛性.
  \paragraph{(1)}
    \[
      \varphi(x) = \frac{1}{3}\left(x^2 +2 -e^x\right).
    \]
    令$\f(x) = x^2 - 3x + 2 -e^2$,$\f(0.25)\approx -0.028 > 0$,
    $\f(0.3)\approx -0.16 < 0$,所以零点$x_*$在$[0.25, 0.3]$中. 而
    在该区间上$\varphi\hp(x) = (2x - e^x)/3 \in (-0.3, -0.2)$.
    所以一阶收敛. $\quad\blacksquare$
  \paragraph{(2)}
    \[
      \varphi(x) = -\frac{1}{20}(x^3+2x^2-20x-20).
    \]
    令$\f(x)=x^3+2x^2+10x-20$,$\f(1)=-7<0$,$\f(1.5)=2.875>0$,
    所以零点$x_*$在$[1, 1.5]$中. 而在该区间上$\varphi\hp(x)
    =-(3x^2 + 4x - 20)/20 \in (0.3, 0.7)$. 所以一阶收敛. $\quad\blacksquare$

\vspace{1cm}
\subsection{Steffensen加速迭代}
\subsubsection{对之前设计的不动点迭代格式,编写直接迭代的程序,命名为direct\_iteration.m,
迭代停止准则为$|x_k - x_{k-1}| < 10^{-8}$;}
  \lstinputlisting{../src/sec121.m}
  \lstinputlisting{../src/direct_iteration.m}

\subsubsection{编写Steffensen加速程序,命名为Steffensen.m, 在相同条件下重复方程(1)的计算:
给出不同初值下,加速前和加速后的迭代步数(列出表格).}
  \lstinputlisting{../src/sec122.m}
  \lstinputlisting{../src/Steffensen.m}
  \begin{table}[htbp]
    \caption{迭代步数比较}
    \centering
    \begin{tabular}{c|cc}
      \toprule
      $x_0$ & direct iteration & Steffensen \\
      \midrule
      0	& 14	& 4 \\
      0.1	& 14	& 4 \\
      0.2	& 13	& 3\\
      0.3	& 13	& 3\\
      0.4	& 14&	3 \\
      0.5	& 14&	4\\
      0.6	&14	&4 \\
      0.7	&15	&4\\
      0.8	&15	&4\\
      0.9	&15	&4 \\
      1	  &15	&4 \\
      \bottomrule
    \end{tabular}
  \end{table}

\vspace{1cm}
\subsection{Newton迭代}
\subsubsection{针对方程(2),讨论其实根的范围,并判断其是单根还是重根;}
  \paragraph{解}
    对于(2)的$\f(x)=\lhs$. 由于$\f(1) = -7 <0$,$\f(2) = 16 > 0$. 所以
    在$(1, 2)$上有一实根. 同时有
    \[
      \f\hp(x) = 3x^2 + 4x + 10
    \]
    其判别式$\Delta<0$,所以$\f\hp(x)>0$恒成立. 从而仅有一实根,且为单根.

\subsubsection{编写基于课本P227式(4.14)的Newton迭代程序,命名为Newton\_iteration.m, 迭代停止准则同前;}
  \lstinputlisting{../src/sec132.m}
  \lstinputlisting{../src/Newton_iteration.m}

\subsubsection{在相同条件下,比较Steffensen加速法与Newton迭代法的迭代步数.}
  \lstinputlisting{../src/sec133.m}
  \begin{table}[htbp]
    \caption{迭代步数比较}
    \centering
    \begin{tabular}{c|cc}
      \toprule
      $x_0$ & Newton iteration & Steffensen \\
      \midrule
      1	& 5	& 8 \\
      1.1&	4&	7\\
      1.2&	4&	7 \\
      1.3&	4&	8\\
      1.4&	4&	5\\
      1.5&	4&	6\\
      1.6&	4&	7\\
      1.7&	5&	9\\
      1.8&	5&	11\\
      1.9&	5&	13\\
      2  & 	5&	15 \\
      \bottomrule
    \end{tabular}
  \end{table}

\subsubsection{选做:若方程有复根,应如何求解?可能的话,求出方程(2)的一个复根.}
  \paragraph{解}
    由于压缩映射定理对于$\R^n$都是成立的,所以对于用于迭代的映射$\varphi$,
    只需要在不动点$x_*$附近满足$\nabla\varphi(x) \le l < 1$,即可保证局部收敛.
    从而只需要选择恰当的$\varphi$和复平面上的恰当初始点$x_0$,即可求得一个复根.
    对于方程(2),取初始点$x_0 = -1+3i$,得复根$x_* \approx -1.6844 + 3.4313i$.


\newpage
\section{问题2的解答}
  多项式求根是一个病态问题,考虑多项式
  \[
    p(x) = (x-1)(x-2)\cdots(x-10) = a_0 + a_1x + \cdots + a_9x^9 + x^{10}
  \]
  求解扰动方程$p(x) + \vep x^9 = 0$.

\subsection{病态性分析}
  任意给定系数,对$\vep = 10^{-6},\,10^{-8},\,10^{-10}$,
  用MATLAB求根函数计算扰动方程的根,分析$\vep$对根的影响(注意MATLAB多项式的存储方式).
  \paragraph{解}
    代码与求解结果如下.
    \lstinputlisting{../src/sec221.m}
    \begin{table}[htbp]
      \centering
      \caption{扰动方程的根}
      \begin{tabular}{c|ccc}
        \toprule
        $\varepsilon$ & $10^{-6}$ & $10^{-8}$ & $10^{-10}$ \\
        \midrule
        $x_{10}$ & 9.99722948588222 & 9.99997244092440 & 9.99999972451783  \\
        $x_9$ & 9.00954408664740 & 9.00009608123229 & 9.00000096030286  \\
        $x_8$ & 7.98668735225904 & 7.99986684514984 & 7.99999866983353  \\
        $x_7$ & 7.00940116158683 & 7.00009342012864 & 7.00000093241953  \\
        $x_6$ & 5.99651655390372 & 5.99996500763840 & 5.99999965125337  \\
        $x_5$ & 5.00067909066280 & 5.00000678245187 & 5.00000006736228  \\
        $x_4$ & 3.99993932906276 & 3.99999939306495 & 3.99999999402577  \\
        $x_3$ & 3.00000195269050 & 3.00000001953720 & 3.00000000018562  \\
        $x_2$ & 1.99999998730198 & 1.99999999987237 & 1.99999999999926  \\
        $x_1$ & 1.00000000000276 & 1.00000000000006 & 0.999999999999987 \\
        \bottomrule
      \end{tabular}
    \end{table}
    对于每一个零点$r_i = i$,相对于$a_9 = -55$的相对条件数成立公式
    \[
      \kappa_i = \frac{|a_9\times r_i^8|}{|p\hp(r_i)|}.
    \]
    所以每个零点对应的条件数为,
    \[
      \kappa_1 = 1.51\times 10^{-4},\quad
      \kappa_2 = 0.3492,\quad\dots,\quad
      \kappa_9 = 9.15\times 10^4,\quad
      \kappa_{10} = 1.51\times 10^4.
    \]
    首先在原多项式的零点处,$p\hp(x)$都没有十分小或十分大,同时$a_9=-55$的大小也可以接收.
    所以条件数的大小主要由$r_i^8$决定.
    当$r_i$较小,为$1$或$2$时,条件数较小,由系数的扰动而产生的误差较小. 而
    当$r_i$较大时,由于有$r_i^8$的存在,条件数十分大,从而有微小扰动产生的误差亦会十分大.

\subsection{解决方案思考}
  选做:可能的话提出一个稳定化求解的策略(可以查阅文献).
  \paragraph{解}
    设多项式被表示为
    \[
      p(x) = \sum_{i=0}^n b_ip_i(x).
    \]
    其中$\{p_i\}$是一组基. 则多项式零点$r$相对于系数$b_i$的相对条件数成立公式
    \[
      \kappa = \frac{|b_i p_i(r)|}{|rp\hp(r)|}.
    \]
    由于$p\hp(x)$是给定的,无法修改,所以考虑选取恰当的基使得在零点处的$p_i(r)$的
    取值较小即可. 从而可以选取Chebyshev多项式作为基,然后测量对应的系数即可.\footnote{
      参考文献: Tsai, Edison, "A Method for Reducing Ill-Conditioning of Polynomial
       Root Finding Using a Change of Basis" (2014). University Honors Theses. Paper 109.
    }


\end{document}
