\section{函数的多项式插值}
\subsection{问题的提出}
  \begin{defi}[插值]
    \label{defi: 插值}
    设函数$y = \f(x)$在$[a, b]$上有定义,且已知在点
    $a\le x_0 < x_1 < \cdots < x_n \le b$处的函数值
    $y_i = \f(x_i)$,若存在一简单函数$P(x)$,成立
    \[
      P(x_i) = y_i,
    \]
    则称$P(x)$为$\f(x)$的插值函数,点$x_1,\,x_2,\,\dots,\,x_n$
    称为插值节点,$[a, b]$称为插值区间,求$P(x)$的方法被称为插值法. \par
    若$P(x) \in P_n$为次数不超过$n$的多项式,则称为多项式插值.
  \end{defi}

  \begin{thm}[唯一性]
    给定满足\defref{defi: 插值}的$n+1$个点上的函数值,则
    次数不超过$n$的插值多项式$P_n(x)$存在且唯一.
  \end{thm}
  \proof
    利用待定系数法,设多项式的系数为$a_0,\,\dots\,a_n$,则
    有线性方程组
    \[
      \left\{
      \begin{gathered}
          a_0 + a_1x_0 + \cdots + a_nx_0^n = y_0,\\
          a_0 + a_1x_1 + \cdots + a_nx_1^n = y_1,\\
          \cdots\cdots \\
          a_0 + a_1x_n + \cdots + a_nx_n^n = y_n,\\
      \end{gathered}
      \right.
    \]
    其系数矩阵为Vandermonde矩阵
    \[
      A =
      \begin{pmatrix}
        1 & x_0 & \cdots & x_0^n \\
        1 & x_1 & \cdots & x_1^n \\
        \vdots & \vdots & & \vdots \\
        1 & x_n & \cdots & x_n^n \\
      \end{pmatrix}
    \]
    根据\defref{defi: 插值}中对于$x_i$的要求,矩阵行列式成立
    \[
      \det A = \prod_{i,j=0,\,i>j}^n (x_i - x_j) \ne 0.
    \]
    所以该方程组有唯一解.
  \remark
    虽然插值多项式是唯一的,但是根据基函数的选取的不同,
    系数是不相同的,所以才需要不同的插值方法.

  \iffalse
    给定一个函数$y = y(x)$在$[a, b]$中的点$x_0<\cdots<x_n$相应的函数值
    $y(x_i)$,设法重构函数$y = y(x)$. \footnote{从系统论的观点出发,
    即:已知映射$K$的一系列输入输出对,如何重构该映射$K$. }\par
    一般情况下解不唯一. 所以设$y = y(x)$位于某一函数内,此时可能存在唯一解.
    例如,设$y(x)\in P_n$,则只需知道$n+1$个点的函数值,即可有唯一解.
      或者设输出$y_i$有测量误差.
      \[
      \widetilde{y}_i = \f(x_i) + \vep_i
      \]
      其中$\vep_i$为误差.
      \[
      y = \f(x) + \vep
      \]
      其中$\vep$正态分布. 根据极大似然估计,等价于最小二乘解,
      (输出最小二乘法)
      \[
      \min_{y\in\pi} \sum_{i=0}^n|y_i - y(x_i)|^2
      \]
      其中$\pi$为某一个函数,有两种情况,
      \begin{enumerate}
      \item $\pi$由物理或化学规律确定;(基于机理建模)
      \item $\pi$由数据决定. (基于数据建模)
      \end{enumerate}
      (求关于权重和阈值的优化问题);(非线性最小二乘解).

    \paragraph{系统论的观点}
  \fi

\subsection{Lagrange插值法}
  \begin{thm}[Lagrange插值法]
    \label{thm: Lagrange插值法}
    定义
    \[\begin{split}
      l_i(x) &= \frac{\prod_{j\ne i}(x - x_j)}{\prod_{j\ne i}(x_i - x_j)} \\
      (L_n\f)(x) &= \sum_{i=0}^ny_il_i(x),
    \end{split}\]
    则$L_n\f$即为$\f$的插值多项式.
  \end{thm}
  \proof
    考虑构造$l_i\in P_n$,满足条件$l_i(x_j) = \delta_{ij}$,这样
    $L_n\f=\sum y_il_i$满足要求. 改写条件为(以$l_0$为例)
    \[\begin{split}
      l_0(x) &= \alpha(x-x_1)\cdots(x-x_n) \\
      l_0(x_0) &= 1
    \end{split}\]
    解得
    \[
      \alpha = \frac{1}{(x_0-x_1)\cdots(x_0-x_n)}\quad\blacksquare
    \]
  \remark
    这样构造插值多项式的动机在于在取定插值节点后,插值实际上相当于
    构造一个从$\mbf{y}=(y_0,\,\dots,\,y_n)\in\R^{n+1}$
    到$y^*(x)\in P_n$的一个映射$\ms{F}$,并且可以证明,
    $\ms{F}$是线性的. 因此成立
    \[
      \ms{F}(\mbf{y}) = \ms{F}(\sum_{i=0}^ny_i\mbf{e}_i)
       = \sum_{i=0}^ny_i\ms{F}(\mbf{e}_i)
       = \sum_{i=0}^ny_i l_i(x).
    \]

  \begin{thm}[Lagrange余项公式]
    设符号含义同\thmref{thm: Lagrange插值法}且$\f$充分光滑,则
    对于每一个固定的$x$成立
    \[
      \f(x) - (L_n\f)(x) =
      \frac{\f^{(n+1)}(\xi)}{(n+1)!}\omega_{n+1}(x)
    \]
    其中$\xi\in(a, b)$且
    \[
      \omega_{n+1}(x) = (x-x_0)\cdots(x-x_n).
    \]
  \end{thm}
  \proof
    固定$x\ne x_i$,定义$R(x)$满足
    \[
      \f(x) - (L_n\f)(x) = R(x)\omega_{n+1}(x).
    \]
    构造辅助函数$\g(t)$
    \[
      \g(t) = \f(t) - (L_n\f)(t) - R(x)\omega_{n+1}(t).
    \]
    根据插值法与$R(x)$的定义,成立
    \[
      \g(x_i) = 0,\quad g(x) = 0,
    \]
    即函数$\g(t)$有$n+2$个零点. 反复应用Rolle定理,可知存在
    $\xi\in(a, b)$,成立$\g^{(n+1)} = 0$,即
    \[\begin{split}
      & \g^{(n+1)}(\xi) = \f^{(n+1)}(\xi) - R(x)(n+1)! = 0 \\
      \Rightarrow\quad& R(x) = \frac{\f^{(n+1)}(\xi)}{(n+1)!}
    \end{split}\]
    结合$R(x)$的定义式即可知命题成立. $\blacksquare$
  \remark
    当已知$\f^{(n+1)}$有界时,可以使用此公式进行估计.

\subsection{Runge现象}
  \begin{thm}
    对于复函数$\f(z)$,如果存在$r_0>\frac32(b-a)$,使得
    $\f(z)$在$B_{r_0}(\frac{a+b}{2})$内解析,则
    $P_n(x) = L_n(x)$在$[a, b]$上一致收敛与$\f(z)$.
    这里$B_{r_0}(\frac{a+b}{2})$为以$\frac{a+b}{2}$为圆心,
    $r_0$为半径的圆.
  \end{thm}

\subsection{Newton插值法}
  \begin{exa}
    $n=2$时问题的求解. \par
    设$y^*(x) = a_0 + a_1(x-x_0) + a_2(x-x_0)(x-x_1)$,
    根据$y^*(x_0) = y_0$,$y^*(x_0) = y_0$,得$a_0 = \f(x_0)$,
    $a_0 + a_1(x_1-x_0) = \f(x_1)$,即
    \[
      a_1 = \frac{\f(x_1)-\f(x_0)}{x_1 - x_0}.
    \]
    可以发现$a_1$为
    割线的斜率. 同理可知
    \[
      a_2 = \frac{\f(x_2)-\f(x_0)-a_1(x_2-x_0)}{(x_2-x_0)(x_2-x_1)}
      = \frac{\dfrac{\f(x_2) - \f(x_0)}{x_2-x_0} - \dfrac{\f(x_1) - \f(x_0)}{x_1-x_0}}{x_2-x_1}
    \]
    为斜率的斜率.
  \end{exa}

  \begin{defi}[差商]
    递归$\f(x)$在$x_{i},\,\dots,\,x_{i+n}$的各阶差商为:
    $\f[x_i] = \f(x_i)$,第$k$阶差商为
    \[
      \f[x_i,\,\dots,\,x_{i+k}] =
      \frac{\f[x_{i+1},\,\dots,\,x_{i+k}] - \f[x_i,\,\dots,\,x_{i+k-1}]}
      {x_{i+k} - x_i}
    \]
  \end{defi}


  \begin{thm}[$n$次Newton插值法]
    \label{thm: Newton插值法}
    $x_0,\dots,x_n$为互异插值点,则函数$\f(x)$满足
    \[\begin{split}
      \f(x) =& \f[x_0] + \f[x_0, x_1](x-x_0) + \f[x_0,x_1,x_2](x-x_0))(x-x_1) + \cdots \\
      & + \f[x_0,\dots,x_n](x-x_0)\cdots(x-x_{n-1}) + R_n(x).
    \end{split}\]
    其中$R_n(x)$为其Newton插值多项式的余项,为
    \[
      R_n(x) = \f[x_0,\dots,x_n,x](x-x_0)\cdots(x-x_n).
    \]
  \end{thm}
  \proof
    根据差商的定义,有
    \[\begin{split}
      \f(x) &= \f[x_0] + \f[x_0, x](x-x_0) \\
      \f[x_0, x] &= \f[x_0, x_1] + \f[x_0, x_1, x](x-x_1) \\
      \cdots & \cdots\cdots \\
      \f[x_0,\dots,x_{n-1}, x] &= \f[x_0,\dots,x_n] +
      \f[x_0,\dots,x_n,x](x-x_n)
    \end{split}\]
    将上述式子反复代入它上面的式子,即得Newton插值公式.
  \remark
    Newton插值法的优点在于,当插值点的个数增加时,无需重新计算原有的系数,
    即Newton插值多项式是可以递归计算的.

  \begin{thm}
    根据Newton插值公式,可以得到如下差商的性质.
    \begin{enumerate}
      \item
      \[
        \f[x_0,\dots,x_m] =
        \sum_{i=0}^m\frac{\f(x_i)}{\prod_{j\ne i}(x_i-x_j)}
      \]
      \item 设$i_0,\dots,i_m$为$0,\dots, m$的任意一个排列,则
      \[
        \f[x_0,\dots,x_m] = \f[x_{i_0},\dots,x_{i_m}].
      \]
      \item 广义Lagrange中值定理
      \[
        \f[x_0,\dots,x_m] = \frac{\f^{(m)}(\xi)}{m!},\quad
        \xi\in(\min\{x_i\}, \max\{x_i\})
      \]
    \end{enumerate}
  \end{thm}
  \proof
    交换插值节点的顺序后,$n$次Newton插值多项式的$n$次项系数不变,
    所以[2.]成立. 同$m-1$次的Lagrange插值多项式比较第$m-1$次
    项系数及其余项,即可得到[1.]和[3.]
  \remark
    根据[3.]可知,对于一个$n$次多项式,$n$阶差商即为其$n$次项系数,
    $k(k>n)$阶差商为零.

  \begin{defi}
    \label{def: 算符}
    给定序列$\{\f_k\}$,$\f_k$表示$\f$在$x=x_k$处的值,定义
    \begin{enumerate}
      \item 前向差分算符$\Delta\f_k = \f_{k+1} - \f_k,$
      \item 移位算符$\sft\f_k = \f_{k+1},$
      \item 恒等算符$\idt\f_k = \f_k.$
    \end{enumerate}
  \end{defi}

  \begin{thm}[算符二项式定理]
    \label{thm: 算符二项式定理}
    对于算符$A$,$B$,若它们可交换,则成立二项式定理
    \[
      (A+B)^m = \sum_{k=0}^m \binom{m}{k} A^kB^{m-k}
    \]
  \end{thm}

  \begin{pos}
    根据\defref{def: 算符}和\thmref{thm: 算符二项式定理}可知
    \[\begin{split}
      \Delta &= \sft - \idt \\
      \Delta^n &= \sum_{k=0}^n\binom{n}{k}(-\idt)^{n-k}\sft^k \\
      \sft^n &= \sum_{k=0}^n\binom{n}{k}\Delta^k
    \end{split}\]
  \end{pos}

  \begin{thm}[均匀插值]
    \label{thm: 均匀插值}
    设$x_0,x_1,\dots,x_n$满足$x_k = x_0 + kh$,则有
    \[\begin{split}
      \f[x_k, x_{k+1}] &= \frac{\Delta\f_k}{h} \\
      \f[x_k, x_{k+1}, x_{k+2}] &=
      \frac{\Delta^2\f_k}{2h^2} \\
      \cdots & \cdots \\
      \f[x_k, \dots,x_{k+m}] &=
      \frac{\Delta^m\f_k}{m!h^m} \\
    \end{split}\]
  \end{thm}

  \begin{thm}[Newton前插公式]
    设记号同\thmref{thm: 均匀插值},另$x = x_0 + th$,$t\in\R$,
    代入\thmref{thm: Newton插值法},则成立
    \[\begin{split}
      N_n(x) =& \f_0 + t\Delta\f_0 + \frac{t(t-1)}{2!}\Delta^2\f_0 + \cdots + \frac{t(t-1)\cdots(t-n+1)}{n!}\Delta^n\f_0.
    \end{split}\]
    其余项为
    \[
      R_n(x) = \frac{t(t-1)\cdots(t-n)}{(n+1)!}h^{n+1}\f^{(n+1)}(\xi),
      \quad \xi\in(x_0, x_n)
    \]
  \end{thm}
  \remark
    利用广义二项式定理,有
    \[\begin{split}
      \f(x) &= \f(x_0 + th) = \sft^t\f(x_0) = (\idt+\Delta)^t\f(x_0) \\
      &= \sum_{k=0}^\infty\binom{t}{k}\Delta^k\f_0
    \end{split}\]

\subsection{Hermite插值}
  \begin{thm}
    设$\f\in C^n[a,b]$,$x_0,x_1,\dots,x_n$为$[a, b]$上的互异节点,
    则$\f[x_0,\dots,x_n,x]$在$[a, b]$上连续.
  \end{thm}

  \begin{thm}[Hermite插值]
    若给出$m+1$个插值条件(含函数值和导数值)可构造出次数不超过$m$次的
    多项式.
  \end{thm}
  \remark
    可以利用待定系数法或者基函数法求的Hermite插值多项式.

\subsection{分段低次多项式插值}
  \paragraph{思路}
    局部入手,整体分析\footnote{例:微分流形}.
    化整为零,以直代曲\footnote{例:Riemann积分}.

  \begin{lemma}[$\omega_n$的估计]
    任给节点$x_0<x_1<\cdots<x_n$,记$h = \max{x_{i+1}-x_i:
    i=0,1,\dots,n}$,则对于任意$x\in[x_0,x_n]$,成立
    \[
      |(x-x_0)\cdots(x-x_n)| \le \frac{n!h^{n+1}}{4}.
    \]
  \end{lemma}

  \begin{thm}[分段线性插值]
    记$a = x_0 < \cdots x_n = b$,$e_k = (x_k, x_{k+1})$,
    $h_k = x_{k+1} - x_k$,$h = \max h_k$. 找函数$y=\f_h(x)$,
    逼近原有函数,使得
    \begin{enumerate}
      \item 满足插值条件,$\f_h(x_k) = \f(x_k)$,
      \item $\f_h(x)$连续,
      \item $\f_h(x)\in P_1$,$x\in e_k$.
    \end{enumerate}
    则$\f_h$的结果为
    \[
      \f_h(x) = \f_h(x_k) + \f_h[x_k, x_{k+1}](x-x_k),
      \quad(x\in e_k).
    \]
    设$M_2$表示二阶导数的上界,则误差$R(x)$满足,
    \[
      R(x) = \left|\frac{\f^{(2)}(\xi)}{2}(x-x_k)(x-x_{k+1})\right|
      \le \frac{1}{8}M_2h^2.
    \]
  \end{thm}
  \remark
    若作整体的Lagrange插值,则余项$R_L(x)$满足
    \[
      \left|\frac{\f^{(n+1)}(\xi)}{(n+1)!}\omega_{n+1}\right|
      \le \frac{M_{n+1}h^{n+1}}{n+1}.
    \]
    而求高阶导数后容易出现Runge现象.

  \begin{thm}[分段三次Hermite插值]
    节点同上,构造$\f_h(x)$使得
    \begin{enumerate}
      \item $\f_h(x_k) = \f(x_k),\,\f_h\hp(x_k) = \f\hp(x_k)$
      \item $\f\in \ms{C}^1[a, b]$
      \item $\f_n(x) \in P_3,\,x\in e_k$
    \end{enumerate}
    对于两插值点的情况,$\f$的结果为
    \[
      \f_h(x) = \f(x_k)\alpha_k + \f(x_{k+1})\alpha_{k+1}
       + \f\hp(x_k)\beta_k + \f\hp(x_{k+1})\beta_{k+1}.
    \]
    其余项$R(x)$满足
    \[
      R(x) = \left|\frac{\f^{4}(\xi)}{4}(x-x_{k})^2(x-x_{k+1})^2\right|
      \le \frac{M_4}{4!}\times\left(\frac{h}{4}\right)^4
      =\frac{1}{384}M_4h^4.
    \]
  \end{thm}

\subsection{三次样条插值}
  \begin{defi}
    \label{defi: 三次样条插值}
    给定控制点$a = x_0 < \cdots < x_n = b$,设函数$y=y^*(x)$满足
    \begin{enumerate}
      \item $y^*(x) = y_k$,
      \item $(y^*)^{(4)} = 0,x\in e_k\,\Leftrightarrow
      \,y^* \in P_3,x\in e_k$,
      \item $y^*\in\ms{C}^2[a, b]$.
    \end{enumerate}
    称满足后两个条件的函数为\tbf{三次样条函数},称满足上述三个条件的函数
    为\tbf{三次样条插值函数}.
  \end{defi}

  \begin{defi}[边界条件]
    $\,$
    \begin{enumerate}
      \item 转角条件:$S\hp(x_0) = \f_0\hp$,$S\hp(x_n) = \f_n\hp$,
      \item 弯矩条件:$S^{\prime\prime}(x_0) = \f_0^{\prime\prime}$,
      $S^{\prime\prime}(x_n) = \f_n^{\prime\prime}$,称
      $S^{\prime\prime}(x_0) = S^{\prime\prime}(x_n) = 0$的特例为
      自然边界条件,
      \item 周期条件:$S(x_0+0)=S(x_n-0)$,$S\hp(x_0+0)=S\hp(x_n-0)$,
      $S^{\prime\prime}(x_0+0)=S^{\prime\prime}(x_n-0)$,
      \item 非纽结条件:$S^{\pr\pr\pr}(x)$在$x=x_1$和$x=x_{n-1}$处
      连续. \footnote{(Not-a-knot end Condition)这是Matlab中spline在$X$和$Y$长度相同时所应用的边界
      条件. }
    \end{enumerate}
  \end{defi}
  \remark
    根据\defref{defi: 三次样条插值},在每个小区间上有$4$个待定系数,
    所以总共有$4n$个待定系数. 而所给条件仅有$n+1$个插值条件,以及在
    中间$n-1$个插值节点处二阶导数连续(从而原函数与一阶导数也连续),
    有$3n-3$个光滑性条件,共$4n-2$个条件,因此需要额外的边界条件来确定
    剩余两个系数.

  \begin{thm}[样条插值的求解]
    设$S^{\pr\pr}(x_i) = M_i$,通过求解$M_i$来确定插值多项式.
    由于$S(x)\in\ms{C}^2$且在每一段上$S(x)\in P_3$,所以$S^{\pr\pr}(x)$
    是分段的线性函数. 设在每一段上,
    \[
      S^{\pr\pr}(x) = M_j\frac{x_{j+1}-x}{x_{j+1}-x_j}
      + M_{j+1}\frac{x-x_j}{x_{j+1}-x_j}.
    \]
    对$S^{\pr\pr}$积分一次、二次,并分别利用光滑性条件、插值条件,以及
    所给定的边界条件,即可求得$M_i$的值.
  \end{thm}

  \begin{thm}
    设$\f(x)\in\ms{C}^4[a, b]$,则三次样条插值函数$S_3(x)$,
    \[
      \max_{a\le x\le b}\left| \f^{(m)}(x) - S^{(m)}(x) \right|
      \le C_m \max_{a\le x\le b}|\f^{(4)}(x)|h^{4-m},\quad
      m=0,1,2,
    \]
    其中$C_0 = 5/384$,$C_1 = 1/24$,$C_2 = 3/8$.
  \end{thm}
