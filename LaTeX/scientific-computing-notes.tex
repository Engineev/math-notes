\documentclass[12pt, a4paper]{article}
\usepackage{ctex}

\usepackage[margin=1in]{geometry}
\usepackage{
  color,
  clrscode,
  amssymb,
  ntheorem,
  amsmath,
  listings,
  fontspec,
  xcolor,
  supertabular,
  multirow
}
\definecolor{bgGray}{RGB}{36, 36, 36}
\usepackage[
  colorlinks,
  linkcolor=bgGray,
  anchorcolor=blue,
  citecolor=green
]{hyperref}
\newfontfamily\courier{Courier}

\theoremstyle{margin}
\theorembodyfont{\normalfont}
\newtheorem{thm}{定理}
\newtheorem{cor}[thm]{推论}
\newtheorem{pos}[thm]{命题}
\newtheorem{lemma}[thm]{引理}
\newtheorem{defi}[thm]{定义}
\newtheorem{std}[thm]{标准}
\newtheorem{imp}[thm]{实现}
\newtheorem{alg}[thm]{算法}
\newtheorem{exa}[thm]{例}
\newcommand{\st}{\text{s.t.}}
\newcommand{\mn}{\mathnormal}
\newcommand{\tbf}{\textbf}
\newcommand{\fl}{\mathnormal{fl}}
\newcommand{\f}{\mathnormal{f}}
\newcommand{\g}{\mathnormal{g}}
\newcommand{\R}{\mathbf{R}}
\newcommand{\Q}{\mathbf{Q}}
\newcommand{\JD}{\textbf{D}}
\newcommand{\rd}{\mathrm{d}}
\newcommand{\str}{^*}
\newcommand{\vep}{\varepsilon}
\newcommand{\lhs}{\text{L.H.S}}
\newcommand{\rhs}{\text{R.H.S}}
\newcommand{\con}{\text{Const}}
\newcommand{\oneton}{1,\,2,\,\dots,\,n}
\newcommand{\aoneton}{a_1a_2\dots a_n}
\newcommand{\xoneton}{x_1,\,x_2,\,\dots,\,x_n}
\newcommand\thmref[1]{定理\ref{#1}}
\newcommand\lemmaref[1]{引理\ref{#1}}
\newcommand\defref[1]{定义\ref{#1}}
\newcommand\equref[1]{(\ref{#1})}
\newcommand{\remark}{\paragraph{评注}}
\newcommand{\example}{\paragraph{例}}
\newcommand{\proof}{\paragraph{证明}}

\title{科学计算 笔记\\MA235}
\author{任云玮}
\date{}

\begin{document}
\lstset{numbers=left,
  basicstyle=\scriptsize\courier,
  numberstyle=\tiny\courier\color{red!89!green!36!blue!36},
  language=C++,
  breaklines=true,
  keywordstyle=\color{blue!70},commentstyle=\color{red!50!green!50!blue!50},
  morekeywords={},
  stringstyle=\color{purple},
  frame=shadowbox,
  rulesepcolor=\color{red!20!green!20!blue!20}
}
\maketitle
\tableofcontents
\newpage

\section{绪论}

\subsection{计算机数值计算基本原理}

\subsubsection{计算机基本工作原理}
  \begin{tabular}{lcr}
    |CPU| &  & |外存储器| \\
    | & 数据总线 & | \\
    |I/O接口| & |I/O接口| & |外存接口| \\
    |输入|    & |输出|    & |外存储器|
  \end{tabular}

\subsubsection{实数的存贮方法}
  \begin{defi}[二进制浮点数系]\footnote{floating Number System}
    实数在计算机内部为\tbf{近似存贮},采用二进制浮点数系
    \[
      F(2,n,L,U)=\{\pm0.\aoneton\times10^m\}\cup\{0\}
    \]
    其中$a_1=1$,$a_i\in\{0,\,1\}$. 指数$m$满足$L\le m\le U$.
    称$n$为其字长,$2$表示采用二进制。
  \end{defi}

  \begin{std}[IEEE]
    $\,$
    \begin{enumerate}
      \item 单精度: $t=24,L=-126,U=127$
      \item 双精度: $t=53,L=-1022,U=1023$
      \item Underflow Limit: $UFL=0.1\times2^L$.
      若$0<x<UFL$,则$\\fl(x)=0$.
      \item Overflow Limit: $OFL=0.11\dots1*2^U$.
      若$x>OFL$,则$\fl(x)=\infty$.
      \item 舍入: 若$UFL\le x\le OFL$,则$\fl(x)$为舍入所得浮点数。
      舍入规则如下:设$x=0.\aoneton\dots\times2^m$. 若$a_{n+1}=1$,
      则$d_t+1$并舍弃其后项;否则直接舍弃其后项。
    \end{enumerate}
  \end{std}

  \begin{defi}[机器精度]
    下仅考虑舍去的情况。
    \[\begin{split}
    x-\fl(x) &= 2^m \times 0.0\dots0a_{n+2}\dots \\
    &=2^m \times [2^{-(t+2)} + 2^{-(t+3)} + \cdots] \\
    &=2^m\times2^{-(t+1)}
    \end{split}\]
    其相对误差满足
    \[
      \frac{x-\fl(x)}{x} < \frac{x-\fl(x)}{0.5\times2^m}=2^{-t}
    \]
    记为$\varepsilon$,称之为机器精度。
  \end{defi}

  \begin{pos}
    \[
      \fl(x)=x(1+\delta),\,\text{其中}\,|\delta|\le\varepsilon
    \]
  \end{pos}

\subsubsection{实数的基本运算原理}
  加法$+$硬件实现$\Rightarrow$四则运算。

  \begin{imp}[$x+y$]
    设$x,\,y$为浮点数,则$x+y$的实现方式如下:
    \begin{enumerate}
      \item 对阶:将指数$m$化为两者中较大者;
      \item 尾数相加;
      \item 舍入;
      \item 溢出分析等……
      \item 结果输出。
    \end{enumerate}
  \end{imp}
  \remark
    由$\fl(x)+\fl(y)=x(1+\delta_x)+y(1+\delta_y)$可知,当一个
    大数与一个小数相加时,小数有可能被忽略,所以应当避免大数小数间的相加。

\newpage
\subsection{误差的来源与估计}

\subsubsection{误差的来源}
 \begin{enumerate}
  \item 模型问题。例:近似地球为球体来计算。
  \item 测量误差。例:测量地球半径时的误差。
  \item 方法误差(截断误差)。
  例:对于$y=\f(x)$,求$\f(x^*)$时使用Taylor展开。
  \item 舍入误差(rounding-off)。例:计算机计算时的误差。
 \end{enumerate}

\subsubsection{误差与有效数字}
  \begin{defi}[绝对误差]
    设$x$为给定实数,$x^*$为其近似值。定义绝对误差为
    \[
      e(x^*) = x^* - x.
    \]
    称$\varepsilon^*$为其误差上界,若
    \[
      |e(x^*)| \le \varepsilon^*
    \]
  \end{defi}

  \begin{defi}[相对误差]
    对于同上的$x$和$x^*$,定义其相对误差
    \[
      e_r(x^*)=\frac{x^* - x}{x}
    \]
    称$\varepsilon_r^*$为其相对误差界,若
    \[
      |e_r(x^*)|\le\varepsilon_r^*
    \]
  \end{defi}
  \remark
    在实际应用中,$x$通常是未知的,所以会采用
    \[
      \bar{e}_r(x^*)=\frac{x^*-x}{x^*}
    \]
    来代替相对误差。对于分子,使用绝对误差界来替代,有如下不等式
    \[
      |\bar{e}_r(x^*)| \le \frac{\varepsilon^*}{|x^*|}.
    \]
    这两种相对误差界间的差别,当$\varepsilon^*\ll 1$时,满足
    \[
      |e_r-\bar{e}_r|=O((\varepsilon_r^*)^2)
    \]

  \begin{defi}[有效数字]
    设$x\in R$,$x^*$为其近似值。称$x^*$相对于$x$有$n$位有效数字,
    若$n$是满足下式的$n$的最大值。
    \[
      |x^* - x| \le \frac{1}{2} \times 10^{m - n}
    \]
  \end{defi}
  \remark
    在实践中,一般可以采用更加简便的方法,对于归一化以后的$x^*$,
    在尾数部分有$n$位,则称其有$n$位有效数字。注意,此方法对于
    错误的舍入结果是不适用的,对于错误的情况,需要再减去一位有效
    数字。

  \begin{thm}[误差与有效数字]
    若$x=0.\aoneton\times10^m$有$n$位有效数字,则
    \[
      \left|\frac{x^*-x}{x}\right| \le
      \frac{1}{2a_1}\times10^{1-n}.
    \]
    反之,若
    \[
      \left|\frac{x^*-x}{x}\right| \le
      \frac{1}{2(1+a_1)}\times 10^{1-n},
    \]
    则$x^*$至少有$n$位有效数字。
  \end{thm}
  \proof
    对于前者,只需利用有效数字的定义,以及利用$x\ge 0.a_1$
    (仅考虑$a_1\ne0$的情况)。对于后者,证明是类似的。

\subsubsection{数值运算的误差估计}
  以下内容都假设运算无误差。
  \begin{thm}[四则运算误差估计]
    $\,$
    \begin{enumerate}
      \item 加/减法: $\varepsilon(x\str\pm y\str)
      \le \varepsilon_x\str + \varepsilon_y\str$
      \item 乘法: $\vep(x\str y\str) \le
      |x\str|\vep\str_y + |y\str|\vep\str_x$
      \item 除法: $\vep(\dfrac{x\str}{y\str}) =
      \dfrac{|x\str|\vep\str_y + |y\str|\vep\str_x}{|y\str|^2}$
    \end{enumerate}
  \end{thm}
  \proof
    考虑加法的误差估计。对于$x$,$y$及其近似值$x^*$,$y^*$,
    计算$x\str\pm y\str$和$x\pm y$间的误差。
    \[\begin{split}
        & |x\str\pm y\str - ( x \pm y)|
        \le |x\str - x| + |y\str - y|
        \le \varepsilon_x\str + \varepsilon_y\str \\
        \Rightarrow\quad& \varepsilon(x\str\pm y\str)
        \le \varepsilon_x\str + \varepsilon_y\str
    \end{split}\]
    对于其他的运算,证明是类似的。(证明中可用$+1-1$技巧)

  \begin{thm}[运算的误差估计]
    设$A = \f(\tbf{x}) = \f(\xoneton)$,$\tbf{x}\str$是
    $\tbf{x}$的估计值。利用带Peano余项的Taylor展开,可知
    $A$的绝对误差满足
    \[\begin{split}
      e(A\str) &= \f(\tbf{x}\str) - \f(\tbf{x}) \\
      &= \sum_{p=1}^q\rd^k\f(\tbf{x}\str) + o(||x\str-x||^q) \\
      &\text{取q=1,则} \\
      &= \sum_{k=1}^n\partial_k\f(\tbf{x}\str)(x\str-x) + + o(||x\str-x||^q)
    \end{split}\]
    利用上式,可知
    \[\begin{split}
      &\vep(A\str) \approx \sum_{k=1}^n\partial_k\f(\tbf{x}\str)\vep(x\str)\\
      &\vep_r(A\str) = \frac{\vep(A\str)}{|A\str|}
    \end{split}\]
  \end{thm}

\subsubsection{数字求和的舍入误差分析}
  \begin{pos}
    $n$个浮点数相加,若将它们从小到大排列后相加,则可以减小
    舍入误差。
  \end{pos}
  \proof
    考虑浮点数的求和$S_n=\sum_i^n a_i$,在计算机中的过程表现为
    \[\begin{split}
      & S_2^* = \fl(a_1+a_2)=(a_1+a_2)(1+\vep_2),\quad|\vep_2|\le\vep=2^{-t}\\
      & \cdots\cdots\\
      & S_n^* = \fl(S_{n-1}^*+a_n)(1+\vep_n),\quad|\vep_n|\le\vep
    \end{split}\]
    对于$S_n\str$的误差,若定义$\vep_1=0$,则
    \[
      S_n\str=\sum_{k=1}^n a_k\prod_{p=k}^n(1+\vep_p)
    \]
    对误差进行估计,舍去高阶无穷小,有
    \[
      \prod_{i=k}^n(1+\vep_k)\approx1+\sum_{i=k}^n\vep_k
    \]
    综合上两式,有
    \[\begin{split}
      S_n\str & \approx\sum_{k=1}^na_k(1+\sum_{p=k}^n\vep_p)\\
      & = S_n + \sum_{k=1}^na_k\sum_{p=k}^n\vep_p
    \end{split}\]
    进行移项,并取绝对值,再利用三角不等式,以及$|\vep_i|\le\vep$,得
    \[
      \left|S_n\str-S_n\right|
      \le \sum_{k=1}^n|a_k|\sum_{p=k}^n|\vep_p|
      \le \vep\sum_{k=1}^n|a_k|(n-k+1)
    \]
    其中$n-k+1$关于$k$单调减少,所以根据排序不等式
    [\lemmaref{lemma: 排序不等式}],即可知命题成立。$\blacksquare$

\newpage
\subsection{避免算法失效的基本原则}
  \begin{thm}[原则]
    $\,$
    \begin{enumerate}
      \item 避免两数相除/相减,否则会严重损失有效数字;
      \item 避免大数与小数相加;
      \item 简化计算步骤。
    \end{enumerate}
  \end{thm}


  \begin{alg}[高效计算$e^A$]
    高效计算$e^A$,其中$A\in\R^{n\times n}$。首先有
    \[
      e^A = e^{(A/2^n)2^n} = B^{2^n}
    \]
    只需要得到$B$,即可以利用倍乘的方法快速得到$B^{2^n}$。
    下对于$B$进行估计。当$x\to0$时,$e^x$有Taylor展开
    \[
      e^x = 1 + x + \cdots + \frac{x^n}{n!} + \cdots
    \]
    而取足够大的$n$,即可以使得$A/2^n\approx0$,则可以对它
    展开得
    \[
      B \approx I + C + \frac{1}{2}C^2,\,\text{其中}
      C = A/2^n
    \]
    而对于倍乘,考虑$B^2$,展开平方得
    \[
      B^2 \approx I + 2(C+\frac{1}{2}C^2) + (C+\frac{1}{2}C^2)^2
    \]
    从右至左相加即可。
  \end{alg}

  \begin{alg}[秦九韶,多项式估值]
    设有多项式\equref{equ: 多项式},计算$p(z), z\in\R$的值。
    \begin{equation}
      \label{equ: 多项式}
      p(x) = a_0x^n + a_1x^{n-1} + \cdots + a_{n-1}x + a_n
    \end{equation}
    定义$b_n$满足
    \[
      b_0 = a_0, \quad b_k = a_k + b_{k-1}z
    \]
    则$b_n$即为所要求的值。并且成立
    \[
      p^{\prime}(z) = \sum_{k=0}^{n-1}b_kz^{n-1-k}
    \]
  \end{alg}
  \proof
    用$x-z$去除$p(x)$,记所得余数为$b_n(x)$,即
    \[
      p(x) = (x-z)q(x) + b_n(x),
    \]
    代入$x=z$,则左侧第一项为$0$,可知$p(z) = b_n(z)$。
    将两边的式子展开,利用对应系数相等,即可得算法中$b_n$
    的递推式。

  \begin{thm}[外推法]
    设$x_0$,$x_1$是$x$的两个估计值,且$x_1$相较于$x_0$更
    接近$x$,则可以通过恰当的权值$\omega$,使得它们的加权平均
    \[
      \bar{x} = x_1 + \omega(x_1-x_0)
    \]
    更加接近精确值$x$。
  \end{thm}

  \begin{alg}[$\pi$的估计]
    考虑单位圆,其面积为$\pi$,设$\pi_n$为单位圆的内接正$2n$边形
    的面积,以及
    \[
      \widetilde{\pi}_n = \frac{1}{3}(4\pi_{2n}-\pi_n)
    \]
    则$\pi_n$与$\widetilde{\pi}_n$与$\pi$的误差满足
    \[
      |\pi_n - \pi| = O(\frac{1}{n^2}), \quad
      |\widetilde{\pi}_n - \pi| = O(\frac{1}{n^4})
    \]
  \end{alg}
  \proof
    对于$\pi_n$.
    \[
      \pi_n=n\sin\frac{\pi}{n} =
      \pi - \frac{pi^3}{3!}\frac{1}{n^2} +
      \frac{pi^5}{5!}\frac{1}{n^4} - \cdots
      \,\Rightarrow \, |\pi_n - \pi| = O(\frac{1}{n^2})
    \]
    对于$\widetilde{\pi}_n$.
    \[\begin{split}
      \widetilde{\pi}_n & = \pi_{2n} + k(\pi_{2n} - \pi_n)
      = (1+k)\pi_{2n} - k\pi_n \\
      & = (1+k)(\pi - \frac{\pi^3}{3!}\frac{1}{4n^2} + \cdots)
      - k(\pi - \frac{\pi^3}{3!}\frac{1}{n^2} + \cdots) \\
      & = \pi - (\frac{k+1}{4} - k)\frac{\pi^3}{3!}\frac{1}{n^2}
      + O(\frac{1}{n^4})
    \end{split}\]
    取$k=\frac{1}{3}$,则成立
    \[
      |\widetilde{\pi}_n - \pi| = O(\frac{1}{n^4})
      \quad\blacksquare
    \]

\newpage
\section{附录}
\subsection{不等式}

  \begin{lemma}[排序不等式]
    \label{lemma: 排序不等式}
    对于满足下述条件的$\{a_n\}$,$\{b_n\}$,
    \[\begin{split}
      & 0 \le a_1\le a_2\le\cdots\le a_n \\
      & 0 \le b_1\le b_2\le\cdots\le b_n
    \end{split}\]
    则同序相乘求和值最大,逆序最小,即
    \[
      \sum_{i=1}^n a_ib_i \ge \sum_{i=1}^n a_ib_{k_i}
      \ge \sum_{i=1}^n a_ib_{n-i+1}
    \]
  \end{lemma}

  \begin{lemma}[算数-几何均值不等式]
    \[
      (a_1a_2\cdots a_n)^{1/n} \le \frac{a_1+a_2+\cdots+a_n}{n}
    \]
    当且仅当$a_1 = a_2 = \cdots = a_n$时等号成立。
  \end{lemma}
  \proof
    因为有齐次性,所以不妨设$\prod a_i=1$,并令
    \[
      a_1=\frac{\alpha_1}{\alpha_2},\quad
      \dots,\quad
      a_{n-1} = \frac{\alpha_{n-1}}{\alpha_n},\quad
      a_n = \frac{\alpha_n}{\alpha_1}
    \]
    则只需证明下式即可。
    \[
      \frac{\alpha_1}{\alpha_2} + \cdots + \frac{\alpha_n}{\alpha_1}
      \ge n
    \]
    不妨设$\alpha_1 \le \alpha_2 \le \cdots \le \alpha_n$,则根据排序不等式
    \[
      \lhs \ge \alpha_1\frac{1}{\alpha_1} + \cdots + \alpha_n\frac{1}{\alpha_n}
       = n \quad\blacksquare
    \]

\end{document}
