\documentclass[12pt, a4paper]{article}
\usepackage{ctex}

\usepackage[margin=1in]{geometry}
\usepackage{
  color,
  clrscode,
  amssymb,
  ntheorem,
  amsmath,
  listings,
  fontspec,
  xcolor,
  supertabular,
  multirow
}
\definecolor{bgGray}{RGB}{36, 36, 36}
\usepackage[
  colorlinks,
  linkcolor=bgGray,
  anchorcolor=blue,
  citecolor=green
]{hyperref}
\newfontfamily\courier{Courier}


\theoremstyle{margin}
\theorembodyfont{\normalfont}
\newtheorem{thm}{定理}
\newtheorem{cor}[thm]{推论}
\newtheorem{pos}[thm]{命题}
\newtheorem{lemma}[thm]{引理}
\newtheorem{defi}[thm]{定义}
\newtheorem{std}[thm]{标准}
\newtheorem{imp}[thm]{实现}
\newcommand{\st}{\text{s.t.}}
\newcommand{\mn}{\mathnormal}
\newcommand{\tbf}{\textbf}
\newcommand{\fl}{\mathnormal{fl}}
\newcommand{\f}{\mathnormal{f}}
\newcommand{\g}{\mathnormal{g}}
\newcommand{\R}{\mathbf{R}}
\newcommand{\Q}{\mathbf{Q}}
\newcommand{\JD}{\textbf{D}}
\newcommand{\rd}{\mathrm{d}}
\newcommand{\str}{^*}
\newcommand{\vep}{\varepsilon}
\newcommand{\con}{\text{Const}}
\newcommand{\oneton}{1,\,2,\,\dots,\,n}
\newcommand{\aoneton}{a_1a_2\dots a_n}
\newcommand{\xoneton}{x_1,\,x_2,\,\dots,\,x_n}
\newcommand\thmref[1]{定理~\ref{#1}}
\newcommand\defref[1]{定义~\ref{#1}}
\newcommand{\remark}{\paragraph{评注}}
\newcommand{\example}{\paragraph{例}}
\newcommand{\proof}{\paragraph{证明}}

\title{科学计算 笔记\\MA235}
\author{任云玮}
\date{}

\begin{document}
\lstset{numbers=left,
  basicstyle=\scriptsize\courier,
  numberstyle=\tiny\courier\color{red!89!green!36!blue!36},
  language=C++,
  breaklines=true,
  keywordstyle=\color{blue!70},commentstyle=\color{red!50!green!50!blue!50},
  morekeywords={},
  stringstyle=\color{purple},
  frame=shadowbox,
  rulesepcolor=\color{red!20!green!20!blue!20}
}
\maketitle
\tableofcontents
\newpage

\section{绪论}

\subsection{计算机数值计算基本原理}

\subsubsection{计算机基本工作原理}
  \begin{tabular}{lcr}
    |CPU| &  & |外存储器| \\
    | & 数据总线 & | \\
    |I/O接口| & |I/O接口| & |外存接口| \\
    |输入|    & |输出|    & |外存储器|
  \end{tabular}

\subsubsection{实数的存贮方法}
  \begin{defi}[二进制浮点数系]\footnote{Floating Number System}
    实数在计算机内部为\tbf{近似存贮},采用二进制浮点数系
    \[
      F(2,n,L,U)=\{\pm0.\aoneton\times10^m\}\cup\{0\}
    \]
    其中$a_1=1$,$a_i\in\{0,\,1\}$. 指数$m$满足$L\le m\le U$.
    称$n$为其字长,$2$表示采用二进制。
  \end{defi}

  \begin{std}[IEEE]
    $\,$
    \begin{enumerate}
      \item 单精度: $t=24,L=-126,U=127$
      \item 双精度: $t=53,L=-1022,U=1023$
      \item Underflow Limit: $UFL=0.1\times2^L$.
      若$0<x<UFL$,则$\fl(x)=0$.
      \item Overflow Limit: $OFL=0.11\dots1*2^U$.
      若$x>OFL$,则$\fl(x)=\infty$.
      \item 舍入: 若$UFL\le x\le OFL$,则$\fl(x)$为舍入所得浮点数。
      舍入规则如下:设$x=0.\aoneton\dots\times2^m$. 若$a_{n+1}=1$,
      则$d_t+1$并舍弃其后项;否则直接舍弃其后项。
    \end{enumerate}
  \end{std}

  \begin{defi}[机器精度]
    下仅考虑舍去的情况。
    \[\begin{split}
    x-\fl(x) &= 2^m \times 0.0\dots0a_{n+2}\dots \\
    &=2^m \times [2^{-(t+2)} + 2^{-(t+3)} + \cdots] \\
    &=2^m\times2^{-(t+1)}
    \end{split}\]
    其相对误差满足
    \[
      \frac{x-\fl(x)}{x} < \frac{x-\fl(x)}{0.5\times2^m}=2^{-t}
    \]
    记为$\varepsilon$,称之为机器精度。
  \end{defi}

  \begin{pos}
    \[
      \fl(x)=x(1+\delta),\,\text{其中}\,|\delta|\le\varepsilon
    \]
  \end{pos}

\subsubsection{实数的基本运算原理}
  加法$+$硬件实现$\Rightarrow$四则运算。

  \begin{imp}[$x+y$]
    设$x,\,y$为浮点数,则$x+y$的实现方式如下:
    \begin{enumerate}
      \item 对阶:将指数$m$化为两者中较大者;
      \item 尾数相加;
      \item 舍入;
      \item 溢出分析等……
      \item 结果输出。
    \end{enumerate}
  \end{imp}
  \remark
    由$\fl(x)+\fl(y)=x(1+\delta_x)+y(1+\delta_y)$可知,当一个
    大数与一个小数相加时,小数有可能被忽略,所以应当避免大数小数间的相加。

\newpage
\subsection{误差的来源与估计}

\subsubsection{误差的来源}
 \begin{enumerate}
  \item 模型问题。例:近似地球为球体来计算。
  \item 测量误差。例:测量地球半径时的误差。
  \item 方法误差(截断误差)。
  例:对于$y=\f(x)$,求$\f(x^*)$时使用Taylor展开。
  \item 舍入误差(rounding-off)。例:计算机计算时的误差。
 \end{enumerate}

\subsubsection{误差与有效数字}
  \begin{defi}[绝对误差]
    设$x$为给定实数,$x^*$为其近似值。定义绝对误差为
    \[
      e(x^*) = x^* - x.
    \]
    称$\varepsilon^*$为其误差上界,若
    \[
      |e(x^*)| \le \varepsilon^*
    \]
  \end{defi}

  \begin{defi}[相对误差]
    对于同上的$x$和$x^*$,定义其相对误差
    \[
      e_r(x^*)=\frac{x^* - x}{x}
    \]
    称$\varepsilon_r^*$为其相对误差界,若
    \[
      |e_r(x^*)|\le\varepsilon_r^*
    \]
  \end{defi}
  \remark
    在实际应用中,$x$通常是未知的,所以会采用
    \[
      \bar{e}_r(x^*)=\frac{x^*-x}{x^*}
    \]
    来代替相对误差。对于分子,使用绝对误差界来替代,有如下不等式
    \[
      |\bar{e}_r(x^*)| \le \frac{\varepsilon^*}{|x^*|}.
    \]
    这两种相对误差界间的差别,当$\varepsilon^*\ll 1$时,满足
    \[
      |e_r-\bar{e}_r|=O((\varepsilon_r^*)^2)
    \]

  \begin{defi}[有效数字]
    设$x\in R$,$x^*$为其近似值。称$x^*$相对于$x$有$n$位有效数字,
    若$n$是满足下式的$n$的最大值。
    \[
      |x^* - x| \le \frac{1}{2} \times 10^{m - n}
    \]
  \end{defi}
  \remark
    在实践中,一般可以采用更加简便的方法,对于归一化以后的$x^*$,
    在尾数部分有$n$位,则称其有$n$位有效数字。注意,此方法对于
    错误的舍入结果是不实用的。

  \begin{thm}[误差与有效数字]
    若$x=0.\aoneton\times10^m$有$n$位有效数字,则
    \[
      \left|\frac{x^*-x}{x}\right| \le
      \frac{1}{2a_1}\times10^{1-n}.
    \]
    反之,若
    \[
      \left|\frac{x^*-x}{x}\right| \le
      \frac{1}{2(1+a_1)}\times 10^{1-n},
    \]
    则$x^*$至少有$n$位有效数字。
  \end{thm}
  \proof
    对于前者,只需利用有效数字的定义,以及利用$x\ge 0.a_1$
    (仅考虑$a_1\ne0$的情况)。对于后者,证明是类似的。

\subsubsection{数值运算的误差估计}
  以下内容都假设运算无误差。
  \begin{thm}[四则运算误差估计]
    $\,$
    \begin{enumerate}
      \item 加/减法: $\varepsilon(x\str\pm y\str)
      \le \varepsilon_x\str + \varepsilon_y\str$
      \item 乘法: $\vep(x\str y\str) \le
      |x\str|\vep\str_y + |y\str|\vep\str_x$
      \item 除法: $\vep(\dfrac{x\str}{y\str}) =
      \dfrac{|x\str|\vep\str_y + |y\str|\vep\str_x}{|y\str|^2}$
    \end{enumerate}
  \end{thm}
  \proof
    考虑加法的误差估计。对于$x$,$y$及其近似值$x^*$,$y^*$,
    计算$x\str\pm y\str$和$x\pm y$间的误差。
    \[\begin{split}
        & |x\str\pm y\str - ( x \pm y)|
        \le |x\str - x| + |y\str - y|
        \le \varepsilon_x\str + \varepsilon_y\str \\
        \Rightarrow\quad& \varepsilon(x\str\pm y\str)
        \le \varepsilon_x\str + \varepsilon_y\str
    \end{split}\]
    对于其他的运算,证明是类似的。(证明中可用$+1-1$技巧)

  \begin{thm}[运算的误差估计]
    设$A = \f(\tbf{x}) = \f(\xoneton)$,$\tbf{x}\str$是
    $\tbf{x}$的估计值。利用带Peano余项的Taylor展开,可知
    $A$的绝对误差满足
    \[\begin{split}
      e(A\str) &= \f(\tbf{x}\str) - \f(\tbf{x}) \\
      &= \sum_{p=1}^q\rd^k\f(\tbf{x}\str) + o(||x\str-x||^q) \\
      &\text{取q=1,则} \\
      &= \sum_{k=1}^n\partial_k\f(\tbf{x}\str)(x\str-x) + + o(||x\str-x||^q)
    \end{split}\]
    利用上式,可知
    \[\begin{split}
      &\vep(A\str) \approx \sum_{k=1}^n\partial_k\f(\tbf{x}\str)\vep(x\str)\\
      &\vep_r(A\str) = \frac{\vep(A\str)}{|A\str|}
    \end{split}\]
  \end{thm}

\end{document}
