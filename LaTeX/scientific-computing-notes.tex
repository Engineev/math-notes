\documentclass[12pt, a4paper]{article}
\usepackage{ctex}

\usepackage[margin=1in]{geometry}
\usepackage{
  color,
  clrscode,
  amssymb,
  ntheorem,
  amsmath,
  listings,
  fontspec,
  xcolor,
  supertabular
}
\definecolor{bgGray}{RGB}{36, 36, 36}
\usepackage[
  colorlinks,
  linkcolor=bgGray,
  anchorcolor=blue,
  citecolor=green
]{hyperref}
\newfontfamily\courier{Courier}


\theoremstyle{margin}
%\theorembodyfont{\normalfont}
\newtheorem{thm}{定理}
\newtheorem{cor}[thm]{推论}
\newtheorem{lemma}[thm]{引理}
\newtheorem{defi}[thm]{定义}
\newtheorem{law}[thm]{定律}
\newcommand{\st}{\text{s.t.}}
\newcommand{\mn}{\mathnormal}
\newcommand{\tbf}{\textbf}
\newcommand{\f}{\mathnormal{f}}
\newcommand{\g}{\mathnormal{g}}
\newcommand{\R}{\mathbf{R}}
\newcommand{\Q}{\mathbf{Q}}
\newcommand{\JD}{\textbf{D}}
\newcommand{\rd}{\mathrm{d}}
\newcommand{\con}{\text{Const}}
\newcommand{\oneton}{1,\,2,\,\dots,\,n}
\newcommand{\aoneton}{a_1a_2\dots a_n}
\newcommand\thmref[1]{定理~\ref{#1}}
\newcommand\defref[1]{定义~\ref{#1}}
\newcommand{\remark}{\paragraph{评注}}
\newcommand{\example}{\paragraph{例}}
\newcommand{\proof}{\paragraph{证明}}

\title{科学计算\\MA235}
\author{任云玮}
\date{}

\begin{document}
\lstset{numbers=left,
  basicstyle=\scriptsize\courier,
  numberstyle=\tiny\courier\color{red!89!green!36!blue!36},
  language=C++,
  breaklines=true,
  keywordstyle=\color{blue!70},commentstyle=\color{red!50!green!50!blue!50},
  morekeywords={},
  stringstyle=\color{purple},
  frame=shadowbox,
  rulesepcolor=\color{red!20!green!20!blue!20}
}
\maketitle
\tableofcontents
\newpage

\section{绪论}

%%%
\subsection{计算机数值计算原理}
\subsubsection{计算机基本工作原理}

\subsubsection{实数的存贮方法}

\subsubsection{实数的基本运算原理}

%%%
\subsection{误差的来源与估计}

\subsubsection{误差的来源}
  \begin{enumerate}
    \item 模型问题。例:近似地球为球体来计算。
    \item 测量误差。例:测量地球半径时的误差。
    \item 方法误差(截断误差)。
    例:对于$y=\f(x)$,求$\f(x^*)$时使用Taylor展开。
    \item 舍入误差(rounding-off)。例:计算机计算时的误差。
  \end{enumerate}

\subsubsection{误差和有效数字}
  \begin{defi}[绝对误差等]
    设$x\in\R$,$x^*$为$x$的近似值。称
    \[
      e(x^*)=x^*-x
    \]
    为$x^*$的绝对误差。其误差上界$\epsilon^*$满足
    \[
      |e(x^*)|\le\varepsilon^*
    \]
  \end{defi}

  \begin{defi}[相对误差等]
  \end{defi}

  \begin{defi}[有效数字]\footnote{The number of significant}
    设$x=+-0.a_1a_2...\times10^m$,若在第$n$位舍入,则得到
    \[
     x^*=+-0.......≥
    \]
    称$x^*$具$n$位有效数字。
  \end{defi}
  \example
    设$\pi=3.1415926\dots$,则$\pi_1=3$的有效数字为$n(\pi_1)=1$,
    而$n(4)=0$,因为它并非正确的舍入所得。$n(3.1416)=5$,
    $n(3.1415)=4$。
  \begin{defi}[有效数字]
    若存在最大的$n$,使得
    \[
      |x^*-x|\le\frac{1}{2}\times10^{m-n}
    \]
    则称有$n$位有效数字。
  \end{defi}

  \begin{thm}[误差与有效数字]
    若$x^*=...\times10^m$有$n$位有效数字,则其相对误差满足
    \[
      \varepsilon_r(x^*)\le\frac{1}{2a_1}\times^{1-n}.
    \]
    若相对误差满足
    \[
      \varepsilon_r(x^*)\le\frac{1}{2(1+a_1)}\times^{1-n},
    \]
    则$x^*$至少有$n$位有效数字。
  \end{thm}
  \proof
    (假设$a_1$是精确的。)//todo

\subsubsection{数值运算误差估计}
  设有$x_1\Rightarrow x_1^*$,$x_2\Rightarrow x_2^*$,
  考虑$x_1+x_2\Rightarrow x_1^*+x_2^*$(假设运算无误差),
  进行估计,有
  \[
    |(x_1+x_2) - (x_1^*+x_2^*)| \le \varepsilon(x_1^*)+...
  \]
  对于乘法是类似的。
  \begin{thm}[运算误差估计]
    设有$\x_n\Rightarrow x_n^*$,...,
    \begin{split}\[
      A&=\f(x)=\f(x_1,\,x_2,\,...,\,x_n)\\
      A^*\f(x^*)=\f(x_1^*,\,x_2^*,\,...,\,x_n^*)
    \]\end{split}

  \end{thm}


\end{document}
