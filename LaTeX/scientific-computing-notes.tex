\documentclass[12pt, a4paper]{article}
\usepackage{ctex}

\usepackage[margin=1in]{geometry}
\usepackage{
  color,
  clrscode,
  amssymb,
  ntheorem,
  amsfonts,
  amsmath,
  listings,
  fontspec,
  xcolor,
  supertabular,
  multirow,
  mathtools,
  mathrsfs,
}
\definecolor{bgGray}{RGB}{36, 36, 36}
\usepackage[
  colorlinks,
  linkcolor=bgGray,
  anchorcolor=blue,
  citecolor=green
]{hyperref}
\newfontfamily\courier{Courier}

\theoremstyle{margin}
\theorembodyfont{\normalfont}
\newtheorem{thm}{定理}
\newtheorem{cor}[thm]{推论}
\newtheorem{pos}[thm]{命题}
\newtheorem{lemma}[thm]{引理}
\newtheorem{defi}[thm]{定义}

\DeclareMathOperator{\rank}{rank}
\DeclareMathOperator{\adj}{adj}
\DeclareMathOperator{\tr}{tr}
\DeclareMathOperator{\diag}{diag}
\DeclareMathOperator{\nul}{null}
\DeclareMathOperator{\range}{range}
\DeclareMathOperator{\spn}{span}
% \DeclareMathOperator{\deg}{deg}

\newcommand{\hp}{^\prime}
\newcommand{\vep}{\varepsilon}
\newcommand{\inv}{^{-1}}
\newcommand{\rd}{\mathrm{d}}

\renewcommand{\Im}{\text{Im}}
\renewcommand{\Re}{\text{Re}}



\title{科学计算 笔记}
\author{任云玮}
\date{}

\begin{document}
\lstset{numbers=left,
  basicstyle=\scriptsize\courier,
  numberstyle=\tiny\courier\color{red!89!green!36!blue!36},
  language=C++,
  breaklines=true,
  keywordstyle=\color{blue!70},commentstyle=\color{red!50!green!50!blue!50},
  morekeywords={},
  stringstyle=\color{purple},
  frame=shadowbox,
  rulesepcolor=\color{red!20!green!20!blue!20}
}
\maketitle
\tableofcontents
\newpage
\input{01-Intro.tex}

\newpage
\input{02-Interpolation.tex}

\newpage
\section{函数的多项式逼近}
\subsection{绪论}
  \begin{defi}[逼近]
    对函数$\f$逼近,即找一简单函数$\g$,使得在某种度量的意义下,
    它们之间的误差最小或足够小.
  \end{defi}

  \begin{thm}[Weierstrass]
    对于定义在$[a, b]$上的连续复函数,存在一列复多项式$\{P_n\}$,
    成立
    \[
      \lim_{n\to\infty}P_n = \f,
    \]
    且是一致的. 若$\f$是实函数,则$P_n$的系数也为实数.
  \end{thm}
  \remark
    Stone-Weierstrass定理
    \footnote{
      \tbf{Theorem(Stone) }
      Suppose $\ms{A}$ is a self-adjoint algebra of complex
      continuous functions on a compact set $K$, $\ms{A}$
      separates points on $K$, and $\ms{A}$ vanishes at no
      point of $K$. Then the uniform closure $\ms{B}$ of
      $\ms{A}$ consists of all complex continuous functions
      on $K$. In other words, $\ms{A}$ is dense in $\ms{C}(K)$.
    }
    保证了至少在最大模的意义下,用多项式
    来逼近函数是可能的.

  \begin{defi}[常用范数]
    对于$\R^n$,常用的范数有
    \begin{enumerate}
      \item $\|\tbf{x}\|_\infty = \max_{1\le i\le n}|x_i|$,
      \item $\|\tbf{x}\|_1 = \sum_{i=1}^n|x_i|$,
      \item $\|\tbf{x}\|_2 = (\sum_{i=1}^nx_i^2)^{1/2}$.
    \end{enumerate}
    对于$\ms{C}[a, b]$,常用的范数有
    \begin{enumerate}
      \item $\|\f\|_\infty = \max_{a\le x\le b}|\f(x)|$,
      \item $\|\f\|_1 = \int_a^b|\f(x)|\rd x$,
      \item $\|\f\|_2 = (\int_a^b\f^2(x) \rd x)^{1/2}$.
    \end{enumerate}
  \end{defi}
  \remark
    通常对于内积空间$X$,可以定义范数为
    \[
      \|\tbf{x}\| = \sqrt{(x, x)}.
    \]

  \begin{defi}[权函数]
    \label{defi: 权函数}
    设$[a,b]$为有限或无限区间\footnote{例子中说$\rho=1$是一个
    常用的权函数,但我没有明白,在无限区间的时候[1.]是如何成立的. }
    ,
    非负函数$\rho(x)$称为$[a,b]$上的权
    函数,若满足
    \begin{enumerate}
      \item $\int_a^b\rho(x)x^k\rd x < \infty$,$k=0,1,2,\dots$,
      \item 对任意非负$\g\in\ms{C}[a, b]$,若$\int_a^b\rho(x)\g(x)\rd x =0$,
      则$g=0$.
    \end{enumerate}
  \end{defi}
  \remark
    利用权函数,可以定义带权内积和范数.

\newpage
\subsection{最佳平方逼近}
  \begin{defi}[最佳平方逼近]
    \label{defi: 最佳平方逼近}
    给定$\f\in\ms{C}[a, b]$和线性无关的函数列$\varphi_0,\dots,
    \varphi_n\in\ms{C}[a, b]$,定义$S_n=\spn\{\varphi_0,
    \dots,\varphi_n\}$,称$\f^*\in S_n$为最佳平方逼近函数,若
    \[
      \|\f^* - \f\| = \min_{\g\in S_n}\|\f-\g\|_2.
    \]
    即$\f^*$是在$2$-范数的含义下,$S_n$中与$\f$最接近的函数.
  \end{defi}
  \remark
    对于离散的情况
    \footnote{
      实际上我们可以利用Riemann-Stieltjes积分定义内积,
      \[\begin{split}
        & (\f,\g) = \int_a^b \f\rd G, \\
        & G(x) =
        \begin{cases}
          \int_a^b\g\rd x,\quad&\text{$\g$为函数,} \\
          \sum_{i=0}^n\f(x_i)I(x-x_i), &\text{$\g$为离散点}
        \end{cases}
      \end{split}\]
      其中$I(x)$为单位阶跃函数. 可以发现,这两种描述的方式是等价的.
      在这样的描述下,对于离散点的$G$实际上是阶梯函数.
    }
    ,可以描述为:给定$x_0,\dots,x_n$处的函数值
    $\f(x_k)$,求$\f^*$,成立
    \[
      \sum_{i=0}^{n}\rho(x_i)|\f(x_j)-\f^*(x_j)|^2 =
      \min_{\g\in S_n}\sum_{i=0}^{n}\rho(x_i)|\f(x_j)-\g(x_j)|^2
    \]

  \begin{thm}[最佳平方逼近的求解]
    设记号同\defref{defi: 最佳平方逼近},设
    \[
      \g(x) = \sum_{i=0}^na_i\varphi_i(x),
    \]
    则可以定义关于$\tbf{a} = (a_0,\dots,a_n)\tr$的函数
    \[\begin{split}
      I(\tbf{a}) = \|\f-\g\|_2^2 &=
      \left\|\f - \sum_{i=0}^na_i\varphi_i\right\|_2^2
      =  \int_a^b \rho\left( \f-\sum_{i=0}^n a_i\varphi_i \right)^2 \rd x.
    \end{split}\]
    根据定义,$I(\tbf{a})$在$\f^*$处取极值,根据Fermat定理,
    在该点各偏导数为零,通常假设$\f$的条件足够好,极限和积分可以换序,即有
    \[\begin{split}
      \frac{\partial I}{\partial a_j} &=
      \int_a^b \frac{\partial}{\partial a_i}
      \rho\left( \f-\sum_{i=0}^n a_i\varphi_i \right)^2 \rd x\\
      &= -2\int_a^b\rho\left( \f-\sum_{i=0}^n a_i\varphi_i \right)\varphi_j \rd x
      = 0.
    \end{split}\]
    即有线性方程组,
    \begin{equation}\begin{cases}
      \label{equ: 法方程}
      (\varphi_0, \varphi_0)a_0 + (\varphi_0, \varphi_1)a_1 + \cdots + (\varphi_0, \varphi_n)a_n = (\f, \varphi_0) \\
      (\varphi_1, \varphi_0)a_0 + (\varphi_1, \varphi_1)a_1 + \cdots + (\varphi_1, \varphi_n)a_n = (\f, \varphi_1) \\
      \dots\dots\\
      (\varphi_n, \varphi_0)a_0 + (\varphi_n, \varphi_1)a_1 + \cdots + (\varphi_n, \varphi_n)a_n = (\f, \varphi_n)
    \end{cases}\end{equation}
    由于$\{\varphi_k\}$线性无关,所以方程组\equref{equ: 法方程}有唯一解.
    设其解为$\tbf{a}^*$,则最佳平方逼近函数即为
    \[
      \f^* = \sum_{i=0}^na^*_i\varphi_i(x),
    \]
  \end{thm}
  \remark
    实际上在计算的时候一般采用Legendre多项式来计算,而非解法方程.
    见\thmref{thm: Legendre多项式的逼近性质}.

  \paragraph{几何描述}
    可以从几何的角度来理解最佳平方逼近. $S_n$是$\{\varphi_k\}$
    张成的空间,而$\f$是$S_n$内或$S_n$外的一个向量,最佳平方逼近
    即找$S_n$中找$\f^*$,使得$\|\f-\f^*\|$最小. 根据几何上的直观,
    $\f-\f^*$应该和$S_n$“垂直”,即与张成$S_n$的向量组中的向量分别
    垂直. 而垂直可以被描述为内积为零. 从而就得到了式\equref{equ: 法方程}.
    (见\figref{fig: 最佳平方逼近几何含义})
    \begin{figure}[htbp]
      \centering
      \includegraphics[height=8cm]{../image/least-square.png}
      \caption{最佳平方逼近几何含义}
      \label{fig: 最佳平方逼近几何含义}
    \end{figure}

\newpage
\subsection{正交多项式·绪论}
  \begin{defi}[正交]
    设函数$\f,\g\in\ms{C}[a, b]$,$\rho$为$[a, b]$上的权函数
    且满足
    \[
      (\f, \g) = \int_a^b\rho\f\g\rd x = 0,
    \]
    则称$\f$和$\g$在$[a, b]$上带权$\rho$\tbf{正交}. 若函数组
    $\{\varphi_k\}_{k=0}^\infty$满足
    \[
      (\varphi_i,\varphi_j) =
      \begin{cases}
        0,&\quad i\ne j \\
        A_k>0,& i=j
      \end{cases}
    \]
    则称$\{\varphi_k\}$为$[a, b]$上的带权$\rho$的\tbf{正交函数组}.
    若$A_k=1$,则称为\tbf{标准正交函数组}.
  \end{defi}

  \begin{defi}[正交多项式]
    设$\{\varphi_k\}_{k=0}^\infty$是首项系数$a_n\ne0$的$n$
    次多项式序列. 若它们正交,则称它们为\tbf{正交多项式序列}.
  \end{defi}

  \begin{alg}[Gram-Schmidt正交化]
    设$\{\varphi_k\}$是内积空间$V$的一组基,定义
    \[\begin{split}
      \psi_0 &= \varphi_0, \\
      \psi_{n} &= \varphi_n - \sum_{i=0}^{n-1}(\varphi_n, \psi_i)\eta_i
    \end{split}\]
    其中$\eta_i = \psi_i / \|\psi_i\|^2$. 则$\{\psi_k\}$为
    $V$的一组正交基.
  \end{alg}
  \remark
    要求$n$次正交多项式组,只需另$\varphi_k = x^k$,再进行
    Gram-Schmidt正交化即可.

  \begin{thm}
    设$\{\varphi_n\}_{n=0}^\infty$是一列正交多项式,
    根据正交性(从而线性无关)可以得到正交多项式的如下性质,
    \begin{enumerate}
      \item $P_n \subset \spn\{\varphi_0, \dots,\varphi_n\}$,
      \item 设$P\in P_{n-1}$,则$\varphi_n$与$P$正交.
    \end{enumerate}
  \end{thm}

  \begin{thm}
    设$\{\varphi_n\}_{n=0}^\infty$是$[a,b]$上带权$\rho$的
    正交多项式,则成立
    \[
      \varphi_{n+1} = (x-\alpha_n)\varphi_n - \beta_n\varphi_{n-1},
      \quad n = 0, 1,\dots,
    \]
    其中
    \[\begin{split}
      &\varphi_0 = 1,\quad \varphi_{-1} = 0,\\
      &\alpha_n = (x\varphi_n, \varphi_n) / (\varphi_n, \varphi_n),\\
      &\beta_n = (\varphi_n, \varphi_n) / (\varphi_{n-1}, \varphi_{n-1}).
    \end{split}\]
  \end{thm}
  \proof
    由于齐次性,不妨设$\varphi_n$首项系数为$1$. 所以成立
    \[
      \varphi_{n+1}-x\varphi_n = \sum_{k=0}^n\gamma_k\varphi_k,
    \]
    对于系数$\gamma_k$,成立
    \footnote{
      设$B$是内积空间$V$的一组正交基,则对于任意
      $x\in V$,成立
      \[
        x = \sum_{\beta\in B}\frac{(x,\beta)}{(\beta,\beta)}\beta
      \]
      另外,根据这里内积的定义,成立$(\varphi_i,\varphi_j)
      =(\varphi_i/x, x\varphi_j)$.
    }
    \[
      \gamma_k = \frac{(\varphi_{n+1}-x\varphi_n,\varphi_k)}
      {(\varphi_k, \varphi_k)}
      = \frac{(\varphi_{n+1},\varphi_k) - (\varphi_n, x\varphi_k)}{(\varphi_k,\varphi_k)}.
    \]
    由于$\{\varphi_n\}$正交,所以当$k<n-1$时,成立$\gamma_k=0$.
    所以有
    \[
      \varphi_{n+1} - x\varphi_n = \gamma_n\varphi_n + \gamma_{n-1}\varphi_{n-1}.
    \]
    再进行一些代换,即可以得到原递推式. $\blacksquare$

  \begin{thm}
    设$\{\varphi_n\}_{n=0}^\infty$为$[a, b]$上带权$\rho$的
    正交多项式,则$\varphi_n$在区间$(a, b)$上有$n$个不同的零点.
  \end{thm}
  \proof
    首先利用权函数的定义,证明零点不可能都是偶数重的. 再假设
    $x_1,\dots,x_l$是$\varphi_n$的奇数重零点,则
    \[
      \left(\varphi_n, (x-x_1)\cdots(x-x_l)) \right) \ne 0,
    \]
    再利用正交性可得$l=n$. $\blacksquare$

\newpage
\subsection{Legendre多项式}
  \begin{defi}[Legendre多项式]
    取区间$[-1, 1]$,$\rho(x)\equiv1$,称由$\{1,x,\dots,x^n,\dots\}$
    正交化而得的多项式为Legendre多项式. 其表达式为
    \[
      P_0(x) = 1,\quad
      P_n(x) = \frac{1}{2^nn!}\frac{\rd^n}{\rd x^n}(x^2-1)^n,
      \quad n=1,2,\dots
    \]
    首项系数为$1$的Legendre多项式为
    \[
      \widetilde{P}_n(x) = \frac{n!}{(2n)!}\frac{\rd^n}{\rd x^n}(x^2-1)^n.
    \]
  \end{defi}

  \begin{thm}[Legendre多项式的性质]
    Legendre多项式有如下性质,
    \begin{enumerate}
      \item 正交性:
      \[
        \int_{-1}^1 P_n(x)P_m(x)\rd x =
        \begin{cases}
            0, &\quad m\ne n, \\
            \dfrac{2}{2n+1},&\quad m = n.
        \end{cases}
      \]
      \item 奇偶性:
        \[P_n(x) = (-1)^nP_n(-x),\]
      \item 递推关系:
      \[
        (n+1)P_{n+1} = (2n+1)xP_n - nP_{n-1},\quad
        n = 1, 2,\dots
      \]
    \end{enumerate}
  \end{thm}
  \proof todo

  \begin{thm}[Legendre多项式的逼近性质]
    \label{thm: Legendre多项式的逼近性质}
    在区间$[-1, 1]$上,设$\widetilde{L}_n$是首项系数为$1$的
    Legendre多项式,则
    \[
      \|\widetilde{L}_n\|_2 = \min_{P\in P_n}\|P(x)\|_2.
    \]
  \end{thm}
  \remark
    应用方法和说明可以参考Chebyshev多项式的逼近性质.
    (\thmref{thm: Chebyshev多项式的逼近性质})

  \begin{lemma}[前$4$项Legendre多项式]
    \[\begin{split}
      P_0 &= 1,\\
      P_1 &= x,\\
      P_2 &= \frac{3}{2}x^2 - \frac{1}{2},\\
      P_3 &= \frac{5}{2}x^3 - \frac{3}{2}x
    \end{split}\]
  \end{lemma}

\newpage
\subsection{Chebyshev多项式}
  \begin{defi}[Chebyshev多项式]
    取区间$[-1, 1]$,$\rho(x)=(1-x^2)^{-1/2}$,称由$\{1,x,\dots,x^n,\dots\}$
    正交化而得的多项式为Chebyshev多项式. 其表达式为
    \[
      T_n(x) = \cos(n\arccos x),\quad|x|\le 1
    \]
  \end{defi}

  \begin{thm}[Chebyshev多项式的性质]
    Chebyshev多项式有如下性质,
    \begin{enumerate}
      \item 递推关系:
      \[\begin{split}
        &T_0(x) = 1, \quad T_1(x) = x,\\
        &T_{n+1}(x) = 2xT_n(x) - T_{n-1}(x),\quad n =1,2,\dots
      \end{split}\]
      \item 正交性:
      \[
        \int_{-1}^1\frac{T_n(x)T_m(x)}{\sqrt{1-x^2}} \rd x =
        \begin{cases}
          0, &\quad n\ne m,\\
          \pi/2, &\quad n = m \ne 0,\\
          \pi,&\quad n = m = 0.
        \end{cases}
      \]
      \item $T_{2k}(x)$只含$x$的偶次幂,$T_{2k+1}(x)$只含$x$的
      奇次幂.
      \item $T_n$在区间$[-1, 1]$上的$n$个零点为
      \[
        x_k = \cos\frac{2k-1}{2n}\pi,\quad k = 1,2,\dots,n.
      \]
      \item $T_n$的首项系数为$2^{n-1}$.
    \end{enumerate}
  \end{thm}

  \begin{thm}[Chebyshev多项式的逼近性质]
    \label{thm: Chebyshev多项式的逼近性质}
    在区间$[-1,1]$上,设$\widetilde{T}_n$是首项系数
    为$1$的Chebyshev多项式,则
    \[
      \|\widetilde{T}_n\|_\infty = \min_{P\in\widetilde{P}_n}
      \|P(x)\|_\infty = \frac{1}{2^{n-1}}.
    \]
  \end{thm}
  \remark
    这一定理意味着,取区间$[-1, 1]$,$n$次Chebyshev多项式是所有次数
    小于等于$n$的首项为$1$的多项式中,绝对值的最大值最小的一个. 从而,
    若想用$P_{n-1}$中的多项式来逼近$n$次多项式$\f$,只需找
    $\f^*\in P_{n-1}$,使得
    \[
      \f - \f^* = a_n \widetilde{T}_n.
    \]
    其中$a_n$为$\f$的$n$次项系数. 对于一般的在区间$[a, b]$上的情况,
    只需利用平移和伸缩映射到$[-1, 1]$上即可.

  \begin{thm}[Chebyshev零点插值]
    设插值节点$x_0,\dots,x_n$为Chebyshev多项式$T_{n+1}$的
    零点,被插值函数$\f\in\ms{C}^{n+1}[-1, 1]$,则多项式插值
    的余项$R_n$满足
    \[
      |R_n| \le \frac{1}{2^n(n+1)!}\|\f^{(n+1)}(x)\|_{\infty}.
    \]
  \end{thm}
  \proof
    由于插值点是Chebyshev多项式的零点,所以$\omega_{n+1}=
    \widetilde{T}_n$,所以根据\thmref{thm: Chebyshev多项式的逼近性质},成立
    \[
      \omega_{n+1} \le \frac{1}{2^n}.\quad\blacksquare
    \]
  \remark
    这一定理保证了使用Chebyshev多项式的零点插值,至少可以使得
    误差的最大值最小.


\newpage
\section{变分方法与数据拟合}
\subsection{绪论}
  若要求函数$\f(\mbf{x})$,$x\in\R^n$的最值点,根据
  Fermat引理,只需要求出所有成立$\nabla\f = 0$的点再
  逐一验证即可. 变分是这一思想的推广,它所处理的是过程的优化.
  下给出一个优化过程的例子,完整的解答见之后的章节. (todo: ref)

  \begin{exa}[最速降线]
    给定空中的一点$A=(0,0)$,地上一点$B=(x_1,y_1)$,求一条连接
    $A$和$B$的轨迹,使得假设在无阻力情况下,有小球沿轨道从$A$到$B$
    所需要的时间最短. \par
    假设轨道曲线充分光滑,则问题可以转换为,设滑行轨道$y\in\ms{C}^2$的方程为
    \[
      y=y(x), \quad y(0)=0,\quad y(x_1) = y_1.
    \]
    试确定曲线$y$使得时间$T(y)$最小. 根据机械能守恒,小球在$(*,-y)$处
    (见\figref{fig: 最速降线问题})的速度,为
    \[
      v = \sqrt{2gy},
    \]
    同时由速度的定义知
    \[
      v = \frac{\rd s}{\rd t} = \sqrt{1+y^{\pr2}(x)}\frac{\rd x}{\rd t},
    \]
    根据上两式,即有
    \[
      \sqrt{2gy}\rd t = \sqrt{1+y^{\pr2}(x)}\rd x \,\Rightarrow\,
      \rd t = \left( \frac{1+y^{\pr2}}{2gy} \right)^{1/2}\rd x
    \]
    设过程为$A\to B$,$0\to t_1$,$0\to x_1$,则
    \[
      T(y) = t_1 = \int_0^{x_1} \left( \frac{1+y^{\pr2}}{2gy} \right)^{1/2}\rd x,
    \]
    即为所要最小化的$T(y)$.
    \begin{figure}[htbp]
      \centering
      \includegraphics[height=5cm]{../image/brachistochrone.png}
      \caption{最速降线问题}
      \label{fig: 最速降线问题}
    \end{figure}
  \end{exa}

\newpage
\subsection{变分方法}
  \begin{defi}[过程优化]
    \label{def: 过程优化}
    过程的优化即求解
    \[
      \label{equ: 过程优化}
      y^* = \agm_{y\in K}J(y)
    \]
    的过程,其中函数集合
    \[
      K = \{y\in\ms{C}^2[x_0, x_1]\,:\, y(x_0)=y_0,\,y(x_1)=y_1\}.
    \]
  \end{defi}

  \begin{defi}
    \label{defi: K}
    定义函数集合
    \[\begin{split}
      K &= \{\f\in\ms{C}^2[x_0, x_1]\,:\,\f(x_0)=y_0,
      \f(x_1) = y_1\}, \\
      K_0 &=\{\f\in\ms{C}^2[x_0, x_1]\,:\,\f(x_0)=
      \f(x_1) = 0\}.
    \end{split}\]
    对于任意$\f_0\in K$,$\eta \in K_0$定义集合
    \[
      K(\f_0, \eta) = \{\f_0 + \vep\eta\,:\,\vep\in\R\}.
    \]
  \end{defi}

  \begin{defi}[泛函的方向导数]
    \label{def: 泛函的方向导数}
    记号同\defref{defi: K}. 定义泛函
    \[
      J(\f) = \int_a^bL(x, \f(x), \f\hp(x)) \rd x,\quad
      \f \in K.
    \]
    如果$\f^*$是函数$J(\f)$在集合$K$中的最小值点,即
    \[
      \f^* = \agm_{\f\in K}J(\f),
    \]
    则对于任意$\eta\in K_0$,$\f^*$也是$J(\f)$在集合$K(\f^*,\eta)$
    中的最小值点. 所以成立
    \[
      \f^* = \agm_{\vep\in\R}J(\f^* + \vep\eta).
    \]
    由于函数$J(\f^* + \vep\eta)$是关于实数$\vep$的一元函数,所以在
    它最小值点,即$\vep =0$处成立
    \begin{equation}
      \label{equ: 方向导数为零}
      \frac{\rd}{\rd\vep}J(\f^* + \vep\eta)\bigg|_{\vep =0}=0.
    \end{equation}
    称左侧为泛函$J$在$\f^*$处沿$\eta$方向的\tbf{变分}
    或\tbf{方向导数}. 即为$\delta J(\f^*, \eta)$\footnote{一般来讲,
    $\f^*$可以是集合中的任意一点. }.
  \end{defi}
  \remark
    这里采用的是分析学中一个常见思想,将一个在高维空间中的问题
    转化为一个低维空间中的问题. 一个更加简单的例子是,证明若
    $k$维欧式空间中的函数$\f$在凸集$K$中的各偏导数恒为零,则
    它在$K$中为常量. 一个证法是取定$K$中任意一点(向量)$\mbf{x}_0$,
    则对于任意$\mbf{x}\in K$,则对于任意直线段$xx_0$,有方程
    \[
      y = \mbf{x_0}t + \mbf{x}(1-t).
    \]
    上式是一个一元实函数,对它求导并利用一元函数微分学中的知识,可以
    知道在这条直线段上$\f$的函数值不变. 由于$\mbf{x}_0$和$\mbf{x}$
    的选取是任意的,所以在$K$上$\f$的值不变. \par
    在这里采用的是同样的思想,只是把欧式空间换成了一个函数集合而已.

  \begin{lemma}[变分引理 I]
    \label{lemma: 变分引理}
    设$\f\in\ms{C}[a,b]$,且对任意满足$\g(a)=\g(b)=0$的
    $\g\in\ms{C}^2[a,b]$,有
    \[
      \int_{a}^{b}\f\g \rd x =0,
    \]
    则在$[a,b]$上成立$\f\equiv 0$.
  \end{lemma}

  \begin{lemma}[变分引理 II]
    设$\f\in\ms{C}[a, b]$,且对于任意满足$\eta(a) = \eta(b) = 0$的
    $\eta\in\ms{C}^1(a, b)$都成立
    \[
      \int_a^b\f\eta\hp\rd x = 0,
    \]
    则在$[a, b]$上成立$\f\equiv \text{Const}$.
  \end{lemma}

  \begin{thm}[Euler-Lagrange方程]
    \label{thm: Euler-Lagrange方程}
    记号同\defref{def: 泛函的方向导数}. 泛函$J(\f)$在极值点满足
    Euler-Lagrange方程
    \begin{equation}
      \label{equ: Euler-Lagrange方程}
      \frac{\partial L}{\partial\f} - \frac{\rd}{\rd x}\frac{\partial L}{\partial\f\hp} = 0.
    \end{equation}
  \end{thm}
  \proof
    假设满足求导积分换序的条件,则泛函$J(\f)$在
    $\f_*+\vep\eta$处沿$\eta$方向的方向导数为
    \footnote{为了记号的清晰,这里用$\f_*$替代之前的$\f^*$.
    并且这里关于偏导数的记号,应理解为关于分母所表示的那一分量的偏导数. }
    \[\begin{split}
      \frac{\rd}{\rd\vep}J(\f_*+\vep\eta) &=
      \frac{\rd}{\rd\vep}\int_{x_0}^{x_1}L(x,\f_*+\vep\eta,\f_*\hp+\vep\eta\hp)\rd x\\
      &= \int_{x_0}^{x_1} \frac{\partial L}{\partial (\f_*+\vep\eta)}\eta
      +\frac{\partial L}{\partial (\f_*\hp+\vep\eta\hp)}\eta\hp\,\rd x.
    \end{split}\]
    代入$\vep=0$,即得$J$在最小值点$\f_*$处的方向导数,为
    \[\begin{split}
      \frac{\rd}{\rd\vep}J(\f_*) &=
      \int_{x_0}^{x_1} \frac{\partial L}{\partial \f}\eta
      +\frac{\partial L}{\partial \f\hp}\eta\hp\,\rd x \\
      &= \int_{x_0}^{x_1} \frac{\partial L}{\partial\f}\eta\rd x
      + \frac{\partial L}{\partial\f\hp}\eta\bigg|_{x_0}^{x_1}
      - \int_{x_0}^{x_1}\eta\frac{\rd}{\rd x}\frac{\partial L}{\partial \f\hp}\rd x\\
      &= \int_{x_0}^{x_1} \eta\left( \frac{\partial L}{\partial\f}
      - \frac{\rd}{\rd x}\frac{\partial L}{\partial\f\hp} \right) \rd x = 0.
    \end{split}\]
    注意由于$\eta\in K_0$,即有$\eta(x_0) = \eta(x_1) = 0$.
    由于$\eta$的选取是任意的,所以根据\lemmaref{lemma: 变分引理},式子
    \equref{equ: Euler-Lagrange方程}成立. $\blacksquare$
  \remark
    对于$L = L(\f,\f\hp)$,即$L$不显含$x$的情况,
    \equref{equ: Euler-Lagrange方程}是可以精确求解的.

  \begin{thm}[守恒律定理]
    \label{thm: 守恒律定理}
    设$L = L(\f,\f\hp)$,则沿着\equref{equ: 过程优化}的解曲线
    $y^* = \f^*(x)$,成立
    \[
      H = \f\hp\frac{\partial L}{\partial\f\hp} - L = \text{Const}.
    \]
  \end{thm}
  \proof
    \[\begin{split}
     \frac{\rd H}{\rd x} &= \frac{\rd}{\rd x}
     \left( \f\hp\frac{\partial L}{\partial\f\hp}-L(\f,\f\hp) \right)\\
     &= \f^{\pr\pr}\frac{\partial L}{\partial\f\hp}
     + \f\hp\frac{\rd}{\rd x}\frac{\partial L}{\partial\f\hp} -
     \frac{\partial L}{\rd\f}\f\hp -
     \frac{\partial L}{\rd\f\hp}\f^{\pr\pr}.
    \end{split}\]
    根据\thmref{thm: Euler-Lagrange方程},$\rd H /\rd x =0$,所以
    命题成立. $\blacksquare$

  % \begin{exa}[最速降线的求解]
  %   根据之前的讨论,即求解
  %   \[
  %     \f_* = \agm_{\f\in K} T(\f) = \int_{0}^{x_1}
  %     \left( \frac{1+\f\hp}{2g\f} \right)^{1/2}\rd x.
  %   \]
  %   根据\thmref{thm: 守恒律定理},成立
  %   \[
  %
  %   \]
  % \end{exa}

\newpage
\subsection{曲线拟合的正则化方法}
  \begin{defi}[Tikhonov正则化]
    \label{def: Tikhonov正则化}
    对于给定的数据$Y$,定义数据拟合项为$J_1(\f)$,用于表示拟合结果
    相较于原数据的接近程度,同时要求拟合的结果尽可能满足对于结果的
    要求,用$J_2(\f)$来描述$\f$满足要求的程度,则求解拟合结果的
    过程即为求解
    \[
      \f_* = \agm_{\f\in K}(J_1(\f) + \alpha J_2(\f)).
    \]
    其中$\alpha$为\tbf{正则化参数},用于表示拟合的过程中,应更接近
    原数据或是更满足拟合要求. 若取$\alpha=0$,即为插值.
  \end{defi}

  \begin{prob}
    给定函数$y$在样本点$0=x_0 < x_1 < \cdots < x_n = 1$处的
    近似值$\tilde{y}_i$,误差满足
    \[
      |\tilde{y}_i - y(x_i)| \le \delta,
    \]
    试重构$y$的近似函数$\f_*$.\par
    按照\defref{def: Tikhonov正则化}的思想,定义
    \[\begin{split}
      J_1(\f) &= \sum_{i=1}^{n-1}\frac{h_i+h_{i+1}}{2}
      \left( \tilde{y}_i - \f(x_i) \right)^2 \\
      J_2(\f) &= \int_0^1 (\f^{\pr\pr})^2\rd x
    \end{split}\]
    其中
    \[\begin{split}
      h_i &= x_i - x_{i-1},\quad i = 1,2,\dots,n\\
      h &= \max_{1\le i\le n}h_i
    \end{split}\]
    则问题转换为求解
    \begin{equation}
      \label{equ: 正则化曲线拟合}
      \f_* = \agm_{\f\in K}\left( J_1(\f) + \alpha J_2(\f) \right).
    \end{equation}
  \end{prob}
  \remark
    不失一般性的,可以设$\tilde{y}_0 = \f(x_0)$且$\tilde{y}_n=\f(x_n)$.
    否则只需要用
    \[
      Y(x) = y(x)+\tilde{y}_0-y(0)+(\tilde{y}_n-y(1)+y(0)-\tilde{y}_n)x
    \]
    来替代$y$即可. 可以证明
    \begin{enumerate}
      \item $Y(0) = \tilde{y}_0$且$Y(1) = \tilde{y}_n$,
      \item $|\tilde{y}_i - Y(x_i)| \le 4\delta$.
    \end{enumerate}

  \begin{thm}
    对于任意$\alpha>0$,\equref{equ: 正则化曲线拟合}的解为三次样条函数.
  \end{thm}
  \proof
    设$\f_*$为式\equref{equ: 正则化曲线拟合}的解. \par
    对于任意$\eta\in K_0$,$\vep\in R$,
    \[
      J(\f_*+\vep\eta) = \sum_{i=1}^{n-1}\frac{h_i + h_{i+1}}{2}
      [\tilde{y}_i - f_*(x_i) - \vep\eta(x_i)]^2 + \alpha
      \int_0^1(\f_*^{\pr\pr} + \vep\eta^{\pr\pr})^2\,\rd x.
    \]
    求它关于$\vep$的导数,成立
    \[
      \frac{\rd}{\rd\vep}J(\f_*+\vep\eta) =
      -2\sum_{i=1}^{n-1}\frac{h_i+h_{i+1}}{2}[\tilde{y}_i-\f_*(x_i)-\vep\eta(x_i)]\eta(x_i)
      +2\alpha\int_0^1 (\f_*^{\pr\pr}+\vep\eta^{\pr\pr})\eta^{\pr\pr} \,\rd x.
    \]
    根据\defref{def: 泛函的方向导数}中的\equref{equ: 方向导数为零},上式在$\vep=0$时
    值为零,即
    \begin{equation}
      \label{equ: 正则化证明1}
      \frac{\rd}{\rd\vep}J(\f_*+\vep\eta)\bigg|_{\vep=0} =
      -2\sum_{i=1}^{n-1}\frac{h_i+h_{i+1}}{2}[\tilde{y}_i-\f_*(x_i)]\eta(x_i)
      +2\alpha\int_0^1\f_*^{\pr\pr}\eta^{\pr\pr}\,\rd x = 0.
    \end{equation}
    接下来分两步构造出$\f_*$.\\
  \tbf{Step 1. }
    由于$\eta$的选取是任意的,所以我们选择恰当的$\eta\in\ms{C}^\infty(x_i,x_{i+1})$,
    使得成立
    \[
      \eta(x_i) = 0,\quad i = 1,2,\dots,n-1
    \]
    所以根据\equref{equ: 正则化证明1},成立
    \[
      \int_{x_i}^{x_{i+1}}\f_*^{\pr\pr}\eta^{\pr\pr} = 0.
    \]
    对于上式进行两次分部积分,得到
    \[
      0 = \f_*^{\pr\pr}\eta\hp\bigg|_{x_i}^{x_{i+1}} -
      \f_*^{(3)}\eta\bigg|_{x_i}^{x_{i+1}} +
      \int_{x_i}^{x_{i+1}}\f_*^{(4)}\eta\rd x.
    \]
    同样因为$\eta$的选取是任意的,所以可以在原来的基础上,选取$\eta$使得成立
    \[
      \eta(x_i) = \eta\hp(x_i) = 0,\quad i = 1,2,\dots,n-1
    \]
    所以有
    \[
      \int_{x_i}^{x_{i+1}}\f_*^{(4)}\eta\rd x = 0.
    \]
    根据变分引理,即成立
    \begin{equation}
      \label{equ: 正则化证明2}
      \f_*^{(4)}(x) = 0\quad\Rightarrow\quad \f\in P_3,\quad x\in(x_i,x_{i+1}).
    \end{equation}
  \tbf{Step 2. }
    取满足$\eta(0)=\eta(1)=0$的$\eta\in\ms{C}^\infty[0, 1]$,分部积分得
    (注意$\f_*^{(4)} = 0$)
    \[\begin{split}
      \int_0^1 \f_*^{\pr\pr}\eta^{\pr\pr}\rd x =&
      \sum_{i=1}^{n-1}
      \left(
        -\f_*^{(3)}\eta\bigg|_{x_i-1}^{x_i}
        +\f_*^{\pr\pr}\eta\hp\bigg|_{x_i-1}^{x_i}
      \right) \\
      =& \sum_{i=1}^{n-1}
      ( \f_*^{(3)}(x_i+) - \f_*^{(3)}(x_i-))\eta(x_i) +
      \sum_{i=1}^{n-1} ( \f_*^{\pr\pr}(x_i+) - \f_*^{\pr\pr}(x_i-))\eta\hp(x_i)\\
      & + \f_*^{\pr\pr}(1)\eta\hp(1) - \f_*^{\pr\pr}(0)\eta\hp(0)
    \end{split}\]
    将结果带入\equref{equ: 正则化证明1},即
    \begin{equation}\begin{split}
      \label{equ: 正则化证明3}
      0 =&
      \f_*^{\pr\pr}(1)\eta\hp(1) - \f_*^{\pr\pr}(0)\eta\hp(0)\\
      &+\sum_{i=1}^{n-1} \left(
        -\frac{h_i+h_{i+1}}{2}[\tilde{y}_i - \f_*(x_i)]
        + \alpha(\f_*^{(3)}(x_i+) - \f_*^{(3)}(x_i-))
      \right)\eta(x_i) \\
      &+\alpha\sum_{i=1}^{n-1} ( \f_*^{\pr\pr}(x_i+) - \f_*^{\pr\pr}(x_i-))\eta\hp(x_i)
    \end{split}\end{equation}
    由于$\eta$的选取是任意的,即意味着$\eta(x_i)$,$(i=1,2,\dots,n-1)$和
    $\eta\hp(x_i)$,$(i=0,1,\dots,n)$的选取是任意的,所以需要选取适当的$\f$才可以使得
    \equref{equ: 正则化证明3}中的每一项都恒为零. 为使第$1$、$3$项为零,需要满足
    \[\begin{split}
      & \f_*^{\pr\pr}(1) = \f^{\pr\pr}(0) = 0,\\
      & \f_*^{\pr\pr}(x_i+) = \f_*^{\pr\pr}(x_i-).
    \end{split}\]
    同样的,根据不同的$\alpha$,选取恰当的$\f(x_i)$或
    $\f^{(3)}(x_i)$使得第二项为零.
    综合上式以及\equref{equ: 正则化证明2},可知$\f_*$为三次样条函数,
    且当$\alpha=0$时为样条插值函数. $\blacksquare$


\newpage
\input{05-NumericalIntegralAndDiff.tex}

\newpage
\section{非线性方程求根}
\subsection{二分法}
  \begin{thm}[闭区间套定理]
    设$\{[a_n, b_n]\}$满足
    \begin{enumerate}
      \item $[a_n, b_n] \subset [a_{n-1}, b_{n-1}]$,
      \item 当$n\to\infty$时,$|b_n - a_n|\to 0$.
    \end{enumerate}
    则存在$\xi$成立
    \[
      \bigcap_{k=0}^{\infty}[a_n, b_n] = \{\xi\}.
    \]
  \end{thm}

  \begin{thm}[连续函数零点定理]
    \label{thm: 连续函数零点定理}
    设$\f\in\ms{C}[a, b]$且$\f(a)\f(b)<0$,则存在$c\in[a, b]$,
    成立$\f(c) = 0$.
  \end{thm}
  \remark
    这一定理是多种方程求根方法的基础,下给出构造性的证明,这一证明
    本身实际上描述了二分法求根的过程.
  \proof
    构造如下闭区间套,令$[a_0, b_0]=[a, b]$,$c_n=(a_n+b_n)/2$,
    如果$\f(c_n)=0$,则结论成立,否则$c_n$至少与$a_n$和$b_n$中的
    一个异号,取该半个区间为$[a_{n+1}, b_{n+1}]$. 由上述构造可知
    \[
      [a_{n+1}, b_{n+1}]\subset[a_n,b_n],\quad
      |b_n - a_n| = \frac{b-a}{2^n}\to 0.
    \]
    从而存在$\xi\in\bigcap_{k=0}^\infty[a_n, b_n]$.
    若$\f(\xi)\ne 0$,不妨设$\f(\xi) = r>0$,则存在$\delta>0$,
    在$[\xi-\delta,\xi+\delta]$上$\f(x)>0$恒成立. 取足够大的$n$,
    即可使$[a_n,b_n]\subset[\xi-\delta,\xi+\delta]$,其中
    $\f(a_n)\f(b_n)<0$,与$\f$在$[\xi-\delta,\xi+\delta]$保号
    矛盾,从而$\f(\xi)=0$. $\blacksquare$

  \begin{alg}[二分法求解零点]
    对区间$n$等分,对每个满足端点函数值异号的区间$[a_k,a_{k+1}]$,
    按照\thmref{thm: 连续函数零点定理}证明中的方法二分,直到满足
    $b_n-a_n<\vep$,即与零点误差小于$\vep$为止.
  \end{alg}
  \remark
    二分法实现简单,但是在高维情况下,由于不再有“区间端点”的概念,
    所以难以推广.

\subsection{不动点法}
  \begin{defi}[压缩映射]
    设$\f$将$E\subset\R$映射到$E$上. 称$\f$为压缩映射,
    若存在常数$l<1$,对任意$x,y\in E$成立
    \[
      |\f(x) - \f(y)| \le l|x-y|.
    \]
  \end{defi}

  \begin{thm}[压缩映射定理]
    \label{thm: 压缩映射定理}
    设$\varphi$是$[a,b]$上的连续压缩映射,则存在唯一的$x\in[a, b]$,
    成立$\varphi(x) = x$.
  \end{thm}
  \proof
    对于唯一性. 设成立$x_1=\varphi(x_1)$,$x_2=\varphi(x_2)$,则
    \[
      |x_1-x_2| = |\varphi(x_1)-\varphi(x_2)|\le l|x_1-x_2|
      \quad\Rightarrow\quad x_1=x_2.
    \]
    对于存在性. 令$F(x)=x-\varphi(x)\in\ms{C}[a, b]$. 若
    $a=\varphi(a)$或$\varphi(b)$,则得证. 否则由于$\f$映射到$[a,b]$
    自身,所以成立$a<\varphi(a), \varphi(b)<b$. 则成立
    $F(a)F(b)<0$,由\thmref{thm: 连续函数零点定理}可知,存在$\xi\in[a,b]$,
    成立$F(\xi)=0$,即$\xi=\varphi(\xi)$. $\blacksquare$
  \remark
    此定理对于任意的完备度量空间都是成立的,且条件中的连续性要求可以略去,
    证明方法是构造迭代数列$x_{n+1}=\varphi(x_n)$.

  \begin{thm}[余项估计]
    设$\varphi$是$[a, b]$上压缩常数为$l$的连续压缩映射. 则数列
    \[
      x_0 = a,\quad x_{n+1} = \varphi(x_n)
    \]
    收敛至$x_*$. 且有估计式
    \[
      |x_n - x_*| \le \frac{l^n}{1-l}|x_1-x_0|.
    \]
  \end{thm}
  \proof
    下证明$\{x_n\}$为Cauchy序列.
    \[
      |x_{k+1}-x_k| = |\varphi(x_k) - \varphi(x_{k-1})|
      \le l|x_k-x_{k-1}| \le \cdots \le l^k|x_1-x_0|.
    \]
    所以对于任意的$n$和$p>0$,成立
    \[
      |x_{n+p}-x_n|\le \sum_{k=0}^{p-1} |x_{n+k+1}-x_{n+k}|
      \le |x_1-x_0|\sum_{k=0}^{p-1}l^{n+k}.
    \]
    由于$l>0$,根据几何级数的性质,成立
    \[
      |x_{n+p}-x_n| \le \frac{l^n}{1-l}|x_1-x_0|.
    \]
    当$n\to\infty$时,$\rhs\to 0$,从而$\{x_n\}$是Cauchy序列,
    所以收敛. 令$p\to\infty$,即得估计式
    \[
      |x_n-x_*| \le \frac{l^n}{1-l}|x_1-x_0|.
      \quad\blacksquare
    \]

  \begin{defi}[局部收敛]
    设$\varphi(x)$有不动点$x_*$,则对于迭代法
    \begin{equation}
      \label{equ: 迭代法}
      x_{n+1}=\varphi(x_n).
    \end{equation}
    如果存在$x_*$的某个领域$O_\delta(x)$,使得任意$x_0\in O_\delta(x)$,
    \equref{equ: 迭代法}收敛至$x_*$,则称\equref{equ: 迭代法}局部收敛.
  \end{defi}

  \begin{thm}[局部收敛的条件]
    设$x_*$是$\varphi\in\ms{C}^1$的不动点,且$|\varphi\hp(x_*)|<1$,
    则\equref{equ: 迭代法}在$x_*$处局部收敛.
  \end{thm}
  \proof
    由于$\varphi\hp$是连续的,所以存在$O_\delta(x_*)$,对任意的$x\in
    O_\delta(x_*)$,成立
    \[
      |\varphi\hp(x)| < l < 1.
    \]
    从而对于任意的$x,y\in O_\delta(x_*)$,根据中值定理,成立
    \[
      |\varphi(x) - \varphi(y)| = |\varphi\hp(\xi)||x-y| < l|x-y|.
    \]
    即$\varphi(x)$在$O_\delta(x_*)$上是压缩映射,从而对任意$x_0\in
    O_\delta(x_*)$,\equref{equ: 迭代法}收敛至$x_*$. $\blacksquare$

  \begin{defi}[$p$阶收敛]
    设$x_*$是$\varphi(x)$的不动点,记迭代误差$e_k=x_k-x_*$,若
    当$k\to\infty$时,成立
    \[
      \frac{e_{k+1}}{e_k^p} \to C \ne 0.
    \]
    则称\equref{equ: 迭代法}$p$阶收敛.
  \end{defi}
  \remark
    此定义描述了迭代式收敛的速度.

  \begin{thm}[$p$阶收敛条件]
    设$x_*$是迭代过程$x_{n+1}=\varphi(x_n)$的不动点,若对正整数$p$,
    $\varphi^{(p)}$在$x_*$附加连续,且成立
    \[\begin{split}
      \varphi\hp(x_*) = \cdots = \varphi^{(p-1)}(x_*) = 0,\quad
      \varphi^{(p)}(x_*) \ne 0.
    \end{split}\]
    则\equref{equ: 迭代法}在$x_*$附加$p$阶收敛.
  \end{thm}
  \proof
    在$x_*$处将$\varphi$ Taylor展开即可.

  \begin{alg}[不动点法]
    对于方程$\f(x) = 0$,将其变形成等价的$x=\varphi(x)$的形式,
    且$\varphi$满足在零点处局部收敛,则可以利用迭代数列
    $x_{n+1} = \varphi(x_n)$来求解方程的根.
  \end{alg}

\subsection{Newton法}
  


\newpage
\section{附录}
\subsection{不等式}

  \begin{lemma}[排序不等式]
    \label{lemma: 排序不等式}
    对于满足下述条件的$\{a_n\}$,$\{b_n\}$,
    \[\begin{split}
      & 0 \le a_1\le a_2\le\cdots\le a_n \\
      & 0 \le b_1\le b_2\le\cdots\le b_n
    \end{split}\]
    则同序相乘求和值最大,逆序最小,即
    \[
      \sum_{i=1}^n a_ib_i \ge \sum_{i=1}^n a_ib_{k_i}
      \ge \sum_{i=1}^n a_ib_{n-i+1}
    \]
  \end{lemma}

  \begin{lemma}[算数-几何均值不等式]
    \[
      (a_1a_2\cdots a_n)^{1/n} \le \frac{a_1+a_2+\cdots+a_n}{n}
    \]
    当且仅当$a_1 = a_2 = \cdots = a_n$时等号成立.
  \end{lemma}
  \proof
    因为有齐次性,所以不妨设$\prod a_i=1$,并令
    \[
      a_1=\frac{\alpha_1}{\alpha_2},\quad
      \dots,\quad
      a_{n-1} = \frac{\alpha_{n-1}}{\alpha_n},\quad
      a_n = \frac{\alpha_n}{\alpha_1}
    \]
    则只需证明下式即可.
    \[
      \frac{\alpha_1}{\alpha_2} + \cdots + \frac{\alpha_n}{\alpha_1}
      \ge n
    \]
    不妨设$\alpha_1 \le \alpha_2 \le \cdots \le \alpha_n$,则根据排序不等式
    \[
      \lhs \ge \alpha_1\frac{1}{\alpha_1} + \cdots + \alpha_n\frac{1}{\alpha_n}
       = n \quad\blacksquare
    \]

\newpage
\subsection{积分相关公式}
  \begin{lemma}[分部积分]
    设$u,v\in\ms{C}^{n+1}[a, b]$,则成立
    \[
      \int_a^buv^{(n+1)}\rd x =
      [ uv^{(n)} - u\hp v^{(n-1)} + \cdots +  (-1)^nu^{(n)}v]
      \bigg\vert_a^b + (-1)^{n+1}\int_a^bu^{(n+1)}v\rd x.
    \]
  \end{lemma}

\newpage
\subsection{Euler-Maclaurin公式}
  \paragraph{绪论}
   此节内容的主要是对 Apostol, T. M. (1 May 1999).
   "An Elementary View of Euler's Summation Formula"
   的翻译和整理.

  \subsubsection{广义Euler常数}
    \paragraph{绪论}
      本章节仅考虑在$[1,\infty)$上满足$\f>0$,且严格单调递减的函数.

    \begin{defi}[$d_n$]
      定义$d_n$为用积分近似离散求和的误差,即
      \begin{equation}
        \label{equ: d_n}
        d_n = \sum_{k=1}^{n-1}\f(k) - \int_1^n\f(x)\rd x
      \end{equation}
    \end{defi}
    \remark
      注意,$d_n$的定义中,$k$仅遍历$[1, n-1]$,而不包含$n$.
      \begin{figure}[htbp]
        \centering
        \includegraphics[width=15cm]{../image/d_n.png}
        \caption{$d_n$的几何解释}
        \label{fig: d_n的几何解释}
      \end{figure}
      \figref{fig: d_n的几何解释}为$d_n$的几何解释,其中
      在曲线上方的阴影部分即为$d_n$,可以将它们统一移动到图像的最左侧.

    \begin{defi}[广义Euler常数]
      已知$0<d_n<d_{n+1}<\f(1)$,所以可以定义关于$\f$的广义Euler常数
      $C(\f)$为
      \[
        C(\f) = \lim_{n\to\infty}d(n)
      \]
    \end{defi}

    \begin{pos}[余项估计]
      $0 < C(\f) - d_n < \f(n)$.
    \end{pos}

    \begin{thm}
      设$\f$在$[1,\infty)$上为正且严格单调递减,则存在一列
      $\{E_{\f}(n)\}$,满足$0<E_{\f}(n) < \f(n)$,成立
      \begin{equation}
        \sum_{k=1}^n\f(k) = \int_1^n\f\rd x + C(\f) + E_{\f}(n),
        \quad n = 2,3,\dots.
      \end{equation}
    \end{thm}
    \remark
      只需令$\lhs = \sum_{k=1}^{n-1}\f(k) + \f(n)$,在根据$d_n$
      的定义即可证明. 这一定理意味着求和与积分之间的误差仅仅为一个与
      $\f$有关的常数以及一个比$\f(n)$还要小的正数. 从而如果$\f(n)\to 0$,
      则$E_{\f}(n)\to 0$,此时成立
      \begin{equation}
        C(\f) = \lim_{n\to\infty}\left(
          \sum_{k=1}^n \f(k) - \int_1^n\f(x)\rd x
        \right).
      \end{equation}

    \begin{defi}[经典Euler常数]
      取$\f(x) = 1/x$,它满足$\lim_{x\to\infty}\f(x) = 0$,则有
      \[
        \gamma = C(\frac{1}{x}) = \lim_{n\to\infty}\left(
          \sum_{k=1}^n\frac{1}{k} - \ln n
        \right).
      \]
    \end{defi}

  \subsubsection{Euler求和公式}
    \paragraph{绪论}
      在此章节不再限制$\f$需要为正且单调减,在目前仅要求$\f\in\ms{R}[1, n]$
      (在之后会添加诸如连续可导等条件). 在此基础上重新定义$d_n$(与上一节保持一致)
      并推导出类似于上一节中的结论.

    \begin{defi}[$d_n$]
      定义$d_n$为
      \begin{equation}
        d_n = \sum_{k=1}^{n-1}I(k) =
        \sum_{k=1}^{n-1} \int_{k}^{k+1}\{ \f(k) - \f(x) \}\rd x.
      \end{equation}
    \end{defi}
    \remark
      可以发现这一定义和上一节中的定义,在$\f>0$且单调减的情况下是一致的.
      实际上这一定义只是将每一点处的差值积分起来而已.

    \begin{pos}
      对于任意常数$c$,由于成立$\rd x= \rd(x+c)$,所以可以取$c=-(k+1)$,并分部
      积分,则有
      \[
        I(k) = \int_{k}^{k+1} (x-[x]-1)\f\hp(x)\rd x.
      \]
      从而
      \[
        d_n = \int_1^n(x-[x])\f\hp(x)\rd x + \f(1) - \f(n).
      \]
    \end{pos}

    \begin{thm}[一次导数形式的Euler求和公式]
      对于任意$\f\in\ms{C}^1[1,n]$,成立
      \begin{equation}
        \label{equ: 一次导数Euler求和公式}
        \sum_{k=1}^n\f(k) = \int_1^n\f(x)\rd x +
        \int_1^n(x-[x])\f\hp(x)\rd x + \f(1).
      \end{equation}
    \end{thm}
    \remark
      对于误差项中的$\f\hp$,如果它不变号,即$\f$单调的情况下,
      可以考虑让它乘一个变号的因数,从而使得误差更小. 例如,可以
      用$x-[x]-\frac{1}{2}$来代替$x-[x]$. 则有了如下推论.

    \begin{cor}
      \label{cor: 一次Euler求和公式}
      对于任意$\f\in\ms{C}^1[1,n]$,定义一次Bernoulli函数为
      \[
        P_1(x) =
        \begin{cases}
          x - [x] - \frac{1}{2},\quad& x \notin \mbf{N} \\
          0, & x\in\mbf{N}.
        \end{cases}
      \]
      则可以重写\equref{equ: 一次导数Euler求和公式}为
      \begin{equation}
        \sum_{k=1}^n\f(k) = \int_1^n\f(x)\rd x +
        \int_1^nP_1(x)\f\hp(x)\rd x + \frac{1}{2}(\f(n) + \f(1)).
      \end{equation}
      若假设$\int_1^{\infty}P_1\f\rd x$存在,那么上式可以继续写为
      \[
        \begin{split}
        \sum_{k=1}^n\f(k) &= \int_1^n\f(x)\rd x + C(\f) + E_{\f}(n) \\
        C(\f) &= \frac{1}{2}\f(1) + \int_1^{\infty}P_1\f\hp\rd x\\
        E_{\f}(n) &= \frac{1}{2}\f(n) - \int_n^{\infty}P_1\f\hp\rd x
      \end{split}\]
    \end{cor}
    \remark
      注意,在此处$C(\f)$和$E_{\f}(n)$的定义和之前依然是一致的. 并且只需要
      $\f(x)$在$[1, \infty)$上反常绝对可积,就可以保证
      $\int_1^\infty P_1\f\rd x$的存在性.\footnote{我暂时并没有明白,是
      什么条件保证了$\f(n)\to 0$,从而$C(\f) = \lim_{n\to\infty}d_n$存在.}

    \begin{cor}[Euler常数]
      将$\f(x) = 1/x$,代入\corref{cor: 一次Euler求和公式},得
      经典Euler常数$\gamma$为
      \[
        \gamma = \frac{1}{2} = \int_1^\infty \frac{P_1(x)}{x}\rd x.
      \]
    \end{cor}

  \subsubsection{余项分析}
    \paragraph{绪论}
      这一章节通过对
      \begin{equation}
        \label{equ: Euler求和公式余项1}
        \int_n^{n+1}P_1(x)\f(x)\rd x
      \end{equation}
      分部积分来进一步分析余项,并为此引入了更高次的Bernoulli函数.

    \begin{defi}[二次Bernoulli函数]
      定义二次Bernoulli函数$P_2(x)$为
      \[
        P_2(x) = 2\int_0^xP_1(t)\rd t + \frac{1}{6}.
      \]
    \end{defi}
    \remark
      如此定义二次Bernoulli函数的动机来自于对
      \equref{equ: Euler求和公式余项1}的进一步分部积分. 这样就需要有
      \[
        P_2(x) = 2\int_0^xP_1(t)\rd t + c
      \]
      之所以要有系数$2$是为了之后公式的简洁. 由于$P_1(x)$有周期性,从而
      \[
        P_2(x+1) - P_2(x) = 2\int_x^{x+1}P_1(t)\rd t = 0,
      \]
      即$P_2(x)$也有周期性. 为了让按照同样方式定义的$P_3$也具有周期性,
      所以希望有
      \[
        \int_0^1P_2(x)\rd x= 0 \,\Rightarrow\, c = \frac{1}{6}.
        \quad\blacksquare
      \]

    \begin{thm}[二次导数形式的Euler求和公式]
      设$\f\in\ms{C}^2[1, n]$,成立
      \[
        \sum_{k=1}^n\f(k) = \int_1^n\f\rd x
        - \frac{1}{2}\int_1^nP_2\f^{\pr\pr}\rd x
        + \frac{1}{2}P_2(0)(\f\hp(n)-\f\hp(1))
        + \frac{1}{2}(\f(n) + \f(1)).
      \]
      并且,如果$\f^{\pr\pr}$在$[1, \infty)$上反常绝对可积,则有
      \[\begin{split}
        \sum_{k=1}^n\f(k) &= \int_1^n\f\rd x + C(\f) + E_{\f}(n) \\
        C(\f) &= \frac{1}{2}\f(1) - \frac{1}{2}P_2(0)\f\hp(1)
        - \frac{1}{2}\int_1^\infty P_2\f^{\pr\pr}\rd x \\
        E_{\f}(n) = \frac{1}{2}\f(n) + \frac{1}{2}P_2(0)\f\hp(n) +
        \frac{1}{2}\int_n^\infty P_2\f^{\pr\pr}\rd x.
      \end{split}\]
    \end{thm}

  \subsubsection{Bernoulli数和Euler求和公式的一般形式}
    \begin{defi}
      定义\tbf{Bernoulli周期函数}为
      \begin{equation}
          P_k(x) = k\int_0^x P_{k-1}(t)\rd t + B_k,\quad k\ge 2
      \end{equation}
      其中$B_k$称为\tbf{Bernoulli数},它使得下式成立
      \[
        \int_0^1P_k(x)\rd x = 0.
      \]
      显然$P_k(x)$在$k\ge 2$的时候在$[0, 1]$上是$k$次多项式,称
      它为\tbf{Bernoulli多项式}.
    \end{defi}

    \begin{thm}[等价定义]
      Bernoulli数和Bernoulli多项式的另一种常见定义为,
      \[\begin{split}
        & B_0 = 1,\quad \sum_{k=0}^{m+1} {{m+1}\choose{k}} B_k = 0 \\
        & B_n(x) = \sum_{k=0}^n{{n}\choose{k}}B_{n-k}x^k.
      \end{split}\]
    \end{thm}
    \remark
      对于大于$1$的奇数$k$,$B_k=0$,且有
      \[
        |B_{2k}(x)| \le |B_{2k}|,\quad
        |B_{2k+1}(x)| \le (2k+1)|B_{2k}|.
      \]

    \begin{thm}[Euler求和公式的一般形式]
      设$\f\in\ms{C}^{2m+1}[1, n]$,成立
      \[\begin{split}
        \sum_{k=1}^n\f(k) = & \int_1^n\f\rd x +
        \frac{1}{(2m+1)!}\int_1^n P_{2m+1}(x)\f^{(2m+1)}(x)\rd x\\
        & + \sum_{r=1}^m \frac{B_{2r}}{(2r)!}(\f^{(2r-1)}(n)-\f^{(2r-1)}(1))
        + \frac{1}{2}(\f(1) + \f(n)).
      \end{split}\]
      且若$\f^{(2m+1)}$在$[1, \infty)$上反常绝对可积,则成立
      \[\begin{split}
      \sum_{k=1}^n\f(k) &= \int_1^n\f\rd x + C(\f) + E_{\f}(n) \\
      C(\f) &= \frac{1}{2}\f(1) - \sum_{r=1}^m\frac{B_{2r}}{(2r)!}\f^{(2r-1)}(1)
      + \frac{1}{(2m+1)!}\int_1^{\infty}P_{2m+1}\f^{(2m+1)}\rd x \\
      E_{\f}(n) &= \frac{1}{2}\f(n) + \sum_{r=1}^m\frac{B_{2r}}{(2r)!}\f^{(2r-1)}(n)
      - \frac{1}{(2m+1)!}\int_n^{\infty}P_{2m+1}\f^{(2m+1)}\rd x
      \end{split}\]
    \end{thm}


\end{document}
