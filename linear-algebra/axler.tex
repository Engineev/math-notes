\section{Linear Algebra Done Right}

\setcounter{subsection}{7}
\subsection{Operators on Complex Vector Spaces}
  \paragraph{p243. 8.4}
    这一定理表明,如果$T^{n+k}u=0$,则$T^nu=0$,其中$k\ge 0$.
  % end

  \paragraph{p243. 8.5}
    一般而言,$V=\nul T\oplus\range T$是不成立的,注意到虽然$\dim V=\dim\nul T+\dim
    \range T$始终成立,但是$\nul T$和$\range T$之间可能有非零的重合。所以实际上这一定理的
    内容主要是$\nul T^n\cap \range T^n = \{0\}$.
  % end

  \paragraph{p247. The proof of 8.13}
    注意有$(T-\lambda_1I)(T-\lambda_2I)=((T-\lambda_2I)(T-\lambda_1I))$. 我们的思
    路在于逐个证明$a_j=0$. 要做到这一点,首先要去掉其他的项并保留当前$a_j$项,要消去其他项,
    只需利用$v_k\in G(\lambda, T)=\nul(T-\lambda_kI)^n$,即利用$(T-\lambda_2I)^n
    \cdots(T-\lambda_mI)^n$即可. 但是这样仅能得到
    \[
      0 = a_1(T-\lambda_2I)^n\cdots(T-\lambda_mI)^nv_1.
    \]
    由于我们不能确保上式右侧的,不考虑系数的向量最后不为零,所以不能直接得出$a_1=0$. 因此我们
    希望给$v_1$再作用一个$(T-pI)^q$形式的线性算子(以确保仍可交换)使得它一定不为零。在此注意
    到如果$w$是$T$的非广义特征向量,则可以满足条件,同时我们有$(T-\lambda_1I)^kv_1$是$T$的
    一个非广义特征向量。
  % end

% end