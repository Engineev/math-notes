\section{Multilinear Algebra and Applications}

\subsection{Introduction}
  \paragraph{换基公式}
    设$\mathcal{B}$和$\wtilde{\mathcal{B}}$为两组基而$\wtilde{\mathcal{B}}=
    \mathcal{B}L_{\wtilde{\mathcal{B}}\mathcal{B}}$. 则$[v]_{\wtilde{
    \mathcal{B}}}=L\inv[v]_{\mathcal{B}}$.
  % end
% end

\subsection{Review of Linear Algebra}
  \paragraph{换基公式}
    设$L_{\wtilde{\mathcal{B}}\mathcal{B}}=[L^i_j]$且$\Lambda=L\inv$. 则
    $\tilde{b}_j=L^i_jb_i$且$b_j=\Lambda_j^i\tilde{b}_i$. 设线性变换$T$在这两组基下
    的矩阵分别为$A$和$\wtilde{A}$,则有$\wtilde{A}=L\inv AL$. 
  % end
% end

\subsection{Multilinear Forms}
  \paragraph{p25. }
    For every $\alpha\in V^*$, $\alpha=\alpha(b_i)\beta^i$.
  % end

  \paragraph{p25. 换基公式(线性泛函)}
    设$\alpha\in V^*$且$\wtilde{b}_j = b_iL^i_j$,则
    $[\alpha]_{\wtilde{\mathcal{B}}^*}=[\alpha]_{\mathcal{B}^*}L$. 即线性泛函的坐
    标为covariant的.
  % end

  \paragraph{p30.}
    $\,$\\
    \begin{tabular}{|l|l|}
      \hline
      covariance of a basis                    
      & contravariance of the dual basis  \\ \hline
      contravariance of the coordinate vectors 
      & covariance of linear forms \\ \hline
    \end{tabular}
  % end
  
  \paragraph{p34. 换基公式(二次型)}
    $\tilde{B} = L^TBL$.
  % end
% end



