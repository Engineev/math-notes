\section{Multilinear Algebra and Applications}

\subsection{Introduction}
  \paragraph{换基公式}
    设$\mathcal{B}$和$\wtilde{\mathcal{B}}$为两组基而$\wtilde{\mathcal{B}}=
    \mathcal{B}L_{\wtilde{\mathcal{B}}\mathcal{B}}$. 则$[v]_{\wtilde{
    \mathcal{B}}}=L\inv[v]_{\mathcal{B}}$.
  % end
% end

\subsection{Review of Linear Algebra}
  \paragraph{换基公式}
    设$L_{\wtilde{\mathcal{B}}\mathcal{B}}=[L^i_j]$且$\Lambda=L\inv$. 则
    $\tilde{b}_j=L^i_jb_i$且$b_j=\Lambda_j^i\tilde{b}_i$. 设线性变换$T$在这两组基下
    的矩阵分别为$A$和$\wtilde{A}$,则有$\wtilde{A}=L\inv AL$. 
  % end
% end

\subsection{Multilinear Forms}
  \paragraph{p25. }
    For every $\alpha\in V^*$, $\alpha=\alpha(b_i)\beta^i$.
  % end

  \paragraph{p25. 换基公式(线性泛函)}
    设$\alpha\in V^*$且$\wtilde{b}_j = b_iL^i_j$,则
    $[\alpha]_{\wtilde{\mathcal{B}}^*}=[\alpha]_{\mathcal{B}^*}L$. 即线性泛函的坐
    标为covariant的.
  % end

  \paragraph{p30.}
    $\,$\\
    \begin{tabular}{|l|l|}
      \hline
      covariance of a basis                    
      & contravariance of the dual basis  \\ \hline
      contravariance of the coordinate vectors 
      & covariance of linear forms \\ \hline
    \end{tabular}
  % end
  
  \paragraph{p34. 换基公式(双线性)}
    $\tilde{B} = L^TBL$.
  % end
% end

\subsection{Inner Products}
  \paragraph{p41. 双线性,二次型,内积}
    \begin{enumerate}
      \item 双线性:$\varphi:V\times V\to\mathbb{R}$,对于各分量线性。
      \item 内积:$g:V\times V\to\mathbb{R}$,满足对称性和正定性的双线性函数。
      \item 二次型:$\mathbb{R}^n\to\mathbb{R}$,对于双线性取$V=\mathbb{R}^n$,同时
      取相同的第一第二分量,即$\varphi(v,v)$. 可证明,任意二次型都为$v^TSv$的形式。通常
      要求$S$为对称阵,即要求原双线性函数为对称的。
    \end{enumerate}
  % end

  \paragraph{p42. 内积与标准正交基}
    设$\mathcal{B}\subset V$是内积$g$下的一组标准正交基,则$g$在$B$下的矩阵为单位阵。由于
    标准正交基$\mathcal{B}$是一定存在的,所以任意内积都有形式
    \[
      g(u,v) = [u]_{\mathcal{B}}^T [v]_{\mathcal{B}} = v^iw_i
    \]
  % end

  \paragraph{p48. Reciprocal Basis}
    Let $g:V\times V\to\mathbb{R}$ be an inner product, $\mathcal{B}=\{b_1,
    \dots,b_n\}$ a basis of $V$ and $\mathcal{B}^g=\{b^1,\dots,b^n\}$ the
    reciprocal basis of $\mathcal{B}$. Put $G=[g]_{\mathcal{B}}=[g_{ij}]$ and 
    $M=[g]_{\mathcal{B}^g}=[g^{ij}]$. Then
    \begin{enumerate}
      \item $MG=I$.
      \item $\mathcal{B}^g=\mathcal{B}M$ and \textit{vice versa}. Namely, 
            $M=L_{\mathcal{B}^g\mathcal{B}}$.
      \item The reciprocal basis is contravariant and the vector coordinates 
            w.r.t to the reciprocal basis if covariant. Namely, $[v]_{\wtilde{
            \mathcal{B}}^g}^T=[v]_{\mathcal{B}^g}^TL_{\wtilde{\mathcal{B}}
            \mathcal{B}}$.
      \item (Change of basis to the reciprocal basis) TODO
    \end{enumerate}
  % end
% end

% end

