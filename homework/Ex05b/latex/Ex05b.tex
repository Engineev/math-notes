\documentclass[12pt, a4paper]{article}
\usepackage{ctex}

\usepackage[margin=1in]{geometry}
\usepackage{
  color,
  clrscode,
  amssymb,
  ntheorem,
  amsmath,
  listings,
  fontspec,
  xcolor,
  supertabular,
  multirow,
  mathtools,
  mathrsfs
}
\definecolor{bgGray}{RGB}{36, 36, 36}
\usepackage[
  colorlinks,
  linkcolor=bgGray,
  anchorcolor=blue,
  citecolor=green
]{hyperref}
\newfontfamily\courier{Courier}

\theoremstyle{margin}
\theorembodyfont{\normalfont}
\newtheorem{thm}{定理}
\newtheorem{cor}[thm]{推论}
\newtheorem{pos}[thm]{命题}
\newtheorem{lemma}[thm]{引理}
\newtheorem{defi}[thm]{定义}
\newtheorem{std}[thm]{标准}
\newtheorem{imp}[thm]{实现}
\newtheorem{alg}[thm]{算法}
\newtheorem{exa}[thm]{例}
\newtheorem{prob}[thm]{问题}
\DeclareMathOperator{\sft}{E}
\DeclareMathOperator{\idt}{I}
\DeclareMathOperator{\spn}{span}
\DeclareMathOperator*{\agm}{arg\,min}
\newcommand{\pr}{\prime}
\newcommand{\tr}{^\intercal}
\newcommand{\st}{\text{s.t.}}
\newcommand{\hp}{^\prime}
\newcommand{\ms}{\mathscr}
\newcommand{\mn}{\mathnormal}
\newcommand{\tbf}{\textbf}
\newcommand{\mbf}{\mathbf}
\newcommand{\fl}{\mathnormal{fl}}
\newcommand{\f}{\mathnormal{f}}
\newcommand{\g}{\mathnormal{g}}
\newcommand{\R}{\mathbf{R}}
\newcommand{\Q}{\mathbf{Q}}
\newcommand{\JD}{\textbf{D}}
\newcommand{\rd}{\mathrm{d}}
\newcommand{\str}{^*}
\newcommand{\vep}{\varepsilon}
\newcommand{\lhs}{\text{L.H.S}}
\newcommand{\rhs}{\text{R.H.S}}
\newcommand{\con}{\text{Const}}
\newcommand{\oneton}{1,\,2,\,\dots,\,n}
\newcommand{\aoneton}{a_1a_2\dots a_n}
\newcommand{\xoneton}{x_1,\,x_2,\,\dots,\,x_n}
\newcommand\thmref[1]{定理~\ref{#1}}
\newcommand\lemmaref[1]{引理~\ref{#1}}
\newcommand\defref[1]{定义~\ref{#1}}
\newcommand\posref[1]{命题~\ref{#1}}
\newcommand\secref[1]{节~\ref{#1}}
\newcommand\equref[1]{(\ref{#1})}
\newcommand\figref[1]{图 \ref{#1}}
\newcommand\corref[1]{推论~\ref{#1}}
\newcommand\exaref[1]{例~\ref{#1}}
\newcommand\algref[1]{算法~\ref{#1}}
\newcommand{\remark}{\paragraph{评注}}
\newcommand{\example}{\paragraph{例}}
\newcommand{\proof}{\paragraph{证明}}


\title{科学计算作业$\,$练习$5b$}
\author{\small 任云玮\\\small2016级ACM班\\\small516030910586}
\date{}

\begin{document}
\lstset{
  numbers=left,
  basicstyle=\scriptsize,
  numberstyle=\tiny\color{red!89!green!36!blue!36},
  language=Matlab,
  breaklines=true,
  keywordstyle=\color{blue!70},commentstyle=\color{red!50!green!50!blue!50},
  morekeywords={},
  stringstyle=\color{purple},
  frame=shadowbox,
  rulesepcolor=\color{red!20!green!20!blue!20}
}
\maketitle

\begin{lemma}
  \label{lemma: 变分引理2}
  设$\f\in\ms{C}[a, b]$,且对于任意满足$\eta(a) = \eta(b) = 0$的
  $\eta\in\ms{C}^1(a, b)$都成立
  \[
    \int_a^b\f\eta\hp\rd x = 0,
  \]
  则在$[a, b]$上成立$\f\equiv \text{Const}$.
\end{lemma}
\paragraph{证明}
  由于$\f\in\ms{C}[a, b]$,所以$\f\in\ms{R}[a, b]$,令$c\in\R$满足
  \[
    \int_a^b (\f(x)-c) \rd x = 0.
  \]
  同时令
  \[
    \eta(x) = \int_a^x \left( \f(t)-c \right)\rd t.
  \]
  显然$\eta$满足$\eta(a) = \eta(b) = 0$且$\eta\in\ms{C}^1(a, b)$.
  则根据条件,成立
  \[
    0 = \int_a^b(\f-c)\eta\hp\rd x = \int_a^b(\f-c)^2\rd x
    \,\Rightarrow\,
    \f \equiv c.\quad\blacksquare
  \]


\vspace{2cm}
\noindent1. 已知$\f\in\ms{C}[a, b]$,……
\proof
  假设存在$x_0\in(a, b)$,成立$\f(x_0)\ne0$,不妨设$\f(x_0)>0$.
  则由于$\f\in\ms{C}[a, b]$,所以存在足够小的$\delta>0$,使得
  \[
    \f(x) > 0,\quad x\in I = (x_0-\delta, x_0+\delta).
  \]
  不妨设$I = (-1, 1)$,否则只需另
  $t = -1 + \frac{1}{\delta}(x-x_0+\delta)$即可. 令$\g$
  为冲击函数
  \[
    \g(x) =
    \begin{cases}
      e^{\frac{1}{x^2-1}},\quad &x\in(-1, 1) \\
      0,                         & |x|\ge 1.
    \end{cases}
  \]
  可知$\g\in\ms{C}^\infty(\R)$,且在$(-1, 1)$外取值为零. 从而
  \[
    \int_a^b\f\g\rd x = \int_{-1}^{1}\f\g\rd x > 0.
  \]
  与已知条件矛盾,从而假设不成立,即有
  \[
    \f(x) = 0, \quad x\in[a, b].\quad\blacksquare
  \]

\vspace{1cm}
\par\noindent2. 证明如下变分问题无解……
\proof
  (反证)令
  \[
    J(\f) = \int_{-1}^{1}x^2\left( \f\hp(x)\right)^2\rd x.
  \]
  假设$J$有极值,设极值点为$\f_*$,对于任意$\eta\in\ms{C}_0^1[-1,1]$,
  成立
  \[\begin{split}
    0 = \frac{\rd}{\rd \vep}J(\f_*+\vep\eta)\bigg|_{\vep=0} = 2\int_{-1}^1 x^2\f_*\hp\eta\hp\rd x
  \end{split}\]
  又因为$x^2\f_*\hp\in\ms{C}[-1, 1]$,且$\eta$的选取是任意的,
  由\lemref{lemma: 变分引理2}可知,
  \[
    x^2\f_*\hp \equiv \text{Const}.
  \]
  令$x=0$,得$c = 0$,从而$\f_*\hp\equiv 0$,即$\f$为常值函数.
  与已知$\f(\pm 1) = \pm 1$矛盾,从而假设不成立,即该变分问题无解.
  $\blacksquare$


\vspace{1cm}
\par\noindent3. 对于求解最速降线问题……
\ans
  \[
    y(1+(y\hp)^2) = c\,\Rightarrow\,
    y\hp = \sqrt{\frac{c-y}{y}} \,\Rightarrow\,
    \sqrt{\frac{y}{c-y}}\rd y = \rd x.
  \]
  对两边积分,得
  \[\begin{split}
    x = \int_0^y \sqrt{\frac{t}{c-t}}\rd t
  \end{split}\]
  令$u = \sqrt{t/(c-t)}$,即$t=c(1-(1+u^2)^{-1})$,同时令$u=\tan v$,则
  \[\begin{split}
    x &= 2c\int_0^{\sqrt{\frac{y}{c-y}}} \frac{u^2}{(1+u^2)^2}\rd u \\
      &= 2c\int_0^{\arctan\sqrt{\frac{y}{c-y}}} \frac{\tan^2v}{1+\tan^2v}\frac{1}{\cos^2v}\rd v \\
      &= 2c\int_0^{\arctan\sqrt{\frac{y}{c-y}}} \sin^2v\rd v \\
      &= c\left( x - \frac{1}{2}\sin 2x \right)\bigg|_0^{\arctan\sqrt{y/(c-y)}}\\
      &= c\arctan\sqrt{\frac{y}{c-y}}
      - \frac{c}{2}\sin\left( 2\arctan\sqrt{\frac{y}{c-y}} \right).
  \end{split}\]
  其中$c$满足
  \[
    x_1 = c\arctan\sqrt{\frac{y_1}{c-y_1}}
    - \frac{c}{2}\sin\left( 2\arctan\sqrt{\frac{y_1}{c-y_1}} \right).\quad\blacksquare
  \]

\vspace{1cm}
\par\noindent4. 已知函数$y=\f(x)$在节点……
\ans
  不妨设在两端,近似值精确成立.
  设$h_i = x_i-x_{i-1}$,定义
  \[
    J_1(\f) = \sum_{i=1}^{n-1}\frac{h_i+h_{i+1}}{2}
    \left( \tilde{y}_i - \f(x_i) \right)^2, \quad
    J_2(\f) = \int_0^1 (\f\hp(x))^2\rd x,
  \]
  下最小化
  \[
    J(\f) = J_1(\f) + \alpha J_2(\f).
  \]
  设$\f_*$是所要求的解,则对于任意$\eta\in\ms{C}_0^2[0, 1]$,成立
  \begin{equation}\begin{split}
    \label{equ: 导数为零}
    0 &= \frac{\rd}{\rd\vep}J(\f_*+\vep\eta)\bigg|_{\vep=0} =
    -2\sum_{i=1}^{n-1}\frac{h_i+h_{i+1}}{2}[\tilde{y}_i-\f_*(x_i)]\eta(x_i)
    +2\alpha\int_0^1\f_*\hp\eta\hp\rd x.
  \end{split}\end{equation}
\tbf{Step I. }
  对于每一个区间$I_k = [x_{k-1}, x_k]$,
  取满足$\eta(x_{k-1}) = \eta(x_k) =0$的$\eta$,则对于每个区间
  分部积分得,
  \[
    0 = \int_{x_{i-1}}^{x_i}\f_*\hp\eta\hp\rd x
    = \f_*\hp\eta\bigg|_{x_{i-1}}^{x_i}
     - \int_{x_{i-1}}^{x_i}\f_*^{\pr\pr}\eta\rd x
     = - \int_{x_{i-1}}^{x_i}\f_*^{\pr\pr}\eta\rd x.
  \]
  根据变分引理,成立
  \[
    \label{equ: 线性}
    \f_*^{\pr\pr} = 0 \,\Rightarrow\,
    \f\in P_1,\quad x\in(x_{i-1}, x_i).
  \]
  从而在区间$I_k$上,成立
  \[
    \f_*\hp = c_k,\quad k = 1,2,\dots,n.
  \]
\tbf{Step II. }
  取$\eta\in\ms{C}^2_0[0, 1]$,成立
  \[\begin{split}
    \int_0^1 \f_*\hp\eta\hp \rd x &=
    \sum_{i=1}^{n-1} \left(
        \f_*\hp\eta\bigg|_{x_{i-1}}^{x_i}
    \right)\,\rd x = \sum_{i=1}^{n=1} \left(
      c_{i-1} - c_i
    \right)\eta(x_i)
  \end{split}\]
  代入\equref{equ: 导数为零},由于$\eta(x_i)$,$(i=1,\dots,n-1)$
  的选取是任意的,所以成立
  \[
    \alpha(c_{i-1}-c_i) - \frac{h_i+h_{i+1}}{2}[\tilde{y}_i - \f_*(x_i)] = 0
  \]
\tbf{结论 }
  综上,$\f_*$是一个分段线性的函数,设每段上斜率为$c$,则$\f_*$满足
  \[\begin{split}
    &\f_*(0) = y_0,\quad f_*(1) = y_1, \\
    &\alpha(c_{i-1} - c_i) - \frac{h_i+h_{i+1}}{2}[\tilde{y}_i-\f_*(x_i)] = 0,
    \quad i = 1,2,\dots,n-1\quad\blacksquare
  \end{split}\]

\end{document}
