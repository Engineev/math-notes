\documentclass[12pt, a4paper]{article}
\usepackage{ctex}

\usepackage[margin=1in]{geometry}
\usepackage{
  color,
  clrscode,
  amssymb,
  ntheorem,
  amsmath,
  listings,
  fontspec,
  xcolor,
  supertabular,
  multirow,
  mathtools,
  mathrsfs
}
\definecolor{bgGray}{RGB}{36, 36, 36}
\usepackage[
  colorlinks,
  linkcolor=bgGray,
  anchorcolor=blue,
  citecolor=green
]{hyperref}
\newfontfamily\courier{Courier}

\theoremstyle{margin}
\theorembodyfont{\normalfont}
\newtheorem{thm}{定理}
\newtheorem{cor}[thm]{推论}
\newtheorem{pos}[thm]{命题}
\newtheorem{lemma}[thm]{引理}
\newtheorem{defi}[thm]{定义}
\newtheorem{std}[thm]{标准}
\newtheorem{imp}[thm]{实现}
\newtheorem{alg}[thm]{算法}
\newtheorem{exa}[thm]{例}
\newtheorem{prob}[thm]{问题}
\DeclareMathOperator{\sft}{E}
\DeclareMathOperator{\idt}{I}
\DeclareMathOperator{\spn}{span}
\DeclareMathOperator*{\agm}{arg\,min}
\newcommand{\pr}{\prime}
\newcommand{\tr}{^\intercal}
\newcommand{\st}{\text{s.t.}}
\newcommand{\hp}{^\prime}
\newcommand{\ms}{\mathscr}
\newcommand{\mn}{\mathnormal}
\newcommand{\tbf}{\textbf}
\newcommand{\mbf}{\mathbf}
\newcommand{\fl}{\mathnormal{fl}}
\newcommand{\f}{\mathnormal{f}}
\newcommand{\g}{\mathnormal{g}}
\newcommand{\R}{\mathbf{R}}
\newcommand{\Q}{\mathbf{Q}}
\newcommand{\JD}{\textbf{D}}
\newcommand{\rd}{\mathrm{d}}
\newcommand{\str}{^*}
\newcommand{\vep}{\varepsilon}
\newcommand{\lhs}{\text{L.H.S}}
\newcommand{\rhs}{\text{R.H.S}}
\newcommand{\con}{\text{Const}}
\newcommand{\oneton}{1,\,2,\,\dots,\,n}
\newcommand{\aoneton}{a_1a_2\dots a_n}
\newcommand{\xoneton}{x_1,\,x_2,\,\dots,\,x_n}
\newcommand\thmref[1]{定理~\ref{#1}}
\newcommand\lemmaref[1]{引理~\ref{#1}}
\newcommand\defref[1]{定义~\ref{#1}}
\newcommand\posref[1]{命题~\ref{#1}}
\newcommand\secref[1]{节~\ref{#1}}
\newcommand\equref[1]{(\ref{#1})}
\newcommand\figref[1]{图 \ref{#1}}
\newcommand\corref[1]{推论~\ref{#1}}
\newcommand\exaref[1]{例~\ref{#1}}
\newcommand\algref[1]{算法~\ref{#1}}
\newcommand{\remark}{\paragraph{评注}}
\newcommand{\example}{\paragraph{例}}
\newcommand{\proof}{\paragraph{证明}}


\title{科学计算作业$\,$练习$1$}
\author{\small 任云玮\\\small2016级ACM班\\\small516030910586}
\date{}

\begin{document}
\maketitle

\noindent1、$\sqrt{7}$可由下列迭代算法计算……
\ans
  下证引理
  \begin{lemma}
    \label{lemma1}
    对于任意正数$x$,设$x_1>\sqrt{x}$,对于$n>1$,满足
    \begin{equation}
      \label{equ1}
      x_{n+1} = \frac12(x_n + \frac{x}{x_n}),
    \end{equation}
    则$\lim_{n\to\infty}x_n = \sqrt{x}$. 记$\vep_n = x_n - \sqrt{x}$,
    则成立
    \[
      \vep_{n+1} = \frac{\vep_n^2}{2x_n}
    \]
  \end{lemma}
  \textbf{证明 }
    对于$n>0$,根据基本不等式,成立
    \[
      x_{n+1}=\frac{1}{2}(x_n+\frac{x}{x_n})\ge\sqrt{x}.
    \]
    并且,
    \[\begin{split}
      x_n-x_{n+1} &= x_n - \frac{1}{2}(x_n+\frac{x}{x_n}) \\
      &= \frac12(x_n - \frac{x}{x_n}) \\
      &> \frac12(\sqrt{x} - \frac{x}{\sqrt{x}}) = 0
    \end{split}\]
    所以$\{x_n\}$单调减且下有界,所以存在极限,记为$A$,对\equref{equ1}
    两边取极限,成立
    \[
      A = \frac{1}{2}(A+\frac{x}{A})\quad\Rightarrow\quad
      A = \sqrt{x}.
    \]
    对于余项$\vep_n$,成立
    \[\begin{split}
      \vep_{n+1} - \sqrt{x} &= \frac{1}{2}(x_n + \frac{x}{x_n}) - \sqrt{x} \\
      &= \frac{x_n^2 - 2\sqrt{x}x_n + x}{2x_n} \\
      &= \frac{(x_n - \sqrt{x})^2}{2x_n} = \frac{\vep_n^2}{2x_n}.
      \quad\blacksquare
    \end{split}\]
  (1) 易知$x_1>\sqrt{7}$成立。应用\lemref{lemma1},取$x=7$,
  得$\lim x_n = x^* = \sqrt{7}$.\\
  (2) 设$x_n = 0.a_1a_2\dots a_n\times10$. $x_n$有$n$位
  有效数字,即成立
  \[
    \vep_n \le \frac12\times^{1-n}
  \]
  应用\lemref{lemma1},则对于$\vep_{n+1}$,成立
  \[\begin{split}
    \vep_{n+1} &\le \frac{\frac{1}{4} \times 10^{2-2n}}{2x_n}
    < \frac{5\times10^{1-2n}}{4\sqrt{7}}
    < \frac{1}{2} \times 10^{1-2n}
  \end{split}\]
  所以$x_{n+1}$至少具有$2n$位有效数字。 $\blacksquare$

\par\noindent2、已知$(\sqrt{2}-1)^6$……
\ans
  设$x=\sqrt{2}$,$x^*=1.4$为其近似值。并且
  \[\begin{split}
    \f_1(x) &= (x-1)^6,\quad
    \f_2(x) = (3-2x)^3,\quad
    \f_3(x) = 99-70x, \\
    \f_4(x) &= \frac{1}{(1+x)^6},\quad
    \f_5(x) = \frac{1}{(3+2x)^3},\quad
    \f_6(x) = \frac{1}{99+70x}
  \end{split}\]
  它们的导数在$x=1.4$处的取值分别为
  \[\begin{split}
    &\f_1\hp(1.4) = 0.06144,\quad
    \f_2\hp(1.4) = -0.24,\quad
    \f_3\hp(1.4) = -70,\\
    &\f_4\hp(1.4) \approx -0.1308,\quad
    \f_5\hp(1.4) \approx 5.3020 \times 10^{-3},\quad
    \f_6\hp(1.4) \approx -1.804 \times 10^{-3}
  \end{split}\]
  根据公式$\vep(\f(x^*)) \approx |\f\hp(x^*)|\vep(x^*)$可知,
  采用最后一个算式所得结果最精确。$\blacksquare$

\par\noindent3、试改变下列表达式使计算结果比较精确……
\ans
  (1)
  \[
    \frac{1}{1+2x} - \frac{1-x}{1+x} =
    \frac{1+x - (1-x)(1+2x)}{(1+2x)(1+x)} =
    \frac{2x^2}{(1+2x)(1+x)}\quad\blacksquare
  \]
  (2)
  \[
    \sqrt{x+\frac{1}{x}}-\sqrt{x-\frac{1}{x}} =
    \frac{x+\frac{1}{x} - (x-\frac{1}{x})}
    {\sqrt{x+\frac{1}{x}}+\sqrt{x-\frac{1}{x}}}
    = \frac{2}{x
    (\sqrt{x+\frac{1}{x}} + \sqrt{x-\frac{1}{x}})}
    \quad\blacksquare
  \]

\par\noindent4、找至少两种方法计算
  \[
    \f(x) = \frac{1-\cos x}{x^2},\,x\ne0,\,|x|\ll1.
  \]
\ans
  \[
    \f(x) = \frac{2\sin^2(x/2)}{x^2}
  \]
  对$\sin y$进行Taylor展开,至第$n$项,则有
  \[
    \f_n(x) = \frac{1}{2}
    \left(\sum_{k=0}^n (-1)^k\frac{x^{2k}}{(2k+1)!} + O(x^{2n+1})\right)^2
    \quad\blacksquare
  \]
\ans
  直接对$\cos x$进行Taylor展开,至第$n$项,则有
  \[\begin{split}
    \f(x) &= \frac{1}{x^2} - \frac{1}{x^2}
    \left( 1-\frac{x^2}{2} +
    \sum_{k=2}^n\frac{(-1)^kx^{2k}}{(2k)!} + O(x^{2n+1})
    \right) \\
    &= \frac{1}{2} - \sum_{k=2}^n\frac{(-1)^kx^{2k}}{(2k)!} + O(x^{2n+1})
    \blacksquare
  \end{split}\]

\par\noindent5、设$Y_0 = 28$,按递推公式……
\ans
  (1) 记$\vep(x^*) = \sqrt{783} - 27.982$. 因为$\{Y_n\}$为
  等差数列,所以可知
  \[\begin{split}
    Y_{100} &= Y_0 - \sqrt{783} \\
    Y_{100}^* &= Y_0 - x^*
  \end{split}\]
  所以
  \[
    |Y_{100} - Y_{100}^*| = |\sqrt{783} - x^*| = \vep(x^*).
  \]
  (2)
  \[\begin{split}
    &Y_n = Y_{n-1}-\frac{1}{100}\sqrt{783}
    \,\Rightarrow\,
    Y_n - \frac{1}{100}\sqrt{783} = 2\left(Y_{n-1} - \frac{1}{100}\sqrt{783}\right)\\
    \Rightarrow\, &
    Y_n = \left(28 - \frac{1}{100}\sqrt{783}\right)2^n
    + \frac{1}{100}\sqrt{783}
  \end{split}\]
  对于取$\sqrt{783} = 27.982$的情况同理,所以有
  \[\begin{split}
    Y_{100} &= \left(28 - \frac{1}{100}\sqrt{783}\right)2^{100}
    + \frac{1}{100}\sqrt{783} \\
    Y_{100}^* &= \left(28 - \frac{1}{100}27.982\right)2^{100}
    + \frac{1}{100}27.982
  \end{split}\]
  从而对于误差,成立
  \[
    |Y_{100}-Y_{100}^*| =
    \vep(x^*)\frac{2^{100}}{100} + \frac{\vep(x^*)}{100}
    \quad\blacksquare
  \]

\par\noindent6、给定$n$个矩阵……
\ans
  设矩阵$A_1,\,A_2,\,\dots,A_n$的大小分别为$a_0\times a_1,\,
  a_1\times a_2,\,\dots,a_{n-1}\times a_n.$ 大小为$a\times b$
  和$b\times c$的矩阵相乘,操作次数$\f(a, b, c)$为
  \[
    \f(a, b, c) = ac(2b-1)
  \]
  设将第$i$个至第$j$个矩阵相乘所需的最小操作次数为$F(i, j)$,则有
  \[
    F(i, j) =
    \begin{cases}
      0,\quad & i = j \\
      \min\limits_{i \le k < j}
      \{ F(i, k) + F(k+1, j) + \f(a_{i-1}, a_k, a_j) \},
      \quad & i \ne j
    \end{cases}
  \]
  所以$F(1,n)$即为最少操作次数。在递归求解$F(1,n)$的同时,设
  $k_{ij}$为取出$F(i, j)$中$\min$所对应的$k$,则计算顺序为
  最后计算$A_1A_2\cdots A_{k_{1n}}$和$A_{k_{1n}+1}\cdots A_n$
  的乘积,其次最后计算$A_1\cdots A_{k_{1k_{1n}}}$和
  $A_{k_{1k_{1n}}+1}\cdots A_{k_{1n}}$的乘积。依此类推。
  $\blacksquare$

\end{document}
