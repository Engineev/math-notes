\documentclass[12pt, a4paper]{article}
\usepackage{ctex}

\usepackage[margin=1in]{geometry}
\usepackage{
  color,
  clrscode,
  amssymb,
  ntheorem,
  amsfonts,
  amsmath,
  listings,
  fontspec,
  xcolor,
  supertabular,
  multirow,
  mathtools,
  mathrsfs,
}
\definecolor{bgGray}{RGB}{36, 36, 36}
\usepackage[
  colorlinks,
  linkcolor=bgGray,
  anchorcolor=blue,
  citecolor=green
]{hyperref}
\newfontfamily\courier{Courier}

\theoremstyle{margin}
\theorembodyfont{\normalfont}
\newtheorem{thm}{定理}
\newtheorem{cor}[thm]{推论}
\newtheorem{pos}[thm]{命题}
\newtheorem{lemma}[thm]{引理}
\newtheorem{defi}[thm]{定义}

\DeclareMathOperator{\rank}{rank}
\DeclareMathOperator{\adj}{adj}
\DeclareMathOperator{\tr}{tr}
\DeclareMathOperator{\diag}{diag}
\DeclareMathOperator{\nul}{null}
\DeclareMathOperator{\range}{range}
\DeclareMathOperator{\spn}{span}
% \DeclareMathOperator{\deg}{deg}

\newcommand{\hp}{^\prime}
\newcommand{\vep}{\varepsilon}
\newcommand{\inv}{^{-1}}
\newcommand{\rd}{\mathrm{d}}

\renewcommand{\Im}{\text{Im}}
\renewcommand{\Re}{\text{Re}}



\title{科学计算作业$\,$练习$10$}
\author{\small 任云玮\\\small2016级ACM班\\\small516030910586}
\date{}

\newcommand{\y}{\mbf{y}}
\newcommand{\J}{\mbf{J}}
\newcommand{\D}{\mbf{D}}

\begin{document}
\lstset{
  numbers=left,
  basicstyle=\scriptsize,
  numberstyle=\tiny\color{red!89!green!36!blue!36},
  language=Matlab,
  breaklines=true,
  keywordstyle=\color{blue!70},commentstyle=\color{red!50!green!50!blue!50},
  morekeywords={},
  stringstyle=\color{purple},
  frame=shadowbox,
  rulesepcolor=\color{red!20!green!20!blue!20}
}
\maketitle

\noindent 2. 设线性方程组……
\proof
(1) 即证:对于$\mbf{J}=\mbf{D}^{-1}(\mbf{L}+\mbf{U})$,
  $\mbf{G}=(\mbf{D}-\mbf{L})^{-1}\mbf{U}$,$\rho(\mbf{J})<1$
  成立当且仅当$\rho(\mbf{D})<1$成立. 因为
  \[\begin{split}
    &\mbf{J}=
    \begin{pmatrix}
      a_{11} & 0\\
      0 & a_{22}
    \end{pmatrix}^{-1}
    \begin{pmatrix}
        0 & -a_{12} \\
        -a_{21} & 0
    \end{pmatrix}
    = \begin{pmatrix}
         0 & -a_{12}/a_{11} \\
        -a_{21}/a_{22} & 0
    \end{pmatrix} \quad\Rightarrow\quad
    \lambda_1^2 = \frac{a_{12}a_{21}}{a_{11}a_{22}}, \\
    &\mbf{G}=
    \begin{pmatrix}
      a_{11} & 0 \\
      a_{21} & a_{22} \\
    \end{pmatrix}^{-1}
    \begin{pmatrix}
        0 & -a_{12} \\
        0 & 0
    \end{pmatrix} =
    \begin{pmatrix}
      0 & -a_{12}/a_{11} \\
      0 & a_{12}a_{21} / (a_{11}a_{22})
    \end{pmatrix}\quad\Rightarrow\quad
    \lambda_2 = 0 \text{ 或 }\lambda_2 = \frac{a_{12}a_{21}}{a_{11}a_{22}}.
  \end{split}\]
  所以$|\lambda_1| < 1$,当且仅当$|\lambda_2| = |\lambda_1^2| < 1$,
  所以Jacobi迭代法和GS迭代法同时收敛或发散. $\blacksquare$\\
(2) 两者的渐进收敛速度之比为
  \[
    \frac{R(\mbf{J})}{R(\mbf{G})} = \frac{1}{2}.
    \quad\blacksquare
  \]

\vspace{1cm}
\par\noindent 4. 设$A=$……
\ans
  \[\begin{split}
    \J = \D^{-1}(\mbf{L}+\mbf{U}) &=
    \begin{pmatrix}
      0.1 & & \\
      & 0.1 & \\
      && 0.2
    \end{pmatrix}
    \begin{pmatrix}
      0 & -a & 0 \\
      -b & 0 & -b \\
      0 & -a & 0
    \end{pmatrix} =
    \begin{pmatrix}
      0 & -0.1a & 0 \\
      -0.1b & 0 & -0.2b \\
      0 & -0.1a & 0
    \end{pmatrix}\\
    \Rightarrow\quad
    0 = |\lambda\mbf{E} - \J| &=
    \begin{vmatrix}
        \lambda & 0.1a & 0 \\
        0.1b & \lambda & 0.2b \\
        0 & 0.1a & \lambda
    \end{vmatrix} =
    \lambda^3 - \frac{3}{100}ab \quad\Rightarrow\quad
    \lambda = \sqrt[3]{\frac{3ab}{100}}
  \end{split}\]
  从而Jacobi迭代法收敛当且仅当$|ab| < 100/3$.\par
  \[\begin{split}
    \mbf{G} = (\D-\mbf{L})^{-1}\mbf{U} &=
    \begin{pmatrix}
      0.1 & & \\
      -0.01b & 0.1 & \\
      0.002ab& -0.02a & 0.2
    \end{pmatrix}
    \begin{pmatrix}
      & a & \\
      &  & b\\
      & &
     \end{pmatrix} =
     \begin{pmatrix}
        0 & 0.1a & 0 \\
        0 & -0.01ab & 0.1b \\
        0 & 0.002a^2b & -0.02ab
     \end{pmatrix} \\ \quad\Rightarrow\quad
     0 = |\lambda\mbf{E}-\mbf{G}| &=
     \begin{vmatrix}
       \lambda & -0.1a & 0 \\
       0 & \lambda + 0.01ab & -0.1b \\
       0 & -0.002a^2b & \lambda + 0.02ab
     \end{vmatrix} = \lambda^2(\lambda+0.03ab)
  \end{split}\]
  从而GS迭代法收敛当且仅当$|ab|<100/3.\quad\blacksquare$

\vspace{1cm}
\par\noindent 6. 用Jacobi迭代与GS迭代解线性方程组……
\proof
  \[\begin{split}
    \mbf{J} =
    \begin{pmatrix}
      0 & 0 &  -2/3 \\
      0 & 0 & 1/2 \\
      -1. & 1/2 & 0
    \end{pmatrix} \quad&\Rightarrow\quad
    \lambda(\mbf{J}) = 0\,\text{或}\,\pm\frac{\sqrt{33}}{6},\\
    \mbf{G} =
    \begin{pmatrix}
      0 & 0 & -2/3 \\
      0 & 0 & 1/2 \\
      0 & 0 & 11/12
    \end{pmatrix} \quad&\Rightarrow\quad
    \lambda(\mbf{G}) = 0 \,\text{或}\,\frac{11}{12}.
  \end{split}\]
  从而两种迭代法都收敛. 且$\rho(\mbf{G}) < \rho(\mbf{J})$,从而GS迭代法
  收敛更快. $\quad\blacksquare$

\vspace{1cm}
\par\noindent 9. 设线性方程组$\A\x=\mbf{b}$,其中$\A$为对称正定阵…
\proof
  迭代式为
  \begin{equation}
    \label{equ9}
    \x^{(k+1)} = \left(\mbf{E}-\omega\A\right)\x^{(k)}+\omega\x^{(k)}.
  \end{equation}
  设$\lambda$为$\mbf{B}=\mbf{E}-\omega\A$的特征值,则
  \[
    (\mbf{E}-\omega\A)\x=\lambda\x\quad\Rightarrow\quad
    \A\x = \frac{1-\lambda}{\omega}\x.
  \]
  即$(1-\lambda)/\omega$为$\A$的特征值. 由于$\A$正定,所以
  \[
    \frac{1-\lambda}{\omega} > 0 \quad\Rightarrow\quad
    \lambda < 1.
  \]
  同时又有
  \[
    \frac{1-\lambda}{\omega} \le\beta \quad\Rightarrow\quad
    1-\lambda \le \beta\omega < 2 \quad\Rightarrow\quad \lambda > -1.
  \]
  综上,$|\lambda| < 1$,即$\rho(\mbf{B})<1$,从而\equref{equ9}收敛. $\blacksquare$

\end{document}
