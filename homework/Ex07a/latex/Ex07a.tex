\documentclass[12pt, a4paper]{article}
\usepackage{ctex}

\usepackage[margin=1in]{geometry}
\usepackage{
  color,
  clrscode,
  amssymb,
  ntheorem,
  amsmath,
  listings,
  fontspec,
  xcolor,
  supertabular,
  multirow,
  mathtools,
  mathrsfs
}
\definecolor{bgGray}{RGB}{36, 36, 36}
\usepackage[
  colorlinks,
  linkcolor=bgGray,
  anchorcolor=blue,
  citecolor=green
]{hyperref}
\newfontfamily\courier{Courier}

\theoremstyle{margin}
\theorembodyfont{\normalfont}
\newtheorem{thm}{定理}
\newtheorem{cor}[thm]{推论}
\newtheorem{pos}[thm]{命题}
\newtheorem{lemma}[thm]{引理}
\newtheorem{defi}[thm]{定义}
\newtheorem{std}[thm]{标准}
\newtheorem{imp}[thm]{实现}
\newtheorem{alg}[thm]{算法}
\newtheorem{exa}[thm]{例}
\newtheorem{prob}[thm]{问题}
\DeclareMathOperator{\sft}{E}
\DeclareMathOperator{\idt}{I}
\DeclareMathOperator{\spn}{span}
\DeclareMathOperator*{\agm}{arg\,min}
\newcommand{\pr}{\prime}
\newcommand{\tr}{^\intercal}
\newcommand{\st}{\text{s.t.}}
\newcommand{\hp}{^\prime}
\newcommand{\ms}{\mathscr}
\newcommand{\mn}{\mathnormal}
\newcommand{\tbf}{\textbf}
\newcommand{\mbf}{\mathbf}
\newcommand{\fl}{\mathnormal{fl}}
\newcommand{\f}{\mathnormal{f}}
\newcommand{\g}{\mathnormal{g}}
\newcommand{\R}{\mathbf{R}}
\newcommand{\Q}{\mathbf{Q}}
\newcommand{\JD}{\textbf{D}}
\newcommand{\rd}{\mathrm{d}}
\newcommand{\str}{^*}
\newcommand{\vep}{\varepsilon}
\newcommand{\lhs}{\text{L.H.S}}
\newcommand{\rhs}{\text{R.H.S}}
\newcommand{\con}{\text{Const}}
\newcommand{\oneton}{1,\,2,\,\dots,\,n}
\newcommand{\aoneton}{a_1a_2\dots a_n}
\newcommand{\xoneton}{x_1,\,x_2,\,\dots,\,x_n}
\newcommand\thmref[1]{定理~\ref{#1}}
\newcommand\lemmaref[1]{引理~\ref{#1}}
\newcommand\defref[1]{定义~\ref{#1}}
\newcommand\posref[1]{命题~\ref{#1}}
\newcommand\secref[1]{节~\ref{#1}}
\newcommand\equref[1]{(\ref{#1})}
\newcommand\figref[1]{图 \ref{#1}}
\newcommand\corref[1]{推论~\ref{#1}}
\newcommand\exaref[1]{例~\ref{#1}}
\newcommand\algref[1]{算法~\ref{#1}}
\newcommand{\remark}{\paragraph{评注}}
\newcommand{\example}{\paragraph{例}}
\newcommand{\proof}{\paragraph{证明}}


\title{科学计算作业$\,$练习$7a$}
\author{\small 任云玮\\\small2016级ACM班\\\small516030910586}
\date{}

\begin{document}
\lstset{
  numbers=left,
  basicstyle=\scriptsize,
  numberstyle=\tiny\color{red!89!green!36!blue!36},
  language=Matlab,
  breaklines=true,
  keywordstyle=\color{blue!70},commentstyle=\color{red!50!green!50!blue!50},
  morekeywords={},
  stringstyle=\color{purple},
  frame=shadowbox,
  rulesepcolor=\color{red!20!green!20!blue!20}
}
\maketitle

\begin{lemma}
  \label{lemma}
  设$x_*$是$\varphi(x)\in\ms{C}^1[a, b]$的不动点,若
  $\varphi(x)$在$[a, b]$上单调且$|\varphi\hp(x_*)|\ge h>1$,则
  $\varphi(x)$不局部收敛.\footnote{
    关于这一引理的条件:之所以我要加上单调(单射)条件,是因为我觉得可能会出现,
    在$x_*$的某一领域中的点,被映射到某个这一领域外的点集内,而这个点集内的点都被直接映射到$x_*$上
    的情况.
    而之所以加上常数$h>1$而非直接大于$1$,主要是为了证明方便,否则在说明迭代足够多次
    后,有$|x_m-x_*|>\vep$时,可能会出现$\varphi\hp(\xi)\to 1$,则可能虽然每一次
    误差确实都在增大,但最后的结果却并没有超过$\vep$的情况,即
    $\prod_{k=1}^\infty \varphi(\xi_k) = \text{Constant}$的情况.
  }
\end{lemma}
\tbf{证明: }
  由于$\varphi\hp(x_*)>1$且连续,所以存在$\delta>0$,使得在$O_\delta(x_*)$
  中成立
  \begin{equation}
    \label{h}
    |\varphi(x)-x_*| = |\varphi\hp(\xi)(x-x_*)| \ge h|x-x_*|.
  \end{equation}
  所以对任意的$0<\vep<\delta$,对任意的$x_0\in O_\delta(x_*)
  \backslash\{x_*\}$,迭代
  足够多次后成立$|x_n-x_*|>\vep$. 由于$\varphi(x)$单调,所以仅有
  $x=x_*$满足$\varphi(x)=x_*$. 从而若假设存在$x_0\in[a, b]$,使迭代数列收敛
  至$x_*$,则对任意$\vep>0$,存在$N$,使得任意$n>N$,有$x_n\in O_{\vep}(x_*)\backslash\{x_*\}$.
  对于这样的$n$,根据\equref{h},继续迭代足够多次后,成立$|x_m-x_*|>\vep$,与收敛矛盾,从而不存在
  $x_0\in[a, b]$,使得迭代数列收敛. $\blacksquare$

\vspace{1cm}
\noindent2. 为求方程$x^3-x^2-1=0$……
\ans
  令$\f(x) = x^3-x^2-1$,则$\f(1.4)=-0.216<0$,
  $\f(1.5)=0.125>0$,由于$\f$在$[1.4,1.5]$上连续,所以在
  $[1.4, 1.5]$中存在零点$x_*$.
\paragraph{(1)}
  \[
    \varphi(x) = 1 + \frac{1}{x^2} \quad\Rightarrow\quad
    \varphi\hp(x) = -\frac{2}{x^3}.
  \]
  在$[1.4, 1.5]$上,$|\varphi\hp(x)|\le|\varphi\hp(1.4)|<0.75$. 从而
  $\varphi\hp(x_*)\le 0.75<1$且连续,从而在$x_*$处局部收敛. $\blacksquare$
\paragraph{(2)}
  \[
    \varphi(x) = \sqrt[3]{1+x^2}\quad\Rightarrow\quad
    \varphi\hp(x) = \frac{2}{3}x(1+x^2)^{-2/3}.
  \]
  在$[1.4, 1.5]$上,$|\varphi\hp(x)|\le 0.5$且连续,从而
  在$x_*$处局部收敛. $\blacksquare$
\paragraph{(3)}
  若$x_0\le 1$或$x_0\ge 2$,则$x_{k+1}$或$x_{k+2}$不存在,所以
  仅需考虑$x_0\in(1, 2)$. 由于$\varphi(x)$在$(1, 2)$上单调减,
  且$\varphi\hp(x) = -0.5(x-1)^{-3/2}$,在$[1.4, 1.5]$上恒
  大于$1.1$,从而$\varphi\hp(x_*)>1.1$. 根据\lemref{lemma},
  $\varphi(x)$不局部收敛. $\blacksquare$
\paragraph{计算}
  设$\varphi(x) = \sqrt[3]{1+x^2}$,取$x_0 = 1.5$,压缩常数
  $l=0.5$,从而约需要迭代$10$次,计算过程见\tabref{table},结果为
  \[
    \tilde{x} = 1.46558.\quad\blacksquare
  \]
  \begin{table}[htbp]
    \centering
    \caption{$n-x_n$表}
    \label{table}
    % \begin{adjustbox}{width=1\textwidth}
    \begin{tabular}{c|ccccccccccc}
    \hline
    $n$    & 0   & 1        & 2  & $\cdots$ & 8       & 9       & 10      \\ \hline
    $x_n$ & 1.5 & 1.48125 & 1.47271 & $\cdots$ & 1.46563 & 1.46560 & 1.46558 \\ \hline
    \end{tabular}
  % \end{adjustbox}
  \end{table}


\vspace{1cm}
\par\noindent 5. 用Steffensen迭代法计算……
\ans
  计算结果见\tabref{计算结果}.
  \begin{table}[htbp]
    \centering
    \caption{计算结果}
    \label{计算结果}
    \begin{tabular}{c|ccc}
      \toprule
      \multicolumn{4}{c}{$x_{k+1} = \sqrt[3]{1+x^2}$} \\
      \midrule
      $k$ & $x_k$ & $y_k$ & $z_k$ \\ \hline
      $0$ & $1.5$ & $1.481248$ & $1.472706$ \\
      $1$ & $1.465558$ & $1.465565$ & $1.465569$ \\
      $2$ & $1.465571$ & $1.465571$ &	$1.465571$ \\
      $3$ & $1.46557$ && \\ \bottomrule
    \end{tabular}
    $\qquad$
    \begin{tabular}{c|ccc}
      \toprule
      \multicolumn{4}{c}{$x_{k+1} = 1/\sqrt{x_k-1}$} \\
      \midrule
      $k$ & $x_k$ & $y_k$ & $z_k$ \\ \hline
      $0$ & $1.500000$ & $1.414214$ & $1.553774$ \\
      $1$ & $1.467342$ & $1.462792$	& $1.469966$ \\
      $2$ & $1.465576$ & $1.465564$ &	$1.465583$ \\
      $3$ & $1.46557$ & & \\ \bottomrule
    \end{tabular}
  \end{table}

\vspace{1cm}
\par\noindent 6. 设$\varphi(x) = $……
\ans
  设$\f(x_*)=0$,则只需要满足$\varphi\hp(x_*) = 0$且
  $\varphi^{\pr\pr}(x_*) = 0$,即至少三阶收敛.
  \[\begin{split}
     & \varphi\hp(x) = 1-p\hp\f - p\f\hp - q\hp\f^2 - 2q\f\f\hp \\
     \Rightarrow\quad &
     \varphi\hp(x_*) = 1 - p(x_*)\f\hp(x_*) = 0 \\
     \Rightarrow\quad &
     p = \frac{1}{\f\hp}.
  \end{split}\]
  将结果代回,得
  \[\begin{split}
    & \varphi\hp(x) = \frac{\f^{\pr\pr}}{(\f\hp)^2}\f - q\f^2-2q\f\f\hp \\
    \Rightarrow\quad &
    \varphi^{\pr\pr}(x_*) = \left(\frac{\f^{\pr\pr}}{\f\hp} - 2q(\f\hp)^2\right)
    \bigg|_{x=x_*} = 0 \\
    \Rightarrow\quad &
    q = \frac{\f^{\pr\pr}}{2(\f\hp)^3}.
  \end{split}\]
  综上
  \[
    \varphi = x - \frac{\f}{\f\hp} - \frac{\f^{\pr\pr}\f^2}{2(\f\hp)^3}.
    \quad\blacksquare
  \]

\vspace{1cm}
\par\noindent 10. 对于$\f(x)=0$的牛顿公式……
\proof
  由于Newton公式二阶收敛,所以有
  \begin{equation}
    \label{sim}
    \frac{x_{k+1}-x_*}{x_k-x_*} \to 0 \quad\Rightarrow\quad
    \frac{x_{k+1}-x_k}{x_k-x_*} \to -1.
  \end{equation}
  记$U_{k+1} = (x_{k+1}-x_*)/(x_k-x_*)^2$,则根据\equref{sim}
  \begin{equation}
    \label{RU}
    \begin{split}
    \frac{R_k}{U_{k-1}} &= \frac{x_k-x_{k-1}}{(x_{k-1}-x_{k-2})^2}
    \frac{(x_{k-2}-x_*)^2}{x_{k-1}-x_*} \\
    & = \frac{x_k-x_{k-1}}{x_{k-1}-x_*}\left(
      \frac{x_{k-2} - x_*}{x_{k-1} - x_{k-2}}
    \right)^2 \to -1\times(-1)^2 = -1.
  \end{split}\end{equation}
  而对于$U_{k+1}$,存在$\xi$位于$x_*$和$x_k$之间,成立
  \begin{equation}
    \label{lim}
    \lim_{k\to\infty} U_{k+1} = \lim_{\xi\to x_*}\frac{\varphi^{\pr\pr}(\xi)}{2}
    = \frac{\varphi^{\pr\pr}(x_*)}{2} = \frac{\f^{\pr\pr}(x_*)}{2\f\hp(x_*)}.
  \end{equation}
  结合\equref{RU}和\equref{lim},有
  \[
    \lim_{k\to\infty}R_k = -\lim_{k\to\infty}U_k = -\frac{\f^{\pr\pr}(x_*)}{2\f\hp(x_*)}.
    \quad\blacksquare
  \]

\vspace{1cm}
\par\noindent 14. 应用Newton法于方程……
\ans
\paragraph{(1)}
  \[
    \f(x) = x^n - a \quad\Rightarrow\quad
    \varphi(x) = \frac{1}{n}\left( (n-1)x + \frac{a}{x^{n-1}} \right).
  \]
  $\varphi$在$\sqrt[n]{x}$处的一阶、二阶导数分别为
  \[\begin{split}
    \varphi\hp(\sqrt[n]{a}) &= \left(1-\frac{1}{n}\right)\left(1-\frac{a}{x^n}\right)
    \bigg|_{x=\sqrt[n]{a}} = 0, \\
    \varphi^{\pr\pr}(\sqrt[n]{a}) &= \frac{(n-1)a}{x^{n+1}}\bigg|_{x=\sqrt[n]{a}} \ne 0.
  \end{split}\]
  从而极限为
  \[
    \lim_{k\to\infty}\frac{\sqrt[n]{a} - x_{k+1}}{(\sqrt[n]{a}-x_k)^2}
    = -\lim_{\xi\to\sqrt[n]{a}}\frac{\varphi^{\pr\pr}(\xi)}{2}
    = \frac{1-n}{2\sqrt[n]{a}}.\quad\blacksquare
  \]
\paragraph{(2)}
  \[
    \f(x) = 1 - \frac{a}{x^n} \quad\Rightarrow\quad
    \varphi(x) = \frac{a(n+1)x - x^{n+1}}{an}.
  \]
  $\varphi$在$\sqrt[n]{x}$处的一阶、二阶导数分别为
  \[\begin{split}
    \varphi\hp(\sqrt[n]{a}) &= \frac{(n+1)(a-x^n)}{an}\bigg|_{x=\sqrt[n]{a}} = 0, \\
    \varphi^{\pr\pr}(\sqrt[n]{a}) &= \frac{(n+1)(a-nx^{n-1})}{an}\bigg|_{x=\sqrt[n]{a}} \ne 0.
  \end{split}\]
  从而极限为
  \[
    \lim_{k\to\infty}\frac{\sqrt[n]{a} - x_{k+1}}{(\sqrt[n]{a}-x_k)^2}
    = -\lim_{\xi\to\sqrt[n]{a}}\frac{(n+1)(a-nx^{n-1})}{an}
    = \frac{n+1}{n}\left(\frac{n}{\sqrt[n]{a}} -1\right).\quad\blacksquare
  \]

\end{document}
