\documentclass[12pt, a4paper]{article}
\usepackage{ctex}

\usepackage[margin=1in]{geometry}
\usepackage{
  color,
  clrscode,
  amssymb,
  ntheorem,
  amsmath,
  listings,
  fontspec,
  xcolor,
  supertabular,
  multirow,
  mathtools,
  mathrsfs
}
\definecolor{bgGray}{RGB}{36, 36, 36}
\usepackage[
  colorlinks,
  linkcolor=bgGray,
  anchorcolor=blue,
  citecolor=green
]{hyperref}
\newfontfamily\courier{Courier}

\theoremstyle{margin}
\theorembodyfont{\normalfont}
\newtheorem{thm}{定理}
\newtheorem{cor}[thm]{推论}
\newtheorem{pos}[thm]{命题}
\newtheorem{lemma}[thm]{引理}
\newtheorem{defi}[thm]{定义}
\newtheorem{std}[thm]{标准}
\newtheorem{imp}[thm]{实现}
\newtheorem{alg}[thm]{算法}
\newtheorem{exa}[thm]{例}
\newtheorem{prob}[thm]{问题}
\DeclareMathOperator{\sft}{E}
\DeclareMathOperator{\idt}{I}
\DeclareMathOperator{\spn}{span}
\DeclareMathOperator*{\agm}{arg\,min}
\newcommand{\pr}{\prime}
\newcommand{\tr}{^\intercal}
\newcommand{\st}{\text{s.t.}}
\newcommand{\hp}{^\prime}
\newcommand{\ms}{\mathscr}
\newcommand{\mn}{\mathnormal}
\newcommand{\tbf}{\textbf}
\newcommand{\mbf}{\mathbf}
\newcommand{\fl}{\mathnormal{fl}}
\newcommand{\f}{\mathnormal{f}}
\newcommand{\g}{\mathnormal{g}}
\newcommand{\R}{\mathbf{R}}
\newcommand{\Q}{\mathbf{Q}}
\newcommand{\JD}{\textbf{D}}
\newcommand{\rd}{\mathrm{d}}
\newcommand{\str}{^*}
\newcommand{\vep}{\varepsilon}
\newcommand{\lhs}{\text{L.H.S}}
\newcommand{\rhs}{\text{R.H.S}}
\newcommand{\con}{\text{Const}}
\newcommand{\oneton}{1,\,2,\,\dots,\,n}
\newcommand{\aoneton}{a_1a_2\dots a_n}
\newcommand{\xoneton}{x_1,\,x_2,\,\dots,\,x_n}
\newcommand\thmref[1]{定理~\ref{#1}}
\newcommand\lemmaref[1]{引理~\ref{#1}}
\newcommand\defref[1]{定义~\ref{#1}}
\newcommand\posref[1]{命题~\ref{#1}}
\newcommand\secref[1]{节~\ref{#1}}
\newcommand\equref[1]{(\ref{#1})}
\newcommand\figref[1]{图 \ref{#1}}
\newcommand\corref[1]{推论~\ref{#1}}
\newcommand\exaref[1]{例~\ref{#1}}
\newcommand\algref[1]{算法~\ref{#1}}
\newcommand{\remark}{\paragraph{评注}}
\newcommand{\example}{\paragraph{例}}
\newcommand{\proof}{\paragraph{证明}}


\title{科学计算作业$\,$练习$6c$}
\author{\small 任云玮\\\small2016级ACM班\\\small516030910586}
\date{}

\begin{document}
\lstset{
  numbers=left,
  basicstyle=\scriptsize,
  numberstyle=\tiny\color{red!89!green!36!blue!36},
  language=Matlab,
  breaklines=true,
  keywordstyle=\color{blue!70},commentstyle=\color{red!50!green!50!blue!50},
  morekeywords={},
  stringstyle=\color{purple},
  frame=shadowbox,
  rulesepcolor=\color{red!20!green!20!blue!20}
}
\maketitle

\noindent11. 用$n=2,3$的Gauss-Legrende公式计算……
\ans
  令$t=x-2$,则所求积分为
  \[
    I = \int_{-1}^1 \f(x)\rd x = \int_{-1}^1 e^{x+2}\sin(x+2)\rd x.
  \]
  \paragraph{$\mbf{n = 2}$}
    在$[-1, 1]$上的$3$次Legrende多项式及对应零点为
      \[
        P_3 = \frac{5}{2}x^3 - \frac{3}{2}x \quad\Rightarrow\quad
        x_0 = -\frac{\sqrt{15}}{5},\,x_1 = 0,\,x_2 = \frac{\sqrt{15}}{5}
      \]
    对应的函数值及系数为
    \[
      \begin{cases}
        \f(x_0) = e^{-\sqrt{15}/5+2}\sin(-\frac{\sqrt{15}}{5}+2),\\
        \f(x_1) = e^{2}\sin 2 \\
        \f(x_2) = e^{\sqrt{15}/5+2}\sin(\frac{\sqrt{15}}{5}+2),
      \end{cases}
      \begin{cases}
        A_0 = 5/9, \\
        A_1 = 8/9, \\
        A_2 = 5/9
      \end{cases}
    \]
    从而
    \[
      Q_3(\f) \approx 10.94840.\quad\blacksquare
    \]
  \paragraph{$\mbf{n = 3}$}
      在$[-1, 1]$上的$4$次Legrende多项式及对应零点为
        \[
          P_4 = \frac{1}{8}\left( 35x^4 - 30x^2 + 3 \right) \quad\Rightarrow\quad
          x_{0(3)} = \mp0.86114,\, x_{1(2)} = \mp0.33998
        \]
      对应的函数值及系数为
      \[
        \begin{cases}
          \f(x_0) = 2.83636,\\
          \f(x_1) = 5.23850,\\
          \f(x_2) = 7.45854,\\
          \f(x_3) = 4.83870
        \end{cases}
        \quad
        \begin{cases}
          A_0 = A_3 = 0.347855\\
          A_1 = A_2 = 0.652145
        \end{cases}
      \]
      从而
      \[
        Q_4(\f) \approx 10.95011.\quad\blacksquare
      \]

\vspace{1cm}
\par\noindent12. 地球卫星轨道是一个椭圆……
\ans
  根据条件,有
  \[
    a = 7782.5,\quad c = 972.5,
  \]
  即轨道周长为
  \[
    S = 31130 \int_0^{\pi/2}\sqrt{1-0.015615\sin^2\theta}\rd\theta.
  \]
  设所需达到的精度为$\vep = 1(\text{km})$. 用复化Simpson方法计算,
  结果如下表
  \[\begin{array}{ c c c c}
    \hline
    \quad n\qquad & \qquad h_n\qquad & \qquad S_n\qquad &\qquad |S_n - S_{n-1}| \quad\\ \hline
    1 & \pi/2 & 48707.50  & \\
    2 & \pi/4 & 48707.44  & 0.06 \\ \hline
  \end{array}\]
  即
  \[
    S = 48707\,(km)\quad\blacksquare
  \]

\end{document}
