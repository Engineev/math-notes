\documentclass[12pt, a4paper]{article}
\usepackage{ctex}

\usepackage[margin=1in]{geometry}
\usepackage{
  color,
  clrscode,
  amssymb,
  ntheorem,
  amsmath,
  listings,
  fontspec,
  xcolor,
  supertabular,
  multirow,
  mathtools,
  mathrsfs
}
\definecolor{bgGray}{RGB}{36, 36, 36}
\usepackage[
  colorlinks,
  linkcolor=bgGray,
  anchorcolor=blue,
  citecolor=green
]{hyperref}
\newfontfamily\courier{Courier}

\theoremstyle{margin}
\theorembodyfont{\normalfont}
\newtheorem{thm}{定理}
\newtheorem{cor}[thm]{推论}
\newtheorem{pos}[thm]{命题}
\newtheorem{lemma}[thm]{引理}
\newtheorem{defi}[thm]{定义}
\newtheorem{std}[thm]{标准}
\newtheorem{imp}[thm]{实现}
\newtheorem{alg}[thm]{算法}
\newtheorem{exa}[thm]{例}
\newtheorem{prob}[thm]{问题}
\DeclareMathOperator{\sft}{E}
\DeclareMathOperator{\idt}{I}
\DeclareMathOperator{\spn}{span}
\DeclareMathOperator*{\agm}{arg\,min}
\newcommand{\pr}{\prime}
\newcommand{\tr}{^\intercal}
\newcommand{\st}{\text{s.t.}}
\newcommand{\hp}{^\prime}
\newcommand{\ms}{\mathscr}
\newcommand{\mn}{\mathnormal}
\newcommand{\tbf}{\textbf}
\newcommand{\mbf}{\mathbf}
\newcommand{\fl}{\mathnormal{fl}}
\newcommand{\f}{\mathnormal{f}}
\newcommand{\g}{\mathnormal{g}}
\newcommand{\R}{\mathbf{R}}
\newcommand{\Q}{\mathbf{Q}}
\newcommand{\JD}{\textbf{D}}
\newcommand{\rd}{\mathrm{d}}
\newcommand{\str}{^*}
\newcommand{\vep}{\varepsilon}
\newcommand{\lhs}{\text{L.H.S}}
\newcommand{\rhs}{\text{R.H.S}}
\newcommand{\con}{\text{Const}}
\newcommand{\oneton}{1,\,2,\,\dots,\,n}
\newcommand{\aoneton}{a_1a_2\dots a_n}
\newcommand{\xoneton}{x_1,\,x_2,\,\dots,\,x_n}
\newcommand\thmref[1]{定理~\ref{#1}}
\newcommand\lemmaref[1]{引理~\ref{#1}}
\newcommand\defref[1]{定义~\ref{#1}}
\newcommand\posref[1]{命题~\ref{#1}}
\newcommand\secref[1]{节~\ref{#1}}
\newcommand\equref[1]{(\ref{#1})}
\newcommand\figref[1]{图 \ref{#1}}
\newcommand\corref[1]{推论~\ref{#1}}
\newcommand\exaref[1]{例~\ref{#1}}
\newcommand\algref[1]{算法~\ref{#1}}
\newcommand{\remark}{\paragraph{评注}}
\newcommand{\example}{\paragraph{例}}
\newcommand{\proof}{\paragraph{证明}}


\title{科学计算作业$\,$练习$9a$}
\author{\small 任云玮\\\small2016级ACM班\\\small516030910586}
\date{}

\newcommand{\y}{\mbf{y}}

\begin{document}
\lstset{
  numbers=left,
  basicstyle=\scriptsize,
  numberstyle=\tiny\color{red!89!green!36!blue!36},
  language=Matlab,
  breaklines=true,
  keywordstyle=\color{blue!70},commentstyle=\color{red!50!green!50!blue!50},
  morekeywords={},
  stringstyle=\color{purple},
  frame=shadowbox,
  rulesepcolor=\color{red!20!green!20!blue!20}
}
\maketitle

\noindent 2. 设$\A=(a_{ij})_n$是对称正定阵……
\ans
(1) 因为$\A$正定,所以对于任意$\x\ne 0$,成立$\x\tr\A\x >0$. 取
    $\x=\mbf{e}_i$,则有
    \[
      a_{ii} = \mbf{e}_i\tr\A\mbf{e}_i > 0.\quad\blacksquare
    \]
(2) 由于$\A$为对称阵,所以有
    \begin{equation}
      \label{equ2}
      \begin{pmatrix}
        a_{11}  &  \mbf{a}_1\tr \\
        \mbf{0} & \A_2
      \end{pmatrix} =
      \begin{pmatrix}
          1        &  \mbf{0} \\
          -\mbf{a}_1/a_{11} &  \mbf{E}_{n-1}
      \end{pmatrix}
      \begin{pmatrix}
        a_{11}    & \mbf{a}_1\tr \\
        \mbf{a}_1 & \A_1
      \end{pmatrix}
      = \mbf{P}\tr\A
    \end{equation}
    因此成立
    \[\begin{split}
      \A_2
       = -\frac{1}{a_{11}}\mbf{a_1}\mbf{a}_1\tr + \A_1. \\
    \end{split}\]
    同时$\A_1\tr = \A_1$,所以$\A_2$是对称阵. 对\equref{equ2}两边
    同乘$\mbf{P}$,有
    \[
      \begin{pmatrix}
        a_{11} & \mbf{0} \\
        \mbf{0} & \A_2
      \end{pmatrix} =
      \mbf{P}\tr\A\mbf{P}
    \]
    考虑$\mbf{P}\tr\A\mbf{P}$的特征值$\lambda$和对应的特征向量
    $(x_1, \mbf{x}\tr)\tr$. 对于$\lhs$,成立
    \[
      \begin{pmatrix}
        a_{11} & \mbf{0} \\
        \mbf{0} & \A_2
      \end{pmatrix}
      \begin{pmatrix}
        x_1 \\
        \x
      \end{pmatrix}=
      \begin{pmatrix}
        a_{11}x_1 \\
        \A_2\x
      \end{pmatrix} = \lambda
      \begin{pmatrix}
        x_1 \\
        \x
      \end{pmatrix}
    \]
    即$\lambda$也是$\A_2$的特征值. 对于$\rhs$,成立
    \[
      \mbf{P}\tr\A\mbf{P}\x = \lambda\x
      \quad\Rightarrow\quad
      \x\tr\mbf{P}\tr\A\mbf{P}\x = \x\tr\lambda\x
      \quad\Rightarrow\quad
      0 < \lambda\|\x\|_2^2
      \quad\Rightarrow\quad
      \lambda > 0.
    \]
    又由于$\A_2$是对称的,所以$\A_2$是对称正定阵. $\quad\blacksquare$

\vspace{1cm}
\par\noindent 4. 试推导矩阵$\A$的Crout分解……
\ans
  只需要利用$\A\tr = \mbf{U}\tr\mbf{L}\tr$,即可归化为Doolittle分解.
  因此只需把对应公式中的$u$和$l$,以及下标互换即可. 所以计算公式为,
  \[\begin{split}
    & l_{i1} = a_{i1},\quad u_{1i} = a_{1i} / l_{11},\\
    & l_{ir} = a_{ir} - \sum_{k=1}^{r-1} u_{kr}l_{ik}, \\
    & u_{ri} = \left( a_{ri} - \sum_{k=1}^{r-1}u_{ki}l_{rk} \right) / l_{rr}.
    \quad\blacksquare
  \end{split}\]

\vspace{1cm}
\par\noindent 16. 设$\A$为非奇异矩阵,求证……
\proof
  因为$\A$非奇异,所以对任意$0\ne\x\in\R^n$,存在$\y\ne 0$,
  成立$\A\y = \x$,同时当$\x$取遍$\R^{n}\backslash\{0\}$时,
  $\y$也取遍$\R^{n}\backslash\{0\}$,且反之亦成立. 同时由于
  $\|\A\|_{\infty} = \max_{1\le i\le n}\sum_{j=1}^n|a_{ij}|$,
  所以最值总是可以取到的,且$\A$非奇异,所以不为零. 因此有
  \begin{equation}
    \label{equ1}
    \|\A^{-1}\|_\infty = \max_{\x\ne 0}\frac{\|\A^{-1}\x\|_\infty}{\|\x\|_\infty}
     = \max_{\y\ne 0}\frac{\|\A^{-1}(\A\y)\|_\infty}{\|\A\y\|_\infty}
     = \max_{\y\ne 0}\frac{\|\y\|_\infty}{\|\A\y\|_\infty}.
  \end{equation}
  根据\equref{equ1},有
  \[
    \frac{1}{\|A^{-1}\|_{\infty}}
    = \min_{\y\ne 0}\frac{\|\A\y\|_{\infty}}{\|\y\|_{\infty}}.
    \quad\blacksquare
  \]

\vspace{1cm}
\par\noindent 20. 是$\A,\mbf{B}\in\R^{n\times n}$……
\proof
  \[\begin{split}
    \text{cond}(\A\mbf{B}) &= \|(\A\mbf{B})^{-1}\|\|(\A\mbf{B})\| \\
    &\le \|\A^{-1}\|\|\A\|\,\|\mbf{B}^{-1}\|\|\mbf{B}\|
    = \text{cond}(\A)\text{cond}(\mbf{B}).
    \quad\blacksquare
  \end{split}\]


\end{document}
