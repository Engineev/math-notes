\documentclass[12pt, a4paper]{article}
\usepackage{ctex}

\usepackage[margin=1in]{geometry}
\usepackage{
  color,
  clrscode,
  amssymb,
  ntheorem,
  amsmath,
  listings,
  fontspec,
  xcolor,
  supertabular,
  multirow,
  mathtools,
  mathrsfs
}
\definecolor{bgGray}{RGB}{36, 36, 36}
\usepackage[
  colorlinks,
  linkcolor=bgGray,
  anchorcolor=blue,
  citecolor=green
]{hyperref}
\newfontfamily\courier{Courier}

\theoremstyle{margin}
\theorembodyfont{\normalfont}
\newtheorem{thm}{定理}
\newtheorem{cor}[thm]{推论}
\newtheorem{pos}[thm]{命题}
\newtheorem{lemma}[thm]{引理}
\newtheorem{defi}[thm]{定义}
\newtheorem{std}[thm]{标准}
\newtheorem{imp}[thm]{实现}
\newtheorem{alg}[thm]{算法}
\newtheorem{exa}[thm]{例}
\newtheorem{prob}[thm]{问题}
\DeclareMathOperator{\sft}{E}
\DeclareMathOperator{\idt}{I}
\DeclareMathOperator{\spn}{span}
\DeclareMathOperator*{\agm}{arg\,min}
\newcommand{\pr}{\prime}
\newcommand{\tr}{^\intercal}
\newcommand{\st}{\text{s.t.}}
\newcommand{\hp}{^\prime}
\newcommand{\ms}{\mathscr}
\newcommand{\mn}{\mathnormal}
\newcommand{\tbf}{\textbf}
\newcommand{\mbf}{\mathbf}
\newcommand{\fl}{\mathnormal{fl}}
\newcommand{\f}{\mathnormal{f}}
\newcommand{\g}{\mathnormal{g}}
\newcommand{\R}{\mathbf{R}}
\newcommand{\Q}{\mathbf{Q}}
\newcommand{\JD}{\textbf{D}}
\newcommand{\rd}{\mathrm{d}}
\newcommand{\str}{^*}
\newcommand{\vep}{\varepsilon}
\newcommand{\lhs}{\text{L.H.S}}
\newcommand{\rhs}{\text{R.H.S}}
\newcommand{\con}{\text{Const}}
\newcommand{\oneton}{1,\,2,\,\dots,\,n}
\newcommand{\aoneton}{a_1a_2\dots a_n}
\newcommand{\xoneton}{x_1,\,x_2,\,\dots,\,x_n}
\newcommand\thmref[1]{定理~\ref{#1}}
\newcommand\lemmaref[1]{引理~\ref{#1}}
\newcommand\defref[1]{定义~\ref{#1}}
\newcommand\posref[1]{命题~\ref{#1}}
\newcommand\secref[1]{节~\ref{#1}}
\newcommand\equref[1]{(\ref{#1})}
\newcommand\figref[1]{图 \ref{#1}}
\newcommand\corref[1]{推论~\ref{#1}}
\newcommand\exaref[1]{例~\ref{#1}}
\newcommand\algref[1]{算法~\ref{#1}}
\newcommand{\remark}{\paragraph{评注}}
\newcommand{\example}{\paragraph{例}}
\newcommand{\proof}{\paragraph{证明}}


\title{科学计算作业$\,$练习$2$}
\author{\small 任云玮\\\small2016级ACM班\\\small516030910586}
\date{}

\begin{document}
\maketitle

\begin{thm}[唯一性]
  \label{thm1}
  设$p, q\in P_n$在$x_i(i=0,\dots,n)$处相等,则$p\equiv q$.
\end{thm}
\textbf{证明 }
  设$r(x) = p(x) - q(x)$,它在$x=x_i$处值为$0$,设
  $r(x) = \sum_{i=0}^na_ix^i$,则有
  \[
    AX =
    \begin{pmatrix}
      1 & x_0 & \cdots & x_0^n \\
      1 & x_1 & \cdots & x_1^n \\
      \vdots & \vdots & & \vdots \\
      1 & x_n & \cdots & x_n^n \\
    \end{pmatrix}
    \begin{pmatrix}
      a_0 \\
      a_1 \\
      \vdots \\
      a_n
    \end{pmatrix} = 0
  \]
  $A$为Vandermonde矩阵,由于$x_i$互异,所以$\det A\ne0$,从而$X=0$,
  即$p(x) - q(x) = r(x)\equiv 0$,$p\equiv q$。$\blacksquare$

\begin{lemma}
  \label{lemma2}
  若$\f(x)$是次数为$n$的多项式,则当$k>n$时,$\f[x_0,\dots,x_k] = 0$.
  且$\f[x_0,\dots,x_n] = a_n$,$a_n$为$n$次项的系数。
\end{lemma}
\textbf{证明 }
  对$\f(x)$插值,由于$\f\in P_n$,所以成立
  \[\begin{split}
    \f(x) =& \f[x_0] + \cdots + \f[x_0,\dots,x_n](x-x_0)\cdots(x-x_{n-1}) \\
    &+ \f[x_0,\dots,x_{n+1}](x-x_0)\cdots(x-x_n) + \cdots \\
    &+ \f[x_0,\dots,x_k](x-x_0)\cdots(x-x_{k-1}).
  \end{split}\]
  因为$\f\in P_n$,所以$x^k(k>n)$项系数为零,同时比较$n$次项系数,得
  \[\begin{split}
    \f[x_0,\dots,x_k] &= 0,\\
    \f[x_0,\dots,x_n] &= a_n.\quad\blacksquare
  \end{split}\]

\vspace{1cm}
\noindent4. 设$x_j$为互异节点……
\ans
 (1) 成立$$\lhs(x_i)=\sum_{j=0}^nx_j^k\delta_{ij}=x_i^k=\rhs(x_i),$$
  即两边在$x_i$处相等,根据\thmref{thm1},$\lhs\equiv\rhs.\quad\blacksquare$\\
 (2) 成立
  \[
    \lhs(x_i) = \sum_{j=0}^n(x_j-x)^k\delta_{ij} = (x_j-x_j)^k = 0
    = \rhs(x_i),
  \]
  即两边在$x_i$处相等,根据\thmref{thm1},$\lhs\equiv\rhs.\quad\blacksquare$

\vspace{1cm}
\par\noindent8. $\f(x) = x^7 + x^4 + 3x + 1$,……
\ans
  设$x_k = 2^k(k = 0,\dots,8)$,对$\f$进行插值,有
  \[\begin{split}
    \f(x) =& \f[x_0] + \f[x_0,x_1](x-x_0) + \cdots +
    \f[x_0,\dots,x_7](x-x_0)\cdots(x-x_6)\\
    &+ \f[x_0,\dots,x_8](x-x_0)\cdots(x-x_7)
  \end{split}\]
  由于$\f$为$7$次多项式,所以根据\lemref{lemma2},得
  \[
    \f[x_0,\dots,x_7] = 1,\quad
    \f[x_0,\dots,x_8] = 0.\quad\blacksquare
  \]

\vspace{1cm}
\par\noindent9. 证明$\Delta(\f_k\g_k)=$……
\ans
  \[\begin{split}
    \Delta(\f_k\g_k) &= \f_{k+1}\g_{k+1} - \f_k\g_k
    = \f_{k+1}\g_{k+1} - \f_k\g_{k+1} + \f_k\g_{k+1} - \f_k\g_k \\
    &= \g_{k+1}(\f_{k+1}-\f_k) + \f_k(\g_{k+1} - \g_k)
    = \f_k\Delta\g_k + \g_{k+1}\Delta\f_k.\quad\blacksquare
  \end{split}\]

\vspace{1cm}
\par\noindent10. 证明Abel变换
\ans
  \[\begin{split}
    \sum_{k=0}^{n-1}\f_k\Delta\g_k &=
    \f_{n-1}\g_n + \sum_{k=0}^{n-2}\f_k\g_{k+1}
    - \sum_{k=1}^{n-1}\f_k\g_k - \f_0\g_0 \\
    &= (\f_{n-1}-\f_n+\f_n)\g_n + \sum_{k=0}^{n-2}\f_k\g_{k+1}
    - \sum_{k=0}^{n-2}\f_{k+1}\g_{k+1} - \f_0\g_0 \\
    &= -\Delta\f_{n-1}\g_n + \f_n\g_n +
    \sum_{k=0}^{n-2}(\f_k - \f_{k+1})\g_{k+1} - \f_0\g_0 \\
    &= \f_n\g_n - \f_0\g_0 - \sum_{k=0}^{n-1}\g_{k+1}\Delta\f_k.
    \quad\blacksquare
  \end{split}\]
\correct
  此题更加方便的证明方法是利用第9题的结论,移项后两边求和. 这一
  做法实际上是由“Abel变换是离散分部积分”而自然导出的.

\vspace{1cm}
\par\noindent12. 若$\f(x) = a_0 + a_1x + \cdots + a_nx^n$有……
\ans
  根据条件可知,$\f(x_j) = 0$,$\f(x) = A(x-x_1)\cdots(x-x_n)$.
  \[\begin{split}
   \lhs &= \sum_{j=1}^m\lim_{x\to x_j} \frac{x_j^k(x-x_j)}{\f(x)-\f(x_j)} \\
   &= \sum_{j=1}^m\lim_{x\to x_j}\frac{x_j^k(x-x_j)}{A(x-x_1)\cdots(x-x_n)} \\
   &= \frac{1}{A}\sum_{j=1}^m\frac{x_j^k}{\prod_{i\ne j}(x_j-x_i)}
  \end{split}\]
  记$\g_k(x) = x^k/A$,则$\lhs = \g_k[x_1,\dots,x_n]$,因为
  $\g_k\in P_k$,所以根据\lemref{lemma2},当$k<n-1$时候,
  $\lhs = 0$. 当$k=n-1$时,$\g_k[x_1,\dots,x_n]$为$x^k$项系数,即$\lhs=A^{-1}$.
  又
  \[
    A(x-x_1)\cdots(x-x_n) = a_0 + \cdots + a_nx^n,
  \]
  对上式两端求$n$阶导数,得到$A = a_n$. 从而此时$\lhs=a_n^{-1}.\quad\blacksquare$

\vspace{1cm}
\par\noindent1. 设$\f(x)\in P_n$,且对$k = 0,\,1,\,\dots,\,n$成立
  $\f(k) = \dfrac{k}{k+1}$,求f(x)。
\ans
  设
  \[
    \g(x) = (x+1)\f(x) - x\in P_{n+1}.
  \]
  根据条件,可知$0,1,\dots,n$是$\g(x)$的$n+1$个零点,所以有
  \[
    \g(x) = Ax(x-1)\cdots(x-n).
  \]
  根据上面两式,成立
  \[
    (x+1)\f(x) = Ax(x-1)\cdots(x-n) + x
  \]
  下确定恰当的$A$,使得$x=-1$是$\rhs$的一个根,从而$\rhs/(x+1)\in P_n$.
  令$\rhs(-1)=0$,得
  \[
    A(-1)(-2)\cdots(-1-n) - 1 = 0 \quad\Rightarrow\quad
    A = \frac{(-1)^{n+1}}{(n+1)!}
  \]
  综上,
  \[
    \f(x) = \frac{\frac{(-1)^{n+1}}{(n+1)!}x(x-1)\cdots(x-n)+x}{x+1}.
    \quad\blacksquare
  \]

\vspace{1cm}
\noindent2.任给节点$x_0 < x_1 <\cdots< x_n$,记$h = \max$……
\ans
  设$x\in[x_k, x_{k+1}]$,则对于该区间成立
  \begin{equation}
    \label{equ: x_k}
    |(x-x_k)(x-x_{k+1})| \le \left[\frac{(x_{k+1}-x) + (x-x_k)}{2}\right]^2
    \le \frac{(x_{k+1}-x_k)^2}{4} \le \frac{h^2}{4}.
  \end{equation}
  对于除去$k$和$k+1$以外的$n-1$个$j$,成立
  \begin{equation}
    \label{equ: x_j}
    |x-x_j| \le
    \begin{cases}
      (k-j+1)h,&\quad(j<k)\\
      (j-k)h,&\quad(j>k+1)
    \end{cases}
  \end{equation}
  分别取遍$2h,\dots,(k+1)h$和$2h,\dots,(n-k)h$。
  根据式\equref{equ: x_k}和\equref{equ: x_j},成立
  \[\begin{split}
    & |(x-x_0)(x-x_1)\cdots(x-x_k)(x-x_{k+1})\cdots(x-x_n)| \\
    \le& \frac{h^2}{4}\times n!h^{n-1} = \frac{n!h^{n+1}}{4}
    \quad\blacksquare
  \end{split}\]


  由于结论只与$x_k$之间的相对位置有关,所以不妨设$x_0 = 0$,下证等距的情况。
  根据齐次性,不妨设$h = 1$(否则令$y = x / h$即可)。所以只需证明
  \[
    x(x-1)\cdots(x-n) \le \frac{n!}{4}
  \]
  对于$n=1$的情况,根据均值不等式,成立,
  \[
    |x(x-1)| = x(1-x) \le \frac{1}{4}.
  \]
  假设对于$n$的情况成立,考虑$n+1$的情况。若$x\in[0, n]$,则
  \[
    \lhs \le \frac{n!}{4}\times(n+1-x) \le \frac{(n+1)!}{4}.
  \]
  若$x\in[n, n+1]$,结合$n=1$的情况,成立
  \[
    \lhs \le (n+1)n(n-1)\cdots\frac{1}{4} = \frac{(n+1)!}{4}
  \]

\end{document}
