\documentclass[12pt, a4paper]{article}
\usepackage{ctex}

\usepackage[margin=1in]{geometry}
\usepackage{
  color,
  clrscode,
  amssymb,
  ntheorem,
  amsmath,
  listings,
  fontspec,
  xcolor,
  supertabular,
  multirow,
  mathtools,
  mathrsfs
}
\definecolor{bgGray}{RGB}{36, 36, 36}
\usepackage[
  colorlinks,
  linkcolor=bgGray,
  anchorcolor=blue,
  citecolor=green
]{hyperref}
\newfontfamily\courier{Courier}

\theoremstyle{margin}
\theorembodyfont{\normalfont}
\newtheorem{thm}{定理}
\newtheorem{cor}[thm]{推论}
\newtheorem{pos}[thm]{命题}
\newtheorem{lemma}[thm]{引理}
\newtheorem{defi}[thm]{定义}
\newtheorem{std}[thm]{标准}
\newtheorem{imp}[thm]{实现}
\newtheorem{alg}[thm]{算法}
\newtheorem{exa}[thm]{例}
\newtheorem{prob}[thm]{问题}
\DeclareMathOperator{\sft}{E}
\DeclareMathOperator{\idt}{I}
\DeclareMathOperator{\spn}{span}
\DeclareMathOperator*{\agm}{arg\,min}
\newcommand{\pr}{\prime}
\newcommand{\tr}{^\intercal}
\newcommand{\st}{\text{s.t.}}
\newcommand{\hp}{^\prime}
\newcommand{\ms}{\mathscr}
\newcommand{\mn}{\mathnormal}
\newcommand{\tbf}{\textbf}
\newcommand{\mbf}{\mathbf}
\newcommand{\fl}{\mathnormal{fl}}
\newcommand{\f}{\mathnormal{f}}
\newcommand{\g}{\mathnormal{g}}
\newcommand{\R}{\mathbf{R}}
\newcommand{\Q}{\mathbf{Q}}
\newcommand{\JD}{\textbf{D}}
\newcommand{\rd}{\mathrm{d}}
\newcommand{\str}{^*}
\newcommand{\vep}{\varepsilon}
\newcommand{\lhs}{\text{L.H.S}}
\newcommand{\rhs}{\text{R.H.S}}
\newcommand{\con}{\text{Const}}
\newcommand{\oneton}{1,\,2,\,\dots,\,n}
\newcommand{\aoneton}{a_1a_2\dots a_n}
\newcommand{\xoneton}{x_1,\,x_2,\,\dots,\,x_n}
\newcommand\thmref[1]{定理~\ref{#1}}
\newcommand\lemmaref[1]{引理~\ref{#1}}
\newcommand\defref[1]{定义~\ref{#1}}
\newcommand\posref[1]{命题~\ref{#1}}
\newcommand\secref[1]{节~\ref{#1}}
\newcommand\equref[1]{(\ref{#1})}
\newcommand\figref[1]{图 \ref{#1}}
\newcommand\corref[1]{推论~\ref{#1}}
\newcommand\exaref[1]{例~\ref{#1}}
\newcommand\algref[1]{算法~\ref{#1}}
\newcommand{\remark}{\paragraph{评注}}
\newcommand{\example}{\paragraph{例}}
\newcommand{\proof}{\paragraph{证明}}


\title{概率论$\,$笔记}
\author{任云玮}
\date{}

\begin{document}
\maketitle
\tableofcontents
\newpage

\section{样本空间与概率}

\subsection{概率模型}

  \begin{defi}
    $\,$
    \begin{enumerate}
      \item 对于一次实验,定义其可能产生的结果的全体为\tbf{样本空间}$\Omega$.
      \item 称一个集合$A$为\tbf{事件},若它是样本空间$\Omega$的一个子集.
    \end{enumerate}
  \end{defi}
  \remark
    对于样本空间,在选取的时候需要注意结果需要是良定义的(无歧义的),同时需要
    实验的所有结果都在$\Omega$中.

  \begin{defi}[概率律]
    \label{defi: 概率律}
    设$\Omega$是一个样本空间,称定义在$\Omega$中事件全体上的函数$\p$
    为\tbf{概率律},若它成立
    \begin{enumerate}
      \item 非负性. 对任意事件$A$,$\p(A)\ge 0$.
      \item 可加性. 对任意不相交的$A$和$B$,成立$\p(A\cup B)=\p(A)+P(B)$.
        或者更一般的,对于两两互不相交的$\{A_n\}_{n=1}^\infty$,成立
        $\p(A_1\cup A_2\cup\cdots) = \p(A_1) + \p(A_2) + \cdots$.
      \item 归一性. $\p(\Omega)=1$.
    \end{enumerate}
  \end{defi}
  \remark
    显然成立$\p(\varnothing) = 0$. 另外,一般在讨论概率律的时候,
    不区分只包含一个结果的事件和该结果本身.

  \begin{thm}[概率律的性质]
    \label{thm: 概率律的性质}
    给定概率律$\p$,事件$A$,$B$,$C$,则
    \begin{enumerate}
      \item 若$A\subset B$,则$\p(A)\le \p(B)$.
      \item $\p(A\cup B) = \p(A) + \p(B) - \p(A\cap B)$.
      \item $\p(A\cup B) \le \p(A) + \p(B)$.
      \item $\p(A\cup B\cup C) = \p(A) + \p(A^c\cap B) +
        \p(A^c \cap B^c \cap C)$.
    \end{enumerate}
  \end{thm}
  \remark
    这些式子的证明都是trivial的,其中[3.]至少可以推广至有限个事件. 对于
    [4.],它实际上演示了一个将重合的事件拆分成不相交事件的方法.

  \begin{lemma}[Bonferroni不等式]
    设有事件$A_1, A_2, \dots, A_n$,则成立
    \[
      \p\left(\bigcap_{i=1}^n A_i\right) \ge \sum_{i=1}^n\p(A_i) - (n-1).
    \]
  \end{lemma}
  \proof
    对$n$施归纳法,利用\thmref{thm: 概率律的性质} [2.]即可.

  \begin{thm}[容斥原理]
    设$A_1,A_2,\dots, A_n$为样本空间$\Omega$的事件,则
    \[
      \p\left( \bigcup_{i=1}^n A_i \right) =
      \sum_{k=1}^n (-1)^k
      \sum_{\substack{I\subset{1,\dots,n}\\|I|=k}}\p(A_I).
    \]
  \end{thm}

  \begin{thm}[连续概率]
    设有事件序列$\{A_n\}_{n=1}^\infty$,成立$A_n\subset A_{n+1}$. 令
    $A=\bigcup_{n=1^\infty}A_n$,则$\p(A)=\lim_{n\to\infty}\p(A_n)$.
  \end{thm}
  \proof
    考虑将$\p(\bigcup_{n=1}^\infty A_n)$拆分成级数的形式. 定义$B_0=\varnothing$,
    $B_n = A_n - A_{n-1}$,则只需证明$\bigcup_{k=1}^\infty B_k = A$,再利用
    \defref{defi: 概率律}拆分$\lhs$即可.$\quad\blacksquare$
  \remark
    对于“单调减”的事件序列,把并换成交,可以有类似的结论.

\end{document}
